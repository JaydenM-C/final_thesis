\chapter[Introduction]{Introduction}
\label{Chap:Intro}

% ***************************************************
% Introduction
% ***************************************************

\hypertarget{intro-opening}{%
\section{Phonotactics and phylogeny}\label{intro-opening}}

Can the minutiae of sound sequences in human language encode something of human history? And how can we tell? This thesis presents a methodology for interrogating novel data sources in quantitative historical linguistics and finds that the answer may be yes, though extracting it is a complex challenge.

Quantitative advances in historical linguistics mean that researchers are able to extract fine-grained historical patterns from large datasets and infer increasingly detailed phylogenies of the world's language families \autocites[see][]{dunn_language_2015}{bowern_computational_2018}. However, quantitative historical linguistics remains predominantly reliant on lexical cognate data. Lexical cognate identification has been a cornerstone of historical linguistic methodology for over a century and a half, but requires extensive human expertise and manual labour, creating a bottleneck in current quantitative historical linguistic practice. It also represents but one subpart of the human linguistic system. There has been some interest in phylogenetic tree inference using structural features of languages \autocites[e.g.][]{dunn_structural_2005}{dunn_structural_2008}{reesink_explaining_2009} but the limited parameter space and apparent fast rate of change of many grammatical features can cloud historical signal and make phylogenetic inference difficult \autocites{reesink_systematic_2012}{greenhill_evolutionary_2017}.

This thesis presents an empirical evaluation of the inferential capacity of phonotactics---the system of constraints defining how phonemes may fit together in linear sequences in a language---in quantitative historical linguistics. The motivation to consider phonotactics as a data source in historical linguistic study is two-fold. Firstly, there is a practical element. Phonotactic information can be extracted relatively simply by examining the sequences of segments that appear in a language's lexicon. Given a set of comparably segmented wordlists from a variety of languages, a comparative dataset can be harvested rapidly using automated means. In the simplest case, this could be a dataset simply coding the presence or absence of sequences of two segments, or \emph{biphones}. Frequencies of sequences can also be extracted, which are shown to contain finer grained, informative variation in Chapter \textbf{CHAP REF}. It would also be possible to extract automatically more sophisticated forms of data, encoding syllable structure for example, though this is beyond the scope of this thesis (though see Chapter \textbf{CHAP REF} for further discussion of future directions). This relatively low resource requirement is advantageous in under-described parts of the world, such as Australia, which is the focus of this thesis. The second motivation is more theoretical. A prerequisite for reconstructing phylogeny from some novel data source is, naturally, that the data source contains historical signal \autocite{dunn_language_2015}. The hypothesis that phonotactics might potentially be a source of historical signal is based on evidence that languages often adapt loanwords to fit within the existing phonotactic system such that novel words will reflect phonotactic tendencies that are present in the already-existing lexicon \autocites{hyman_role_1970}{silverman_multiple_1992}{crawford_adaptation_2009}{kang_loanword_2011}, and thus a language's phonotactic system could be expected to remain historically conservative even in instances of lexical borrowing, which is a source of noise in lexical data. A language's phonotactic system is also unaffected by instances of undetected semantic shift, which is another source of noise in lexical data. Finally, with regards to phonotactic frequencies specifically, there is evidence that speakers are sensitive to phonological frequencies generally, and this has a role in lexical innovation \autocites{coleman_stochastic_1997}{albright_rules_2003}{hayes_stochastic_2006}[see also][pp.~20--21]{gordon_phonological_2016}.

The question this thesis seeks to address, in simple terms, is this: Does phonotactic data help or hinder the task of phylogenetic tree inference? As just described, there are reasons to expect that it should help. But this immediately raises the question: What does it mean to `help' phylogenetic tree inference, and how can we know whether it helps or not? Unlike, say, the evaluation of a novel drug or vaccine, where it is possible to link a test condition to an observable outcome, it is not as immediately clear how to measure the effect of a novel methodology in linguistic phylogenetics, where the target outcome (i.e.~the true historical phylogeny) is in the non-observable past. In service of the main aims, this thesis presents a step-by-step process for empirically interrogating new data in a linguistic phylogenetic context, moving from theoretical speculation that the data contains phylogenetic signal to mathematical operationalisation of language change processes, implementation in phylogenetic models and, finally, evaluation using model comparison methods.

This thesis concentrates on languages of Australia, starting with an Australia-wide sample of 166 languages in Chapter \textbf{CHAP REF} and narrowing to 44 languages of the western branch of the Pama-Nyungan family in \textbf{CHAP REF}. The methodological reasons for shrinking the language sample will be elaborated on in the relevant chapters ahead. The Pama-Nyungan family, far and away the largest and most geographically expansive language family in Australia, has been well established on traditional historical linguistic grounds and its phylogenetic structure has been inferred using computational phylogenetic methods and lexical cognate data \autocites{bowern_computational_2012}{bouckaert_origin_2018}. However, deep-time relationships between Pama-Nyungan and the various smaller Australian language families in the north are not well established. Even more ancient relationships between Australian languages and the rest of Sahul---the continent of Australia and New Guinea, which has constituted a single landmass for most of human history---is subject to intrigue and conjecture. I discuss this background further in Chapter \textbf{CHAP REF} and argue that linguistics is an important component within the multidisiciplinary efforts that will be required to further complete this picture. Returning to the present study, though, the primary goal here is not to increase our understanding of the phylogeny of Pama-Nyungan or Australian languages per se. Rather, Australian languages and Pama-Nyungan languages in particular serve as an excellent test case since data availability is good \autocites[lexical data in][]{bowern_chirila_2016}{round_ausphon_2017} and established phylogenies exist to serve as a yardstick or point of comparison \autocites{bowern_computational_2012}{bouckaert_origin_2018}. Theoretically too, Australian languages serve as an interesting test case. Australian languages have been described frequently as being remarkably homogenous in their phonologies and phonotactics \autocites{round_segment_2021}{round_phonotactics_2021} (see \textbf{CHAP REF} for more discussion and references). This should make for a deliberately difficult test case, since such homogeneity would be expected to cloud phylogenetic signal.

This thesis finds that phonotactics, extracted at the relatively simple level of binary and frequency-based biphone characters, do show evidence of containing phylogenetic signal. However, a more complex story unfolds amidst the challenge of extracting it. On the main question of whether phonotactic data `helps' phylogenetic inference, the results at this stage are indeterminate. The phylogenetic models with phonotactic data that I test largely fail to converge, and produce marginal likelihood estimates (which indicate model fit) that are unreliable. There are two possible interpretations of this, which are not mutually exclusive. One is that the phonotactic data are genuinely non-tree-like and introduce noise to the model. The other is that the implementation of an evolutionary model for phonotactic data is inadequate. The latter seems possible, even likely, since the evolutionary model had to be compromised a great deal in an effort to keep the project computationally feasible. The challenge of designing an evolutionary model that is simultaneously faithful to the real-world evolutionary processes of phonotactics while remaining computationally realistic is difficult though perhaps not entirely intractable. I outline further directions on this topic in Chapter \textbf{CHAP REF}.

This thesis nevertheless contributes to the field a paradigm for empirical data innovation in linguistic phylogenetics. The set of steps employed in the following chapters can be generalised to any novel kind of linguistic data for which there is a reasoned hypothesis that historical signal might be present. This starts with a sound statistical (re-)evaluation of how the data is distributed and how this fits causal linguistic processes. Next, phylogenetic signal needs to be quantified in an area where an independent reference phylogeny already exists as a yardstick. The detection of phylogenetic signal does not in itself indicate the data can be used for phylogenetic tree inference. But, in combination with a well-founded hypothesis on causal evolutionary processes, it would justify further testing. Next, the causal processes that shape the data need to be operationalised in an evolutionary model, then phylogenetic tree inference can take place. The overall fit of a phylogenetic tree model which includes the novel data can be compared to the fit of an equivalent model with separate tree elements for novel and existing data, giving an indication of whether the novel data strengthens phylogenetic tree inference.

\hypertarget{intro-research-question}{%
\section{Research question}\label{intro-research-question}}

The primary research question this thesis seeks to address is whether or not phonotactic data, in conjunction with lexical cognate data, can strengthen the inference of linguistic phylogenies. However, since it is not possible to directly observe the past and see whether a phylogenetic tree inferred with phonotactic data more closely approaches the true historical phylogeny, this presents a challenge of testability. What does it mean to strengthen phylogenetic tree inference? In this thesis, I operationalise this question in terms of model fit. The main research question can then be restated more precisely as follows. Does the fit of a phylogenetic model improve when the phylogenetic tree is inferred from cognate data and phonotactic data together, compared to an equivalent model in which trees are inferred from cognate data and phonotactic data separately? Model fit, in this context, is inferred by estimating marginal likelihoods and, secondarily, comparing support values for clades in each tree.

Underlying this is the more general question of how to approach and evaluate empirically novel data sources in linguistic phylogenetic research, in particular continuous-valued characters such as frequencies. If one has a well-reasoned hunch that some data may contain informative phylogenetic information, how should one test this hypothesis? How should one progress from this hypothesis to implementation in a phylogenetic model and eventual tree inference, if no one has inferred trees from this kind of data before? In the chapters that follow, I illustrate this progression. In the course of doing so, I address some specific research questions along the way: (1) What mathematical distribution best characterises the observed frequencies of a language's phonemes? And what links, if any, can be inferred between phoneme frequency distributions and causal sound/language change processes that shaped them? (2) What methods exist for identifying phylogenetic information (or, more precisely, \emph{phylogenetic signal}) in comparative datasets? (3) Does phonotactic frequency data contain phylogenetic signal? And, finally, (4) do marginal likelihoods and clade support values support a phylogeny inferred from lexical cognate data and phonotactic frequency data jointly over a comparable model in which trees are inferred from cognate data and phonotactic data separately? What is the effect of adding phonotactic frequency data to a phylogenetic tree model?

\hypertarget{intro-chapter-outline}{%
\section{Chapter outline}\label{intro-chapter-outline}}

This thesis presents four research papers, completed by a literature review chapter at the start, and discussion and conclusion chapters at the end. Each of the four research chapters are either published, accepted for publication or in preparation for publication independently from one another. Together, however, these papers demonstrate a stepwise process of empirically evaluating and implementing a novel source of data in quantitative historical linguistics. The discussion chapter ties these parts together and outlines a future research paradigm for data innovation in quantitative historical linguistics. I briefly introduce each remaining chapter below.

Chapter \textbf{CHAP REF} places the thesis within its scientific domain by reviewing the history of the historical linguistics field, particularly with regards to methodological development and the rise of phylogenetic linguistic methods in recent decades. Subsequently, I give a background overview of Australian languages and their phonological systems. The final section widens the scope to frame the previous discussion within the context of the deep-time human history of Sahul---the continent of Australia and New Guinea. This is a fascinating area of multidisciplinary endeavour, and it will be shown that historical linguistic investigation of Australian languages is an essential component. The chapter concludes with a discussion of how present limitations of existing works motivate the research aims of this thesis.

Chapter \textbf{CHAP REF} presents a re-evaluation of phoneme frequencies in the lexicons of 166 Australian languages. The motivation for this re-evaluation is two-fold. Firstly, phoneme frequency distributions are the stationary outcomes of diachronic, causal processes. Developing a sound understanding of these distributions can, therefore, improve our understanding of the processes that shaped them. Secondly, the identification of putative power law distributions, which includes the famous Zipfian distribution, has undergone major revision across the sciences in recent times as traditional methods for power law identification were shown to be unreliable \autocite{clauset_power-law_2009}. I use a more reliable maximum likelihood statistical framework to evaluate whether phoneme frequencies are fitted best by a power law distribution or one of several alternative, heavy-skewed distributions (lognormal, exponential and Poisson). The results largely confirm an earlier study of phoneme frequencies in mostly European languages \autocite{tambovtsev_phoneme_2007} but also add qualification and nuance. This is followed by reasoned discussion on potential links between these results and causal processes in phonological evolution.

Chapter \textbf{CHAP REF} reviews literature on the topic of phylogenetic autocorrelation---similarity due to phylogenetic relatedness---in linguistics and comparative biology. The paper compares and contrasts methodological approaches to phylogenetic autocorrelation in both fields, introduces phylogenetic comparative methods and discusses the concept of phylogenetic signal---the tendency of closely related languages/species to appear similar to one another to a greater degree than expected by chance. I argue for the continued uptake of phylogenetic comparative methods in comparative fields of linguistics and foreshadow the methodological approach that follows in Chapter 5.

Chapter \textbf{CHAP REF} measures phylogenetic signal in several kinds of phonotactic data harvested from the lexicons of 111 Pama-Nyungan languages. Three datasets are tested using diagnostic statistics developed by \textcite{blomberg_testing_2003}: (1) Binary variables recording the presence or absence of biphones (two-segment sequences) in a lexicon, (2) frequencies of transitions between segments, and (3) frequencies of transitions between natural sound classes. The results indicate that a statistically significant phylogenetic signal is present in all three datasets, but in varying degrees. The binary dataset (unsurprisingly) contains the weakest phylogenetic signal. The finer-grained frequency datasets contain stronger signal, and signal in the sound class dataset in particular is strongest. The presence of non-random, phylogenetic tree-like structure within these relatively simple representations of phonotactic systems gives support to the notion that phonotactic data could be used profitably for novel phylogenetic tree inference in linguistics.

Chapter \textbf{CHAP REF} puts into practice the findings of previous chapters and attempts to test the main research question of this thesis. I attempt to infer a phylogeny of 44 western Pama-Nyungan languages using a combination of existing lexical cognate data from \textcite{bouckaert_origin_2018}, binary biphone data and sound class transition frequency data. Two phylogenetic models are evaluated using a Bayesian Markov chain Monte Carlo process: (1) A `linked' model in which a single tree is inferred from all three datasets together, and (2) a `separate' model in which two trees are inferred---one from cognate data only and the other from the two phonotactic datasets only. The two models are compared firstly by comparing marginal likelihood estimates, which give an overall indication of model fit, and secondly by comparing maximum clade credibility trees and posterior clade support values. The present study returns results that are indeterminate overall. Inferring trees using continuous valued characters at scale is computationally non-trivial and I am unable to infer consistent, stable results within current constraints of the model. I provide a breakdown of these limitations and detail the steps required to investigate the role of phonotactics in phylogenetics further.

Tying together each of the previous chapters, Chapter \textbf{CHAP REF} proposes a future research program for the empirical interrogation of novel data sources in linguistic phylogenetics. Beyond the immediate steps necessary to further investigate phonotactics specifically, I suggest there is a more generalisable set of principles which could be applied to linguistic frequency data of other kinds. This would start with a well-reasoned hypothesis for how the frequency characters would change, tested empirically through statistical evaluation of the shape of frequency distributions. The next prerequisite is a well-reasoned hypothesis for why the frequency characters are expected to be historically conservative, followed by a statistical evaluation of whether this is the case via phylogenetic comparative methods. Finally, knowledge of how the frequency characters shift through time needs to be built into a phylogenetic model for eventual implementation in phylogenetic tree inference.

The thesis concludes with a brief summary in Chapter \textbf{CHAP REF}.

Text-based supplementary information, such as bibliographic details for original language sources, explanatory notes on code and data and ancillary information on methods, is given in the appendices. Extra supplementary information, such as data, code and raw results files, are made available in online repositories. Where applicable, the details of these repositories are given at the start of each chapter. Each study is designed to be fully reproducible and uses only free, open source software throughout.

% ***************************************************