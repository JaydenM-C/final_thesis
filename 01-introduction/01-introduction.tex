\chapter[Introduction]{Introduction}
\label{Chap:Intro}

% ***************************************************
% Introduction
% ***************************************************

A study of the evolution of structural linguistic diversity. In this instance, phonotactics.

This will be a short, sharp introduction. No more than a few pages, opening with an enticing research question. Est. 2 days' work to write and polish.

Thesis statement

Rationale and position of the work in its field

\hypertarget{chapter-outline}{%
\subsection{Chapter outline}\label{chapter-outline}}

Chapter 1 introduces the thesis and outlines the motivations for a phylogenetic phonotactic research program. Chapter 2 places the subsequent papers within two centuries of historical linguistic tradition and also broadly contextualises this research within studies of early human history.

Chapter 3 presents a re-evaluation of the distributions claimed to characterize phoneme frequencies, in the wake of a major debate in the sciences around power-law hypotheses and the unreliability of earlier methods of evaluating them. This paper compares the fit of power laws and three alternative distributions to 168 Australian languages, using a maximum likelihood framework. It presents evidence supporting earlier results, but also qualifying and nuancing them. Most notably, phonemic inventories appear to have a Zipfian-like frequency structure among their most-frequent members (though perhaps also a lognormal structure) but a geometric (or exponential) structure among the least-frequent. These findings hold implications for accounts of the causal mechanisms that generate phoneme frequency distributions, which are considered when evaluating evolutionary models for frequency-based phonotactic data in later chapters (in particular, Chapter 6).

Chapter 4 turns attention to the task of comparison when phylogenetic relationships exist between entities being compared. Phylogeny has long been recognised as a source of potential bias to control in linguistic typology. This chapter critically reviews the sampling methods traditionally used to control for phylogeny in linguistic typology and compares and contrasts these approaches to sampling methods and more recent computational methods in comparative biology. It describes phylogenetic comparative methods and, in particular, methods for measuring phylogenetic signal, which is an important first step in a phylogenetic comparative approach. It advocates for increased phylogenetic awareness in comparative linguistics and, in particular, increased uptake of phylogenetic comparative methods over balanced sampling methods.

Chapter 5 presents an implementation of the phylogenetic signal detection methods outlined in Chapter 4. It shows how a phylogenetic approach opens the possibility of gaining historical insights from entirely new kinds of linguistic data---in this instance, statistical phonotactics. The chapter presents results of testing for phylogenetic signal in 3 datasets, each extracted from a sample of 111 Pama-Nyungan languages: (1) binary variables recording the presence or absence of \emph{biphones} (two-segment sequences) in a lexicon (2) frequencies of transitions between segments, and (3) frequencies of transitions between natural sound classes. Phylogenetic signal is higher in finer-grained frequency data than in binary data, and highest in natural-class-based data. These results demonstrate the viability of employing a new source of readily extractable data in historical and comparative linguistics.

Chapter 6 takes the encouraging findings of Chapter 5 and attempts to put the same data into practice, in the service of phylogenetic tree inference. The study tests the question of whether phylogenetic tree inference is strengthened with the inclusion of phonotactic data. A phylogeny of the Pama-Nyungan family is inferred twice using Bayesian computational methods: Once in which the phylogeny is inferred jointly using phonotactic data and lexical cognate data partitions, and once more using two distinct tree partitions, one inferred using phonotactic data only and one inferred using cognate data only (all other model parameter settings being kept the same). Support for the combined model over the separated model is compared by calculating a Bayes factor from the log marginal likelihood estimates for each model. The study finds that combination of phonotactic data with lexical data \textbf{does/does not} significantly strengthen tree inference.

In Chapter 7, all the lessons of the previous four chapters are drawn together to outline short-term future research priorities and a longer-term program of linguistic phylogenetic research with phonotactics. The papers presented in this thesis largely serve as methodological groundwork, in which results are compared to pre-existing benchmarks. Chapter 7 discusses how to expand this research in the pursuit of new insights into deep-time linguistic and human history in Sahul and beyond. Chapter 8 concludes the thesis.

% ***************************************************