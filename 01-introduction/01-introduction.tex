\chapter[Introduction]{Introduction}
\label{Chap:Intro}

% ***************************************************
% Introduction
% ***************************************************

A study of the evolution of structural linguistic diversity. In this instance, phonotactics.

This will be a short, sharp introduction. No more than a few pages, opening with an enticing research question. Est. 2 days' work to write and polish.

Thesis statement

Rationale and position of the work in its field

\hypertarget{chapter-outline}{%
\subsection{Chapter outline}\label{chapter-outline}}

Chapter 2 places thesis in academic context. Historical linguistics, linguistic phylogenetics, science of human history. Gives background on Sahul, linguistic area under investigation.

Chapter 3, Re-evaluating phoneme frequencies.

Chapter 4, phylogenetic comparative methods (PCMs) in linguistics. Introduces phylogenetic signal, its importance for comparative study and methods to detect it in difference kinds of data.

Chapter 5 measures phylogenetic signal in phonotactic data.

Chapter 6 uses said phonotactic data to infer Pama-Nyungan phylogeny jointly with cognate data.

Chapter 7 oulines future research program based on these findings.

Chapter 8 concludes.

% ***************************************************