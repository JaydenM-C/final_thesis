% ***************************************************
% Discussion
% ***************************************************
\chapter[Towards phylogenetic phonotactics]{Towards a phylogenetic phonotactic research program}
\label{Chap:Discussion}

% ********* Enter your text below this line: ********
The motivating research question posed at the start of this thesis was whether or not phylogenetic tree inference could be strengthened by complementing traditional, lexical cognate data with phonotactic data. Given the indeterminate results of the previous chapter, the answer to this question remains elusive. However, there is still much to be gleaned from this experience and general lessons that can be applied to future quantitative interrogations of novel data sources in historical linguistics. This chapter begins with a brief summary of each of the previous chapters, followed by an explanation of how the four separate studies presented cohere into a stepwise interrogation of a novel data source, phonotactics, in linguistic phylogenetics. I present the novel insights that can be drawn from this evaluation. Follow that, I address limitations of the thesis. I turn attention to future research directions and point to possible solutions to limitations that could be pursued in subsequent study. I note how the scope of this present work could be expanded to cover the Sahul continent and the challenges associated with this. Finally, I sketch an outline of the future of methodological development in linguistic phylogenetics and propose a research paradigm in which this could be done.

\hypertarget{discussion-intro}{%
\section{Summary so far}\label{discussion-intro}}

The preceding five chapters began with a literature review followed by four papers (three experimental studies and one research essay). The first paper presented a critical re-evaluation of frequency data in phonology, starting at the most basic level of individual phoneme frequencies. The second examined the role of phylogenetic comparative methods in linguistics and presented some use cases for measuring phylogenetic signal. The third, put those methods for measuring phylogenetic signal into practice, detecting significant phylogenetic signal in phonotactic data. The fourth inferred a Pama-Nyungan phylogeny using phonotactic data combined with and separately from previously published lexical cognate data. I summarise each in turn now.

Chapter \textbf{CHAP REF} critically examined the history of phylogenetic thinking in linguistics and the burgeoning subfield of linguistic phylogenetics. One limitation of linguistic phylogenetic tree inference presently is an overreliance on lexical data. A relatively restricted number of studies have tried inferring trees with typological characters \autocite[e.g.][]{sicoli_linguistic_2014}, but these tend to suffer from a limited amount of data \autocite{yanovich_phylogenetic_2020}, logical dependencies between characters \autocite{yanovich_phylogenetic_2020} and rapid evolutionary rates and homology within a restricted state space \autocite{greenhill_evolutionary_2017}. Linguistic phylogenetics stands to benefit from an increased volume of data, including data drawn from more diverse parts of language. Separately, with regards to phylogenetic methods themselves, it is recognised that language evolution is not directly analogous to biological evolution and thus linguistic evolutionary processes warrant methodological consideration in their own right. Furthermore, as the discussion in Chapter \textbf{CHAP REF} emphasises, linguistic phylogenetics commands its own position in evolutionary sciences. Linguists should not be seen to be rushing to catch up to biologists by taking off-the-shelf methods from a more mature field. There is a great deal of scope for adapting biological tools and methods for inferring linguistic evolution, since biological and linguistic evolution share certain similarities in evolutionary processes. But this should be done by firstly building a ground-up understanding of linguistic evolutionary processes. In this spirit, I seek to evaluate the utility of phonotactic data in linguistic phylogenetics in this thesis, in particular the frequencies of biphones---sequences of two phonological segments.

Chapter \textbf{CHAP REF} presented the first of a four-part investigation. The output of a phylogenetic methodology is only as good as the data fed to it. Before simply pouring frequency statistics into a phylogenetic algorithm, I first examined the nature of phonological frequencies and the causative processes that the shape of those frequencies represents. Previous literature has looked at the frequencies of phonemes in particular languages and found that they tend to form heavily skewed distributions (such that there are a few high frequency phonemes and a long tail of many low frequency phonemes), though not skewed in a way that is well described by Zipf's law, which characterises word frequencies. I also found, however, that phoneme frequencies warranted a re-evaluation in light of advances in statistical methods for identifying distributions and demonstrations showing that previous methods were unreliable \autocite{clauset_power-law_2009}. In addition, previous studies tended to be dominated by European languages and Australian languages were either absent or dramatically underrepresented. I used the maximum likelihood framework suggested by \textcite{clauset_power-law_2009} to evaluate the distributions of phoneme frequencies in the lexicons of 168 Australian languages. I found weak support for a Zipfian-like power law structure among more frequent phonemes (though perhaps also a lognormal structure) and weak support for a geometric (or exponential) structure among less frequent phonemes. In the subsequent discussion in Chapter \textbf{CHAP REF}, I offer some tentative comments on causative processes. These comments remain speculative, because of the inherent uncertainty in fitting distributions to 20--30 observations (the number of unique phonemes in a given language). However, exponential distributions may be linked to birth-death processes and power law distributions may be linked to preferential attachment processes.

In Chapter \textbf{CHAP REF}, I break from discussions phonological data and turn attention to phylogenetic methodologies in comparative study. Phylogenetic methods might be most immediately associated with phylogenetic tree inference in historical linguistics. However, phylogenetic thinking is essential in all comparative fields of linguistics. As Chapter \textbf{CHAP REF} demonstrates, phylogenetic effects have long been a consideration in linguistic typology and also comparative biology where, similar to linguistic typologists, biologists are not immediately concerned with the task of tree inference but must consider historical relatedness as a confounding factor in their results. Both fields share similarities but also significant differences in how they deal with phylogeny as a confounding effect in comparative study, more properly termed phylogenetic autocorrelation. I argue in favour of the uptake of phylogenetic comparative methods in linguistics, in which phylogenetic information is incorporated directly into the statistical model. The critical advantage of this approach is that it enables a comparative study to include a maximal amount of data and the maximal extent of existing phylogenetic knowledge. This is in contrast to methods that involve discarding languages in order to create a phylogenetically balanced sample, and/or making limited phylogenetic assumptions, e.g.~assuming equal phylogenetic distance between subgroups of languages in a sample (effectively assuming a star-like phylogenetic structure). Phylogenetic comparative methods require pre-existing phylogenies of languages to implement. Crucially, however, it is possible to use limited phylogenetic information (i.e.~trees that are not fully resolved into bifurcating branches) and it is possible to incorporate phylogenetic uncertainty. And, as I argue, it is better to incoporate some phylogenetic information even if not perfect, rather than neglect phylogenetic autocorrelation altogether and effectively, perhaps unwittingly, assume a star phylogeny. One other use of phylogenetic comparative methods is to evaluate various questions of evolutionary dynamics, as I went on to demonstrate in Chapter \textbf{CHAP REF}.

Chapter \textbf{CHAP REF} quantifies the phylogenetic information content in phonotactics, a novel data source which has hitherto not featured in linguistic phylogenetic tree inference. I implement offshoot methods of the phylogenetic comparative methods discussed in the previous chapter to measure phylogenetic signal---the tendency of more closely related languages to resemble one another more than more distantly related languages. I extract phonotactic data at the most simple, basic level, firstly by coding the presence or absence of biphones and secondly by extracting biphone transition frequencies from 112 lexicons of Pama-Nyungan languages, corresponding to 112 tips in Claire Bowern's Pama-Nyungan phylogeny. I found a statistically significant degree of phylogenetic signal in all datasets, but particularly strong signal in frequency data, which provides a finer-grained level of information than the relatively low-yield binary data. This result demonstrated the potential utility of phonotactic data in phylogenetic tree inference, which is perhaps contrary to what one might have expected given the ostensibly high degree of phonotactic homogeneity among Australian languages.

Following on from the conclusions of Chapter \textbf{CHAP REF}, in Chapter \textbf{CHAP REF} I present the first instance of phylogenetic tree inference with the addition of the biphone data evaluated in the previous chapter. I combine the binary biphone dataset and biphone frequency transition dataset with a partition of lexical cognate data from \textcite{bouckaert_origin_2018}. For an evolutionary model of lexical cognates, I reproduce as close as possible the model that \textcite{bouckaert_origin_2018} found to be best supported. I use a simple Brownian motion model of evolution for logit transformed biphone frequency data. For binary biphone data, I run an independent preliminary test of 16 different iterations of several evolutionary parameters and then use the best supported iteration as the evolutionary model for this data partition in the main study. Finally, I calibrated the age of the Wati subgroup and overall age of the tree to the date ranges best supported in \textcite{bouckaert_origin_2018}. I then inferred Pama-Nyungan phylogenies using a Markov Chain Monte Carlo (MCMC) process twice. In the first run, the Pama Nyungan tree was inferred jointly from cognate and phonotactic data. In the second run, trees were inferred from cognate data and phonotactic data separately. After discarding an initial burn-in period, a stepping stone process \autocite{baele_accurate_2013} was used to estimate an overall marginal likelihood for each analysis and, from this, a Bayes Factor was calculated to test whether the cognate-phonotactics model produced a significantly better supported tree. This would be evidence for the hypothesis that Bayesian computational inference of linguistic phylogenies can be improved by including partitions of phonotactic data with more traditional lexical cognate data. {[}Edit here in light of final result{]}

Altogether, the results of these papers demonstrate that large volumes of phonotactic information can be extracted from language wordlists, the distribution of this phonotactic information is the outcome of diachronic processes that lend themselves to a phylogenetic model and, therefore, this phonotactic information is phylogenetically informative and can be implemented profitably in phylogenetic tree inference studies. Moreover, this thesis illustrates a stepwise procedure for evaluating the utility of other kinds of linguistic data from other parts of language in phylogenetic tree inference. This study is not an attempt to add novel data into a phylogenetic black box and see what happens. One can imagine an alternative, inadvisable approach beginning with the erroneous assumption that historical linguistic methods are outdated and linguists need to catch up to biologists by adopting computational phylogenetic methods. Noticing the relatively diminutive scale of linguistic datasets, one could extract as many linguistic data points as possible and, without necessarily understanding the methodological intricacies, input the data into some phylogenetic software and present the results as a novel linguistic phylogeny, inferred from a new, expanded dataset. The problem with this approach is summed up by the old adage `junk in, junk out'. Even if the resulting phylogeny were to seem plausible based on existing historical linguistic work, the problem is that it would be impossible to say whether any novel aspects of the results (including novel subgrouping, but especially also secondary, more tentative results such as dating and homeland detection) constitute genuine novel insights or artefacts of the data (or unexpected interactions between the data and assumptions of the model or computational implementation). This is a somewhat caricatured, hypothetical example. But aspects of this kind of approach may have contributed to a particular limitation of linguistic phylogenetics over the past two decades. That is, although phylogenetic studies have contributed a great deal of new information on questions that historical linguistics had not yet solved, e.g.~detailed internal branching structure within families/subgroups, no one yet has been able to increase the time-depth of existing historical linguistic methods and infer novel macrofamilial relationships. Even as the field of linguistic phylogenetics matures, much of the present work focuses on demonstrating the validity of the method by reproducing existing historical linguistic results. This might be due, in part, to a lack of understanding of what it really means when the output of phylogenetic methods diverges from existing knowledge and whether this constitutes genuine novel insight or some kind of noise. In contrast to the above `junk in, junk out' scenario, I have attempted in this thesis to present a careful, considered, ground-up evaluation of the data and methods before attempting to infer a phylogeny with phonotactic data in Chaper \textbf{CHAP REF}. This process is generalisable to any kind of linguistic data that would be new to phylogenetic tree inference and it can be summarised in an idealised sense as follows.

\begin{enumerate}
\def\labelenumi{\arabic{enumi}.}
\item
  Consider the theoretical motivations for testing the novel data source. Simply testing everything because it is available invites the possibility of spurious correlations and false positives. Instead, one should consider the linguistic theory that would motivate the novel data source's use in linguistic phylogenetic methods. What is known about the historical processes behind the data? Is there reason to suspect that this part of language is historically conservative or resistant to horizontal transmission between languages?
\item
  Consider the structure and nature of the data. How is the data distributed? Are there correlations or logical dependencies between variables? What does this say about the diachronic processes that produced the present distribution of observations?
\item
  Find a case study of languages for which there is a high quality pre-existing phylogeny, quantify the tree-like signal in the data, and evaluate its evolutionary dynamics. Is there phylogenetic signal?If the data is distributed randomly with regards to phylogeny (no phylogenetic signal) then some non-phylogenetic process is at play and the data will only contribute noise during phylogenetic tree inference. A negative result here would not mean the data is uninteresting---it could say something interesting about geography, ecology, language contact or something else---but it will not suit a tree model. Alternatively, if phylogenetic signal is detected, ideally this would be followed up with a subsequent evaluation of evolutionary dynamics. For example, is a Brownian motion evolutionary model appropriate or is a punctuated equillibrium model better supported?
\item
  At this point, there is well-founded, theoretically grounded evidence that the novel data patterns phylogenetically, a rigorous understanding of the data's distributional and logical structure, and a theoretical and empirical demonstration of its phylogenetic information content. The final step in this proof-of-concept is to attempt to infer a linguistic phylogeny with the addition of the novel data source. A Bayes factor can be calculated to test whether the addition of the novel data source adds sufficient phylogenetic information to infer the phylogeny with greater confidence.
\end{enumerate}

If the novel data source passes each of these steps, then we can be more confident about using the same kind of data to infer linguistic phylogenies in other contexts where less is known, and potentially make new knowledge claims about the results. In the case of phonotactics, evaluated within the confines of the Pama-Nyungan family in this thesis, the most immediately obvious (if ambitious) avenue for future work is expansion to the rest of Australia and the continent of Sahul. I outline the prospects for Sahul expansion below. However, there are further methodological considerations and subsequent evaluation that should occur before phylogenetic expansion with phontactics can take place. I turn attention now to some of the limitations of the work presented in the previous chapters of this thesis and the outstanding questions that still need to be considered.

\hypertarget{current-limitations-and-future-work}{%
\section{Current limitations and future work}\label{current-limitations-and-future-work}}

By outlining a general framework for assessing the phylogenetic potential of novel linguistic data sources, I wish to convey that work on the phylogenetics of phonotactics has now begun in earnest, but is far from complete. There is a lot of work still to be done in the phylogenetic phonotactics space. I now discuss some of the limitations and unanswered questions from the studies presented in this thesis, interleaved with research priorities for future work which could address them. These research priorities form three main avenues. One concerns expansion of the language sample to the rest of Australia, Sahul, or elsewhere in the world. The second avenue concerns the expansion of phonotactic datasets. The third avenue involves developing an improved understanding of the evolutionary dynamics of phonotactics.

One analytical choice which has been questioned by reviewers pertains to the language sample, which was restricted to the Pama-Nyungan family in Chapters \textbf{CHAP REF}--\textbf{CHAP REF} and Australia broadly in (\textbf{CHAP REF}). One response to this point is that historically an overrepresentation of European languages (especially western ones) in linguistics and under-documentation in other parts of the world has meant that linguistic generalisastions have been made often based on evidence from a single continent, or at least overrepresentation by a single continent. An example of this is \textcite{tambovtsev_phoneme_2007}, which attempts to generalise about phoneme frequencies based on a dataset of 90 Eurasian languages, 2 Oceanic languages,and just a single language from each of Australia, Africa and South America. Another point to make is that although the studies in this thesis are genealogically and geographically restricted, within the Pama-Nyungan family (or Australia generally in the case of Chapter \textbf{CHAP REF}), I have been able to include a maximal sample of all the languages for which there is sufficient high quality lexical data available. There has been no need to discard data in order to balance the sample between subgroups or geographic regions. All of this is to say that the restricted scope of the language sample in this thesis may not be quite as large a caveat to the results as initially might be assumed. Nevertheless, future expansion of the study to other linguistic regions would be beneficial and I turn to the question of how to do that now.

I have sought as much as possible to make the studies in this thesis reproducible with regards to code. An advantage of this is that expansion of this research to other parts of the world should be relatively straightforward in cases where there exists i) comparably segmented and phonemically standardised wordlists and ii) a high quality reference phylogeny, ideally for a sample of at least 30 languages. Reproduction in this way would be a useful validation of the results presented here. Beyond this, a priority for future work is to expand phylogenetic tree inference with phonotactics to the rest of Australia and potentially Sahul. The linguistic phylogeny of Pama-Nyungan is relatively well-studied now in the sense that we have large, high quality phylogeny for the family, with dates calibrated against archaeological evidence and even a phylogeographical component. however, this is only a fraction of Australia's linguistic diversity. No detailed phylogeny yet exists linking the many families and isolates in Australia's Top End, nor linking Australia's non-Pama-Nyungan families to Pama-Nyungan under a putative proto-Australian root. Expansion to the rest of Australia is a particularly promising possibility for expanded phylogenetic studies with phonotactics, because although there is enormous lexical diversity between the languages, non-Pama-Nyungan languages tend to share similar phonemic inventories to their Pama-Nyungan counterparts and, therefore, there should be plenty of characters with non-missing data for most or all languages. There is also scope to improve our understanding of the phylogenetics of less well-studied subgroups or subgroups with ostensibly high rates of horizontal transfer within Pama-Nyungan. An example is Cape York, where some progress has been made processing archival materials from the linguist Bruce Sommer \autocite{hollis_cape_2016}. Cape York languages show more divergent phonology from the rest of Australia and so expanded coverage in this area would constitute an interesting challenge case for phylogenetic phonotactics.

Potential expansion of phylogenetic tree inference with phonotactics to Sahul deserves special consideration. The first challenge in this enterprise will be sourcing data from New Guinea. Two sources of lexical data are the \emph{TransNewGuinea} database \autocite{greenhill_transnewguinea_2015} and a dataset of Mandang comparative wordlists \autocite{zgraggen_comparative_1980}. The Mandang dataset holds promise, albeit within a restricted area (the Mandang Province). It contains good-sized wordlists (ranging from 242--300 lexical items per language) which are also phonemicised, for 98 languages. As for the \emph{TransNewGuinea} database, the scope and scale of the database is enormous. It contains lexical data for 1,028 language varieties, mostly but not entirely belonging to the Trans-New Guinea family. There are two issues pertaining to the use of \emph{TransNewGuinea} data in phylogenetic phonotactic studies. The first is that the wordlists tend to be too short for these kinds of studies, averaging 142 lexical items per language. There are many sufficiently longer wordlists that could be used, but most of the languages in the database would fail to meet the 250-item threshold used in this thesis. The second is that, although the wordlists are transcribed in phonemic form, it would take a considerable amount of work to ensure these forms are phonemically standardised such that each individual segment is comparable across languages. One possible subject for future study, then, is to build a comparative database of New Guinea phonologies, following the principles of AusPhon \autocite{round_ausphon-lexicon_2017}. Towards this goal, there is some preliminary work on automated parsing of language grammars in New Guinea, however, this remains in an embryonic stage of development \autocite{round_automated_2020}. In addition, advances in computational linguistic tools for grapheme-to-phoneme conversion processes may help \autocite[see][]{salesky_corpus_2020}. Once lexical data from New Guinea has been acquired and segmented, the next major challenge will be dataset sparsity. This is because the phonemic inventories of languages in New Guinea are diverse. This results in a greater quantity of mismatches between languages, where a biphone character for a biphone present in one language is recorded as missing for another language because the other language lacks a segment from its inventory entirely. Biphone frequency transition data, in the form used in Chapter \textbf{CHAP REF}, might only give low level phylogenetic information between smaller families/subgroups that share similar phonemic inventories. Matches between more distantly related languages are more likely to be spurious due to the segments being generally common cross-linguistically. For example, just about every language will have the sequence /ma/, but whether the frequency of a transition from /m/ to /a/ is indicative of deep-time relationships between hitherto unrelated languages is untested (and perhaps doubtful on the face of it). There is no straightforward way around this challenge, but there are a couple of possibilities. Firstly, future evaluation of the evolutionary dynamics of phonotactics should attempt to quantify empirically where in the tree phylogenetic signal is coming from (particular clades or particular levels of time-depth) rather than simply quantifying phylogenetic signal throughout the tree overall. Secondly, one potential way to overcome dataset sparsity is to include different kinds of phonotactic variables with less mismatch between languages. This approach would, however, likely come with the trade-off of reducing the size of the dataset and/or introducing new issues of independence between variables.

Turning now to the second major limitation of phylogenetic phonotactics, there is the issue that biphones are an extremely simple representation of the phonotactic system of a language. Biphone characters only capture the effects of phonotactic restrictions on immediately adjacent segments, they do not account for phonotactic restrictions on longer strings, e.g.~clusters of three consonants, nor phonotactic rules concerning morpheme boundaries. This might be especially significant in Australia where, as discussed in Chapter \textbf{CHAP REF}, word structure has been traditionally described using a disyllabic template. Furthermore, there is no distinction between vowels and consonants in biphone characters, nor distinct consonant slots (e.g. \(C_1\), \(C_2\) and \(C_3\) in the disyllabic template discussed in Chapter \textbf{CHAP REF}). In response to these limitations, future studies should consider alternative phonotactic variables. Perhaps the most straightforward response would be to consider trigrams, or even longer n-gram sequences. This would capture phonotactic dependencies across longer distances and would greatly expand the size of the dataset, but it would face the same challenges of dataset sparsity discussed above. Other possibilities include the extraction of transition frequencies between consonants, ignoring vowels. Vowel-to-vowel transitions, ignoring intervening consonants, would also be a possibility. Recording transitions across morpheme boundaries and syllable boundaries would be a possibility, although it would require syllabified lexical data. AusPhon lexical data would require syllabification, with all the challenges of determining precise syllable boundaries that that entails. Lastly, there is the option of incorporating natural sound class information in the dataset, such as the dataset of frequency transitions between place and manner sound classes tested in Chapter \textbf{CHAP REF}. This helps solve the problem of dataset sparsity, since languages are likely to share many of the same places and manners of articulation, even if they do not share many directly comparable phonemes. However, it also introduces issues associated with binning of data into larger categories. Phonemes share overlapping phonological features and, therefore, may be binned into different overlapping natural-class-based variables and counted different numbers of times. Again, there is no easy solution to this issue. However, further consideration of evolutionary models and how they apply to phonotactic datasets is likely to help. I turn to discussion of evolutionary models now.

The third limitation to discuss is the sole reliance on a Brownian motion model of biphone frequency evolution in this thesis. As discussed in Chapter \textbf{CHAP REF}, Brownian motion is a simple model to implement and makes a natural starting point for investigation. But future studies should consider whether the Brownian motion model or another model is more appropriate, with consideration of our existing knowledge of sound change. Brownian motion, in which a continuous character value can wander up or down with equal probability, is well suited to shifts in biphone frequency associated with lexical changes. As languages gain or lose individual lexical items, the frequencies of transitions from segments in that lexical item will shift minutely. Over time, these minute shifts will accumulate, resulting in shifts in character values that resemble Brownian motion. The biggest problem for this model is that lexical replacement and innovation are not the only diachronic processes affecting biphone transition frequencies. Sound change processes will create sudden jumps in biphone character values, often to 0 or 1, as well as having the effect of both creating and eliminating new biphone data points in a language. Consider an assimilation process in a particular environment, for example, place assimilation of nasals to match a following stop. This will cause biphone transition frequencies for heterorganic nasal+stop sequences to jump to 0 and homorganic nasal+stop sequences to jump to 1. Relatedly, continuous models frequently cannot account for absolute 0 and 1 frequency values, which is problematic since 0 and 1 frequency values are meaningful in phonotactics, given binary phonotactic rules that place absolute restrictions on certain phonemes in certain environments or the kinds of assimilation processes that guarantee certain segments in certain environments as just described. With regards to sound change, phonemic splits and mergers will create new biphone data points and eliminate others completely for a language. In addition, sound change can cause erroneous matches and mismatches between biphone characters for different languages, owing to the direct matching of comparable phonemes. For example, a series of fricatives in language A might be historically related to a series of voiced stops in language B. In the methods used in this thesis, biphone characters involving fricatives in language A would not be matched to biphone characters involving voiced stops in language B, despite their common origin. The problem of sound change is genuinely difficult and deserves prioritisation in future work. Notwithstanding efforts at automated sequence alignment, a true computational model of sound change remains elusive. With regards to evolutionary models beyond Brownian motion, an immediate short-term priority is to test whether Brownian motion is best supported over other off-the-shelf models such as punctuated equilibrium or the Ornstein-Uhlenbeck model. Moreover, future work should incorporate a Lévy process which allows for sudden, discontinuous jumps. There is some preliminary work in this space \autocites{landis_phylogenetic_2012}{landis_pulsed_2017}{blomberg_beyond_2020}. One potential path for future evaluation would be to work with simulated data. That is, rather than working with natural language data, various sound change and lexical change processes could be simulated on a wordlist, and the resulting frequency distributions could be compared to actual observed frequency distributions using a maximum likelihood framework, as in Chapter \textbf{CHAP REF}, or similar.

A related concern for future research concerns the independence of phonotactic variables and inference of evolutionary parameters. Throughout this thesis, biphone characters have been treated as independent. This meant that phyogenetic signal was inferred for each character individually in Chapter \textbf{CHAP REF} and, likewise, independent evolutionary processes were inferred for each individual character in Chapter \textbf{CHAP REF}. However, the independence of biphone transition frequencies is not a tenable assumption, due to the tendency of phonemes to form natural classes. Phonotactic rules tend to apply to natural classes of segments, rather than an individual segment, and will therefore manifest in the frequencies of a whole group of biphone characters. Likewise, as discussed above, sound change processes will also tend to apply to whole classes of phonemes and therefore affect the frequencies of whole classes of biphone characters. One simple way of dealing with this is to bin phonemes into natural classes and calculate transition frequencies between classes rather than individual segments, as illustrated in Chapter \textbf{CHAP REF}. This is not entirely satisfying, however, due to the overlapping nature of natural classes and considerable reduction in dataset size. Another approach is to treat all biphone characters as part of a single, multidimensional system. Methods for doing this are being developed in the field of morphometrics, in which one often has to work with complex, multivariate datasets of interrelated measurements (e.g.~various measurements capturing the shape of a skull). I have made one early attempt to measure phylogenetic signal using such a method \autocite{macklin-cordes_phylogeny_2018}, however, this reduces an entire dataset of phonotactic variation to a single distance matrix. Treating biphone data as a single, multivariate system also renders phylogenetic tree inference impractical. For example, the 2,236 biphone characters used in Chapter \textbf{CHAP REF} would constitute a single multivariate diffusion model of 2,236 dimensions, which is too computationally intensive to be repeated millions of times over in a Markov Chain Monte Carlo process. Between these two extremes, there are options for partitioning biphone data into \(k\) independent groups. The issue here is how to decide on how the number of independent groups and how to delineate the boundaries between them. This is a challenge and an active area of development in evolutionary biology as well, particularly given the rise of morphometrics. One approach being developed currently is phylogenetic factor analysis \autocites{tolkoff_phylogenetic_2018}{hassler_inferring_2020}, which infers groupings itself (removing the need for the analyst to partition the data beforehand) but still requires the analyst to define the \(k\) number of groups beforehand. Besides following the methodological development in this space, future work towards understanding the patterns of correlation and interaction between phonotactic characters will be useful.

\hypertarget{future-directions}{%
\section{Future directions}\label{future-directions}}

Given everything we know about phonology, phonotactics, sound change and language change generally, what would an ideal evolutionary model for phonotactics look like? Extended discussion on this.

Ultimately, it's highly unlikely to be a simple matter of adding some phonotactic frequencies and, hey presto, it enables us to push back time depth such that we can identify deep-time linguistic relationships across Sahul. But if it helps us build up more solid lower level relationships, help solve ambiguities/disputes, helps infer better branch lengths perhaps, then that kind of information can be used to constrain future deep-time analyses with more hardcore data and methods, ancient DNA and that.
% ***************************************************