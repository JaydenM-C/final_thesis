% ***************************************************
% Discussion
% ***************************************************
\chapter[Towards phylogenetic phonotactics]{Towards a phylogenetic phonotactic research program}
\label{Chap:Discussion}

% ********* Enter your text below this line: ********
This chapter requires a lot of fleshing out. However, writing it should be relatively free flowing, since there ae many well-developed ideas I've accumulated to discuss over the last few years. It's been the subject of a good deal of thinking, discussions with Erich and old notes. The ideas are all there, it's just a matter of getting them down on the page. Writing should be relatively quick too since it's predominantly about my own work and ideas and so requires less referencing and follow-up reading than, say, Ch.2. Total expected chapter length is around 10pp, \textasciitilde{}4-5k words. Expected writing time is 2 weeks (\textless{}500 words per day, which should be very doable given the nature of this writing in particular).

\hypertarget{discussion-intro}{%
\section{Introduction}\label{discussion-intro}}

Brief recap of study motivation, research questions. How we got here.

The preceding four chapters presented four papers, starting with a critical re-evaluation of frequency data in phonology, starting at the most basic level of individual phoneme frequencies. The second examined the role of phylogenetic comparative methods in linguistics and presented some use cases for measuring phylogenetic signal. The third, put those methods for measuring phylogenetic signal into practice, detecting significant phylogenetic signal in phonotactic data. The fourth inferred a Pama-Nyungan phylogeny using phonotactic data combined with and separately from previously published lexical cognate data.

This chapter will discuss implications, limitations and future directions for this kind of work. Aim is to have a clear outline of future program of phylogenetic research involving phonotactic frequency data.

\hypertarget{limitations}{%
\subsection{Limitations}\label{limitations}}

Addressing more straightforward limitations of present studies. One is that we still don't have a good understanding of where in the tree phylogenetic signal is coming from. In other words, does the phylogenetic signal in phonotactic data come from shallow nodes in the tree or further back in time?

Another more immediate term priority for future research is to explore more `off the shelf' evolutionary models beyond Brownian motion.

Something else to consider in future study is correlated evolution between phonotactic characters. The characters considered in the last two chapters are far from independent. There are dependencies between binary characters and frequency characters, as discussed in Chapter 6, and there are dependencies within datasets too, since sound changes and phonotactic restrictions tend to affect whole natural classes of sounds rather than individual segments. One approach, which we tried in Chapter 5, is to consider sound class characters that code the frequency of one sound class being followed by another. This isn't very satisfying though. For one, sound classes are overlapping so different segments end up being counted different amounts of times. Secondly, it greatly shrinks the size of the dataset. For example, if we used sound class dataset in Chapter 6, it would have been XX\% (small) proportion of the total datset when combined with cognate characters. Another approach is to combine characters into multidimensional structures and then calculate a single matrix of distances between languages, a morphometric approach. This was tried in ALS2018 talk and does generate strong phylogenetic signal. But it ultimately reduces an entire dataset to just one variable.

One other concern is the series of choices one makes in matching a phoneme in language A to a phoneme in language B. E.g. matching vowels in languages with and without length distinctions. Do you treat a long vowel as a sequence of two short vowels or a separate segment? These are questions on which two linguists can and do disagree on the basis of different objective criteria. We discuss this a little in Chapter 5 and we list phonemic normalisation decisions in Appendix. Our concern has been mainly with keeping things consistent between languages without wading too deeply into the merits of certain analyses over others. But what effect might these decisions have on studies such as ours? This is something that, thanks to Ausphonlex, we can parameterise fairly easily. A future study task would be to, e.g.~replicate phylogenetic signal tests while cycling through different normalisation options to see what impact, if any, those decisions have on results.

\hypertarget{future-directions}{%
\subsection{Future directions}\label{future-directions}}

Big picture question: Given everything we know about phonology, phonotactics, sound change and language change generally, what would an ideal evolutionary model for phonotactics look like? Extended discussion on this.

\hypertarget{sahul}{%
\subsubsection{Sahul}\label{sahul}}

Expansion plans. Pama-Nyungan's linguistic phylogeny is relatively well-studied now in the sense that we have large, high quality phylogeny which has calibrated dates and the geographical element. This is still only a subset of Australia's linguistic diversity though. Still tons of undetected or weakly supported links between non-Pama-Nyungan language families. Cape York still has a treasure trove of possibilities too. Nod to Sommer archival materials, some early work on that. Interesting challenge because of wild phonology.

New Guinea. Another thing all together, and will have to resolve dataset sparsity challenges because of greater phonological variation there. Identify need to build on existing lexical databases (e.g.~Simon's TNG database) with comparative database of phonologies, built with Ausphonlex-like principles. Maybe aided by automated parsing of language grammars (a stretch, but that was one of the original aims of that work so worth a mention). Short wordlists will likely be a problem.

Ultimately, it's highly unlikely to be a simple matter of adding some phonotactic frequencies and, hey presto, it enables us to push back time depth such that we can identify deep-time linguistic relationships across Sahul. But if it helps us build up more solid lower level relationships, help solve ambiguities/disputes, helps infer better branch lengths perhaps, then that kind of information can be used to constrain future deep-time analyses with more hardcore data and methods, ancient DNA and that.

% ***************************************************