% ***************************************************
% Conclusion
% ***************************************************
\chapter[Conclusion]{Conclusion}
\label{ch-conclusion}

% ********* Enter your text below this line: ********

What do the minutiae of sound sequences in human language encode about the forces of human history that shaped them? This thesis has shown that the minutiae of sound sequences in human language do appear to contain a signal of human history. Presently, however, their integration in a successful phylogenetic model for the purposes of inferring phylogenetic trees remains elusive. Nevertheless, it is valuable to conclude by recalling the purpose of this investigation into phonotactics in linguistic phylogenetics, and to emphasise the lessons it holds for future data innovation in this field?

The first two decades of the 21st century in historical linguistics have been characterised by the mainstream adoption of quantitative methods, particularly Bayesian phylogenetic methods for inferring phylogenetic trees and testing other evolutionary questions \autocites[see][]{dunn_language_2015}{bowern_computational_2018}. Notwithstanding the many successes of this quantitative turn, a rate-limiting step is the acquisition of lexical cognate data which is typically resource-intensive and at least partially, if not entirely, a manual process. A consequence, also, of the predominance of lexical cognate data in modern linguistic phylogenetics is that most of the resulting linguistic phylogenies inferred in recent years have been inferred from a single source of linguistic evidence representing only one sub-part of the linguistic system. Although it seems unlikely that lexical cognate data, which has been the bedrock of the historical linguistic Comparative Method for over 150 years, will be surpassed as the primary linguistic data source any time soon, it is timely to consider whether modern quantitative historical linguistics could benefit from data from other parts of language as well.

This thesis presented an interrogation of phonotactic data, in particular phonotactic frequency data extracted from Australian language wordlists, in linguistic phylogenetics. This focus on phonotactics stemmed from the observation that a language's phonotactic system tends to remain conservative in the face of language contact and semantic shift events which create noise in the historical signal in lexical data. Frequency data was considered, since Australian languages contain fine-grained variation at the frequency level that is not immediately apparent at a more categorical or binary level, and further, frequencies seem to have a psychological reality in the minds of speakers. One final motivation is the efficiency with which phonotactic data can be extracted from wordlists, which is advantageous in parts of the world where language documentation and research is relatively sparse.

Following the introduction and overview in Chapter \textbf{CHAP REF}, the thesis began with a literature review in Chapter \textbf{CHAP REF}. This covered the history of phylogenetic thinking in historical linguistics. Historical linguistics has been argued to be an inherently phylogenetic field \autocite{dunn_language_2015} and has a history of phylogenetic tree inference extending back even further than Darwin's seminal work in biology. However, the uptake of computational phylogenetic methods in linguistics is relatively recent. Over the past 20 years, linguistic phylogenies have been inferred using these methods for numerous language families the world over and the subfield sits increasingly within the mainstream of historical linguistics. This discussion was followed by an overview of Australian languages in Section \textbf{CHAP REF}. Australia is a linguistically diverse continent, home to at least 250--300 distinct languages at the time of European arrival (of which around 120 remain presently with living speakers). By far the largest Australian language family is Pama-Nyungan, which covers nearly 90\% of the continent. Some 27 other language families and isolates are packed into Australia's Top End and Kimberley regions, while the linguistic picture in Tasmania to the south, which suffered disproportionately from colonisation, remains unclear. One area in which Australian languages curiously have been described as lacking in diversity is phonology, extending to phonotactics. Australian languages have been described as sharing similar, consonant-heavy phonemic inventories, varying along certain restricted lines. Recent studies that consider frequency data both challenge and nuance these characterisations, however \autocites{gasser_revisiting_2014}{round_phonology_2021}{round_phonotactics_2021}. The chapter concluded with Section \textbf{CHAP REF}, which placed Australian languages and linguistic phylogenetics within the context of broader interest in the human history of Sahul, the continent of Australia and New Guinea. Linguistic evidence will continue to be an important component for increasing our understanding of Sahul's human history, particularly from the early Holocene onwards.

The next chapter, Chapter \textbf{CHAP REF}, presented a re-evaluation of the shape of rank-frequency distributions of phonemes in a language, as extracted from the wordlists of 166 Australian languages. Phoneme frequencies were re-evaluated by using a maximum likelihood statistical framework \autocite{clauset_power-law_2009} to test whether observed distributions of phoneme frequencies are consistent with several similar, competing mathematical distributions. This method has been shown to be more reliable than the statistical methods used to evaluate phoneme frequency distributions in the past. Individual languages in the sample show a high degree of variation and there is a wide range of uncertainty due to the small number of observations in each language (limited to the number of contrastive phonemes in the language). An overall tendency does emerge, however. The most frequent phonemes in a language tend to follow a Zipfian-like power law distribution (though perhaps also a lognormal distribution) while the least frequent phonemes tend to follow a geometric (or exponential) distribution. Mathematically, power laws are associated with preferential attachment processes and geometric distributions are associated with birth-death processes. The chapter concludes with reasoned speculation that phoneme frequency distributions are consistent with a variety of sound changes (e.g.~phonemic mergers and splits), operationalised as birth-death stochastic processes, with a separate preferential attachment dynamic in which phones tend to shift into numerically stronger categories.

Chapter \textbf{CHAP REF} followed by making the case for phylogenetic comparative methods in linguistics. Comparative linguistics, which includes historical linguistics but also typology, is necessarily phylogenetic due to the shared histories between languages descended from a common ancestor. This phenomenon, phylogenetic autocorrelation, has long been recognised in comparative linguistics. However, other comparative fields of science have diverged from linguistics in recent decades towards quantitative phylogenetic comparative methods for controlling for phylogenetic autocorrelation. By contrast, linguistics largely tends to continue controlling for phylogenetic autocorrelation indirectly through, for example, creating phylogenetically balanced language samples. I argued that with the proliferation of detailed linguistic phylogenies that have been inferred in the last 20 years, a priority for comparative linguistics should be a continued shift towards phylogenetic comparative methods. An offshoot of phylogenetic comparative methods is a selection of statistics for measuring phylogenetic signal \autocite[e.g.][]{blomberg_testing_2003}, which can quantify to what degree some observed variation is attributable to the phylogenetic distances between languages or species. This is useful for quantifying the degree to which phylogeny needs to be controlled for as a confound in a synchronic, comparative study. However, it is also useful as a diagnostic tool for determining whether a comparative dataset contains phylogenetic signal which could help infer phylogenetic trees. In this latter context, I employ some of these tools for measuring phylogenetic signal in Chapter \textbf{CHAP REF}

In Chapter \textbf{CHAP REF}, I tested the hypothesis that phonotactics could contain phylogenetic signal. Phylogenetic signal was measured in three datasets coding the presence or absence of biphones, frequencies of transitions between segments and frequencies of transitions between natural sound classes. The datasets were extracted from 111 Pama-Nyungan languages and phylogenetic signal was measured using a pre-existing, independent reference phylogeny inferred from traditional lexical cognate data. Phylogenetic signal was detected in all three datasets. Not unexpectedly, the signal strength was relatively reduced in the binary data and greater in the finer-grained frequency data. The strongest phylogenetic signal was detected in the sound class transition frequency dataset. These results supported the hypothesis that sound sequences might encode phylogenetic information and motivated further evaluation of phonotactics in phylogenetic tree inference.

Chapter \textbf{CHAP REF} presented an attempt to infer a phylogeny of 44 western Pama-Nyungan languages, using phonotactic data in conjunction with pre-existing lexical cognate data. The chapter started with a preliminary test to evaluate the best evolutionary model parameter settings for binary biphone characters and a second preliminary test to reproduce earlier work \autocite{bouckaert_origin_2018} using only lexical cognate data. A simple Brownian motion model was applied to sound class transition frequency data. Independent evolutionary rates and likelihood scores were inferred for each frequency character due to the heavy computational demands of a multivariate continuous character model. A Bayesian MCMC process was run on two phylogenetic models. The first model, termed the `linked' model, included a single tree inferred from the combination of lexical and phylogenetic data. The second, termed the `separate' model, included two trees---one inferred from lexical data alone and the other inferred from phonotactic data separately. Marginal likelihood estimates, which give an overall indication of model fit and enable model comparison, were inferred for each model. However, the marginal likelihood estimates in this study proved to be unreliable. Some limited insight could be gleaned from comparing the maximum clade credibility trees generated by the linked model and the cognate-only model from the preliminary section, suggesting that some non-treelike signal possibly was being introduced by the phonotactic data, but firm conclusions were difficult to draw. The difficulty I encountered is that it is very computationally expensive to infer a phylogenetic tree from a highly multidimensional partition of continuous-valued characters. Treating each continuous character as evolutionarily independent simplifies the model computationally but, firstly, introduces orders of magnitude more free parameters to estimate and, secondly, makes the model less realistic. The combined effect is that the models with frequency-based phonotactic characters fail to converge consistently and fail to produce reliable marginal likelihood estimates. Looking ahead, I suggested that future interrogations of phonotactics in phylogenetics should consider more carefully the frequency distributions of biphone-based characters (not just frequencies of single segments, as in Chapter \textbf{CHAP REF}) and re-evaluate whether the Brownian motion model of character evolution is ideal or should be modified. Secondly, patterns of co-evolution among phonotactic characters should be evaluated. We expect that phonotactic characters will not be completely independent from one another but the models in this study do not account for this. Between the extremes of a single highly dimensional partition of continuous characters and completely independent partitions for each continuous character, there may be a middle ground in which characters are partitioned into smaller groups that both make sense linguistically and keep the model computationally practical.

Finally, the previous chapter, Chapter \textbf{CHAP REF}, drew together the four previous chapters into an overall assessment of phonotactics in phylogenetics. Notwithstanding the unreliability of the results in Chapter \textbf{CHAP REF}, and the ambiguity it leaves us with, it is worth reconfirming what this foray into phonotactic phylogenetics has uncovered. There are two parts to this. Most immediately, the results highlight the non-triviality of computationally inferring a phylogeny with a large dataset of continuous-valued characters. A `plug in and play' approach will simply not work from a practical standpoint, even if it were acceptable in a linguistic sense. I have detailed the future steps for model development that will be necessary to address this. More generally though, I hope to have made the following contribution: A set of steps for the empirical interrogation of linguistic data in phylogenetics, which is generalisable to any set of linguistic data, the application of which is novel in this context. These steps proceed as follows:

\begin{enumerate}
\def\labelenumi{(\arabic{enumi})}
\tightlist
\item
  Articulate a well-reasoned hypothesis on how the linguistic characters in question are expected to change through time.
\item
  Operationalise the language change process as a stochastic mathematical process and statistically evaluate whether the characters' observed frequency distributions fit the associated mathematical distribution.
\item
  Scan the data for phylogenetic signal, using an independent reference phylogeny.
\item
  Create a phylogenetic model, paying attention to the modelling of both the evolution of individual characters and the relationships between characters.
\item
  Evaluate whether the addition of the novel data strengthens phylogenetic tree inference, by comparing the model fit of two otherwise equivalent models, one in which the novel data is combined with traditional lexical cognate data and the other in which the novel data and lexical data are kept separate.
\end{enumerate}

For the sake of under-resourced parts of the linguistic world, or parts where the lexical historical signal is noisy, data innovation should remain a priority for linguistic phylogenetics. With this thesis, I hope to have demonstrated one empirical approach to this task.

% ***************************************************