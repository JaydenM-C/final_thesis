% ***************************************************
% LaTeX Packages
% ***************************************************
% This file defines the document design.
% Usually it is not necessary to edit this file, but you can use it to change aspects of the design if you want.

%There are essential packages that are contained within the uqthesis.cls which are integral to the template - These must not be deleted.  A list of these packages can be found in the README.tet file

%The packages below are optional, please add or alter as required.

%\usepackage{cite}				 %Allows abbreviated numerical citations.
\usepackage{pdfpages}			 %Allows you to include full-page pdfs.
\usepackage{wrapfig}			 %Lets you wrap text around figures.
\usepackage{bm} 				 %Bolded maths characters.
\usepackage{upgreek}			 %Upright Greek characters.
\usepackage{dsfont}				 %Double-struck fonts.
\usepackage{simplewick}			 %For typesetting Wick contractions.
\usepackage{mathtools}		     %Can be used to fine-tune the maths presentation.	
\usepackage{framed}			     %For boxed text.
\usepackage{microtype}			 %pdfLaTeX will fix your kerning.
\usepackage{marvosym}			 %Include symbols (like the Euro symbol, etc.).
\usepackage{color}				 %Nice for scalable pdf graphics using InkScape.
\usepackage{transparent}	     %Nice for scalable pdf graphics using InkScape.
\usepackage{placeins}			 %Lets you put in a \FloatBarrier to stop figures floating past this command.
\usepackage{mdframed,mdwlist}    %Use these for nice lists (less white space).
\usepackage{graphicx}            %Enhanced support for graphics.
\usepackage{float}               %Improved interface for floating objects. 
\usepackage{longtable}           %Allow tables to flow over page boundaries.
\usepackage{mathdots}            %Changed the basic LaTeX and plain TeX commands.
\usepackage{eucal}               %Font shape definitions to use the Euler script symbols in math mode.
\usepackage{array}               %Extending the array and tabular environments.
\usepackage{stmaryrd}            %The StMary’s Road symbol font.
\usepackage{amsthm}              %St Mary Road symbols for theoretical computer science. 
\usepackage{pifont}              %Access to PostScript standard Symbol and Dingbats fonts.
\usepackage{lipsum}              %Easy access to the Lorem Ipsum dummy text.
\usepackage{enumerate}           %Enumerate with redefinable labels. 
\usepackage[all]{xy}             %This is a special package for drawing diagrams.
\usepackage{amsmath}             %ATypesetting theorems (AMS style).
\usepackage{amssymb}             %Provided an extended symbol collection.
\usepackage[utf8]{inputenc}      %Allowed all displayable utf8 characters to be available as input.
\usepackage{fancyhdr}            %Extensive control of page headers and footers.
\usepackage{blindtext}           %Produced 'blind' text for testing.
\usepackage{tikz}                %To create graphic elements.
\usepackage[figuresright]{rotating}	%Allows large tables to be rotated to landscape.
\usetikzlibrary{shapes.geometric, arrows}

%You can add more packages here if you need
\usepackage[backend=biber, style=biblatex-sp-unified, citestyle=sp-authoryear-comp]{biblatex}
\addbibresource{References/bibliography.bib}
\addbibresource{References/my_publications.bib}
\usepackage{threeparttable}

%This defines some macros that implement Latin abbreviations
%COMMENT OUT OR DELETE IF UNDESIRED.
\newcommand{\via}{\textit{via}} %Italicised via.
\newcommand{\ie}{\textit{i.e.}} %Literally.
\newcommand{\eg}{\textit{e.g.}} %For example.
\newcommand{\etc}{\textit{etc.}} %So on...
\newcommand{\vv}{\textit{vice versa}} %And the other way around.
\newcommand{\viz}{\textit{viz}.} %Resulting in.
\newcommand{\cf}{\textit{cf}.} %See, or 'consistent with'.
\newcommand{\apr}{\textit{a priori}} %Before the fact.
\newcommand{\apo}{\textit{a posteriori}} %After the fact.
\newcommand{\vivo}{\textit{in vivo}} %In the flesh.
\newcommand{\situ}{\textit{in situ}} %On location.
\newcommand{\silico}{\textit{in silico}} %Simulation.
\newcommand{\vitro}{\textit{in vitro}} %In glass.
\newcommand{\vs}{\textit{versus}} %James \vs{} Pete.
\newcommand{\ala}{\textit{\`{a} la}} %In the manner of...
\newcommand{\apriori}{\textit{a priori}} %Before hand.
\newcommand{\etal}{\textit{et al.}} %And others, with correct punctuation.
\newcommand{\naive}{na\"\i{}ve} %Queen Amidala is young and \naive{}.