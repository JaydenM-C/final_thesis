%\documentclass[instructions]{uqthesis}
\documentclass[final]{uqthesis} 


%*************************************
% FOR YOUR FINAL THESIS
%*************************************

%IMPORTANT! 
%The default document class (above - line 1 & 2) for the template is \documentclass[instructions]{uqthesis} - this document class will show instructional material and examples relevant to the preliminary material in the compiled PDF preview. THESE INSTRUCTIONS ARE FOR YOUR REFERENCE ONLY AND ARE NOT TO BE INCLUDED IN YOUR FINAL THESIS! 

%To turn off these instructions in your final thesis you MUST use the document class \documentclass[final]{uqthesis} 
%To activate the final thesis document class you must UN-COMMENT THIS DOCUMENT CLASS (remove the % from the start of line 2) and comment out the instructional document class on line 1 (add % to the start of line 1). 

%*************************************
% Introduction to template
%*************************************
%This is The University of Queensland Graduate School Official LaTeX Thesis template.

%Be sure to observe the content of comments within the source code, these are prefaced with a percentage symbol.
%Most important instructions have been CAPITALISED.
%To uncomment an inactive command (if required) remove the % from in front of the command.

%Please see the README for more information.

%This file loads the necessary packages, sets the page styles, and defines required macros.
%Edit this if you are comfortable with LaTeX.

%Other tweaks can be made in uqthesis.cls, but change these at your own risk!

%See README for version.

%You must have the memoir class installed.

% ***************************************************
% LaTeX Packages
% ***************************************************
% This file defines the document design.
% Usually it is not necessary to edit this file, but you can use it to change aspects of the design if you want.

%There are essential packages that are contained within the uqthesis.cls which are integral to the template - These must not be deleted.  A list of these packages can be found in the README.tet file

%The packages below are optional, please add or alter as required.

%\usepackage{cite}				 %Allows abbreviated numerical citations.
\usepackage{pdfpages}			 %Allows you to include full-page pdfs.
\usepackage{wrapfig}			 %Lets you wrap text around figures.
\usepackage{bm} 				 %Bolded maths characters.
\usepackage{upgreek}			 %Upright Greek characters.
\usepackage{dsfont}				 %Double-struck fonts.
\usepackage{simplewick}			 %For typesetting Wick contractions.
\usepackage{mathtools}		     %Can be used to fine-tune the maths presentation.	
\usepackage{framed}			     %For boxed text.
\usepackage{microtype}			 %pdfLaTeX will fix your kerning.
\usepackage{marvosym}			 %Include symbols (like the Euro symbol, etc.).
\usepackage{color}				 %Nice for scalable pdf graphics using InkScape.
\usepackage{transparent}	     %Nice for scalable pdf graphics using InkScape.
\usepackage{placeins}			 %Lets you put in a \FloatBarrier to stop figures floating past this command.
\usepackage{mdframed,mdwlist}    %Use these for nice lists (less white space).
\usepackage{graphicx}            %Enhanced support for graphics.
\usepackage{float}               %Improved interface for floating objects. 
\usepackage{longtable}           %Allow tables to flow over page boundaries.
\usepackage{mathdots}            %Changed the basic LaTeX and plain TeX commands.
\usepackage{eucal}               %Font shape definitions to use the Euler script symbols in math mode.
\usepackage{array}               %Extending the array and tabular environments.
\usepackage{stmaryrd}            %The StMary’s Road symbol font.
\usepackage{amsthm}              %St Mary Road symbols for theoretical computer science. 
\usepackage{pifont}              %Access to PostScript standard Symbol and Dingbats fonts.
\usepackage{lipsum}              %Easy access to the Lorem Ipsum dummy text.
\usepackage{enumerate}           %Enumerate with redefinable labels. 
\usepackage[all]{xy}             %This is a special package for drawing diagrams.
\usepackage{amsmath}             %ATypesetting theorems (AMS style).
\usepackage{amssymb}             %Provided an extended symbol collection.
\usepackage[utf8]{inputenc}      %Allowed all displayable utf8 characters to be available as input.
\usepackage{fancyhdr}            %Extensive control of page headers and footers.
\usepackage{blindtext}           %Produced 'blind' text for testing.
\usepackage{tikz}                %To create graphic elements.
\usepackage[figuresright]{rotating}	%Allows large tables to be rotated to landscape.
\usetikzlibrary{shapes.geometric, arrows}

%You can add more packages here if you need
\usepackage[backend=biber, style=biblatex-sp-unified, citestyle=sp-authoryear-comp]{biblatex}
% bib sources added in order of priority (i.e. in case of duplicates, delete duplicate from lower-ordered bib file)
\addbibresource{References/my_publications.bib}
%\addbibresource{References/bibliography.bib}
\addbibresource{References/wordlistsources.bib}
%\addbibresource{References/references_discussion.bib}
%\addbibresource{References/references_litrev.bib}
\addbibresource{References/references_thesis.bib}
\usepackage{threeparttable}

%This defines some macros that implement Latin abbreviations
%COMMENT OUT OR DELETE IF UNDESIRED.
\newcommand{\via}{\textit{via}} %Italicised via.
\newcommand{\ie}{\textit{i.e.}} %Literally.
\newcommand{\eg}{\textit{e.g.}} %For example.
\newcommand{\etc}{\textit{etc.}} %So on...
\newcommand{\vv}{\textit{vice versa}} %And the other way around.
\newcommand{\viz}{\textit{viz}.} %Resulting in.
\newcommand{\cf}{\textit{cf}.} %See, or 'consistent with'.
\newcommand{\apr}{\textit{a priori}} %Before the fact.
\newcommand{\apo}{\textit{a posteriori}} %After the fact.
\newcommand{\vivo}{\textit{in vivo}} %In the flesh.
\newcommand{\situ}{\textit{in situ}} %On location.
\newcommand{\silico}{\textit{in silico}} %Simulation.
\newcommand{\vitro}{\textit{in vitro}} %In glass.
\newcommand{\vs}{\textit{versus}} %James \vs{} Pete.
\newcommand{\ala}{\textit{\`{a} la}} %In the manner of...
\newcommand{\apriori}{\textit{a priori}} %Before hand.
\newcommand{\etal}{\textit{et al.}} %And others, with correct punctuation.
\newcommand{\naive}{na\"\i{}ve} %Queen Amidala is young and \naive{}.

% ***************************************************
% Title page
% ***************************************************
%***THESIS TITLE***
%Use Sentence Case (capitalise only the first word and proper nouns).
\title{The evolution of phonotactic variation: (working title)}

%***YOUR NAME***
%Do not include initials or middle names. Do not include your supervisor(s)' name(s).
\author{Jayden Macklin-Cordes}
%***YOUR CURRENT DEGREES***
%Use abbreviations. Do not include the date or location of your degree. Do not include the degree for which this thesis is being submitted.
\currentdegrees{B.A. (Hons.)}

%***ORCID ID***
%Add and hyperlink your ORCID
\orcid{\href{https://orcid.org/0000-0003-1910-3969}{0000-0003-1910-3969}}

%***YEAR OF SUBMISSION***
\date{2020}
%***TYPE OF DEGREE***
\submittedfor{Doctor}
%If this thesis is being submitted for Masters, comment out the above line (add % before line) and remove comment (by removing %) from line below
%\submittedfor{Masters}

%***YOUR SCHOOL***
%Use Title Case (capitalise every word which is not a conjunction or preposition).
%See - http://blog.apastyle.org/apastyle/2012/03/title-case-and-sentence-case-capitalization-in-apa-style.html - for help.
\school{School of Languages and Cultures}

\begin{document}

\frontmatter
% Assemble title page
\maketitle
\clearpage

% ***************************************************
% Preface
%****************************************************
\section{Abstract}
\normalfont
%Open abstract.tex to edit
% ***************************************************
% Abstract
% ***************************************************
% TO PRODUCE A STAND-ALONE PDF OF YOUR ABSTRACT, uncomment this section and the \end{document} at the end of the file by removing the % from the start of each line.

%\documentclass[12pt, a4paper]{memoir}

%% ***************************************************
% LaTeX Packages
% ***************************************************
% This file defines the document design.
% Usually it is not necessary to edit this file, but you can use it to change aspects of the design if you want.

%There are essential packages that are contained within the uqthesis.cls which are integral to the template - These must not be deleted.  A list of these packages can be found in the README.tet file

%The packages below are optional, please add or alter as required.

%\usepackage{cite}				 %Allows abbreviated numerical citations.
\usepackage{pdfpages}			 %Allows you to include full-page pdfs.
\usepackage{wrapfig}			 %Lets you wrap text around figures.
\usepackage{bm} 				 %Bolded maths characters.
\usepackage{upgreek}			 %Upright Greek characters.
\usepackage{dsfont}				 %Double-struck fonts.
\usepackage{simplewick}			 %For typesetting Wick contractions.
\usepackage{mathtools}		     %Can be used to fine-tune the maths presentation.	
\usepackage{framed}			     %For boxed text.
\usepackage{microtype}			 %pdfLaTeX will fix your kerning.
\usepackage{marvosym}			 %Include symbols (like the Euro symbol, etc.).
\usepackage{color}				 %Nice for scalable pdf graphics using InkScape.
\usepackage{transparent}	     %Nice for scalable pdf graphics using InkScape.
\usepackage{placeins}			 %Lets you put in a \FloatBarrier to stop figures floating past this command.
\usepackage{mdframed,mdwlist}    %Use these for nice lists (less white space).
\usepackage{graphicx}            %Enhanced support for graphics.
\usepackage{float}               %Improved interface for floating objects. 
\usepackage{longtable}           %Allow tables to flow over page boundaries.
\usepackage{mathdots}            %Changed the basic LaTeX and plain TeX commands.
\usepackage{eucal}               %Font shape definitions to use the Euler script symbols in math mode.
\usepackage{array}               %Extending the array and tabular environments.
\usepackage{stmaryrd}            %The StMary’s Road symbol font.
\usepackage{amsthm}              %St Mary Road symbols for theoretical computer science. 
\usepackage{pifont}              %Access to PostScript standard Symbol and Dingbats fonts.
\usepackage{lipsum}              %Easy access to the Lorem Ipsum dummy text.
\usepackage{enumerate}           %Enumerate with redefinable labels. 
\usepackage[all]{xy}             %This is a special package for drawing diagrams.
\usepackage{amsmath}             %ATypesetting theorems (AMS style).
\usepackage{amssymb}             %Provided an extended symbol collection.
\usepackage[utf8]{inputenc}      %Allowed all displayable utf8 characters to be available as input.
\usepackage{fancyhdr}            %Extensive control of page headers and footers.
\usepackage{blindtext}           %Produced 'blind' text for testing.
\usepackage{tikz}                %To create graphic elements.
\usepackage[figuresright]{rotating}	%Allows large tables to be rotated to landscape.
\usetikzlibrary{shapes.geometric, arrows}

%You can add more packages here if you need
\usepackage[backend=biber, style=biblatex-sp-unified, citestyle=sp-authoryear-comp]{biblatex}
% bib sources added in order of priority (i.e. in case of duplicates, delete duplicate from lower-ordered bib file)
\addbibresource{References/my_publications.bib}
%\addbibresource{References/bibliography.bib}
\addbibresource{References/wordlistsources.bib}
%\addbibresource{References/references_discussion.bib}
%\addbibresource{References/references_litrev.bib}
\addbibresource{References/references_thesis.bib}
\usepackage{threeparttable}

%This defines some macros that implement Latin abbreviations
%COMMENT OUT OR DELETE IF UNDESIRED.
\newcommand{\via}{\textit{via}} %Italicised via.
\newcommand{\ie}{\textit{i.e.}} %Literally.
\newcommand{\eg}{\textit{e.g.}} %For example.
\newcommand{\etc}{\textit{etc.}} %So on...
\newcommand{\vv}{\textit{vice versa}} %And the other way around.
\newcommand{\viz}{\textit{viz}.} %Resulting in.
\newcommand{\cf}{\textit{cf}.} %See, or 'consistent with'.
\newcommand{\apr}{\textit{a priori}} %Before the fact.
\newcommand{\apo}{\textit{a posteriori}} %After the fact.
\newcommand{\vivo}{\textit{in vivo}} %In the flesh.
\newcommand{\situ}{\textit{in situ}} %On location.
\newcommand{\silico}{\textit{in silico}} %Simulation.
\newcommand{\vitro}{\textit{in vitro}} %In glass.
\newcommand{\vs}{\textit{versus}} %James \vs{} Pete.
\newcommand{\ala}{\textit{\`{a} la}} %In the manner of...
\newcommand{\apriori}{\textit{a priori}} %Before hand.
\newcommand{\etal}{\textit{et al.}} %And others, with correct punctuation.
\newcommand{\naive}{na\"\i{}ve} %Queen Amidala is young and \naive{}.

%\begin{document}

%\begin{center}
	%\textbf{\large Your title goes here}

	%\textbf{Abstract}

	%Your Name, The University of Queensland, 20??
%\end{center}

% ********* Enter your text below this line: ********
Historical linguistics has been uncovering historical patterns of shared linguistic descent with considerable success for nearly two centuries. In recent decades, historical linguistics has been augmented with quantitative phylogenetic methods, increasing the scope and scale of linguistic phylogenetic trees that can be inferred and, by extension, the kinds of historical questions that can be investigated. Throughout this period of rapid methodological development, lexical cognate data has remained the main currency of interest. However, there is an immense array of structural variation throughout the world's languages---the outcome of thousands of years of linguistic evolution---raising the prospect of additional sources of historical signal which could help infer shared linguistic pasts.

This thesis investigates the prospect of making linguistic phylogenetic inferences using quantitative phonotactic data, semi-automatically extracted from wordlists. The primary research question motivating the study is as follows. Can a linguistic phylogeny be inferred with greater confidence when lexical cognate data is combined with phonotactic data? Underlying this is the more general question of how to approach and evaluate novel empirical data sources in linguistic phylogenetic research.

I begin with a literature review, situating current linguistic phylogenetic research within historical linguistics broadly and elucidating the motivations for looking further into phonotactics as a data source. I also introduce Australian languages and their phonological profiles, as these provide the language sample for the rest of the thesis.

Four papers follow. The first re-evaluates phoneme frequencies. The deceptively simple question of which statistical distribution best characterises the frequencies of phonemes in a language is crucial, because particular distribution types can be suggestive of particular causal processes that generated them. Using a maximum likelihood framework, I find modest evidence that more frequent phonemes are characterised by a power law distribution while less frequent phonemes are characterised by an exponential distribution. This is followed by discussion of theoretical implications.

The second paper reviews the place of phylogenetic methods in linguistics. It considers the scientific task of comparison and the potential confound of phylogenetic autocorrelation. I discuss comparative approaches from within linguistics and other domains of science and I advocate for the increased uptake of methods which control for phylogeny directly over various balanced sampling methods commonly employed in linguistic typology.

The third paper draws on the phylogenetic comparative methods discussed above to quantify the degree of phylogenetic signal in phonotactic data. Three simple data representations of phonotactics are extracted from wordlists: binary biphone data coding whether or not any given sequence of two phonological segments is present, biphone frequency data, coding the relative frequencies of a segment being preceded or followed by another segment, and sound class biphone data, coding the relative frequencies of a natural sound class being preceded or followed by another sound class. I find some degree of phylogenetic signal in all datasets, with the strongest signal in the sound class dataset.

Finally, the fourth paper tests the primary research question of this thesis. Using a Bayesian computational approach, I infer and compare phylogenies of 44 western Pama-Nyungan languages using two models: one in which the tree is inferred using binary biphone data, sound class data and pre-existing cognate data together, versus one in which two trees are inferred, one using the phonotactic data and the other using the cognate dataset alone. Within the bounds of current computational limits, I do not find evidence that phonotactic data strengthen support for the resulting phylogeny. Although marginal likelihood estimates seem to offer prima facie support for the hypothesis, there is evidence that the evolutionary model for phonotactic data is insufficient and the resulting calculations are unreliable.

I conclude the thesis with a discussion, drawing together the previous papers and detailing a methodological scaffold for investigating novel sources of linguistic data in quantitative historical linguistics in future. There are clear steps for future research to pursue with phonotactics. More generally, however, I present an empirical, step-wise process for methodological development in quantitative historical linguistics, from the initial link between language change processes and observable data distributions, to implementation within complex, expressive phylogenetic models.

% ***************************************************

%\end{document}

\clearpage
% ***************************************************
\section*{Declaration by author}
%DO NOT EDIT.
% ***************************************************
% Declaration by Author
% ***************************************************
% This is the DECLARATION BY AUTHOR
% All candidates to reproduce this section in their thesis verbatim
% DO NOT EDIT!
%
\begin{instructional}
\textit{(All candidates to reproduce this section in their thesis verbatim)\\}
\end{instructional}

\noindent
This thesis is composed of my original work, and contains no material previously published or written by another person except where due reference has been made in the text. I have clearly stated the contribution by others to jointly-authored works that I have included in my thesis.\\

\noindent
I have clearly stated the contribution of others to my thesis as a whole, including statistical assistance, survey design, data analysis, significant technical procedures, professional editorial advice, financial support and any other original research work used or reported in my thesis. The content of my thesis is the result of work I have carried out since the commencement of my higher degree by research candidature and does not include a substantial part of work that has been submitted to qualify for the award of any other degree or diploma in any university or other tertiary institution. I have clearly stated which parts of my thesis, if any, have been submitted to qualify for another award.\\

\noindent
I acknowledge that an electronic copy of my thesis must be lodged with the University Library and, subject to the policy and procedures of The University of Queensland, the thesis be made available for research and study in accordance with the Copyright Act 1968 unless a period of embargo has been approved by the Dean of the Graduate School. \\

\noindent
I acknowledge that copyright of all material contained in my thesis resides with the copyright holder(s) of that material. Where appropriate I have obtained copyright permission from the copyright holder to reproduce material in this thesis and have sought permission from co-authors for any jointly authored works included in the thesis.

\clearpage
%YOU MUST EDIT THIS DOCUMENT.
% ***************************************************
% PRELIMINARY PAGES
% ***************************************************
% The instructions contained within this part of the thesis template need to be suppressed from the final thesis. There are instructions on how to do this in the MainThesis.tex file.

% To ensure your work is not suppressed with the instructions please add your text only where instructed.


%***Publications included in this thesis***
\section*{Publications included in this thesis}

\begin{instructional}

	Start this section on a new page [this template will automatically handle this].\\
	
	\noindent
	If you choose to include publications as part of your thesis as described in UQ policy (\href{http://ppl.app.uq.edu.au/content/4.60.07-alternate-thesis-format-options}{\color{blue}{PPL 4.60.07 Alternative Thesis Format Options}}) use this section to detail accepted or in press publication/s using the standard citation format for your discipline. \\
    
    \noindent
	Papers submitted for publication and awaiting review should appear in the next section, \textbf{Submitted manuscripts included in this thesis}.\\
    
    \noindent
	On the page immediately preceding the chapter that includes your publication, in no more than one (1) page, describe your contribution to the authorship if you are not a sole author. In describing your contribution, you must satisfy the University's authorship policy (\href{http://ppl.app.uq.edu.au/content/4.20.04-authorship}{\color{blue}{PPL 4.20.04 Authorship}}). Authorship is based on having made a substantive contribution to at least one, and usually more than one, of the following activities:
	%
	\begin{enumerate}
		\item	conception and design of the project;
		\item	analysis and interpretation of the research data on which the publication is based;
		\item	drafting significant parts of the publication or critically reviewing it so as to contribute to the interpretation.
	\end{enumerate}
	
	\noindent
	As an author, you must have participated sufficiently in the publication to take public responsibility for at least that part of the work that you contributed.\\
    
    \noindent
	It may be useful to refer to specific parts of the methods, analyses, results, or discussion to illustrate your contribution to the paper.\\
    
    \noindent
	If you have not included any of your publications in the thesis then state ``No publications included''.\\
	
	\textbf{Example:}
	\begin{enumerate}

    \item \cite{DumyCitationKey} \textbf{Your Name}, Co-author 1, and Final Author, \href{linktoyourpaper}{Title of your paper}, \textit{Journal}, Issue, Number, Year

    \item \cite{DumyCitationKey} \textbf{Your Name}, Co-author 1, and Final Author, \href{linktoyourpaper}{Title of your paper}, \textit{Journal}, Issue, Number, Year

    \end{enumerate}
	
\end{instructional}

% ********* Enter your text below this line: ********
\nocite{macklin-cordes_phylogenetic_2021,
        macklin-cordes_re-evaluating_2020}
\printbibliography[keyword={included_publication},heading=none]

% ***************************************************


%***Submitted manuscripts included in this thesis***
\section*{Submitted manuscripts included in this thesis}

\begin{instructional}
	List manuscript/s submitted for publication here. As described above for \textbf{Publications included in the thesis}, on the page immediately preceding the chapter that includes the submitted manuscript, in no more than one (1) page, detail your contribution to the authorship if you are not the sole author.\\
    
    \noindent
    If you have no submitted manuscripts from your candidature then state ``No manuscripts submitted for publication''.\\
    
    \textbf{Example:}
    \begin{enumerate}

    \item \cite{DumyCitationKey} \textbf{Your Name}, Co-author 1, and Final Author, Title of your paper, submitted to \textit{Journal} on 4th June 2018.

    \end{enumerate}
\end{instructional}

% ********* Enter your text below this line: ********

No manuscripts submitted for publication.

% ***************************************************


%***Other publications during candidature***
\section*{Other publications during candidature}

\begin{instructional}
    List other publications arising during your candidature using the standard citation format for your discipline. Divide your publications into sub-sections as appropriate in your discipline \eg{} peer-reviewed papers, book chapters, conference abstracts. Papers submitted for publication and awaiting review are not considered publications and cannot be included in this section.\\
    
    \noindent
    If you have no publications from your candidature then state ``No other publications''.\\
    
    \textbf{Example:}
    \subsection*{Conference abstracts}

    \begin{enumerate}

    \item \cite{DumyCitationKey} \textbf{Your Name}, Co-author 1, and Final Author, Title of your conference paper, \textit{Proceedings of Conference}, other details.

    \end{enumerate}

    \subsection*{Book chapters}

    \begin{enumerate}

    \item \cite{DumyCitationKey} \textbf{Your Name}, Co-author 1, and Final Author, Title of your chapter, Book, editor, \etc{}.

    \end{enumerate}

\end{instructional}

% ********* Enter your text below this line: ********

%\fullcite{round_automated_2020}
\nocite{round_automated_2020}
\printbibliography[keyword={post2016_publication},heading=none]

\subsection*{Conference presentations}

\nocite{macklin-cordes_zipfs_2020,
        macklin-cordes_historical_2019,
        round_clouded_2019,
        macklin-cordes_phylogeny_2018,
        macklin-cordes_robots_2017,
        macklin-cordes_fine-grained_2017,
        macklin-cordes_high-definition_2017,
        macklin-cordes_evaluating_2016,
        hollis_cape_2016,
        macklin-cordes_reflections_2016,
        round_grammar_2016,
        macklin-cordes_high-definition_2016,
        macklin-cordes_high-definition_2016-1}
 \printbibliography[keyword={post2016_conference},heading=none]

% ***************************************************


%***Contributions by others to the thesis***
\section*{Contributions by others to the thesis}

\begin{instructional}
	List the significant and substantial inputs made by others to the research, work and writing represented and/or reported in the thesis. These could include significant contributions to: the conception and design of the project; non-routine technical work; analysis and interpretation of research data; drafting significant parts of the work or critically revising it so as to contribute to the interpretation. \\
    
    \noindent
	If no one contributed significantly then state ``No contributions by others''.
\end{instructional}

% ********* Enter your text below this line: ********

\noindent
Data for this research comes from the Ausphon Lexicon database \autocite{round_ausphon-lexicon_2017}, which is designed and maintained by Erich Round and extends on the CHIRILA database \autocite{bowern_chirila_2016}.\\

\noindent
Additional data, methodological advice and feedback on manuscripts comes from Claire Bowern, Simon Greenhill, Gabriel Hassler, David Nash, Caleb Everett, Steven Moran and two anonymous reviewers. These contributions are detailed further at the beginning of each chapter, where applicable.

% ***************************************************


%***Statement of parts of the thesis submitted to qualify for the award of another degree***
\section*{Statement of parts of the thesis submitted to qualify for the award of another degree}

\begin{instructional}
    The thesis must be comprised only of research undertaken while enrolled in the HDR program unless otherwise approved by the Dean, Graduate School in advance of submission.\\
    
    \noindent
    If you have been given permission to include your previous work that has been used towards another degree, you must list the relevant parts of the thesis that incorporates this work including, the degree name, year and institution, and the outcome of the submission of material. \\
    
    \noindent
    If no parts of the thesis have been submitted in this way then state ``No works submitted towards another degree have been included in this thesis''.
\end{instructional}

% ********* Enter your text below this line: ********

No works submitted towards another degree have been included in this thesis.

% ***************************************************


%***Research involving human or animal subjects***
\section*{Research involving human or animal subjects}

\begin{instructional}
	All research involving human or animal subjects requires prior ethical review and approval by an independent review committee. At UQ, the relevant committee for research involving human subjects is the \href{http://www.uq.edu.au/research/integrity-compliance/human-ethics}{\color{blue}{Human Ethics Unit}} and the relevant committee for research involving animal subjects is the relevant \href{http://www.uq.edu.au/research/integrity-compliance/animal-welfare}{\color{blue}{Animal Ethics Committee}}.  Please provide details of any ethics approvals obtained including the ethics approval number and name of approving committees.  A copy of the ethics approval letter must be included in the thesis appendix.\\
    
    \noindent
	If no human or animal subjects were involved in this research please state: ``No animal or human subjects were involved in this research''.
\end{instructional}

% ********* Enter your text below this line: ********

No animal or human subjects were involved in this research.

% ***************************************************


%***Acknowledgements***
\clearpage
\section*{Acknowledgments}

\begin{instructional}
    Start this section on a new page [the template will handle this for you].\\
    
    \noindent
    Acknowledgements recognise those who have been instrumental in the completion of the project.  Acknowledgements should include any professional editorial advice received including the name of the editor and a brief description of the service rendered.
\end{instructional}

% ********* Enter your text below this line: ********
First and foremost, I wish to thank my phenomenal supervisor, colleague and friend, Erich Round. To Erich, thank you so much for your invaluable mentorship, all the tea times, and your willingness to indulge my occasional thought bubbles. Thank you also to fellow supervisor, Simon Greenhill, for your guidance and expertise and for the incredible opportunities afforded to me. I feel so fortunate to have worked with two such exceptional people. Thank you also to Ilana Mushin and Barbara Hanna for your support and feedback, and keeping my candidature on track. Naturally, any outstanding deficiencies in my work remain my own.

Undoubtedly, one of the most rewarding aspects of this PhD program has been the wonderful friends and colleagues I've spent time with along the way. Thank you to Mitch Browne, Brian Collins, Jacqui Cook, Vivien Dunn, Tom Ennever, Claire Gourlay, Amanda Hamilton-Holloway, Jordan Hollis, Celeste Humphris, Edith Kirlew, David Osgarby, Martin Schweinberger, Janet Watts and Ruihua Yin. Thank you also to Rikker Dockum, Vanya Kapitonov, Nay San, Sasha Wilmoth, and fellow friends and classmates from the 2016 Quantitative Methods Spring School, 2017 LSA Summer Institute, CoEDL Summer Schools, and various conferences. Inevitably, there will be more people that I've missed, but I truly appreciate every one.

I owe endless thanks to my family for their boundless love and support, without which none of this would be possible. Credit must also go also to my friends outside of academia for their role in keeping me healthy, happy and grounded. This thesis could not have happened without a break for parkrun and pancakes on a Saturday morning.

Final thanks are reserved for my wife, Brittany. Thank you for your heroic support, patience and everything else.

% ***************************************************


%***Financial Support***
\clearpage
\section*{Financial support}

\begin{instructional}
    Start this section on a new page [the template will handle this for you].\\
    
    \noindent
    If you are the recipient of an Australian Government Research Training Program (RTP) scholarship, you are required to acknowledge this contribution.  Please include the text below:\\
    
    \noindent
    ``This research was supported by an Australian Government Research Training Program Scholarship''\\
    
    \noindent
    If you received any other financial support for your project, you are also required to acknowledge the funding body/bodies in this section.\\
    
    \noindent
    If no financial provided then state ``No financial support was provided to fund this research''.
\end{instructional}

% ********* Enter your text below this line: ********

This research was supported by an Australian Government Research Training Program Scholarship.\\

\noindent
Summer Institute attendance, lab visits and conference travel were supported by the UQ School of Languages and Cultures, the Max Planck Institute for the Science of Human History, the Linguistic Society of America, the ARC Centre of Excellence for the Dynamics of Language and the Yale University Linguistics Department.\\

\noindent
I gratefully acknowledge all the generous support that made this thesis possible.

% ***************************************************


%***Keywords***
\section*{Keywords}

\begin{instructional}
	Maximum 10 words; use lower case throughout, separating words/phrases with commas. For example: word, word word, word, word, word word
\end{instructional}
% ********* Enter your text below this line: ********

Australian languages, Pama-Nyungan, linguistic phylogenetics, phoneme inventories, frequency distributions, phylogenetic comparative methods, phylogenetic signal, quantitative methods, phylogenetic tree inference, phonology

% ***************************************************


%***Australian and New Zealand Standard Research Classifications (ANZSRC)***
\section*{Australian and New Zealand Standard Research Classifications (ANZSRC)}

\begin{instructional}
    Provide data that links your thesis to the disciplines and discipline clusters in the Federal Government’s Excellence in Research for Australia (ERA) initiative.\\
    
    \noindent
    Please allocate the thesis a \textbf{maximum of 3} \href{http://www.abs.gov.au/Ausstats/abs@.nsf/Latestproducts/6BB427AB9696C225CA2574180004463E?opendocument}{\color{blue}{Australian and New Zealand Standard Research Classifications (ANZSRC) codes}} at the \textbf{6 digit level} and include the descriptor and a percent weighting for each code. Total percent must add to 100.\\


\textbf{Example:}\\


    ANZSRC code: 060101, Analytical Biochemistry, 60\% \\
    \indent ANZSRC code: 060104, Cell Metabolism, 20\% \\
    \indent ANZSRC code: 060199, Biochemistry and Cell Biology not elsewhere classified, 20\%
\end{instructional}

% ********* Enter your text below this line: ********

ANZSRC code: 200406, Language in Time and Space, 60\% \\
ANZSRC code: 200408, Linguistic Structures, 30\% \\
ANZSRC code: 200402, Computational Linguistics, 10\%

% ***************************************************


%***Fields of Research (FoR) Classification***
\section*{Fields of Research (FoR) Classification}

\begin{instructional}
    Allows for categorisation of the thesis according to the field of research. \\
    
    \noindent
    Please allocate the thesis a \textbf{maximum of 3} \href{http://www.abs.gov.au/Ausstats/abs@.nsf/Latestproducts/6BB427AB9696C225CA2574180004463E?opendocument}{\color{blue}{Fields of Research (FoR) Codes}} at the \textbf{4 digit level} and include the descriptor and a percent weighting for each code. Total percent must add to 100. \\

\textbf{Example:}\\

FoR code: 0601, Biochemistry and Cell Biology, 80\% \\
\indent FoR code: 0699, Other Biological Sciences, 20\%
\end{instructional}

% ********* Enter your text below this line: ********

FoR code: 2004, Linguistics, 100\% \\

% ***************************************************


%***Order of remaining thesis content***
\begin{instructional}
\section*{Order for the Remainder of the Thesis}
\noindent
    Remainder of the thesis should be in the following order

    \begin{itemize}
        \item Dedications (if applicable)
        \item Table of Contents
        \item List of Figures and Tables
        \item List of Abbreviations used in the thesis
        \item Main text of the thesis
        \item Bibliography or List of References
        \item Appendices
    \end{itemize}

\noindent
\textbf{Date of thesis template release:} 22 March 2019
\end{instructional}
\clearpage


%***Dedication***
%If you wish to add a dedication (if appropriate), do so here.
%If not, comment out from here...
	\rmfamily
	\normalfont

	\begin{vplace}[1]
		\begin{center}
			For GGD.
		\end{center}
	\end{vplace}

%... to here.

\clearpage
\pagestyle{headings}


%***Table of Contents***
%These generate the table of contents, list of figures, and list of tables from items tagged with a \label{} command.
\tableofcontents
	\clearpage
\listoffigures
	\clearpage
\listoftables
\newpage
%%*************************************
% List of abbreviations
%*************************************
% You can make a list of abbreviations here.
%
% There are LaTeX packages available to take care of these things, but you will
% need to manually add these to the template at this stage (support may be added
% in future releases).

%CHOOSE AN APPROPRIATE TITLE.
%\chapter[List of abbreviations]{List of abbreviations}
\chapter[List of Abbreviations and Symbols]{List of Abbreviations and Symbols}

%If the auto-sizing of the tables annoys you, consider the tabularx package.

%List of abbreviations.
\begin{center}
	\small
	\begin{longtable}{ll}
	\toprule
	Abbreviations & {} \\
	\bottomrule
	AC				& Alternating Current \\
	AFM				& Atomic Force Microscopy/Microscope \\
	\etc{}		&	\etc{} \\
	\hline
	\end{longtable}
\end{center}

%*************************************
% List of symbols
%*************************************
%List of symbols. REMOVE IF NOT NEEDED.
\begin{center}
	\small
	\begin{longtable}{ll}
	\toprule
	Symbols & {} \\
	\bottomrule
	$\hat{\rho}$		& Density operator \\
	\etc{}					& \etc{} \\
	\hline
	\end{longtable}
\end{center}

%***End of list of symbols and abbreviations*** %List of symbols. REMOVE IF NOT NEEDED.

%***End of front matter***

% ***************************************************
% Thesis Content
%****************************************************
\mainmatter

%Each chapter is a separate .tex file. Use \input to load them here, as demonstrated below for Chapter 1 and Chapter 2.
%We recommend keeping each in a separate subfolder, with its accompanying figures, etc. This is how the template is currently structured.
%If you wish to divide your thesis into parts (each containing multiple chapters), us the \part{} command.

%CHAPTER 1
\chapter[Introduction]{Introduction}
\label{Chap:Intro}

% ***************************************************
% Introduction
% ***************************************************

A study of the evolution of structural linguistic diversity. In this instance, phonotactics.

This will be a short, sharp introduction. No more than a few pages, opening with an enticing research question. Est. 2 days' work to write and polish.

Thesis statement

Rationale and position of the work in its field

\hypertarget{chapter-outline}{%
\subsection{Chapter outline}\label{chapter-outline}}

Chapter 2 places thesis in academic context. Historical linguistics, linguistic phylogenetics, science of human history. Gives background on Sahul, linguistic area under investigation.

Chapter 3, Re-evaluating phoneme frequencies.

Chapter 4, phylogenetic comparative methods (PCMs) in linguistics. Introduces phylogenetic signal, its importance for comparative study and methods to detect it in difference kinds of data.

Chapter 5 measures phylogenetic signal in phonotactic data.

Chapter 6 uses said phonotactic data to infer Pama-Nyungan phylogeny jointly with cognate data.

Chapter 7 oulines future research program based on these findings.

Chapter 8 concludes.

% ***************************************************

%CHAPTER 2
\chapter[Literature review]{Literature Review}
\label{Chap:lit-review}

% ***************************************************
% Literature Review
% ***************************************************

This chapter still requires a bit of work, probably the most work of all the chapters in terms of pure writing. I realise this is a bit back-to-front for a PhD thesis. Without wanting to make excuses, this is partly because I already had to incorporate a lot of literature-review-like content in the next three chapters themselves (so they could stand alone as publications). It's also partly a consequence of the literature itself evolving so rapidly even during my candidature. For example, known timespan of human occupation in Australia has roughly doubled since I started my PhD program.

Nevertheless, the thesis requires a broad contextualisation. Needless to say, I've got a big library of references and lot of bits of scattered writing to pull together. Writing will be a bit slower due to the nature of literature review writing (following up secondary references etc.). Realistically, it might take me 1 month to pull together a really high quality, in-depth literature from here. Maybe less in a pinch.

General structure:

\begin{itemize}
\tightlist
\item
  background, interest

  \begin{itemize}
  \tightlist
  \item
    Sahul context
  \end{itemize}
\item
  historical linguistics
\item
  linguistic comparison
\item
  phylogenetic methods
\end{itemize}

Simon's old comments to remember:

\begin{itemize}
\tightlist
\item
  incorporate targeted historical linguistics
\item
  incorporate criticism of phylogenetic methods
\end{itemize}

\emph{Note to self:} Think of better chapter name.

\hypertarget{motivations}{%
\section{Motivations}\label{motivations}}

\hypertarget{sahul-context}{%
\subsection{Sahul context}\label{sahul-context}}

This thesis concentrates on languages of Australia, and in Chapters 5 and 6 on Pama-Nyungan languages specifically. However, it's really laying out some groundwork in the service of bigger questions. We don't especially learn anything new about Pama-Nyungan's phylogeny in this thesis, but the family serves as a test case from which we could springboard off into the rest of the continent where less is known phylogenetically.

Brief note on the geology and geography of Sahul and how it has changed through the ages.

History of human occupation. Theories for the peopling of Sahul and existing work triangulating archaeological, linguistic and genomic evidence.

The linguistic prehistory of Sahul, the continent of Australia and New Guinea, is an enigma which is mostly yet to be cracked by historical linguistics. The question of if and how Australian languages might relate to their Papuan neighbours to the north has been the subject of speculation for many years \autocites[e.g.][]{ogrady_languages_1966}{wurm_papuan_1975}. But, besides a handful of tentative cognates \autocite{foley_papuan_1986} and shared structural features \autocite{nichols_sprung_1997} Reesink, Singer, \& Dunn, 2009), conclusive evidence for a connection remains elusive. Notwithstanding the many small-scale language families (plus two large ones---Pama-Nyungan and Trans-New Guinea) which have been successfully identified via the standard comparative method in historical linguistics, the diagnosis of deeper historical relations is handicapped by several factors in this part of the world: A lack of adequate description \autocite{bowern_computational_2012}, absence of pre-colonial written sources \autocite{foley_papuan_1986}, apparent homogeneity of phonological systems \autocite{baker_word_2014} and millennia of extensive horizontal diffusion \autocites{foley_papuan_1986}{dixon_australian_2002} to name a few. Even in ideal circumstances, the scope of the linguistic comparative method is commonly cited as being limited to approximately ten thousand years, at which point historical signal becomes indistinguishable from noise in lexical data \autocite{nichols_sprung_1997}. This limits the method to a window of time well after rising seas inundated Lake Carpentaria approximately 12,000 years ago and only just before the inundation of Torres Strait approximately 8000 years ago.

Recent advances give cause for optimism that we may be able to extend the time-depth at which historical linguistics can operate and unravel something of the prehistory of Sahul. Methodological advances include the adaptation of Bayesian phylogenetic methods to infer the evolutionary history of language families. For example, \textcite{bowern_computational_2012} and subsequently \textcite{bouckaert_origin_2018} use phylogenetic methods, in combination with cognate identification via the traditional comparative method, to infer the internal branching of the Pama-Nyungan family. As for data, and the possible erosion of historical signal after 10,000 years in lexical data specifically, two observations motivate the approach of the present study: Firstly, linguistics has well and truly entered `the age of big data' and computational methods now enable us to extract minute threads of significance from large volumes of data. This enables us to compare more data points across wider groups of languages quickly and efficiently, and identify trends and correlations which would otherwise escape the attention of even the most highly trained human eye. This thesis evaluates the profitability of complemneting lexical data (which continues to form the backbone of most historical linguistic work) with other kinds of data too.

\hypertarget{phylogenetic-thinking-in-historical-linguistics}{%
\section{Phylogenetic thinking in historical linguistics}\label{phylogenetic-thinking-in-historical-linguistics}}

This section is old and needs refinement.

The first to suggest an analogous relationship between the evolution of languages through time and Darwin's theory of evolution in the biological world was, in fact, Darwin himself \autocite{darwin_origin_1859}. Although we now know a great deal more about the important differences between linguistic and biological evolution, Darwin's \autocite{darwin_descent_1888} description of the two processes as ``curiously parallel'' remains an oft quoted starting point for discussion of evolutionary matters in linguistics \autocite[e.g.][]{atkinson_curious_2005}. Interestingly, the representation of linguistic history in the form of a tree-like structure arguably predates such representations of biological history, with \textcite{schleicher_darwinsche_1863} pointing out that he made use of tree-like representations of languages himself, some years prior to Darwin.

Methodologies in linguistic and biological fields would subsequently diverge in the twentieth century. Methodological developments, such as the utilization of Mendelian inheritance values to model evolution and the eventual mapping of the structure of DNA, resulted in evolutionary biology taking a distinctly quantitative turn. Increasingly, researchers would turn to powerful algorithms and large volumes of data to infer phylogenetic relationships {[}\textcite{atkinson_curious_2005}. In contrast, historical linguistics has tended towards more qualitative methods and datasets which rely to a large extent on the manual, expert judgements by the linguist \autocite{nunn_comparative_2011}. Illustrating this is the linguistic comparative method, arguably the bedrock historical linguistics to this day. This method, to summarise very briefly following \textcite{thomason_language_1992}, consists of a number of painstaking steps requiring a good deal of time and expertise in the languages of study: First, sound correspondences are established. This enables the identification of cognates---words of shared historical origin---which then form the basis for the reconstruction of proto-languages and establishment of diversification patterns through time.

One quantitative method which stands out in this period is the lexicostatistical method, developed by \textcite{swadesh_lexico-statistic_1952}. `Lexicostatistics' infers relatedness among languages by quantifying percentages of cognates and using these figures with clustering algorithms. \textcite{swadesh_towards_1955} also established `glottochronology' as a mechanism for dating linguistic divergence events, which calculates the length of branches on a family tree by assuming an accumulation of lexical replacement at a constant rate through time. These methods were ultimately disfavoured, owing in part to disproval of Swadesh's `universal constant' theory of lexical replacement, a key assumption underlying glottochronology \autocite{blust_why_2000}, and an inability to quantify a level of uncertainty in the results \autocite{atkinson_curious_2005}.

Since the turn of the 21st century, there has been a resurgence of interest in the adaptation of phylogenetic methods from evolutionary biology to answer questions of historical linguistics, driven by factors including technological advances and accessibility, availability of large and open data, and a general desire for greater empirical rigour in the field \autocites{atkinson_curious_2005}{mcmahon_finding_2003}{nunn_comparative_2011}. However, these developments have not been embraced by all in the field. Perhaps an artifact of glottochronology's scarred legacy, \textcite[p.~520]{atkinson_curious_2005} describe a ``curious aversion'' to large-scale quantitative studies and the adoption of phylogenetic methods in historical linguistics. A reoccurring stream of criticism is predicated on the limitations (or even total inadequacy) of a family-tree based model which only shows vertical patterns of inheritance and not horizontal diffusion, e.g.~through linguistic contact and borrowing (\textcite{bateman_speaking_1990}; \textcite{donohue_new_2012}; \textcite{gould_urchin_2010}). Countering these concerns, \textcite{bowern_historical_2010} claims that this is confusing the family tree, which is essentially a visualization tool, with the methods themselves. Further, while there is still much work to be done (and not only in linguistics), phylogenetic methods are capable of modelling assumptions about horizontal admixture, not to mention quantifying uncertainty. In addition, \textcite{greenhill_does_2009} conduct a simulation study to test whether phylogenetic methods are invalidated by horizontal diffusion. They find that, to the contrary, their Bayesian phylogenetic method of choice is quite robust to areal borrowing between languages (albeit with some caveats relating to dating phylogenetic branches).

An additional criticism relates to the suitability of linguistic data for phylogenetic algorithms: Of specific concern is the ability (or otherwise) to reduce hugely complex linguistic systems into neat datasets of numerical or binarized characters without excessively compromising the dataset's ability to be meaningful and informative in a linguistic sense; and further, the question of whether linguistic variables can be assumed to be independent and equivalent to the degree required of phylogenetic algorithms, given the complex interplay between linguistic features and non-random, directional patterns of linguistic change that can be observed \autocite{heggarty_interdisciplinary_2006}. These are, to a large extent, live concerns which need further consideration and theoretical development. This thesis aims to contribute in this regard, by paying close attention to these questions as they relate to comparative study of languages in Sahul.

In summary, similarities between linguistic evolution and biological evolution have been noted for as long as phylogenetic trees have existed---indeed, the history of these academic fields is somewhat intertwined. However, only in more recent times have linguists returned to the methods of evolutionary biology, harnessing the inferential economy of sophisticated phylogenetic algorithms and large datasets. This is a rapidly developing field with potential to generate certain historical linguistic hypotheses which lay out of reach of the traditional comparative methods. Nevertheless, debate continues over the validity of applying phylogenetic methods to linguistic data \autocite[e.g.][]{heggarty_interdisciplinary_2006}. This discussion is important, and need not result in dichotomized `for' and `against' camps of thought. This thesis follows \textcite[p.~2299]{greenhill_does_2009}, who suggest that rather than engaging in ``armchair speculation'' on the suitability of phylogenetic methods ``on a priori grounds'', it is more beneficial to engage with these methods and linguistic theory, crunch the numbers, and quantify how useful (or otherwise) given methods are for testing various historical hypotheses.

% ***************************************************

%CHAPTER 3
%If you are presenting work which has been previously published, acknowledge this here.
% ***************************************************
% How to introduce a previously published chapter
% ***************************************************
%This is an example of how you might introduce a chapter that has been published previously. 
\cleartoevenpage
\pagestyle{empty}	
%Use this command (above) to suppress the header from the preceding chapter.

\noindent
The following publication has been incorporated as Chapter~\ref{Chap:label}.

\noindent
\fullcite{DumyCitationKey}

\begin{table}[h]
	\centering
	\begin{tabular}{clr}
		\toprule
		Contributor & Statement of contribution & \% \\
		\midrule
		\textbf{Your Name}				& writing of text 					& 70\\
															& proof-reading							& 60 \\
															& theoretical derivations 	& 70\\
															& numerical calculations 		& 100\\
															& preparation of figures 		& 80 \\
															& initial concept						& 10 \\
		\midrule
		Co-author 1								& writing of text 					& 20\\
															& proof-reading							& 10 \\
															& supervision, guidance 		& 20\\
															& theoretical derivations 	& 10\\
															& preparation of figures 		& 20 \\
															& initial concept						& 10 \\
		\bottomrule
	\end{tabular}
\end{table}

If your task breakdown requires further clarification, do so here. Do not exceed a single page.


% ***************************************************
% Example of an internal chapter
% ***************************************************
%This is an internal chapter of the thesis.
%If you have a long title, you can supply an abbreviated version to print in the Table of Contents using the optional argument to the \chapter command.
\chapter[Abbreviated title]{Full title}
\label{Chap:label}	%CREATE YOUR OWN LABEL.
\pagestyle{headings}

% ********* Enter your text below this line: ********
Causal processes can give rise to distinctive distributions in the linguistic variables that they affect. Consequently, a secure understanding of a variable's distribution can hold a key to understanding the forces that have causally shaped it. A storied distribution in linguistics has been Zipf's law, a kind of power law. In the wake of a major debate in the sciences around power-law hypotheses and the unreliability of earlier methods of evaluating them, here we re-evaluate the distributions claimed to characterize phoneme frequencies. We infer the fit of power laws and three alternative distributions to 168 Australian languages, using a maximum likelihood framework. We find evidence supporting earlier results, but also qualifying and nuancing them. Most notably, phonemic inventories appear to have a Zipfian-like frequency structure among their most-frequent members (though perhaps also a lognormal structure) but a geometric (or exponential) structure among the least-frequent. We highlight implications for causal accounts.

% ***************************************************


\section{Introduction}
\label{Sec:label}	%CREATE YOUR OWN LABEL.

% ********* Enter your text below this line: ********

Linguistic theorists seek to reveal causal mechanisms which explain the observable diversity of human language. Good causal hypotheses are often suggested by the mathematical distribution that a linguistic variable is described by, owing to the fact that the distribution can be understood as an emergent outcome of some underlying causal process, and that a given mathematical distribution will be consistent with only certain mathematical kinds of underlying processes. Consequently, it is important for the development of theory that proposed claims about distributions be as sounds as possible. For instance, one of the most famous distributions in linguistics is the Zipfian distribution, which technically speaking is a kind of power law. Recently, however, the evaluation of putative power laws across the sciences have come under intense scrutiny and often been found wanting. In response, methodologists have developed more rigorous and secure methods for diagnosing power laws and for distinguishing them from similar but significantly different distributions. For linguists, this creates an opportunity, to re-examine our own putative power law distributions, and by doing so to improve the pathway to sound explanatory theorising.

Here, we re-evaluate the status of mathematical distributions for characterising phoneme frequencies. Previous studies have proposed that phoneme frequencies follow a particular member of the power law family, the Yule-Simon distribution \autocites{martindale_comparison_1996}{tambovtsev_phoneme_2007}. But in the wake of a recent, major debate across the sciences regarding power laws, a reconsideration of this earlier research is timely. In this paper, we apply state-of-the-art maximum likelihood methods for the detection and assessment of power laws, to derive a better understanding of the distributions that do and do not describe phoneme frequencies well. Our results clear the way for more informed research into the ultimate, processual causes behind the frequency patterns of phonemes in human language.

\hypertarget{distributions}{%
\subsection{Distributions}\label{distributions}}

Distributions are properties of variables. A variable can be defined as the set of values that characterize something, be it a \emph{sample} (e.g.~a set of languages), or a \emph{real population} that the sample is drawn from (e.g.~the set of all current languages), or even an \emph{idealized population} which the real population is believed to approximate (e.g.~the set of all possible languages). Often, the ultimate object of scientific interest is an idealized population, and thus its distribution. Nevertheless, in empirical work, we cannot directly access this ultimate object of interest, and so we rely on real populations, or very often, a sample. Consequently, although we may have direct access only to the literal distribution of a sample, with its many idiosyncrasies, we tend to be more concerned with an overall pattern which we believe it approximates, one which often is elegantly characterized by a distribution which mathematically is relatively simple.

With these motivations in mind, how do we then decide that a certain distribution characterizes a variable satisfactorily? One method is visual inspection. This typically involves observing a close match between two histogram plots: one of the data and one of the candidate distribution, or plotting a regression of the data against the candidate distribution. In work on phoneme frequencies, visual inspection has been the primary method of assessing candidate distributions (see Section \ref{power-laws-linguistics}). A more rigorous alternative is to apply quantitative, statistical tests to evaluate how well a particular distribution model (such as the normal distribution) fits a data sample. The purpose of these tests is not to prove definitively that some variable follows a particular distribution, but rather to quantify the degree to which a sample's distribution is consistent with its having been drawn from a population of a particular distribution.\footnote{Within frequentist statistics, there are tests which test against the null hypothesis that data are drawn from a particular distribution (Shapiro-Wilk, Kolmogorov-Smirnov, Pearson's chi-squared test, to name a few). Within a Bayesian framework \autocites{spiegelhalter_omnibus_1980}{farrell_comprehensive_2006}, the approach is to calculate the likelihood of observing the data given a distribution and a set of the distribution's parameters. Bayes factors can be computed to compare the relative likelihoods of observing the data given competing kinds of distributions and parameter sets.} Some of these tests will be the subject of the sections that follow.

A strong motivation for testing the consistency of observed data with a particular distribution, is that many distributions can be described as the \emph{outcome} of certain kinds of processes. Thus there is a direct link between the quality of our evaluation of distributions, and the reasonableness of the causal, explanatory hypotheses we subsequently entertain. Consider for example a so-called \emph{preferential attachment process}, also known as a Yule process or rich-get-richer process. This can be imagined as having a set of urns, into which balls are added one at a time. Specifically, the urn to which a new ball is added is selected with a probability proportional to the number of balls already in the urn. This simple process has an interesting outcome. Initially, each urn is equally likely to be selected, but the distribution will soon skew, as urns with more balls accumulate additional balls faster than the others. If we \emph{rank} the urns in terms of which has the most balls, then with time, the relationship between an urn's rank and how many balls it contains will come to obey a \emph{power law}. A power law is a mathematical a relationship between two quantities where one varies as a power of the other (we discuss power laws further in Section \ref{power-laws}). Consequently, if, a variable can be shown to be consistent with a power law distribution, then this is consistent with there existing a preferential attachment process as the causal mechanism underlying the behaviour of variable. Udny Yule \autocites*{yule_mathematical_1925}[see also][]{albert_species_2011} made this connection a century ago. Yule showed that among flowering plants the level of species richness within a genus follows a power law, and linked that observation to a preferential attachment mechanism.

\hypertarget{power-laws}{%
\subsection{Power law distributions}\label{power-laws}}

Since Yule's first demonstration of the link between power law distributions and preferential attachment processes, power laws have been used to characterize the distributions of a diverse array of phenomena in the natural and physical world and in human society \autocite[p.~661]{clauset_power-law_2009}. City populations \autocites{gabaix_zipfs_1999}{levy_gibrats_2009}{malevergne_testing_2011}, authorship of scientific publications, income distribution \autocite{simon_class_1955}, the superstar phenomenon in the music industry \autocite{chung_stochastic_1994} and the network topology of the Internet \autocite{faloutsos_power-law_1999} are but a few of the phenomena for which power laws have been proposed (see \textcite{newman_power_2005}, pp.~327--329 for further examples). And in linguistics, the Zipfian distribution has been used to characterize word frequencies in text corpora \autocites{estoup_gammes_1916}{zipf_selective_1932}{zipf_human_1949}.

Given the apparent pervasiveness of power laws in a diverse range of unrelated contexts, it is little surprise that there is a rich vein of literature dedicated to evaluating power laws \autocite[p.~662]{clauset_power-law_2009} as well as a century of theorising on the mechanistic processes by which they arise \autocites[see][pp.~336--348]{newman_power_2005}[pp.~230--243]{mitzenmacher_brief_2004}. However, verifying the presence of a power law is not a straightforward task \autocite[p.~666]{stumpf_critical_2012}, and validation of earlier power law proposals using increasingly robust and powerful statistical methods is an active line of inquiry across many fields of science \autocite{malevergne_testing_2011}.\footnote{The question of whether certain phenomena are characterized best by a power law model or some other distribution can be contentious. See, for example, the debate between \textcite{eeckhout_gibrats_2004} and \textcite{levy_gibrats_2009} on the distribution of city population sizes (the former favouring a lognormal model, the latter favouring a power law). Another example concerns the distribution of computer file sizes, where \textcite{barford_generating_1998} and \textcite{barford_changes_1999} argue in favour of a power law model and \textcite{downey_structural_2001} argues in favour of a lognormal model.}

The traditional approach to power law validation, following Pareto's \autocite*{pareto_cours_1897} work on wealth distribution, was visual inspection. When visually inspecting a histogram plotted on log-log scales, a straight line would suggest the presence of a power law. The defining shape parameter (see below), \(\alpha\), could then be obtained by calculating the slope of the straight line using standard linear regression \autocites[p.~665]{clauset_power-law_2009}[p.~254]{urzua_testing_2011} and the \(R^2\) statistic could give an indication of the goodness of fit of the model. However, it has since been demonstrated that this traditional approach can be systematically unreliable \autocite[p.~665]{clauset_power-law_2009}. The unreliability becomes particularly acute when there is a small number of observations, since the ability to distinguish a power law distribution from other similar distributions, including the log-normal, using \(R^2\) is reduced \autocite[p.~691]{clauset_power-law_2009}. To remedy this, \textcite{clauset_power-law_2009} developed power law validation procedures within a more rigorous, maximum likelihood framework. These procedures have since been adopted widely in the literature \autocites[for example,][]{touboul_can_2010}{cho_friendship_2011}{brzezinski_power_2014}[and][]{lee_change_2018}, but have not yet been applied to phoneme frequencies.

At this point, some brief mathematical preliminaries are necessary.

When we refer to distributions here, we are referring to mathematically-defined \emph{functions}, that relate one quantity to another. Those functions may in addition have \emph{free parameters} which can be varied in order to produce a family of closely related distributions. A power law is a relationship between two quantities where one varies as a fixed power of the other, for example \(y = x^3\), or \(y = x^{-2}\) (which can also be written \(y = 1/x^2\)). For present purposes, where we will not be concerned with negative quantities or zeroes, we will use a more narrow definition by \textcite[p.~662]{clauset_power-law_2009}, who define a power law as a relationship in which a quantity, \(x\), is drawn from the distribution defined in Equation \eqref{eq:power-law}, where the free parameter \(\alpha\) is greater than zero and the variable \(x\) likewise is greater than zero.\footnote{For the area under the distribution curve to integrate properly to 1, the power function \(1/x^{\alpha}\) must be multiplied by a normalization constant (denoted \(C\) in the probability density function \(p(x) = C/x^{\alpha}\)). The normalization constant will be calculated differently depending on the value of \(\alpha\) and whether \(x\) is continuous or discrete. \textcite[p.~664]{clauset_power-law_2009} give some examples.} (The symbol `\(\propto\)' means `is proportional to'.) For example, \(x\) might denotes items' frequencies, while \(p(x)\) is the probability that a given item has a frequency of \(x\).

\begin{equation}
p(x) \propto \frac{1}{x^{\alpha}}
\label{eq:power-law}
\end{equation}

In practice, \textcite[p.~662]{clauset_power-law_2009} observe that the exponent (or `scaling parameter'), \(\alpha\), typically, though not exclusively, falls in the range \(2 < \alpha < 3\). They also observe that, in practice, many phenomena will not actually obey a power law for all values of \(x\). Rather, the power law will apply to values only above some minimum threshold value, \(x_{min}\). For example, in frequency data, it may be that only items whose frequencies meet or exceed a lower threshold will follow a power law. More generally, power law distributions come in a variety of specific forms, with different numbers of free parameters. We encounter some of these in Section \ref{power-laws-linguistics} below.

A distinction can be made between power laws that apply to continuous variables and those that apply to discrete ones. Frequency data, including the phoneme frequencies used in this study, are typically discrete. Zipf's Law \eqref{eq:zipfs-law} applies to a discrete number of \(n\) observations whose values, \(x\), are ranked by descending magnitude \(x_1 \geq x_2 \geq \ldots \geq x_n\). For example, \(x\) may be the token frequency of \(n\) types, with \(x_k\) the frequency of the \(k\)th-ranked type. In \eqref{eq:zipfs-law}, the quantity \(p(x_k)\) is the relative frequency of the \(k\)th-ranked type (i.e., its frequency scaled such that the \(n\) relative frequencies sum to 1).\footnote{This is equivalent to the probability that a token selected at random belongs to the \(k\)th-ranked type.}

\begin{equation}
p(x_k) \propto \frac{1}{k^{\alpha}}
\label{eq:zipfs-law}
\end{equation}

\hypertarget{power-laws-linguistics}{%
\subsection{Power laws in linguistics}\label{power-laws-linguistics}}

Investigation of power laws in the linguistic sphere has a long history. One of the oldest and best-known examples of a power law in any discipline is the distribution of word frequencies in text corpora, first noted by \textcite{estoup_gammes_1916} and subsequently described by Zipf \autocites*{zipf_selective_1932}{zipf_human_1949}. Zipf's Law, as it has come to be known, is a power law distribution for discrete data. Its exponent parameter, \(\alpha\), is typically very close to 1, in which case, the second ranked item will be approximately half as frequent as the first, the third ranked item will be one third as frequent as the first, and so on. Zipf's Law continues to garner considerable attention, for example in \textcite{kucera_computational_1967}, \textcite{montemurro_beyond_2001}, and more recently in Baayen \autocites*{baayen_word_2001}{baayen_analyzing_2008}. Various modifications to Zipf's formula have been suggested \autocite[notably][]{mandelbrot_structure_1954} and theoretical explanations put forward \autocites{li_random_1992}{naranan_information_1992}{naranan_models_1998}.

Power laws have also been proposed to describe the distribution of phoneme frequencies. The use of Zipf's Law to model the frequencies of phonological segments initially appears to be an attractive prospect \autocite[pp.~565--566]{witten_source_1990}. Nevertheless, a selection of alternative, non-power law distributions has also been suggested. \textcite{sigurd_rank-frequency_1968} is an early study evaluating the fit of a Zipfian distribution to phoneme frequencies, where the exponent, \(\alpha\), is set to 1. His evaluation method is a simple visual inspection, comparing observed phoneme frequencies in five languages (selected for their variety in segmental inventory size) with their expected frequencies assuming a Zipfian rank-frequency relationship. \textcite[p.~8]{sigurd_rank-frequency_1968} observes that the phoneme frequency distributions do not approximate a Zipfian curve, particularly for the most common segments. Rather, \textcite{sigurd_rank-frequency_1968} finds better approximations using a geometric series equation, where the frequencies of any two successively-ranked segments follow a common ratio, i.e., \(x_{k+1} = \lambda.x_k\) where \(x_k\) is the frequency of the \(k\)th-ranked segment and the parameter \(\lambda\) is the rate, or degree of diminution at each step, giving the discrete distribution:

\begin{equation}
p(x_k) \propto \lambda^k
\label{eq:sigurds-geometric}
\end{equation}

\textcite[p.577]{good_statistics_1969} suggests an alternative method of approximation: following \textcite{whitworth_choice_1901}, Good calculates the expected frequencies of each phoneme given a process whereby a unit interval probability space \([0,1]\) is divided into \(n\) parts at random (where \(n\) is the number of phonemes in the language), following a uniform distribution. This is equivalent to a so-called stick-breaking process: imagine a stick, which represents the unit interval probability space. The stick is broken into \(n\) parts; the \(n-1\) places along the stick at which a break is made are selected randomly and all at once, with any place along the stick equally likely to be selected as any other. When these parts are rearranged by size, from smallest to largest, their expectation follows the expression in \eqref{eq:whitworth-dist}:

\begin{equation}
\frac{1}{n^2},\quad \frac{1}{n} \left( \frac{1}{n} + \frac{1}{n-1} \right),\quad \frac{1}{n} \left( \frac{1}{n} + \frac{1}{n-1} + \frac{1}{n-2} \right),\quad ...
\label{eq:whitworth-dist}
\end{equation}

Giving the discrete distribution:

\begin{equation}
p(x_k) \propto \sum_{i=k}^{n} \frac{1}{i}
\label{eq:whitworth-rank-dist}
\end{equation}

In support of this model, \textcite[p.577]{good_statistics_1969} provides a table of observed versus expected frequencies of both graphemes and phonemes in English, however the sample size is modest (1000 words) and does not extend to any other languages. Furthermore, there is no visual or statistical evaluation of the goodness-of-fit. \textcite{good_statistics_1969} intends for the results to be taken as a curious observation only, with no strong theoretical position or claim of generalisability.

In subsequent work, \textcite[pp.~563--566]{witten_source_1990} examine the frequencies of single graphemes, graphemic bigrams and trigrams in the Brown Corpus and compare the fit of Good's distribution to Zipf's Law by comparing expected entropy values for each model to observed entropy scores. They find that the quality of the fit of Good's model declines with bigrams and trigrams compared to single graphemes, although the observed distribution curves are broadly of the same shape (and resembling the shape of Good's distribution rather than the Zipfian distribution). When assessed using metrics based on entropy, Good's distribution fits better than or around equally as well as the Zipfian distribution for all three datasets. Good's distribution also has the advantage of parsimony, since it is parameter-free: knowing how many unique items (phonemes, graphemes, bigrams, etc.), \(n\), are in the dataset is sufficient to calculate their expected distribution of frequencies---there are no additional parameters to estimate such as \(\alpha\) in \eqref{eq:zipfs-law} or \(\lambda\) in \eqref{eq:sigurds-geometric}.

\textcite{gusein-zade_distribution_1988} and \textcite{borodovsky_general_1989} present a similar distribution, defined in Equation \eqref{eq:borodovsky-rank-dist}. They visually evaluate the distribution's fit to the graphemes of English, Estonian, Russian and Spanish. This equation is also used to describe the distribution of DNA codons \autocite{borodovsky_general_1989}.\footnote{Of course, the statistics of graphemes are different from the statistics of phonological segments. As \textcite[pp.~136--137]{bloomfield_language_1935} rather emphatically points out: ``If we take a large body of speech, we can count out the relative frequencies of phonemes and of combinations of phonemes. This task has been neglected by linguists and very imperfectly performed by amateurs, who confuse phonemes with printed letters.'' Nevertheless, the frequencies of graphemes has been of interest historically in many applications; for example, in traditional printing, the development of Morse code, and library cataloguing \autocite[pp.~550--551]{witten_source_1990}.}

\begin{equation}
p(x_k) \propto \log \frac{(n+1)}{k}
\label{eq:borodovsky-rank-dist}
\end{equation}

\textcite{martindale_comparison_1996} compare the fit of four different distributions to frequencies of both graphemes and phonemes in text corpora from 18 languages. Using the \(R^2\) statistic in a linear regression, they compare the fit of the parameter-free equation of \textcite{borodovsky_general_1989} to the three related equations: the Zipfian distribution; Sigurd's geometric series distribution; and the Yule-Simon distribution \autocites{yule_mathematical_1925}{simon_class_1955}, which can be written:

\begin{equation}
p(x_k) \propto \frac{1}{k^\alpha}.\lambda^k
\label{eq:yule}
\end{equation}

The Yule-Simon equation in \eqref{eq:yule} is the product of the power law in \eqref{eq:zipfs-law} and the geometric equation in \eqref{eq:sigurds-geometric}. Because of the differing rates at which the two parts of the equation decay as \(k\) increases, equation \eqref{eq:yule} produces a distribution which is more like a power law \eqref{eq:zipfs-law} for low values of \(k\) (and thus for high-frequency items, for instance) and more like the geometric \eqref{eq:sigurds-geometric} for high values of \(k\) (low-frequency items) \autocite{simon_class_1955}.

The Yule-Simon equation in \eqref{eq:yule} has not just one free parameter but two, the exponent \(\alpha\) and the rate \(\lambda\), and the Zipfian and Sigurd equations are effectively special cases of it, each with one parameter fewer. The Zipfian distribution is equivalent to \eqref{eq:yule} with \(\lambda\) set to 1 (so that \(\lambda^k = 1\)), while the geometric equation is equivalent to \eqref{eq:yule} with \(\alpha\) set to zero (so that \(1/k^\alpha = 1\)). This is important, since as a general fact, if distribution A is a special case of distribution B, with fewer free parameters than it, then B will always perform at least as well as A when fitting the same set of data. Thus, the Yule-Simon distribution will necessarily fit the same set of data at least as well as the Zipfian distribution, and Sigurd's geometric distribution.

\textcite{martindale_comparison_1996} find that the Yule-Simon distribution fits best, for both graphemes and phonemes. They find that the Zipfian distribution tends to overestimate both high- and low-frequency items, although the differences they observe between models are only small. On this basis, they conclude that it is ``a matter of taste'' whether one opts for the more precise Yule-Simon distribution or simpler models with fewer parameters to estimate \autocite[p.~111]{martindale_comparison_1996}. \textcite{tambovtsev_phoneme_2007} greatly expand Martindale et al.'s \autocite*{martindale_comparison_1996} study to include phoneme frequencies in 95 languages from around the world. The sample is divided into four language groups (Indo-European, Altaic and Yukaghir-Uralic--plus a miscellaneous group) and a series of pairwise sign tests are conducted to test whether the difference in mean \(R^2\) is significant between different distributions for each language group. Again, they find that the Yule-Simon distribution fits best overall.\footnote{Although in their statistical tests they do not adjust their signficance levels to correct for multiple hypothesis testing.}

Obtaining a better fit by using a distribution with an additional parameter may be relatively trivial mathematically speaking, but this does not mean it is uninteresting. The extra parameter may work to capture a significant real-world nuance in an underlying causal process or describe the effect of a secondary process. A compelling causal explanation of a complex distribution might therefore be formulated by identifying some real-world factor and explaining how its mathematical effect on the distribution is expected to match what we find. It is important to consider the possibility of equifinality, too---the fact that multiple, different real-world phenomena may have equivalent mathematical effects. Tests of goodness-of-fit examine only the mathematical aspect, and cannot distinguish between different phenomena whose detectable mathematical contribution is equivalent.

\textcite[p.~111]{martindale_comparison_1996} and \textcite[p.~9]{tambovtsev_phoneme_2007} note that frequencies of phonological segments follow a Zipfian distribution less well than frequencies of words do, in part because the highest-frequency phonemes are not frequent enough. They speculate that this may be so, because if the most-frequent phonemes did pattern in a Zipfian way, then perception problems could arise for language users owing to the small size of a phonological inventory. This speculation does not meet the criteria for a compelling causal explanation though. It is not clarified what the linguistic mechanism is, that acts to prevent such perceptual problems, and thus we do not have a real-world phenomenon whose mathematical properties could be interrogated. Nor is it explained why, if such a mechanism exists, its mathematical effect would be to contribute something like the extra geometric term \(\lambda^k\) that differentiates the Yule-Simon distribution \eqref{eq:yule} from the Zipfian \eqref{eq:zipfs-law}. It will be recalled that the Yule-Simon equation, which \textcite{martindale_comparison_1996} and \textcite{tambovtsev_phoneme_2007} find to be a superior fit, describes a distribution which is most similar to a power law for high-frequency (low \(k\)) items, and most like the geometric for low-frequency (high \(k\)). The claim that its superior fit is due to \emph{non}-power-law-like behaviour of high-frequency items is therefore hard to reconcile with the mathematics. We return to the topic of the two-parameter Yule-Simon distribution in Section \ref{discussion}.

\hypertarget{the-need-for-re-evaluation}{%
\subsection{The need for re-evaluation}\label{the-need-for-re-evaluation}}

Despite the long history of studying power laws in linguistics, the methodological limitations of previous studies and the renewed, general scientific interest in power law phenomena motivate the re-evaluation of a power law model with respect to phoneme frequencies. The evaluation of statistical support for a power law relationship is far from straightforward and remains topical across a wide range of scientific fields \autocite{stumpf_critical_2012}. Although several different models have been compared for their goodness-of-fit to the frequencies of phonological segments \autocites{martindale_comparison_1996}{tambovtsev_phoneme_2007}, the method used to measure fit (using the \(R^2\) statistic) has been shown to be systematically unreliable \autocite{clauset_power-law_2009}. The goal of this paper is to verify the presence or otherwise of power law behaviour in the frequency distributions of phonological segments in the lexicons of Australian languages. It is, to the best of our knowledge, the first attempt to validate a power law model for the phonological segments using a maximum likelihood framework as suggested by \textcite{clauset_power-law_2009}, rather than the traditional and less reliable method of visual inspection.

\hypertarget{methodology}{%
\section{Methodology}\label{methodology}}

Here, we test for the presence or absence of a power law in the distributions of phonological segments following the maximum likelihood framework described by \textcite{clauset_power-law_2009}.

\hypertarget{data}{%
\subsection{Data}\label{data}}

As our data, we take phoneme frequencies in the lexicons of 168 language varieties of Australia. Readers familiar with Australian phonologies may at first find this a curious choice, since Australian languages are known to have similar phonemic inventories across the continent \autocites{capell_new_1956}{dixon_languages_1980}{busby_distribution_1982}{hamilton_phonetic_1996}{baker_word_2014}{round_phonemic_2019}{round_segment_2020}. We prefer to see Australia as an ideal controlled experiment. The phoneme inventories \emph{per se} may be similar, but the phonemes themselves exhibit considerable variation in their frequency distributions \autocites{gasser_revisiting_2014}{macklin-cordes_high-definition_2015}. Likewise, phonemic bigram frequencies in the large Pama-Nyungan family exhibit diversity with a strong phylogenetic signal \autocite{macklin-cordes_phylogenetic_2020}, suggesting that variations in Australian phonological frequencies have evolved over a deep time span.

The phonemic frequencies in this study are extracted from wordlists. Consequently, a difference between our test data and that of earlier work is that because we extract frequencies of phonological segments from lexicons, each unique word is weighted equally (since each word appears once in a language's wordlist),\footnote{There is a significant body of research suggesting that phonological frequencies of this kind are implicitly accessible to speakers and thus psychologically real \autocites[for example,][]{coleman_stochastic_1997}{zuraw_patterned_2000}{ernestus_predicting_2003}{albright_rules_2003}{eddington_spanish_2004}{hayes_stochastic_2006}.} whereas in text corpora, the frequencies of different words can differ radically. This difference is meaningful, since in the latter case phoneme frequencies and word frequencies will not be independent of one another. Phonemes in very high frequency words will have their frequencies boosted due to word frequencies, and consequently some degree of Zipfian-like skewing may appear in the distribution of phonological segments due to the confounding contribution of word frequency. Our method effectively controls for such effects and thus removes a potential confound from the phoneme frequency data.

Our data comes from the Ausphon-Lexicon database, under development by the second author \emph{(Author, 2017)}. Ausphon-Lexicon extends the Chirila resources for Australian languages \autocite{bowern_chirila:_2016}. It adds additional varieties and applies extensive data scrubbing, manual and automatic error-checking, and phonemic conversion using language-specific orthography profiles \autocite{moran_unicode_2018}. A challenge for any typological phonemic research is the long-recognized fact that phonemic analysis itself is non-deterministic \autocites{chao_non-uniqueness_1934}{hockett_problem_1963}{hyman_universals_2008}{dresher_contrastive_2009}. Presented with identical sets of language data, two linguists may produce differing phonological analyses, not due to any error on the part of the linguist but due to differing applications of the multitude of criteria by which decisions are made during the analysis of a phonemic system. As a consequence, cross-linguistic phonological variation can be attributed not only to language facts, but also to variation in linguistic practice. In cross-linguistic research, it is desirable for information to be represented in a comparable way throughout a dataset, and so recent phonological literature has emphasized the value of \emph{normalizing} source descriptions prior to cross-linguistic analysis \autocites{lass_vowel_1984}{hyman_universals_2008}{van_der_hulst_phonological_2017}{round_matthew_2017}{kiparsky_formal_2018}. Phonemic representations in Ausphon-lexicon are normalized in this sense. Section S2, Supplementary Materials, details the normalizations applied, together with bibliographic details of original data sources.

To illustrate an example of the phoneme frequencies in our sample, Figure \ref{fig:walmajarri-freqs} plots the frequencies of phonological segments in the Walmajarri lexicon \autocite{hudson_walmajarri_1993}. Equivalent plots for every language in our sample can be viewed through an interactive visualisation app that we provide in S4 of the Supplementary Materials.

\begin{figure}
\includegraphics[width=1\linewidth]{fig/walmajarri} \caption{Frequency of phonemes in Walmajarri lexicon \autocite{hudson_walmajarri_1993}. Plot (a) displays relative frequencies of each segment type. Plot (b) shows the same frequencies on log-transformed \(x\) and \(y\) axes---the traditional visual device used to identify power laws.}\label{fig:walmajarri-freqs}
\end{figure}



Phonological frequency data differs in some respects from the data types most commonly encountered in scientific power law studies, such as word frequencies or city populations. Typically, in order to understand a population (and some property of it), such as the cities in the United States (and their sizes), or the words of English (and their frequencies), it is impractical to examine every last member of the population, and so the study will examine a sample. Ensuring that a sample is of sufficient size is an important consideration, firstly in order to adequately represent the population and additionally, because a sufficiently large sample size is a key requirement in maximum likelihood estimation \autocites{barndorff-nielsen_inference_1994}{newman_power_2005}. In contrast, the phonemic inventory of any language is relatively small, and it is entirely feasible to examine exhaustive populations of phonemes.\footnote{The probability that we have failed to observe some phoneme that exists in a language is small, and even if we did, the missing segment inevitably will be an especially low-frequency type, unlikely to dramatically alter the overall frequency distribution of segments in that language.} An advantage of this is that the sample is highly representative of the population, but a disadvantage is that the number of observations is small and cannot be increased.

Given a sample of phonemes, we require an estimate, or measurement, of their frequencies. Measurement error arises as a potential concern in this study. Our segment frequencies are calculated from documented wordlists, which are necessarily limited representations of the complete vocabulary of the languages that the wordlists represent. One concern is that the particular morphology of a language's citation forms may cause certain segments in the language to be overrepresented in a wordlist which contains only citation forms. This would represent a bias, that is, a factor that pushes observations in a certain direction. We have attempted to control for this, by removing identifiable citation-form tense morphology from verbal words and noun-class prefixes from nominals. Another source of concern is that wordlists with a smaller number of words will necessarily entail a greater level of uncertainty in the observed segment frequencies. This will be a source of noise in the data. It does not push observed frequencies in any particular direction, but makes them generally less accurate. To address this, in our study, we restrict the language sample to language varieties with a minimum wordlist size of 250 lexical items. We selected 250 lexical items as a cut-off on the basis of \textcite{dockum_swadesh_2019}, who investigate the effect of wordlist size on phonological segment frequencies. \textcite{dockum_swadesh_2019} report accelerating losses in the fidelity of segment frequency estimates as a wordlist drops below 250 items. While more words will always yield better frequency estimates, we select a minimum of 250 as a reasonable compromise. This gives us a sample of 168 Australian language varieties. Wordlist sizes range from 268 to 8742 (median 1072, mean 1438).

\hypertarget{method}{%
\subsection{Statistical framework}\label{method}}

We test for the presence or absence of a power law in the distributions of phonological segments following the maximum likelihood framework described by \textcite{clauset_power-law_2009}. In brief, Clauset \emph{et al.}'s \autocite*[p.~663]{clauset_power-law_2009} proposed procedure consists of three steps:

\begin{enumerate}
\def\labelenumi{\arabic{enumi}.}
\tightlist
\item
  Estimate the parameters \(x_{min}\) and \(\alpha\) of the power law model using the maximum likelihood method \autocites{barndorff-nielsen_inference_1994}{newman_power_2005}\footnote{Maximum likelihood estimation (MLE) is a method for estimating the parameters in a statistical model, given some set of observations by finding the set of parameter values, \(\hat{\theta}\), that maximize a likelihood function, \(P(x\ |\ \hat{\theta})\), where \(x\) is a set of observations. In our case, the parameters, \(\hat{\theta}\), to be estimated are those which define a particular distribution---for example, \(\alpha\) and (optionally) \(x_{min}\) in a power law model.}.
\end{enumerate}

\begin{enumerate}
\def\labelenumi{\arabic{enumi}.}
\setcounter{enumi}{1}
\tightlist
\item
  Calculate the goodness-of-fit between the data and the power law using the Kolmogorov--Smirnov (KS) statistic, where a larger value corresponds to a worse fit. Using a Monte Carlo procedure, a \emph{bootstrapped} \(p\) \emph{value} is calculated\footnote{This is a well-established statistical technique. A large number of simulated datasets are created, with data points drawn from the model power law distribution hypothesized in step 1. Each is then fitted to its own power law model and a KS statistic is calculated for the simulated dataset, relative to this model. The \(p\) value is defined as the fraction of these simulated KS distances larger than the actual, observed KS distance.}, and used to evaluate the plausibility of the power law. Namely, if this \(p\) value falls below a plausibility threshold of 0.1, the power law model is rejected.\footnote{Here we follow the method of Clauset, Shalizi, and Newman (2009), who suggest a threshold of 0.1. Note though, that even when p\textgreater{}0.1, we still do not necessarily accept that the power law is a good fit, rather there is a further round of evaluation (step 3). This use of a `\(p\) value' differs from the more common use case where a null hypothesis is rejected when the p value is above a certain level. The reason for the difference lies in how the hypothesis of interest is related to the null hypothesis. Commonly, the hypothesis of interest is set up as the alternative hypothesis, and low p-values are required to reject the null hypothesis (not of interest). Here, the hypothesis of interest (power law is plausible) is set up as the null hypothesis. Accordingly, it too is rejected when the p-value is low. By allowing it to be rejected all the way up to 0.1 (rather than 0.05, for example), we are setting the bar relatively high. This approach may seem counterintuitive in the context of testing a single distribution hypothesis (where it might seem better to make the distribution of interest deliberately harder to accept than to reject). But in the context of testing which distribution fits the data best among multiple alternatives, it makes sense to make it deliberately hard to reject any particular distribution type.} Otherwise, the power law model remains an initially plausible hypothesis, and we proceed to step 3.
\end{enumerate}

\begin{enumerate}
\def\labelenumi{\arabic{enumi}.}
\setcounter{enumi}{2}
\tightlist
\item
  Compare the power law model with a set of models representing alternative hypotheses. For each alternative model, a bootstrapped \(p\) value is calculated as in steps 1 and 2 above. A likelihood ratio test is performed, comparing the fit of the alternatives with those of the power law model. If the calculated likelihood ratio is significantly different from zero, this indicates a significant difference in plausibility, and its sign (positive or negative) indicates which model is favoured \autocite[p.~680]{clauset_power-law_2009}.
\end{enumerate}

We use the \emph{poweRlaw} package \autocite{gillespie_fitting_2014} in \emph{R} statistical software \autocite{r-core-team_r_2017} to infer all maximum likelihood estimates and conduct bootstrapping to derive \(p\) values. We run 10,000 bootstrap iterations per language, per distribution type.\footnote{We find that 10,000 iterations is sufficient to obtain stable parameter estimates. Beyond 10,000 iterations, estimates will continue to fluctuate but in a tightly proscribed range. Plots of all bootstrapping runs can be viewed in the interactive visualisation app provided in Section S4.}

\hypertarget{methodological-considerations}{%
\subsection{Methodological considerations}\label{methodological-considerations}}

As a brief point of comparison to prior work, we return to the Walmajarri example and plot the linear relationship between phoneme frequencies and rank on a log-log plot. \textcite{tambovtsev_phoneme_2007} find that a Zipfian distribution consistently underestimates the frequency of both high- and low-ranking segments while overestimating the frequency of those in the middle. The dashed black slope on Figure \ref{fig:walmajarri-log-plot} shows a similar pattern. However, when the five lowest-frequency segments (i.e., those with the greatest statistical rank) are removed from the equation, the linear model fits much better (solid blue line). This is consistent with the observation by \textcite{clauset_power-law_2009} that, in practice, power laws are rarely observed across the whole distribution---rather, there is a threshold, the \(x_{min}\) parameter, below which the power law ceases to apply.

\begin{figure}

{\centering \includegraphics[width=0.66\linewidth]{fig/walmajarri_lm} 

}

\caption{Log-log plot of frequencies versus frequency ranks in Walmajarri. When a linear model is fitted to the full distribution (dashed black), high- and low-frequency segments are overestimated and mid-rank segments are underestimated. When lowest-frequency segments are removed from the model (solid blue), the model appears to fit well.}\label{fig:walmajarri-log-plot}
\end{figure}

\textcite{tambovtsev_phoneme_2007} improve the fit of their model by adding an extra geometric term, which causes the low-frequency tail of the distribution to be less power-law-like. On first inspection, it appears the addition of \(x_{min}\) improves the fit similarly well. Unfortunately, attempting to assess the goodness-of-fit of candidate distributions to this data through the application of linear models is problematic. A key assumption when estimating the standard error of the slope (shaded in grey) is that noise in is data are normally distributed, however, this is not the case for the logarithms of frequency data \autocite[ p.~691]{clauset_power-law_2009}. Further, the \(R^2\) statistic commonly used to validate the presence of a power law \autocite[including by][]{tambovtsev_phoneme_2007}, has low statistical power. That is, it often fails to distinguish between data truly drawn from a power law distribution and data drawn from other distribution types, particularly when the sample size is small \autocite[ p.~691]{clauset_power-law_2009}.

Visual inspection of other languages in the dataset indicates that Walmajarri's pattern of phoneme frequencies is common, although there is a good deal of variation (and, consequently, variation in the fit of a linear model). Given the limitations of applying a linear model to a log-log plot, we now turn to more reliable methods for validating the presence of a power law, using the maximum likelihood method outlined above.

\hypertarget{results}{%
\section{Results}\label{results}}

We firstly infer the fit of a power law to the full distribution of phoneme frequencies for each language, without estimating an \(x_{min}\) parameter. In Table \ref{tab:pl-summary} we summarize the maximum likelihood estimates of the power law distribution's defining shape parameter, \(\alpha\), the goodness-of-fit of the estimated power law distribution to the observed distribution of phoneme frequencies, and bootstrapped \(p\) values for the null hypothesis that the data are plausibly drawn from a power law distribution.

\begin{table}

\caption{\label{tab:pl-summary}Power law (with no $x_{min}$) summary.}
\centering
\begin{threeparttable}
\begin{tabular}[t]{lcccc}
\toprule
\textbf{ } & \textbf{Mean} & \textbf{SD} & \textbf{Min} & \textbf{Max}\\
\midrule
$\alpha$ & 1.38 & 0.16 & 1.16 & 2.18\\
goodness-of-fit & 0.35 & 0.07 & 0.15 & 0.53\\
$p$ & 0.01 & 0.03 & 0.00 & 0.25\\
\bottomrule
\end{tabular}
\begin{tablenotes}
\item Summary statistics for $\alpha$, goodness-of-fit and $p$ estimates for the power law distribution, fitted to all segment frequencies in each language variety (with no $x_{min}$ parameter).
\end{tablenotes}
\end{threeparttable}
\end{table}

Mean \(\alpha\) is 1.38 (SD 0.16). As discussed in Section \ref{power-laws}, the standard range of \(\alpha\) is 2 \textless{} \(\alpha\) \textless{} 3 \autocite[p.~662]{clauset_power-law_2009}. \(\alpha\) falls within this range for only 1 language. Furthermore, \(p\) values are very low. Just 2 of the 168 languages gives a \(p\) value above the plausibility threshold.

Throughout this study, the possibility of type I error (false positives) must taken into consideration. By setting our implausibility range at \(p \leq\) 0.1, we accept a one in ten chance of incorrectly rejecting a power law hypothesis which in fact is plausible---this can occur when the distribution's poor fit is due to chance fluctuation alone. Given 168 tests (one test per language), we would therefore expect to reject \(H_0\) incorrectly in around 17 (10\%) of those tests. In this instance though, 99\% of the language sample rejected as implausible. Thus it is clear that the power law distribution is not being deemed implausible just by chance. It is genuinely a poor fit for the vast majority of languages. This result accords well with earlier work which has found that a simple, one-parameter form of the power law distribution poorly characterizes phoneme frequencies \autocites{sigurd_rank-frequency_1968}{martindale_comparison_1996}{tambovtsev_phoneme_2007}.

As discussed in Section \ref{data}, our dataset of frequencies for each language is very likely to contain the complete population of phonemes in the language. At the same time, the number of observations per language is low---ranging from 16 to 34 segments in our language sample (mean 24.5, SD 3.7). Such a small set of observations can be a barrier to highly accurate maximum likelihood estimation. \textcite[p.~669]{clauset_power-law_2009} suggest that a minimum sample size of around 50 is needed to get a maximum likelihood estimate of \(\alpha\) accurate to at least 1\%. This is simply not possible for most of the world's languages (including all languages in this study) due to the limited size of segment inventories. Thus, in phonemic studies such as ours there is likely to be an unavoidable uncertainty in the estimate of \(\alpha\).

\hypertarget{power-law-xmin-results}{%
\subsection{\texorpdfstring{Power law distribution with \(x_{min}\)}{Power law distribution with x\_\{min\}}}\label{power-law-xmin-results}}

If the power law distribution, as inferred above, is inadequate for characterising phoneme frequencies, then what other options are there? There are a couple of approaches to this question. One is to add an additional parameter to improve the fit of the power law; the other is to consider alternative distribution types. In this and the following sections we explore both approaches.

Here, we infer the fit of a power law distribution with an additional \(x_{min}\) parameter. As discussed in Section \ref{methodological-considerations}, the \(x_{min}\) parameter serves to remove some of the least-frequent observations from the sample which is being fitted. As above, we use maximum likelihood to infer the best-fitting \(x_{min}\) threshold for each language. Results are summarized in Table \ref{tab:pl-xmin-summary}.

\begin{table}

\caption{\label{tab:pl-xmin-summary}Power law distribution summary.}
\centering
\begin{threeparttable}
\begin{tabular}[t]{lcccc}
\toprule
\textbf{ } & \textbf{Mean} & \textbf{SD} & \textbf{Min} & \textbf{Max}\\
\midrule
$\alpha$ & 2.72 & 0.59 & 1.80 & 5.74\\
goodness-of-fit & 0.14 & 0.03 & 0.08 & 0.23\\
$p$ & 0.61 & 0.27 & 0.03 & 0.99\\
\bottomrule
\end{tabular}
\begin{tablenotes}
\item Summary statistics for the 174 maximum likelihood estimates (1 per language variety) of $\alpha$ (the shape parameter of the power law distribution), goodness-of-fit (Kolmogorov-Smirnov statistic) and bootstrapped $p$ values.
\end{tablenotes}
\end{threeparttable}
\end{table}

After inferring an \(x_{min}\) parameter, the power law distribution is fitted to an average of only 13.9 segments, though there is a wide degree of variation (mean 13.9, SD 3.8). In percentage terms, the power law distribution is fitted to an average of 57\% of a language's segmental inventory (SD 15\%). 124 languages (74\%) fall within the normal 2--3 range for \(\alpha\). Having only a small number of included observations above the \(x_{min}\) threshold can drive unreasonably high estimates of the \(\alpha\) scaling parameter. A sizeable portion of our sample (39 languages, 23\%) fall in this high range with \(\alpha\) above 3. At the other extreme, 5 languages (3\%) have an unusually low \(\alpha\) under 2. Mean \(\alpha\) is 2.72 (SD 0.59).

When \(x_{min}\) is included, the power law hypothesis is accepted as plausible (though, to emphasize, not necessarily correct) in the 159 of 168 language varieties for which \(p >\) 0.1. \(p\) falls below the 0.1 plausibility threshold in the remaining 9 languages. The lowest \(p\) value for any language is 0.03. This puts the chance of incorrectly rejecting \(H_0\) at around one in thirty, which is still high in a set of 168 tests. Overall, since the number of \(p\) values below 0.1 is considerably fewer than the number we would expect to observe through chance, and since there is a high chance that the lowest \(p\) value, 0.03, is a type I error, we cannot confidently rule out the power law hypothesis for any language in our sample.

Although we have failed to rule out the power law distribution as a plausible one for any language, this still does not mean that the power law distribution is the correct one for our data, and there are some important caveats to our results so far.

A distribution will always fit a set of data at least as well as the same distribution with one less parameter. Thus, the observation that the power law distribution fits better when \(x_{min}\) is added requires some interpretation. Of greatest interest in this respect is the striking degree of improvement in fit, such that the power law distribution shifts from a largely implausible fit against full phoneme inventories, to a largely plausible fit after we exclude the least-frequent observations from samples. This raises the obvious question of why this might be so. We consider this in Section \ref{discussion}, after we have also examined distributional alternatives to power laws.

The inclusion of an \(x_{min}\) parameter when fitting power laws is common practice, but its use is most obviously motivated in contexts where there are very many possible observations. For example, Clauset \emph{et al.} \autocite*[p.~684]{clauset_power-law_2009} fit a power law to frequencies of unique words in Moby Dick and find a best-fitting \(x_{min}\) of 7 (\(\pm2\)). Words occurring fewer than 7 times can be disregarded and this still leaves nearly 3,000 unique words to which the power law distribution can be fitted. In contrast to this typical use case, where a large number of observations remain in play and do fit the power law, our use of \(x_{min}\) with phoneme datasets results in the exclusion of data points from an already small sample, leaving an even smaller set of data being fitted. As a general fact, it is inherently difficult to identify the most appropriate distribution for a small collection of observations. Correspondingly, it is not automatically an insightful finding, that a power law can be plausibly fitted to such small datasets. However, as mentioned just above, it is noteworthy that the same power law did not fit well to the slightly larger datasets that were being used without the \(x_{min}\) parameter. This suggests that it is not the small size of the dataset alone which is causing the good plausibility of the fit.

\(p\) values can be inflated when the sample of observations is small, as it is when investigating phonemes. We have good reason to suspect our \(p\) values are being inflated by the low number of observations per language, the evidence being that the number of \(p\) values we observe below 0.1 is considerably fewer than we would expect by chance. The difficulty we find in ruling out the power law distribution may reflect this.

\hypertarget{alternative-distributions}{%
\subsection{Alternative distributions}\label{alternative-distributions}}

In addition to considering the merits of adding extra parameters to a distribution, we must also consider whether a completely different distribution would provide an equal or better fit to the data. We consider three alternative distributions, which are not part of the power law family and may suggest different underlying generative processes. These are the lognormal, exponential and Poisson distributions. Like the power law distribution, the shape of these distributions can have a sharp initial peak and a rapidly decaying tail. The reader will notice that we do not attempt to fit the two-parameter Yule-Simon distribution. This is because, to our knowledge, there is currently no maximum likelihood estimation procedure available for estimating its parameters. We return to the Yule-Simon distribution in Section \ref{discussion}.

\hypertarget{lognormal-distribution}{%
\subsubsection{Lognormal distribution}\label{lognormal-distribution}}

The lognormal distribution is one where the data form a normal distribution when transformed on a log scale. Once again, we use the \emph{poweRlaw} package \autocite{gillespie_fitting_2014} to estimate parameter values using maximum likelihood. In this instance, the parameters to be estimated are log mean and log standard deviation parameters---the log-scale equivalent of the two parameters that define a normal distribution. We fit the distribution to the whole set of segment frequencies for each language---we do not estimate an \(x_{min}\) parameter at this stage (though see below). The lognormal distribution narrowly construed is a continuous distribution, however the \emph{poweRlaw} package contains a corresponding discretized version, appropriate to phoneme frequency data.

As for the power laws above, we calculate bootstrapped \(p\) values to assess the plausibility of the fit of the lognormal distribution for each language. The \(p\) values obtained are highly variable throughout the dataset. There are 74 languages (44\% of the language sample) for which \(p\) falls in the range of implausibility, below 0.1. This is over twice as many as we would expect if the lognormal distribution were plausible for all languages and \(p \leq\) 0.1 values were due to type I error alone. This result is a little difficult to interpret, given the previously discussed difficulties with small samples of observations per language. What seems clear is that, given the rate of \(p \leq\) 0.1 values is elevated beyond chance, we cannot say that the lognormal distribution plausibly characterizes the segment frequencies of all languages. Nevertheless, for many languages---56\% of languages in our sample---we cannot confidently rule out the lognormal distribution. Overall, this makes the lognormal distribution with no \(x_{min}\) a better fit than the power law distribution with no \(x_{min}\), which we ruled out for up to 99\% of languages in the sample. One caveat to keep in mind is that the lognormal distribution is minimally defined by two parameters rather than one, which potentially puts it at an advantage compared to the single-parameter power law distribution.

\hypertarget{exponential-distribution}{%
\subsubsection{Exponential distribution}\label{exponential-distribution}}

Like the lognormal distribution, the exponential distribution is technically a continuous distribution, though the \emph{poweRlaw} package provides a discrete analogue, namely the geometric series distribution \eqref{eq:sigurds-geometric}, as proposed by \textcite{sigurd_rank-frequency_1968}. As above, we use maximum likelihood to estimate the rate parameter, \(\lambda\), and use the bootstrapping procedure to obtain a \(p\) value.

Bootstrapped \(p\) values are above the 0.1 plausibility threshold for 148 of 168 languages. The number of languages for which \(p \leq\) 0.1 is 20, right on the 17 or so that we would expect from type I errors. This, on the face of it, seems to make the exponential distribution quite a plausible model for phonological segment frequencies more generally. It must be noted, however, that there are a few languages for which the exponential distribution is a very poor fit. The most extreme, Warlmanpa, has goodness-of-fit statistics greater than 0.25 and a \(p\) value of just 0.002. The poor quality of fit is visually evident on a log-log plot (see S4, Supplementary Materials).

\hypertarget{poisson-distribution}{%
\subsubsection{Poisson distribution}\label{poisson-distribution}}

The final distribution we consider is the Poisson distribution, which is related to the exponential distribution. The Poisson distribution is typically used to model the frequency of an event within some interval of time or space. Our case is a bit different since we are modelling the relationship between the frequency of many different events (different phonological segments) and their frequency rank in a language's phonological inventory. As with the exponential distribution, we use maximum likelihood to estimate a single parameter, \(\lambda\), and use bootstrapping to obtain a \(p\) value for the plausibility of the distribution.

The Poisson distribution is totally implausible for all languages in our language sample. Goodness-of-fit statistics range from 0.45 to 0.75 (mean 0.59, SD 0.07). We find \(p\) values indistinguishable from 0 in all cases.

\hypertarget{summary-of-results}{%
\subsection{Summary of results}\label{summary-of-results}}

In Table \ref{tab:results-summary}, we summarize results for the four distribution types evaluated in this study. For each distribution type, we give the number of languages for which the distribution's fit was deemed plausible (\(p >\) 0.1). For completeness, we give results for the exponential, lognormal and Poisson distributions when \(x_{min}\) is included, just as we did for the power law distribution in Section \ref{power-law-xmin-results}. Perhaps most noteworthy is the greatly increased inconclusiveness of the method when applied to the reduced set of data points lying above the \(x_{min}\) threshold. When the fitting task is restricted to a subset of only the most frequent segments in a language, it is possible to plausibly fit all but the Poisson distribution to any language, after type I error is factored in.\footnote{For one language, the bootstrapped \(p\) value estimation procedure failed to converge for the lognormal distribution with \(x_{min}\). This is the only distirubiton we tested which has three free parameters, and in this instance, the algorithmic procedure struggles to differentiate solutions with very similar likelihoods.} One difference is that power law distributions with \(x_{min}\) are fitted on overage to only 57\% of a language's phonemes, whereas the lognormal and exponential distributions are fitted to closer to 80\%. This difference is nuanced further in the next section.

\begin{table}

\caption{\label{tab:results-summary}Summary of results}
\centering
\begin{threeparttable}
\begin{tabular}[t]{lccc}
\toprule
\textbf{ } & \textbf{Without $x_{min}$} & \textbf{With $x_{min}$} & \textbf{Prop. fitted}\\
\midrule
Power law & 2 (1\%) & 159 (95\%) & 57\%\\
Lognormal & 94 (56\%) & 157 (93\%) & 79\%\\
Exponential & 148 (88\%) & 149 (89\%) & 85\%\\
Poisson & 0 (0\%) & 42 (25\%) & 17\%\\
\bottomrule
\end{tabular}
\begin{tablenotes}
\item The four distribution types evaluated in this study are listed, both with and without $x_{min}$. For the two parameterisations, we give the number of languages (as a raw count and percentage of the language sample) for which the uncorrected $p$ value indicates the distribution is a plausible characterisation. For languages whose fits with the $x_{min}$ parameter are plausible, the third column reports the mean proportion of the inventory (with frequency $\geq x_{min}$) that is fitted.
\end{tablenotes}
\end{threeparttable}
\end{table}

\hypertarget{comparison-of-distribution-types}{%
\subsection{Comparison of distribution types}\label{comparison-of-distribution-types}}

As outlined in Section \ref{method}, the third and final step in Clauset \emph{et al.}'s \autocite*{clauset_power-law_2009} framework is a likelihood ratio test. \textcite{clauset_power-law_2009} suggest using Vuong's \autocite*{vuong_likelihood_1989} likelihood ratio test for model selection to determine the best-fitting of two competing models, when there are multiple plausible candidate distributions (as determined through estimating goodness-of-fit and the bootstrapping procedure performed previously). If bootstrapping were to show that only one distribution type plausibly fits the data, a likelihood ratio test would be unnecessary. Recall, though, that just because a distribution is shown to be plausible via the bootstrapping process does not mean that that distribution is the correct one, since there may be other equally or more plausible distributions.

Vuong's \autocite*{vuong_likelihood_1989} test uses the Kullback-Leibler Information Criterion \autocite{kullback_information_1951} to calculate the log likelihood of observing the data given a distribution model, and compares this to the log likelihood of observing the same data given a competing distribution model. The test returns a test statistic, which gives an indication of how strongly one model is favoured over another, and a \(p\) value, indicating whether the difference in the support for each model is statistically significant.

We begin by comparing distributions without the \(x_{min}\) parameter. As summarized in Table \ref{tab:results-summary}, two of these distributions (the power law and Poisson distributions, without \(x_{min}\)) have already been rejected as implausible for all or nearly all languages. Accordingly, we conduct just one likelihood ratio test per language, comparing the fit of the exponential versus lognormal distributions. Overall, we find that Vuong's likelihood ratio test somewhat favours the exponential distribution. Likelihood ratios favour the exponential distribution for 119 languages, and the lognormal distribution for 49 languages. However after Bonferroni correction, the difference in the likelihood of exponential and lognormal models is statistically significant for only two languages, Thaynakwithi and Linngithigh, both of the Northern Paman subgroup of Pama-Nyungan, both favouring the exponential distribution.

Turning to distributions with the \(x_{min}\) parameter, since we have already rejected the Poisson distribution, we conduct likelihood ratio tests pairwise among the remaining three distributions. In order to compare distributions with \(x_{min}\) parameters, it is necessary to set \(x_{min}\) to the same value in both distributions \autocite{gillespie_fitting_2014}. Thus, to make a pairwise comparison, we take the \(x_{min}\) value from distribution A and using it, re-estimate the other parameters of distribution B, and conduct one likelihood ratio test. Then we take \(x_{min}\) from B, use it and re-estimate the other parameters of distribution A, and conduct a second likelihood ratio test, giving two results for each pair of distributions.

Comparing the exponential and lognormal distributions, the likelihood ratios favour the lognormal distribution (141 languages to 27) using \(x_{min}\) from the lognormal fit, and favours the exponential distribution (104 languages to 64) using \(x_{min}\) from the exponential fit, however none of these comparisons reaches significance after Bonferroni correction.

Comparing the power law and lognormal distributions, likelihood ratios favour the lognormal distribution (145 languages to 23) using \(x_{min}\) from the power law fit, and all languages when using \(x_{min}\) from the lognormal fit, however only two of these comparisons reaches significance after Bonferroni correction. Yir Yoront favours the power law when using \(x_{min}\) from the power law fit and Malyangapa favours the lognormal distribution using \(x_{min}\) from the lognormal.

Comparing the power law and exponential distributions, the likelihood ratios favour the power law (138 languages to 30) when taking \(x_{min}\) from the power law fit, though no comparison reaches significance. They favour the exponential distribution 165 languages to 3 when \(x_{min}\) is taken from the exponential fit. Thirteen of those comparisons reach significance.

In sum, we found earlier that when parameterized without \(x_{min}\), only the exponential and lognormal distributions were broadly plausible. Voung's likelihood ratio test marginally favours the exponential test over the lognormal when fitted against entire phonemic inventories, but the difference is at most slight. When parameterized with \(x_{min}\), the power law distribution is fitted to around 60\% of languages' phonemes on average, while the exponential and lognormal are fitted to around 80\% (Table \ref{tab:results-summary}). Pairwise likelihood ratio tests, which apply one distribution's \(x_{min}\) parameter to the other, provide slender evidence of the following. Even when fitted against the small phonemic subsets favoured by the power law, the lognormal distribution may weakly outperform the power law, but the exponential distribution does not. Fitted against the larger subsets favoured by the exponential and lognormal distributions, the power law is outperformed by the exponential and lognormal. The performance of the latter two distributions is indistinguishable. In the next section, we relate these findings to prior literature and return to the question of the two-parameter Yule-Simon distribution. Full results of all likelihood ratio tests described in this section are tabled in Section S3 of the Supplementary Materials.

\hypertarget{discussion}{%
\section{Discussion}\label{discussion}}

Power laws have attracted wide scientific interest and, more recently, debate on their validity \autocites{clauset_power-law_2009}{stumpf_critical_2012}. Power laws have long been proposed for characterising natural language phenomena \autocites{estoup_gammes_1916}{zipf_selective_1932}{zipf_human_1949}. Recently, one kind of power law, the Yule-Simon distribution, has been proposed for characterising phoneme frequencies \autocites{martindale_comparison_1996}{tambovtsev_phoneme_2007}, however the procedure used to deduce its superiority is now known to be unreliable, and the explanation attached to the finding---that high-frequency phonemes are less power-law-like---is not supported by the mathematical shape of the Yule-Simon distribution, which is power-law-like for high-frequency items and geometric-like for low-frequency items. Here, we re-evaluated the plausibility of several distribution types as characterisations of phoneme frequencies, in light of recent debate on power laws generally, using a maximum likelihood statistical framework presented by \textcite{clauset_power-law_2009} and a sample of 168 Australian language varieties.

Using more a robust evaluation procedure than previous investigations, we have confirmed the finding that a basic power law distribution, with a single free parameter, is generally insufficient for characterising phoneme frequencies. Additionally, we reconfirm a result going back to \textcite{sigurd_rank-frequency_1968}, that an exponential (or geometric) distribution, with a single free parameter, is a good plausible fit for full phonemic inventories. Furthermore, we find that a lognormal distribution, with two free parameters, is an additional plausible fit, whereas a Poisson distribution, with a single free parameter, is implausible.

Another novel contribution here was to consider the addition of an \(x_{min}\) parameter, a practice which is now common in power law research. Notably, while power laws are largely implausible fits for entire phoneme inventories, their plausibility is improved strikingly once a subset of the least-frequent phonemes is removed from the sample. This is despite that fact that the full inventories and the reduced ones share the property of comprising notably small samples. The subset removed in order to achieve maximum likelihood is on average large, at just over 40\%. This result indicates that power laws constitute a plausible characterisation for the more-frequent portion of phonemic iventories, and explains why the upper end of a Yule-Simon distribution, which most closely approximates a power law, should be a reasonable fit. We note however, that the lognormal distribution also performs well in this same, high-frequency region of phonemic inventories. Exponential (or geometric) distributions do not fit the higher-frequency portion of inventories as well the power law or lognormal do, but they are good fits for entire inventories, suggesting that they fit particularly well in lower-frequency portions. This would explain why the lower end of a Yule-Simon distribution, which most closely approximates a geometric distribution, should be a reasonable fit.

Using an evaluation procedure which has since been shown to be unreliable, \textcite{martindale_comparison_1996} and \textcite{tambovtsev_phoneme_2007} concluded that the two-parameter Yule-Simon distribution fit whole inventories better than a power law or a geometric distibution. Here we have not been able to directly evaluate the Yule-Simon distribution using a more robust, maximum likelihood method owing to a lack of a maximum likelihood estimation procedure for its parameters. However, we have found evidence supporting a similar conclusion, that the more-frequent and less-frequent portions of phonemic inventories are characterized by different distributional properties. The more-frequent portion better matches a power law, but also a lognormal distribution. This finding serves to clarify and qualify one half of the main finding of \textcite{martindale_comparison_1996} and \textcite{tambovtsev_phoneme_2007}. The less-frequent portion better matches a geometric distribution. This accords with the other half of \textcite{martindale_comparison_1996} and \textcite{tambovtsev_phoneme_2007}'s main finding, but here we have arrived at it by more robust and reliable methods. Furthermore, by estimating \(x_{min}\) parameters, we have provide some estimate of where power-law-like behaviour starts to cut out within a phonemic inventory. To understand what these results entail for theory, we return to the question of causal processes.

\hypertarget{conclusions}{%
\section{Conclusions}\label{conclusions}}

Linguistic theorising will be aided by a sound knowledge of which distributions plausibly characterize a variable \(x\) (such as phoneme frequency), since those distributions will be consistent with only certain mathematical kinds of underlying causal processes. Thus knowledge of distributions helps us by placing an empirical filter upon viable causal explanations. In this paper, we have improved the certainty of our understanding of the distributions of phoneme frequencies, using state-of-the-art statistical methods. By the same token, we should not expect that this one empirical filter will do all the work. In our case, we are not yet able to decide empirically, for example, whether a lognormal or power law distribution better characterizes high-frequency phonemes. However, it may be possible to distinguish between such options on other grounds. For instance, it may be that the causal processes themselves, which generate such distributions, are differentiable in terms of their plausibility, on some basis other than merely the distribution that emerges from them.

It is beyond the scope of this paper to pursue questions of causation that lie behind the distributions we have uncovered. However, a fruitful next step used widely in other sciences is to explicitly consider mathematical families of stochastic processes, using these as a bridge between real-world candidate causal processes and the mathematical implications they have such as observed empirical distributions. For example, many discrete systems can be profitably conceptualized in terms of urn processes, that are associated with characteristic distributions. As \textcite[p.~87]{kuba2012limiting} remark, ``{[}u{]}rn models are simple, useful mathematical tools for describing many evolutionary processes in diverse fields of application''. There exist well-studied urn process which yield many kinds of distributions, and it will be profitable in linguistic research to more clearly relate our own theories of change, including change in phonemic inventories, to these mathematically more generalized processes. By doing so, linguists will be able to tap into related mathematical results (such as relating processes to distributions), that can assist us to further differentiate the theories that are more viable from those that are less so.

In this paper we have also demonstrated a template for future work on distributions themselves. Ideally, such work should begin with critical assessment of links that can be made between existing or new causal hypotheses, including diachronic processes, and particular distributional outcomes. Subsequently, the fit of the hypothesized distribution to real-world data should be evaluated rigorously using robust statistical methods. Lastly, an attempt must be made to rule out competing distribution types and alternative generative mechanisms. As we have demonstrated, this may well be challenging, given the inherent limitations of working with small sets of observations.

The challenges of small datasets should in turn motivate innovation in the kinds of variables that linguists investigate. There may be gains to be made in combating small dataset sizes through methodological innovations around how frequency data is assembled, for example, through paying attention to phonotactic position, natural classes of phonemes, or by creating aggregated datasets for subgroups or language families. In this study we have focused on phoneme frequencies because they have occupied such a prominent place in the history of investigations of distributions. However, by doing so we emphatically do not suggest that phonemes ought to continue to occupy such a prominent place, when more interesting and possibly more tractable phenomena still await investigation.

As linguists increasingly adopt quantitative methods to reveal empirical generalisations and make theoretical advances, we advocate a rigorous approach to evaluating the mathematical distributions with which we characterize linguistic variables. Claims about distributions entail claims about causal explanatory processes. Used critically, they are powerful tools that can be a source of unique and invaluable insight.

\hypertarget{Acknowledgements}{%
\subsubsection{Acknowledgements}\label{Acknowledgements}}

JLM-C is supported by an Australian Government Research Training Program Scholarship. Data compilation was funded by Australian Research Council grant DE150101024 to ER.

% ***************************************************


%CHAPTER 4
%If you are presenting work which has been previously published, acknowledge this here.
% ***************************************************
% How to introduce a previously published chapter
% ***************************************************
%This is an example of how you might introduce a chapter that has been published previously. 
\cleartoevenpage
\pagestyle{empty}	
%Use this command (above) to suppress the header from the preceding chapter.

\noindent
Text from the following publication has been incorporated as Chapter~\ref{ch-pcms}:

\noindent
%\fullcite{DumyCitationKey}
Jayden L. Macklin-Cordes \& Erich R. Round. In prep. Phylogenetic comparative methods in linguistics. \emph{Details TBA}.

\begin{table}[h]
	\centering
	\begin{tabular}{clr}
		\toprule
		Contributor & Statement of contribution & \% \\
		\midrule
		\textbf{Jayden Macklin-Cordes}	& initial concept			& 100 \\
		                                & analysis     	            & 100  \\
		                                & theoretical derivations 	& 100  \\
		                                & writing of text 			& 100  \\
										& proof-reading				& 50  \\
		\midrule
		Erich Round						& supervision, guidance 	& 100 \\
										& proof-reading				& 50  \\
		\bottomrule
	\end{tabular}
\end{table}

The following chapter will form part of a publication which is in preparation. Macklin-Cordes has taken responsibility for the research essay component of this paper, which is what is presented here in this thesis. In the eventual published version, Round will contribute the results of a case study looking at laminal contrasts in Australian languages.

% ***************************************************
% Example of an internal chapter
% ***************************************************
%This is an internal chapter of the thesis.
%If you have a long title, you can supply an abbreviated version to print in the Table of Contents using the optional argument to the \chapter command.
\chapter[Phylogenetic Comparative Methods]{Phylogenetic Comparative Methods in Linguistics}
\label{ch-pcms}	%CREATE YOUR OWN LABEL.
\pagestyle{headings}

% ********* Enter your text below this line: ********

Historical linguistics is increasingly making use of phylogenetic methods. However, phylogeny is consequential for all comparative study, including synchronic typology. Many fields of science, including linguistic typology, have a long history of considering phylogenetic relationships in data sampling. However, a family of statistical methods, \emph{phylogenetic comparative methods}, enable the incorporation of phylogeny directly in a statistical model, with no loss of data. This paper clarifies the logic behind phylogenetic comparative methods, and argues for their applicability in linguistic typology. I make the case for phylogenetic comparative methods firstly by surveying responses to the issue of phylogenetic independence in linguistics and other fields of science. I find that all fields share, in origin, similar lines of development in sampling methodology to create phylogenetically independent samples. However, since embracing quantitative methods, comparative biologists have developed mathematical frameworks for identifying and modelling phylogenetic effects in statistical analysis. Linguists have the opportunity to incorporate this wealth of experience into comparative research design. Latter sections of the paper outline how this can be done, firstly by describing \emph{phylogenetic signal} and methods for its measurement. Secondly, I survey existing applications of phylogenetic signal methods and phylogenetic comparative methods more generally in linguistics. Drawing together previous sections, I argue for the continued uptake of phylogenetic comparative methods in linguistics and describe some implications of this. A desirable outcome for future research would be the collation of high-resolution phylogenetic language trees inferred over the past two decades, brought together into a common data structure. Furthermore, whole-family language sampling in concert with phylogenetic comparative methods should be considered in linguistic typology, beyond language samples that aim for much more sparse, global coverage.

\hypertarget{pcm-intro}{%
\section{Introduction}\label{pcm-intro}}

The comparative language sciences are indispensable in the study of human language and offer a unique contribution to the study of human history. Besides being an intriguing academic pursuit in its own right, historical linguistics can be triangulated with other fields, such as genetics, archaeology and anthropology, to infer early population movements and interactions between cultures \autocites[e.g.][]{hunley_genetic_2008}{gray_language_2009}{bouckaert_mapping_2012}{malaspinas_genomic_2016}{bouckaert_origin_2018}. Similarly, linguistic typology has been combined with fields including ecology and physiology to produce some remarkable theories on the evolution of human language \autocites[e.g.][]{everett_climate_2015}{everett_languages_2017}{bentz_evolution_2018}{blasi_human_2019}.

The fundamental task of comparison underlies both linguistic typology and historical linguistics. In this respect, they are synchronic and diachronic sides of the same coin. In the historical case, extant language data are compared to infer past relationships between languages that cannot be observed directly due to the passing of time. In typology, language data are compared as well, though with different aims. These aims principally concern the nature of human language itself---the limits of possibility and tendencies in language structures, for example. The task of cross-linguistic comparison is complicated, however, by the interwoven patterns of historical descent and language contact that inevitably bind languages to varying degrees and manifest in shared linguistic forms and features observable today. Consequently, shared histories must be taken into account before drawing any inferences from cross-linguistic datasets. The prevalence of a linguistic variable may give the appearance of a particular linguistic tendency if languages are considered independent, but it may also be explained by a single linguistic innovation which was subsequently inherited by a large number of descendant languages. Shared linguistic histories must also be considered in any statistical analysis, since independence of data points is a fundamental assumption of many statistical methods.

Non-independence due to shared histories through descent, termed \emph{phylogenetic non-independence} or, more precisely, \emph{phylogenetic autocorrelation}, is not an unfamiliar concept in linguistics, nor other fields where entities share common paths of descent, such as biology and anthropology. Over a century of thought and methodological development has been dedicated to the topic. However, divergences exist in the lines of thought and development of different fields. This paper considers this old discussion within a cross-disciplinary scope. There are many challenges associated with accounting for phylogenetic autocorrelation in comparative methods, however I find that certain challenges are not as unique to linguistics as often has been assumed. Comparative biologists continue to give a good deal of consideration to methodological challenges of interest to linguists as well. In particular, I find that a family of statistical methods, \emph{phylogenetic comparative methods} (PCMs), are immediately applicable to comparative linguistics. In this paper, I elucidate how and why this is the case and demonstrate empirically the need to account for phylogenetic autocorrelation in variables which might have been assumed to be distributed independently of phylogeny, based on previous descriptions---these variables concern the typology of laminal consonants in the Pama-Nyungan languages of Australia.

This paper proceeds as follows. Section \ref{phylo-autocorrelation} reviews literature on \emph{phylogenetic autocorrelation}---the tendency of languages to show similarities due to phylogenetic relatedness---in linguistics and cognate fields (comparative biology, in particular). The aim is to provide a broad picture of the scientific context that motivates the methodologies discussed later on. While identifying historical signal in a source of data is of clear interest to historical linguistics, it is also of interest to linguistic typology, where phylogenetic autocorrelation is a source of bias that must be controlled. I therefore survey literature from an array of comparative fields of science in which phylogenetic autocorrelation occurs, comparing methodological approaches for identifying and accounting for patterns of historical relatedness among observations in comparative datasets. Section \ref{phylo-sig} outlines some statistical tools for quantifying \emph{phylogenetic signal}, the degree of phylogenetic autocorrelation present in a comparative dataset. Quantifying phylogenetic signal is the first step in a phylogenetically informed comparative methodology---the presence or absence of phylogenetic signal determines the need for phylogenetic comparative methods in subsequent analysis. Then, in Section \ref{pcms-applications}, I discuss examples of studies measuring phylogenetic signal in linguistics. For instance, I consider the example of laminal phonemes in Pama-Nyungan languages (Australia). In this example, \textcite{round_continent-wide_2017} shows that, although place contrasts of laminal consonants are traditionally described as being areally distributed and independent of the Pama-Nyungan family's internal phylogeny, laminal consonants do show phylogenetic patterning when finer-grained variables are examined that capture matters of frequency. A finding like this emphasises the need to consider phylogenetic autocorrelation in linguistic typology, even when studying phenomena that might be considered `safe' from the effects of phylogeny based on prior research. I conclude by advocating for the continued uptake of phylogenetic comparative methods in linguistics. I discuss the implications of this uptake for language sampling strategy in linguistic typology and sketch an outline for future research.

\hypertarget{phylo-autocorrelation}{%
\section{Phylogenetic autocorrelation}\label{phylo-autocorrelation}}

Phylogenetic autocorrelation is common to many comparative fields of science, and linguistics is no exception. Phylogenetic autocorrelation is a potential problem for comparative study, because shared phylogenetic histories limit the independence of observations in a comparative dataset. Observations from more closely related entities will tend to show less variation than more distantly related entities, because they share a longer period of evolutionary history prior to splitting off from their most recent common ancestor, and will have had less time to diverge evolutionarily. If this tendency towards similarity due to shared phylogenetic history is not taken into account, it will introduce bias into the dataset and consequently affect statistical analysis. Different fields have their own lines of literature grappling with this phenomenon extending back many decades. Although there are many similarities between fields, key differences emerge since the uptake of quantitative methods in comparative biology. This section discusses phylogenetic autocorrelation and the history of responses to it in different fields, focusing in particular on linguistics (Section \ref{phylo-auto-ling}) and biology (Section \ref{phylo-auto-bio}).

\hypertarget{phylo-auto-ling}{%
\subsection{Phylogenetic autocorrelation in linguistics and other fields}\label{phylo-auto-ling}}

Both historical linguistics and linguistic typology are comparative fields, in that they necessarily rely on cross-linguistic datasets and the task of making comparisons between different languages is inherent to both. The presence of historical signal in a dataset---where the set of values reflects something of the history of featured languages---will be of interest to any researcher working comparatively, whether they are directly interested in reconstructing the history of those languages (as in historical linguistics) or not (as in linguistic typology). In the case of typology, this is because historical signal represents a kind of statistical non-independence between languages. Statistical non-independence between languages due to shared history is no new revelation in linguistic typology, however there are many possible approaches to dealing with it and a sizeable body of literature on the topic.

We explore typological literature further below, but first I take up the topic of phylogenetic non-independence in more detail. In any cross-linguistic study, there will be some degree of statistical non-independence between languages. This is because languages do not evolve independently, but share various historical connections through time, resulting in shared lexical or grammatical material, whether through inheritance from a common ancestor or borrowing from neighboring languages. Historical relationships between languages are, therefore, an inescapable concern for linguistic typology. Likewise, typological comparison has a place in historical linguistics, as researchers have investigated historical questions by applying statistical methods to typological datasets. A seminal example of this is \textcite{nichols_linguistic_1992}, and more recent examples within linguistic phylogenetics include \textcite{dunn_structural_2005}, \textcite{dunn_structural_2008}, \textcite{rexova_cladistic_2006}, \textcite{reesink_explaining_2009} and \textcite{greenhill_evolutionary_2017}. Non-independence of languages due to historical relationships, and the consequences for cross-linguistic comparison, has direct and indirect implications for both fields.

As noted above, linguistics is far from the only field to face the challenge of phylogenetic non-independence. In comparative anthropology, this issue was noted as early as 1889 by Sir Francis Galton in the context of cross-cultural datasets, which lack independence due to shared histories of cultural innovation and exchange between societies \autocite[p.~15]{naroll_two_1961}. This phenomenon, known as \emph{Galton's Problem}, is now more precisely understood as a form of statistical autocorrelation (similarity between observations as a function of the time lag between them). The same phenomenon has been recognised in comparative biology too. A seminal study concerning comparative studies of phenotypes, \textcite{felsenstein_phylogenies_1985} demonstrates that data from species cannot be assumed to be independently drawn from the same distribution, because species are related to one another via a branching, hierarchical phylogeny, thus, statistical methods that assume independent, identically-distributed observations will inflate the significance of the test (discussed further in Section \ref{phylo-auto-bio} below). Linguists, it has been argued, have been somewhat slower than those in other fields to acknowledge exposure to Galton's problem, or phylogenetic autocorrelation \autocite[p.~293]{perkins_statistical_1989}. Nevertheless, this is a central concern of \textcite[p.~259]{dryer_large_1989} and has been addressed in a considerable body of linguistic literature since then.

Although precise strategies are varied, common to all fields is a history of addressing phylogenetic autocorrelation at the data sampling stage. In linguistics, the use of sampling methods for creating a phylogenetically independent or phylogenetically balanced language sample remains the predominant way of accounting for phylogenetic autocorrelation and literature on this topic extends back several decades. \textcite[pp.~145--149]{bell_language_1978} argues that common strategies which simply ensure equally-weighted representation of ``all major families'' or all continents is inadequate due to differing rates of divergence among families. He estimates the number of language groups separated by more than 3,500 years of divergence and uses it as a heuristic for estimating genetic biases in a selection of proposed language samples. He concludes that European languages tended to be overrepresented and Indo-Pacific languages underrepresented and attributes this to a corresponding over/under-representation among quality language resources, which is a persistent problem for comparative linguistics. Perkins \autocites*{perkins_evolution_1980}{perkins_covariation_1988} creates a sample of 50 languages, later adapted by \textcite{bybee_morphology_1985}, which attempts to account for both genetic and areal biases by selecting no more than one language from each language phylum \autocite[following][]{voegelin_index_1966} and no more than one language from each cultural and geographic area \autocites[following][]{kenny_numerical_1975}{murdock_ethnographic_1967}. This method attempts to account for nonindependence due to areal spread, unlike Bell's heuristic measure which accounts only for genetic bias, however it does not account for differing ages of divergence and size of language phyla in the way Bell does. Crucially for this discussion, these sampling methods inherently have what I refer to as a \emph{lossy} quality. In information technology, lossy methods of data compression involve erasure of parts of the data to create a smaller approximation of the original file (for example, JPEG image files). Similarly, in the language sampling methods thus far described, a smaller approximation of a larger language sample is created by removing languages with certain historical connections that would compromise the independence of observations in the sample.

There are a number of shortfalls associated with lossy sampling strategies. At the crux of these is that it may not be possible to create an independent sample of sufficient size when distant genetic relationships and long-range areal phenomena are considered. There may be uncertainty and scope for disagreement on the independence of languages in a sample. \textcite[p.~261]{dryer_large_1989} refers to the example of the inclusion of three languages in Perkins' sample (Ingassana, Maasai and Songhai) which may be related as part of the Nilo-Saharan family, although these relationships are remote and subject to debate. When the largest proposed areal and genetic groupings are considered, it may simply not be possible to create an independent sample of a sufficient size for generating statistically significant inferences. Another problem is that the maximal extent of presently established language families is partially a product of the extent of adequate documentation and scholarly attention, rather than a completely true reflection of the fullest extent to which the family may be reconstructed. That is to say, two languages which are presently understood to be unrelated, and therefore statistically independent, may in fact belong to a shared larger grouping, which has not yet been identified due to poor documentation or some other factor. \textcite[p.~263]{dryer_large_1989} raises another concern, which is that languages selected on the basis of genetic independence may nonetheless share characteristics due to non-genetic processes---language contact and borrowing. When areal phenomena are considered, Dryer contends, the practicality of constructing a truly independent sample of sufficient size is further stretched. Dryer's proposed solution is to build a sample of languages of approximately equal relative independence (at the level of major subfamilies within Indo-European, such as Romance, Germanic, and so on) for each of five large linguistic areas which are assumed to be independent, or at least sufficiently independent for statistical purposes. Any statistical test can then be applied to each of the five areas and only if the same result is replicated in all five areas is it considered statistically significant. If the same result is replicated in four of five areas, this falls short of statistical significance, although \textcite[pp.~272--273]{dryer_large_1989} considers such cases to be evidence of a ``trend''. Even still, as \textcite[p.~284]{dryer_large_1989} acknowledges, his five linguistic areas may be subject to the same concerns about undetected historical non-independence and it is possible that the whole world may, in effect, function as a single linguistic area, such that the distribution of certain linguistic features may reflect extremely remote areal or genealogical pattern rather than some true tendency of human language.

\textcite[p.~41]{nichols_linguistic_1992} uses Dryer's area-by-area testing method as part of a three-pronged approach. For any given question, Nichols first conducts a chi-square test of the world sample and then re-tests the significance of the finding using either Dryer's method or by running the same test on only the sample of ``New World'' languages (comprising North, Central and South America). \textcite{rijkhoff_method_1993} and \textcite{rijkhoff_language_1998} develop another approach to account for the possibility of non-independence across large linguistic areas and large, as-yet-undetected families. They permit multiple languages within a family to be included but develop a measure, based on the density of nodes in a known language phylogeny, to determine how many languages should be included. In this way, they also aim to account for the fact that some language families will have greater internal diversity than others \autocites[see also][]{bakker_language_2011}{miestamo_sampling_2016}. Another proposed method is to set a minimum threshold of typological distance between languages, calculated from the \emph{World Atlas of Language Structures} (WALS) \autocite{dryer_wals_2013}, such that languages must be sufficiently typologically distinct from others in the sample to warrant inclusion. \textcite{bickel_refined_2009} develops an alternative algorithm based on \textcite{dryer_large_1989}, which allows all uniquely-valued data points within a family to be included in the sample, but then reduces the weighting of data points in the final analysis where a particular value is over-represented within a family. In other words, if all the languages in a particular family share the same value for a variable of interest, those observations may be reduced to a single data point, since this homogeneity is likely the result of a shared retention or innovation.

All up, the developments in typological methodology that have been discussed here demonstrate that historical non-independence between languages has been treated predominantly as a sampling issue in linguistics. Earlier researchers sought to maintain the independence of their sample by maximising the genetic distance between the languages in their sample, such that no two languages were known to belong to the same family. Later, with subsequent acknowledgement of the possibility of non-independence from very large language families, as well as large-scale areal diffusion and effects from as-yet undetected or unconfirmed historical relations, it became apparent that it may be impossible to create a sample which is simultaneously independent and sufficiently large to generate statistical significance. The response to this has been a variety of robustness checks, even bootstrapping-like processes, whereby languages are sampled at an approximately equal relative level of independence and the sample is then subdivided in some way and a statistical test replicated over each subdivision. More recent years have seen the continued evolution of statistics and robustness checking methods \autocite[for an overview, see][]{roberts_robust_2018}, although balanced sampling remains a common element of modern, large-scale comparative linguistic studies \autocites[for example,][]{everett_climate_2015}{everett_languages_2017}{blasi_grammars_2017}.

\hypertarget{phylo-auto-bio}{%
\subsection{Phylogenetic autocorrelation in comparative biology}\label{phylo-auto-bio}}

Comparative biology faces the same issue of phylogenetic autocorrelation as comparative linguistics. Many conventional statistical methods assume that observations are independent and identically distributed, which is problematic in biology since observations come from species, which are related to one another through shared evolutionary histories (as charted graphically in a tree diagram). Further, \textcite[p.~4]{felsenstein_phylogenies_1985} shows that even non-parametric statistics are not immune to violations of this independence assumption. Although both fields face the same phenomenon in essence, linguistics and biology have diverged in their methodological response in recent decades. While linguistics continues to focus on sampling procedures, giving less attention to the methods of subsequent statistical analysis, comparative biologists have shifted towards more direct, statistical solutions.

Earlier approaches to phylogenetic autocorrelation in biology are in a similar vein to the sampling methods discussed in the previous section. \textcite[pp.~346--347]{harvey_comparisons_1982} seek to find a taxonomic level to sample from, which strikes the right balance in terms of being sufficiently statistically independent without being so conservative that sample sizes become prohibitively small. Their proposed solution is to identify and sample from the lowest taxonomic level which can be ``justified on statistical grounds''. One method of doing this is suggested by \textcite[pp.~6--8]{clutton-brock_primate_1977}, who conduct a nested analysis of variance and then select taxonomic level containing the greatest level of variation. Once \textcite{clutton-brock_primate_1977} identify their taxonomic level of interest, they average out data for all species within a given genus for which they have data. In other words, the unit of analysis has shifted from individual species to genera, and each data point represents a genus in the form of an averaged representation of all the species within the genus. Although a similar method in essence, this genus-level averaging process is in marked contrast to balanced sampling methods discussed in the previous section, where an unaltered observation from a single exemplar language is taken as representative of its given family, subfamily or other defined grouping.

\textcite[pp.~85--86]{baker_evolution_1979} discuss the same problem. They use an approach not too dissimilar from the area-by-area robustness checking by \textcite{dryer_large_1989} and \textcite{nichols_linguistic_1992}. \textcite{baker_evolution_1979} replicate their analysis within individual families as well as within different ecological areas, with the assumption that if the same associations are observed within different groups as they are across the dataset as a whole, then one can discount the possibility that the full analysis is simply picking up differences between different families or different ecological groups. A contrasting approach, at least in instances where categorical data are of interest, is to reconstruct ancestral states throughout the phylogeny, enabling one to directly observe whether species which share a common trait do so because of (non-independent) shared inheritance from a common ancestor or whether the traits have evolved independently. \textcite{gittleman_phylogeny_1981} uses a \emph{parsimony model} to reconstruct states in this way. This is where the states of traits at ancestral nodes in a tree are reconstructed in such a way as to minimise the number of evolutionary changes that would need to take place to produce the observed data for extant species on the tips of the tree. There are a couple of shortfalls to this approach. One is that it is only a partial solution to the problem---it tells us which data points are independent and which are not, but besides potentially being used as a tool to identify a taxonomic level with the greatest level of diversity \autocite[proceeding in a similar way to][]{clutton-brock_primate_1977}, there is no indication of how to proceed with comparative study when non-independence has been identified. Further, there are biases in the parsimony method \autocite[p.~7]{felsenstein_phylogenies_1985} which are unlikely to be satisfying for linguists---for example, if a small group of related species (or languages) all share a trait, they will always be reconstructed as inheriting the shared trait from the nearest common ancestor, despite the possibility of parallel evolution (homoplasy), horizontal diffusion, environmental pressures, and so on.

\textcite{felsenstein_phylogenies_1985} contends that it is possible to account for phylogenetic non-independence in a statistical model without the need to remove non-independent data points or compromise the unit of analysis (by, for example, comparing averaged data points representing genera rather than individual species). Felsenstein's breakthrough insight is that this can be achieved not by directly comparing non-independent observations but by comparing \emph{phylogenetically independent contrasts} (PICs) between them. His method has become, by one estimate, the most widespread in comparative biology \autocite[p.~162]{nunn_comparative_2011}. Calculating phylogenetically independent contrasts is possible given a continuously-valued variable of interest, an assumed phylogeny and an assumed model of variable evolution. As a natural starting point, Felsenstein assumes a \emph{Brownian motion} model of evolution. This is where an evolving trait can wander positively or negatively with equal probability, and each new time step is independent from the last, with the resulting effect that displacement of the variable over time will be drawn from a normal distribution with a mean of zero and variance proportional to the amount of elapsed time \autocite[p.~8]{felsenstein_phylogenies_1985}. Consider two sister tips on a phylogenetic tree and two accompanying observations for a continuously-valued variable of interest. The two observations themselves cannot be considered statistically independent, since the two sister tips share much of their evolutionary history through a common point of origin. However, the \emph{contrast} between the two values is independent, because any difference between the two sisters will be the consequence of evolutionary events occurring only along the two \emph{separate} branches linking each sister to their last common ancestor. If the historical evolution of this variable follows a Brownian motion model as described above, then the contrast between the two sister tips will be drawn from a normal distribution with a mean of zero and variance proportional to the time that has elapsed since the two tips split in the tree. An observed contrast can be scaled by dividing it by the standard deviation of its expected variance. This gives a statistically independent contrast of expectation zero and unit variance. This process can be repeated for all adjacent tips in the tree. Contrasts can then be extracted from adjacent nodes in the tree, where the value of the node is an average of the observed values of the tips below it. In the end, there will be a collection of phylogenetic independent contrasts, all of expectation zero and unit variance. It is then possible to apply standard statistical tests to the phylogenetic independent contrasts (rather than directly to observed values) without phylogenetic autocorrelation introducing bias into the results.

One drawback of Felsenstein's method is the reliance on the assumption of Brownian motion as a model of variable evolution. \textcite{grafen_phylogenetic_1989} subsequently devises a similar method, \emph{the phylogenetic regression}, which has the flexibility to incorporate models of evolution other than Brownian motion. Further, Grafen's method is able to be applied in situations where phylogenetic information is incomplete (for example, where the phylogeny is an incomplete work-in-progress rather than an accepted gold-standard). This method is a phylogenetic adaptation of \emph{generalised least squares} (GLS). In this model, the value of a dependent variable, \(y_{i}\), is predicted by the equation \(y_{i} = \alpha + \beta x_{i} + \epsilon\), where \(\alpha\) is the intercept, \(\beta\) is the regression slope, \(x\) is the independent variable and \(\epsilon\) is an error term \autocite[p.~164]{nunn_comparative_2011}. Phylogenetic information can be incorporated into the error term, in the form of a variance-covariance matrix of phylogenetic distances between tips in a tree. PICs and GLS are mathematically equivalent when a Brownian motion evolutionary model is assumed and the reference tree is fully bifurcated, so PICs are essentially a special case of GLS where these assumptions are met \autocite{nunn_comparative_2011}.

There has been some interest in applying phylogenetic comparative methods to cross-linguistic data \autocites[for example,][]{dunn_evolved_2011}{maurits_tracing_2014}{verkerk_diachronic_2014}{birchall_comparison_2015}{zhou_quantifying_2015}{calude_typology_2016}{dunn_dative_2017}{verkerk_phylogenetic_2017}{bentz_evolution_2018}. In contrast to the lossy sampling methods discussed in the previous section, phylogenetic comparative methods are \emph{lossless}. They make it possible to incorporate all available data by directly incorporating phylogenetic history into the statistical model, rather than removing data points in order to balance the sample. Another reason that phylogenetic comparative methods are important is that no comparative study in either biology or linguistics is phylogenetically neutral, no matter what balanced sampling procedure might have been used. One may assume that a set of sampled languages are all sufficiently independent from one another and, accordingly, treat data from those languages as independent observations in statistical analysis, but this is still a phylogenetic assumption: one in which all languages are equally distant from one another in a phylogeny (in other words, all languages are connected to a single node by branches of the same length, with no intermediary structure---a \emph{star phylogeny}) \autocite{purvis_polytomies_1993}. Phylogenetic comparative methods enable the direct inclusion of existing phylogenetic knowledge beyond a simple star phylogeny.

One limitation of phylogenetic comparative methods, and a likely source of criticism in linguistics, is that they rely on access to high-quality, fully-resolved phylogenies complete with branch lengths. This simply is not realistic for families of languages across many parts of the world. Further, phylogenetic comparative methods assume reference phylogenies are accurate, when, in practice, phylogenies are subject to uncertainty. While tree inference in modern biology benefits from advances in high quality, large scale genomic data, it would be natural to balk at this assumption in the context of historical linguistics, where so many phylogenetic relationships within and between language families remain uncertain o unknown. It may come as somewhat of a surprise then, that \textcite[p.~14]{felsenstein_phylogenies_1985} expresses precisely the same concern about phylogenetic uncertainty in the context of comparative biology. Nevertheless, he says ``phylogenies are fundamental to comparative biology; there is no doing it without taking them into account'', and this is unavoidably true of comparative linguistics as well. Further, even if a study does not explicitly consider phylogeny, it is still unavoidably making phylogenetic assumptions. Comfortingly, computational advances make it feasible to incorporate phylogenetic uncertainty into analysis explicitly (and this is an area of continuing active development). Also, there is evidence that even when phylogenies are incomplete, lacking branch length information, or even subject to a degree of error, phylogenetic comparative methods still typically out-perform equivalent (non-phylogentic) comparative methods, which effectively assume a star phylogeny \autocites{grafen_phylogenetic_1989}{purvis_truth_1994}{symonds_effects_2002}.

\hypertarget{phylo-sig}{%
\section{Phylogenetic signal}\label{phylo-sig}}

As discussed in Section \ref{phylo-autocorrelation}, phylogenetic comparative methods are applicable in linguistic typology and any kind of comparative linguistic study where phylogeny is a confound. The previous section described phylogenetic comparative methods, which account for phylogeny. Some variables, however, may not evolve through descent with modification and may not pattern phylogenetically, or may do so only weakly. For example, some biological characteristics may reflect adaptations to a certain ecological niche. Ecological niche hypotheses have been proposed for some phonological variables too \autocites{everett_climate_2015}{everett_languages_2017}{blasi_grammars_2017}. How does one determine, then, whether phylogeny is a confounding factor for a variable of interest? In the last 15 years, an advance in this area has been the advent of methods for quantifying explicitly the degree of \emph{phylogenetic signal} in comparative data \autocites{freckleton_phylogenetic_2002}{blomberg_testing_2003}. Phylogenetic signal refers to the tendency of phylogenetically-related entities to resemble one another \autocites{blomberg_tempo_2002}[p.~717]{blomberg_testing_2003}. This resemblance is more technically defined as statistical non-independence among observation values due to phylogenetic relatedness between taxa \autocite[p.~591]{revell_phylogenetic_2008}. This concept of phylogenetic signal has important applications in comparative linguistics. Here I argue that measuring phylogenetic signal should be considered as a first step in a phylogenetically aware comparative methodology, since it can determine empirically whether phylogenetic comparative methods are required or whether regular statistical methods may suffice. Further, the result of a phylogenetic signal test can contribute to evolutionary hypotheses in its own right, for example by giving evidence for or against a variable following certain modes of evolution.

Rather than assuming phylogenetic non-independence \emph{a priori}, as the phylogenetic comparative methods discussed in Section \ref{phylo-auto-bio} do, or lacking any statistical control for phylogeny and relying on balance sampling alone, measures of phylogenetic signal provide the advantage of being able to quantify explicitly the degree of phylogenetic non-independence in a dataset \autocite[p.~591]{revell_phylogenetic_2008}. Phylogenetic signal may be expected to be strong in some cases or weak in others---given some data and a phylogenetic tree for reference, this can be tested empirically. If a typologist were to find a linguistic variable that is distributed significantly independently of phylogeny, this may be a result of interest in itself---at the least, it will provide an empirical basis that justifies proceeding with regular statistical methods over phylogenetic comparative methods \autocite[as in][]{irschick_comparison_1997}. Measures of phylogenetic signal will also be of interest to historical linguistics, as a diagnostic tool for testing the degree and nature in which some data reflect the phylogenetic history of the languages from which they came. The following section describes some methods for measuring phylogenetic signal in different kinds of variables.

\hypertarget{phylo-sig-quant}{%
\subsection{Quantifying phylogenetic signal in continuous variables}\label{phylo-sig-quant}}

\textcite{blomberg_testing_2003} provide a suite of tools for quantifying phylogenetic signal, which have become somewhat of a standard in the field (cited 3150 times on 31 May 2020, according to Google Scholar). Recent comparative studies using these tools include \textcite{balisi_dietary_2018}, \textcite{hutchinson_contemporary_2018} and \textcite{leff_predicting_2018}. \textcite{blomberg_testing_2003} present a descriptive statistic, \(K\), which is generalisable across phylogeneies of different sizes and shapes. In addition, they provide a randomisation test for checking whether the degree of phylogenetic signal for a given dataset is statistically significant. \(K\) can be calculated using either phylogenetic independent contrasts (PICs) \autocite{felsenstein_phylogenies_1985} or generalised least squares (GLS) \autocite{grafen_phylogenetic_1989} (see Section \ref{phylo-auto-bio}). In a Brownian motion model, where variable values can wander up and down with equal probability through time, PIC variances are expected to be proportional to elapsed time. Among more closely related languages, where there has been less divergence time for variable values to wander, the variance of PICs is expected to be low. The randomisation test works by comparing whether observed PICs are lower than the PIC values obtained by randomly permuting the data across the tips of the tree. The process of permuting data across tree tips at random is repeated many times over. If the real variances, with data in their correct positions on the tree, are lower than 95\% of the randomly permuted datasets, then the null hypothesis of no phylogenetic signal can be rejected at the conventional 95\% confidence level. In other words, closely related languages resemble one another to a statistically significantly greater degree than would be expected by chance.

The descriptive statistic, \(K\), quantifies the strength of phylogenetic signal. As with the randomisation procedure above, the input is a set of observed values, where each observation is associated with a tip of the reference tree. \textcite[p.~722]{blomberg_testing_2003} give an explanation of the calculation of the \(K\) statistic. To recap briefly, \(K\) is calculated by, firstly, taking the mean squared error of the data (\(MSE_0\)), as measured from a phylogenetically-corrected mean\footnote{Simply taking the mean of some variable would be misleading in cases where members of a particularly large clade happen to share similar values at an extreme end of the range. A phylogenetic mean is an estimate of the mean which has been corrected for overrepresentation by larger subclades \autocite[see][]{garland_polytomies_1999}.}, and dividing it by the mean squared error of the data (\(MSE\)), calculated using a variance-covariance matrix of phylogenetic distances between tips in the reference tree (the same variance-covariance matrix of phylogenetic distances incorporated into the error term in GLS-based phylogenetic regression, as discussed in the previous section). This latter value, \(MSE\), will be small when the pattern of covariance in the data matches what would be expected given the phylogenetic distances in the reference tree, leading to a high \(MSE_0/MSE\) ratio and vice versa. Thus, a high \(MSE_0/MSE\) ratio indicates higher phylogenetic signal. Finally, the observed ratio can be scaled according to its expectation under the assumption of Brownian motion evolution along the tree. This gives a \(K\) score which can be compared directly between analyses using different tree sizes and shapes. Where \(K = 1\), this suggests a perfect match between the covariance observed in the data and what would be expected given the reference tree and the assumption of Brownian motion evolution. Where \(K < 1\), close relatives in the tree bear less resemblance in the data than would be expected under the Brownian motion assumption. Notably, \(K > 1\) is also possible---this occurs where there is less variance in the data than expected, given the Brownian motion assumption and divergence times suggested by the reference tree. In other words, relatives bear greater resemblance than would be expected.

As discussed, the assumption of a Brownian motion model of evolution, where a variable is free to wander up or down, with equal probability, as time passes, is central to quantification of phylogenetic signal with the \(K\) statistic. \textcite[pp.~726--727]{blomberg_testing_2003} extend their approach to cover two different modes of evolution as well. This is achieved by incorporating extra parameters into the variance-covariance matrix to reflect different evolutionary processes. The first evolutionary model alternative is the Ornstein-Uhlenbeck (OU) model \autocites{felsenstein_phylogenies_1988}{garland_phylogenetic_1993}{hansen_translating_1996}{lavin_morphometrics_2008} whereby variables are still free to wander up or down at random, but there is a central pulling force towards some optimum value. The second alternative is an acceleration-deceleration (ACDC) model, developed by \textcite{blomberg_testing_2003} where a variable value moves up or down with equal probability (like Brownian motion) but the rate of evolution will either accelerate or decelerate over time.

Other statistics for quantifying phylogenetic signal have been proposed and warrant mention. \textcite{freckleton_phylogenetic_2002} propose using the \(\lambda\) (lambda) statistic, based on earlier work by \textcite{pagel_inferring_1999}. As for \textcite{blomberg_testing_2003}, this approach works with a variance-covariance matrix showing the amount of shared evolutionary history between any two tips in the tree (the diagonal of the matrix, the variances, will indicate the total height of the tree; the off-diagonals, the covariances, will indicate the amount of shared evolutionary history between two given entities, before they diverge in the tree). The statistic, \(\lambda\) is a scaling parameter which can be applied to this variance-covariance matrix. Scaling the values in the matrix by \(\lambda\) transforms the branch lengths of the tree, from \(\lambda = 1\), where branch lengths are left unscaled, to \(\lambda = 0\), where all covariances in the matrix will be zero, in other words, no covariance through shared evolutionary history is indicated between any tips, thus all tips will be joined at the root by branches of equal length (a star phylogeny). \textcite{freckleton_phylogenetic_2002} present a method for finding the \(\lambda\) parameter that maximises the likelihood of a set of observations arising, given a Brownian motion model of evolution. If \(\lambda\) is close to 1, this indicates high phylogenetic signal, where the data closely fit expectation given the shared evolutionary histories in the tree and a Brownian motion model of evolution. Further measures which have been proposed are \(I\) \autocite{moran_notes_1950}, a spatial autocorrelation measure which was adapted for phylogenetic analyses by \textcite{gittleman_adaptation:_1990}, and \(C_{mean}\) {[}abouheif\_method\_1999{]}, which is a test for serial independence \autocite[for an overview, see][]{munkemuller_how_2012}. In an evaluation of different methods \textcite{munkemuller_how_2012} find that, assuming a Brownian motion model of evolution, \(C_{mean}\) and \(\lambda\) generally outperform \(K\) and \(I\). However, \(C_{mean}\) considers only the topology of the reference tree (i.e., the order of the branches from top to bottom), but not branch length information, and the value of the \(C_{mean}\) statistic is partially dependent on tree size and shape, so it lacks comparability between different studies. In addition, \(\lambda\) shows some unreliability with small sample sizes (trees with \textasciitilde{}20 tips).

\hypertarget{phylo-sig-bin}{%
\subsection{Quantifying phylogenetic signal in binary variables}\label{phylo-sig-bin}}

The methods so far described concern continuously-valued data. Other methods have been proposed for quantifying phylogenetic signal in binary and categorical variables too. \textcite{abouheif_method_1999} presents a simulation-based approach for testing whether discrete values along the tips of a phylogeny are distributed in a phylogenetically non-random way. Although this method is useful for testing whether the phylogenetic signal in a set of discretely-valued data is statistically significant, it does not provide a quantification of the level of phylogenetic signal which is comparable between different datasets. Although specific to binary data only, \textcite{fritz_selectivity_2010} present a statistic, \(D\), which quantifies the strength of phylogenetic signal for some binary variable.

The \(D\) statistic is based on the sum of differences between sister tips and sister clades, \(\Sigma d\). To recap, following \textcite{fritz_selectivity_2010}, differences between values at the tips of the tree are summed first (all tips will either share the same value, 0 or 1, with 0 difference; or one will be 0 and the other will be 1, for a difference of 0.5). Nodes immediately above the tips are valued as an average of the two tips below (either 0, 0.5 or 1) and the differences between sister nodes is summed. This process is repeated for all nodes in the tree, until a total sum of differences, \(\Sigma d\), is reached. At two extremes, data may be maximally clumped, such that all 1s are grouped together in the same clade in the tree and likewise for all 0s, or data may be maximally dispersed, such that no two sister tips share the same value (every pair of sisters contains a 1 and a 0, leading to a maximal sum of differences). Lying somewhere in between will be both a phylogenetically random distribution and a distribution that is clumped to a degree expected under a Brownian motion model of evolution. A distribution of sums of differences following a phylogenetically random pattern, \(\Sigma d_r\), is obtained by shuffling variable values among tree tips many times over. A distribution of sums of differences following a Brownian motion pattern, \(\Sigma d_b\) is obtained by simulating the evolution of a continuous trait along the tree, following a Brownian motion process, many times over. Resulting values at the tips above a threshold are converted to 1, values below the threshold are converted to 0. The threshold is set to whatever level is required to obtain the same proportion of 1s and 0s as observed in the real data. Finally, \(D\) is determined by scaling the observed sum of differences to the means of the two reference distributions (the expected sums of differences under a phylogenetically random pattern and under a Brownian motion pattern).

\begin{equation}
D = \frac{\Sigma d_{obs} - mean\left( \Sigma d_{b} \right)}{mean\left( \Sigma d_{r} \right) - mean\left( \Sigma d_{b} \right)}
\end{equation}

Scaling \(D\) in this way provides a standardised statistic which can be compared between different sets of data, with trees of different sizes and shapes, as with \(K\) for continuous variables. One disadvantage of \(D\), however, is that it requires quite large sample sizes (\textgreater{}50), below which it loses statistical power, increasing the chance of a false positive result (type I error).

\hypertarget{phylo-sig-mult}{%
\subsection{Quantifying phylogenetic signal in multivariate and multidimensional data}\label{phylo-sig-mult}}

A notable, more recent development concerns the generalisation of methods for quantifying phylogenetic signal in multivariate and multi-dimensional data. Methods discussed so far quantify phylogenetic signal for a single variable of interest. \textcite{zheng_new_2009} present a generalisation of the \textcite{blomberg_testing_2003} \(K\) statistic for jointly estimating the strength of phylogenetic signal in a collection of variables. In addition, their method allows the incorporation of measurement error or variation within the entities being studied (be they species, languages, etcetera) \autocite[see][]{ives_within-species_2007}. Both of these developments are expected to improve the statistical power of the test, which is an advantage particularly where small sample sizes are concerned, since the original \(K\) statistic requires a minimum sample size of around 20 and lacks sufficient statistical power below this level. This is achieved by standardising the values of each variable to have mean 0 and variance 1 then jointly measuring the MSE for all variables. \textcite{adams_generalized_2014} presents another generalisation of \(K\) for use with `multivariate traits', \(K_{mult}\). Whereas \textcite{zheng_new_2009} estimate phylogenetic signal for a set of multiple \emph{independent} variables simultaneously, a multivariate trait is conceptually a single evolutionary trait but has multiple values associated with it, which are mathematically interrelated and cannot be analyzed independently. The example \textcite{adams_generalized_2014} uses is head shape for a family of salamander species.

\hypertarget{pcms-applications}{%
\section{Phylogenetic signal and PCMs in linguistics}\label{pcms-applications}}

\textcite{macklin-cordes_phylogenetic_2021} presents an application of phylogenetic signal measuring methods in linguistics, specifically a dataset of binary and continuous-valued phonotactic variables from 111 Pama-Nyungan languages using the \(D\) and \(K\) tests described above. Contrary to expectation, given prior descriptions of Pama-Nyungan phonology and phonotactics, strong phylogenetic signal is detected in this phonotactic data. This work has been tentatively expanded by \textcite{macklin-cordes_phylogeny_2018}. Since phonological processes and sound change affect natural classes of sounds rather than individual phonemes, phonotactic variables are subject to complex patterns of non-independence between them. \textcite{macklin-cordes_phylogeny_2018} attempts to account for this by implementing the \(K_{mult}\) method described above and shows that, at first pass, \(K_{mult}\) seems to detect stronger phylogenetic signal than testing several hundred phonotactic variables individually and averaging the results. This work requires further exploration, however. The evaluation of phylogenetic signal is, in itself, the end goal of these studies. They focus on the question of whether or not historical information is present in a novel dataset, which is more straightforwardly of interest for historical linguistic inquiry. However, the results hold secondary implications for the typological matter of non-independence between languages. The detection of phylogenetic signal in the phonotactics of Pama-Nyungan languages would need to be incorporated into any future typological methodology. The need for phylogenetic comparative methods has been established in this particular domain.

In another phylogenetic signal detection example using the same Pama-Nyungan reference phylogeny, \textcite{round_continent-wide_2017} re-examines assumptions about the laminal contrast in Australian languages. The distribution of Australian languages with a single contrastive laminal versus two contrastive laminal places at first seems not to correspond strongly with family and subgroup boundaries, leading researchers to propose that the distribution of one versus two laminal languages has been driven historically by areal diffusion \autocites{dixon_languages_1970}{dixon_languages_1980}{breen_taps_1997}{dixon_australian_2002}. \textcite{round_continent-wide_2017}, however, extracts a finer grained level of variation by considering not just the single, binary presence or absence of a laminal contrast but the frequencies at which those laminals appear in certain contexts before and after different vowels. \textcite{round_continent-wide_2017} detects strong phylogenetic signal in these frequency variables, suggesting that, contrary to previous proposals, laminals do not appear to evolve in a distinctly non-phylogenetic way.

Without delving into detail, other recent examples of phylogenetic comparative methodologies include \textcite{bromham_rate_2015}, which examines the relationship between rates of lexical change to population size in Austronesian languages; \textcite{verkerk_where_2015}, investigating the development of manner verbs and path verbs in Indo-European languages; \textcite{calude_typology_2016}, reconstructing numerals systems in Indo-European languages; \textcite{bentz_evolution_2018}, investigating links between language evolution and ecological factors in a sample of nearly 7,000 languages across the world; and \textcite{moran_investigating_2020}, quantifying differential rates of change in consonants versus vowels in 8 language families across 6 continents.

there are a few limitations of PCMs to note. One limiting assumption is that the tree being used as a reference phylogeny is an accurate representation of the true historical phylogeny. Since the past cannot be observed directly, the best available yardstick is limited to the best available phylogeny that has been inferred independently. In linguistics, this can be a particular problem. The cross-linguistic coverage of linguistic phylogenetic studies has expanded rapidly in the past two decades and some families, such as Indo-European and Austronesian, have been studied extensively. However, high-resolution phylogenetic trees do not exist for large swathes of the globe's linguistic diversity. In the context of Sahul, there is good coverage of the Pama-Nyungan family \autocites{bowern_computational_2012}{bouckaert_origin_2018} but, as yet, few if any studies computationally inferring phylogenies of non-Pama-Nyungan families nor any Papuan language families in New Guinea to the north. There are several large databases of world language classifications and efforts to make them easily available for phylogenetic comparative study \autocite{dediu_making_2018}, but these can lack resolution and branch length information. One partial solution is to incorporate phylogenetic uncertainty into any PCM analysis explicitly by replicating the analysis over a posterior sample of trees rather than a single summary tree, replicating the analysis over separate, competing reference trees, or, in the case of a limited tree structure with lots of polytomies, randomly simulating bifurcating tree structures to simulate uncertainty.

A second key assumption that can prove problematic is that the data being tested are assumed to be completely independent of the data that was used to infer the reference phylogeny. This can be tricky in comparative linguistic data, which tend to contain complex interdependencies. As one example, the phonotactic frequency datasets in \textcite{macklin-cordes_phylogenetic_2021} were extracted from wordlists which contained, as a subset, the words that were used to code cognate data, from which the reference tree was inferred.

One final point to note is that recent research suggests that phylogenetic signal can be inflated when variable values evolve according to a Lévy process, where a variable value can wander as per a Brownian motion process, but with the addition of discontinuous paths (i.e., sudden jumps in the variable's value) \autocite{uyeda_rethinking_2018}. This is particularly concerning for any frequency-based phonological data, which will be subject to sudden shifts caused by phonemic mergers, splits, and other regular sound changes. This is likewise a matter of concern in comparative biology and subject to active development in that field \autocite{uyeda_rethinking_2018}.

\hypertarget{pcms-conclusion}{%
\section{Conclusion}\label{pcms-conclusion}}

Historical and synchronic comparative linguistics are increasingly making use of phylogenetic methods for the same reasons that led biologist to switch to them several decades ago. The central contention of this research essay has been that phylogenetic methods not only give us new ways of studying existing comparative data sets, but open up the possibility to derive insights from new kinds of data.

% ***************************************************


%CHAPTER 5
%If you are presenting work which has been previously published, acknowledge this here.
% ***************************************************
% How to introduce a previously published chapter
% ***************************************************
%This is an example of how you might introduce a chapter that has been published previously. 
\cleartoevenpage
\pagestyle{empty}	
%Use this command (above) to suppress the header from the preceding chapter.

\noindent
The following publication has been incorporated as Chapter~\ref{Chap:phylo-signal}:\\

\noindent
\fullcite{macklin-cordes_phylogenetic_2020}

\begin{table}[h]
	\centering
	\begin{tabular}{clr}
		\toprule
		Contributor & Statement of contribution & \% \\
		\midrule
		\textbf{Jayden Macklin-Cordes}	& initial concept			& 50 \\
		                                & scripting, analysis       & 80 \\
		                                & writing of text 			& 90 \\
										& proof-reading				& 20 \\
										& preparation of figures 	& 75 \\
		\midrule
		Claire Bowern                   & supplying data            & 10 \\
										& supervision, guidance 	& 5  \\
										& proof-reading             & 10 \\
		\midrule
		Erich Round 					& initial concept			& 50 \\
		                                & supplying data            & 90 \\
		                                & scripting, analysis       & 20 \\
		                                & supervision, guidance 	& 95 \\
		                                & writing of text 			& 10 \\
										& proof-reading				& 70 \\
										& preparation of figures 	& 25 \\
										
		\bottomrule
	\end{tabular}
\end{table}

\noindent
Data for this paper comes from the Ausphon Lexicon database, created and maintained by Erich Round and extending on the CHIRILA database created and maintained by Claire Bowern. The Pama-Nyungan reference phylogeny was inferred and supplied by Claire Bowern.\\

\noindent
Responsibility for the study's initial conceptualisation and development is shared approximately equally between Macklin-Cordes and Round, representing the culmination of innumerable discussions over a long period of time.\\

\noindent
Experimental design, scripting and analysis was conducted by Macklin-Cordes (though note that the scripts depend considerably on functions developed by Round). Round contributed text to Section \ref{wordlists}. Macklin-Cordes drafted the remainder of the text. The 'swatch' visual device (e.g. Figure \ref{fig:k-swatch}) was created by Round and minimally adapted by Macklin-Cordes in this paper. Remaining figures were created by Macklin-Cordes with guidance from Round. Macklin-Cordes and Round revised text with feedback from three anonymous reviewers. Bowern assisted with proofreading.

% ***************************************************
% Example of an internal chapter
% ***************************************************
%This is an internal chapter of the thesis.
%If you have a long title, you can supply an abbreviated version to print in the Table of Contents using the optional argument to the \chapter command.
\chapter[Phylogenetic signal in phonotactics]{Phylogenetic signal in phonotactics}
\label{Chap:phylo-signal}	%CREATE YOUR OWN LABEL.
\pagestyle{headings}

% ********* Enter your text below this line: ********

Phylogenetic methods have broad potential in linguistics beyond tree inference. Here, we show how a phylogenetic approach opens the possibility of gaining historical insights from entirely new kinds of linguistic data---in this instance, statistical phonotactics. We extract phonotactic data from 111 Pama-Nyungan vocabularies and apply tests for \emph{phylogenetic signal}, quantifying the degree to which the data reflect phylogenetic history. We test three datasets: (1) binary variables recording the presence or absence of \emph{biphones} (two-segment sequences) in a lexicon (2) frequencies of transitions between segments, and (3) frequencies of transitions between natural sound classes. Australian languages have been characterised as having a high degree of phonotactic homogeneity. Nevertheless, we detect phylogenetic signal in all datasets. Phylogenetic signal is higher in finer-grained frequency data than in binary data, and highest in natural-class-based data. These results demonstrate the viability of employing a new source of readily extractable data in historical and comparative linguistics.

\hypertarget{phy-sig-intro}{%
\section{Introduction}\label{phy-sig-intro}}

A defining methodological development in 21st century historical linguistics has been the adoption of computational phylogenetic methods for inferring phylogenetic trees of languages \autocite{bowern_computational_2018}. The computational implementation of these methods means that it is possible to analyse large samples of languages, thereby inferring the phylogeny (evolutionary tree) of large language families at a scale and level of internal detail that would be difficult, if not impossible, to ascertain manually by a human researcher \autocite[p.~827]{bowern_computational_2012}. There is more to phylogenetics than building trees, and there exists untapped potential to explore the language sciences and human history with a phylogenetic approach. For example, in linguistics, phylogenetic methods have been integrated with geography to infer population movements \autocites{walker_bayesian_2011}{bouckaert_origin_2018}. In comparative biology, phylogenetic methods have been applied profitably to investigations of community ecology \autocite{webb_phylogenies_2002}, ecological niche conservatism \autocite{losos_phylogenetic_2008}, paeleobiology \autocite{sallan_heads_2012} and quantitative genetics \autocite{de_villemereuil_general_2014}. At the heart of these methods, however, is a sound understanding of the evolutionary dynamics of comparative structures. In this paper, we pesent a foundational step by detecting \emph{phylogenetic signal}, the tendency of related species (in our case, language varieties) to share greater-than-chance resemblances \autocite{blomberg_tempo_2002}, in quantitative phonotactic variation.

Throughout recent advances in linguistic phylogenetics, less attention has been paid to methodological development at the stage of data preparation. Large-scale linguistic phylogenetic studies continue, by-and-large, to rely on lexical data which have been manually coded according to the principles of the \emph{comparative method} \autocites[as described by][]{meillet_methode_1925}{campbell_historical_2004}{weiss_comparative_2014}---the comparative method being the long-standing gold-standard of historical linguistic methodology \autocites{chang_ancestry-constrained_2015}{bouckaert_origin_2018}{kolipakam_bayesian_2018}. This article demonstrates that phonotactics can also present a source of historical information. We find that, for a sample of 111 Pama-Nyungan language varieties, collections of relatively simple and semi-automatically-extracted phonotactic variables (termed \emph{characters} throughout) contain phylogenetic signal. This has positive implications for the utility of such phonotactic data in linguistic phylogenetic inquiry, but also introduces methodological considerations for phonological typology.

We discuss firstly the motivations for looking at phonotactics as a source of historical signal, and we give some broader scientific context that motivates the methodological approach we take later on. In Sections \ref{phy-sig-bin}--\ref{phy-sig-classes}, we present tests for phylogenetic signal in phonotactic characters extracted from wordlists for 111 Pama-Nyungan language varieties. Section \ref{materials} details the materials used and reference phylogeny. Section \ref{phy-sig-bin} tests for phylogenetic signal in binary characters that code the presence or absence of biphones (two-segment sequences) in each wordlist, capturing information on the permissibility of certain sequences in a language. Section \ref{phy-sig-cont} also tests for phylogenetic signal in biphones, but extracts a finer-grained level of variation by taking into account the relative frequencies of transitions between segments. Section \ref{phy-sig-classes} groups segments into natural sound classes and tests for phylogenetic signal in characters coding the relative frequencies of transitions between different classes. We conclude with discussion of the limitations of the study design, implications of the results and directions for future research.

\hypertarget{motiv}{%
\subsection{Motivations}\label{motiv}}

There are at least two reasons why consideration of alternative data sources could be fruitful in historical linguistics. The first is that a bottleneck persists in linguistic phylogenetics when it comes to data processing. The data for most linguistic phylogenetic studies are lexical cognate data---typically binary characters marking the presence or absence of a cognate word in the lexicon of each language---which have been assembled from the manual judgements of expert linguists using the traditional comparative method of historical linguistics \autocite[e.g.][]{weiss_comparative_2014}. Although data assembled in this way is likely to remain the gold-standard in historical linguistics for the foreseeable future, it nevertheless constitutes slow and painstaking work \autocites[notwithstanding efforts to automate parts of the process; see][]{list_potential_2017}{rama_are_2018}{list_sequence_2018}. This restricts the pool of languages that can be included in phylogenetic research to those that have been more thoroughly documented, introducing the risk of a sampling bias, where relatively well-studied regions of the global linguistic landscape are over-represented in historical and comparative work.

The second motivation for considering alternative historical data sources is that there are inherent limitations associated with lexical data. Undetected semantic shifts and borrowed lexical items erode patterns of vertical inheritance in a language's lexicon. Put another way, these changes create noise in the historical signal of a language's lexicon. Chance resemblances between non-historically cognate words are another source of noise in lexical data. Eventually, semantic shifts, borrowings and chance resemblances will accumulate to a point where genuine historical signal is indistinguishable from noise. This imposes a maximal cap on the time-depth to which the comparative method can be applied, which is typically assumed to sit somewhere around 10,000 years BP, based on the approximate age of the Afro-Asiatic family \autocite[p.~135]{nichols_sprung_1997}. Some phylogenetic studies have attempted to push back the time-depth limitations of lexical data by using characters that code for a range of grammatical features, under the rationale that a language's grammatical structures should be more historically stable than its lexicon \autocites{dunn_structural_2005}{rexova_cladistic_2006}. However, contrary to expectation, a recent study suggests that grammatical characters evolve faster than lexical data \autocite{greenhill_evolutionary_2017}. Differing rates of evolution are also found in phonology, specifically the rates of change in vowel inventories versus consonant inventories \autocites{moran_differential_2018}{moran_investigating_2020}. An additional issue with grammatical characters is that the space of possibilities for a grammatical variable is often restricted. This means that chance similarities due to \emph{homoplasy} (parallel historical changes) will be much more frequent \autocite[c.f.][]{chang_ancestry-constrained_2015}. For example, many unrelated languages will share the same basic word order by chance, because there is a logical limit on the number of basic word order categories.

\hypertarget{why-phonotactics}{%
\subsection{Phonotactics as a source of historical signal}\label{why-phonotactics}}

The motivation for considering a language's phonotactics as a potential source of historical information is based partly on practical and partly on theoretical observations. From a practical perspective, it is possible to extract phonotactic data with relative ease, at scale, from otherwise resource-poor languages. This is because the bulk of a language's phonotactic system can be extracted directly from phonemicized wordlists. As long as there is a wordlist of suitable length \autocite{dockum_swadesh_2019} and a phonological analysis of the language, phonotactic information can be deduced and coded from the sequences of segments found in the wordlist with a high degree of automation. This modest minimum requirement with regards to language resources is a valuable property in less documented linguistic regions of the world. We detail the process of data extraction for this study in Section \ref{materials} below.

An additional benefit of extracting phonotactic data from wordlists is the potential for expanding the depth of comparative datasets. Although macro-scale studies, including hundreds or even thousands of the world's languages (in other words, \emph{broader} datasets), are increasingly common in comparative linguistics, less attention has been paid to the number of characters per language (dataset \emph{depth}). It is quite a different situation in evolutionary biology, where there has been tremendous growth in whole genome sequencing, thanks to technological advances and falling costs \autocites{delsuc_phylogenomics_2005}{wortley_how_2005}. This, consequently, has led to tremendous growth in the depth of biological datasets. This is an important consideration because the quantity of characters required by modern computational phylogenetic methods can be substantial \autocites{wortley_how_2005}{marin_undersampling_2018}. Certainly, phonotactic data is unlikely to approach the scale of large genomic datasets in biology, but it could effectively deepen historical linguistic datasets.

From a theoretical perspective, there is reason to suspect that the phonotactics of a language preserve a degree of historical signal. There is some evidence that when a borrowed word enters the lexicon of a language, speakers tend to adapt it to suit the phonotactic patterns of that language \autocites{hyman_role_1970}{silverman_multiple_1992}{crawford_adaptation_2009}{kang_loanword_2011}. Consequently, in such a case the historical phonotactic structure of the lexicon remains largely intact, even as particular ancestral words are lost and replaced \autocite[a property termed \emph{pertinacity} by][]{dresher_main_2005}. Similarly, historical phonotactic properties of a language will remain in the phonotactics of a language in the case of an undetected semantic shift.

Laboratory evidence shows that speakers have a high degree of sensitivity to the statistical distribution of phonological segments and structures when producing novel words {[}for example, \textcite{coleman_stochastic_1997}; \textcite{albright_rules_2003}; \textcite{hayes_stochastic_2006}, among others \autocite[see][pp.~20--21]{gordon_phonological_2016}. Lexical innovation then, should have a relatively conservative impact on the frequency distributions of phonotactic characters. Every new word that enters a language's lexicon will have a minute impact on the frequencies of segments and particular sequences of segments in that language. But, over time, the cumulative effect of new lexicon entering a language on phonological and phonotactic frequency distributions will be more modest than if speakers generated new words with no regard for existing frequencies. Thus, there is reason to expect that quantitative phonotactic characters are likely to be conservative.

This is not to say that a language's phonotactic system remains completely immobile over time. Phonotactic systems are affected by sound changes and are not totally immune to borrowing. As mentioned above, frequencies of phonotactic characters will shift, however gradually, with the accumulation of lexical innovations. We make no strong claim about phonotactics being the key to a language's history. We merely note there are grounds to expect that phonotactic data will often be historically conservative, relative to cognate data which contains noise from lexical innovation, borrowing and semantic shift. Correspondingly, our hypothesis is that phonotactic data will contain relatively strong historical signal, which we test in Sections \ref{phy-sig-bin}--\ref{phy-sig-classes} below.

Many kinds of phonotactic structure exist, which could be studied phylogenetically. Here, because we wish to adhere to the basic methodological principle of studying maximally simple and clear cases first before progressing to more complex ones, we limit ourselves to the simplest of phonotactic structures, namely biphones. That being said, there is every reason to expect our results would generalise, perhaps with interesting variations, to other phonotactic structures. Moreover, many of those structures would have the same benefits as our biphones, in terms of their being readily generated in an automated fashion from wordlists. This will be a promising direction for future investigation.

\hypertarget{phylo-sig}{%
\section{Phylogenetic signal}\label{phylo-sig}}

The concept of \emph{phylogenetic signal} \autocites{blomberg_tempo_2002}[p.~717]{blomberg_testing_2003} originates in comparative biology, where it refers to the tendency of phylogenetically related species to resemble one another to a greater degree than would otherwise be expected by chance. This expectation derives from the evolutionary history shared between species. Two closely-related species, which share a relatively recent common ancestor, have had less time in which to diverge evolutionarily. We expect more distantly-related species, whose most recent common ancestor lies much further in the past, to tend to be more different, since they have spent longer on separate evolutionary paths.

Phylogenetic signal manifests itself as \emph{phylogenetic autocorrelation} in comparative studies. That is, species observations in a comparative dataset tend not to behave as independent data points, but rather pattern as a function of the amount of shared evolutionary history between species. For many statistical methods that assume data are independent and identically distributed (i.i.d.), this is a problem. Phylogenetic autocorrelation has long been recognised as an issue in linguistic typology and comparative biology, and both fields share comparable histories of developing sampling methodologies that attempt to correct for or offset phylogenetic relatedness in some way. More recent times have seen the rise of \emph{phylogenetic comparative methods}, statistical methods that directly account for phylogenetic autocorrelation, rather than offsetting it, beginning with foundational works by \textcite{felsenstein_phylogenies_1985} and \textcite{grafen_phylogenetic_1989}\footnote{See \textcite{nunn_comparative_2011} for discussion.}. Although now practically ubiquitous in comparative biology, uptake of phylogenetic comparative methods has been slower in comparative linguistics \autocites[notwithstanding studies such as][]{dunn_evolved_2011}{maurits_tracing_2014}{verkerk_diachronic_2014}{birchall_comparison_2015}{zhou_quantifying_2015}{calude_typology_2016}{dunn_dative_2017}{verkerk_phylogenetic_2017}{widmer_np_2017}{blasi_human_2019}.

Since the turn of the century, methods have been developed for explicitly quantifying the degree of phylogenetic signal in a dataset \autocite[p.~591]{revell_phylogenetic_2008}. Measuring phylogenetic signal can be the first step of a comparative study, to test whether there is sufficient phylogenetic signal to necessitate implementation of a phylogenetic comparative method in a later stage of analysis, or to establish the suitability of standard statistical methods if no phylogenetic signal is detected. Measures of phylogenetic signal can also be used to re-evaluate the validity of older results that pre-date modern phylogenetic comparative methods, as in \textcite{freckleton_phylogenetic_2002}. In other instances, the presence or absence of phylogenetic signal in certain data may be an interesting result in itself. In this study, we present a novel source of linguistic data which traditionally has not been considered a salient source of historical signal for historical linguistic study (indeed, given descriptions of Australian languages, it may have been considered a particularly unlikely source of historical signal; see Section \ref{sample}). We use measures of phylogenetic signal to test the hypothesis that our data contain historical information and, therefore, could contribute to future historical linguistic study.

\textcite{blomberg_testing_2003} provide a set of statistics for measuring phylogenetic signal, which remains prevalent today \autocites[for example,][]{balisi_dietary_2018}{hutchinson_contemporary_2018}{leff_predicting_2018}. We use one of these statistics, \(K\). The \(K\) statistic has the desirable property of being independent of the size and shape of the phylogenetic tree being investigated, which means that studies with different sample sizes can be compared directly. Briefly \autocite[following][p.~722]{blomberg_testing_2003}, the calculation of \(K\) requires three components: (i) character data (i.e., observations for the variable of interest); (ii) a \emph{reference phylogeny}, a phylogenetic tree which has been generated independently from the character data; and (iii) a \emph{Brownian motion} model of evolution.\footnote{A \emph{Brownian motion} model of evolution describes a model of character evolution where the character can move up or down with equal probability as it evolves through time. Under this model of evolution, variance in character values throughout a phylogeny will increase proportionally as time elapses.} These components entail two assumptions of the method: the assumption that the reference phylogeny is an accurate representation of the phylogenetic history of the populations being studied and the assumption that Brownian motion accurately models the evolution of the character data. In practice, the reference phylogeny will be subject to uncertainty. We return to this point in Section \ref{discussion} and evaluate the robustness of our results against phylogenetic uncertainty. Similarly, in practice, the Brownian motion model may not be realistic. Nevertheless, it is a simple model and straightforward to implement, and thus commonly used as a starting point before exploring more complex models of evolution later on. We discuss this further in Section \ref{discussion} and outline possible extensions to the model for future study, taking sound change processes into account. To the extent that Brownian motion fails to model the evolution of phonotactic characters, this should make it more difficult to detect phylogenetic signal.

The \(K\) statistic is then calculated by, firstly, taking the mean squared error of the data (\(MSE_0\)), as measured from a \emph{phylogenetic mean}\footnote{Simply taking the mean of some variable would be misleading in cases where members of a particularly large clade happen to share similar values at an extreme end of the range. A \emph{phylogenetic mean} is an estimate of the mean which takes into account any overrepresentation by larger subclades \autocite[see, for example,][]{garland_jr._polytomies_1999}.}, and dividing it by the mean squared error of the data (\(MSE\)), calculated using a variance-covariance matrix of phylogenetic distances between tips in the reference tree \autocite[see][ for a complete formula]{blomberg_testing_2003}. This latter value, \(MSE\), will be small when the pattern of covariance in the data matches what would be expected given the phylogenetic distances in the reference tree, leading to a high \(MSE_0/MSE\) ratio and vice versa. Thus, a high \(MSE_0/MSE\) ratio indicates higher phylogenetic signal. Finally, the observed \(MSE_0/MSE\) ratio can be scaled according to the expected \(MSE_0/MSE\) ratio given a Brownian motion model of evolution. This gives a statistic, \(K\), which can be compared directly between studies using different trees. When \(K = 1\), this suggests a perfect match between the covariance observed in the data and what would be expected given the reference tree and the assumption of Brownian motion evolution. When \(K < 1\), close relatives in the tree bear less resemblance in the data than would be expected under the Brownian motion assumption. \(K > 1\) is also possible---this occurs where there is less variance in the data than expected, given the Brownian motion assumption and divergence times suggested by the reference tree. In other words, relatives bear closer resemblance than would be expected if the variable evolved along the tree following a Brownian motion model of evolution.

\textcite{blomberg_testing_2003} also present a \emph{randomisation procedure} for testing whether the degree of phylogenetic signal in a dataset is statistically significant. The randomisation procedure utilises Felsenstein's (1985) \emph{phylogenetic independent contrasts} (PICs) method. Felsenstein's insight is that, although two character values (\(x\) and \(y\)) from two sister taxa cannot be considered independent due to phylogenetic autocorrelation, the contrast between them (\(x - y\)) is phylogenetically independent, since these values can only diverge in the time since the two sisters split from their most recent common ancestor. Given a set of character data and a phylogenetic tree, \textcite{felsenstein_phylogenies_1985} presents a method for harvesting a whole set of phylogenetically independent data points, PICs, which can be used for statistical analysis in lieu of the raw set of observations. \textcite{blomberg_testing_2003} take advantage of the expectation that, given a Brownian motion model of evolution, PIC variance is expected to be proportional to time. PICs among more closely-related taxa will tend to be lower than more distant relatives, since they have had less time to diverge from common ancestors. The randomisation procedure first extracts PICs for a given character and records the variance. Then, it extracts PICs and records the variance after randomly shuffling character data among taxa (thereby destroying phylogenetic signal). PIC variance is recorded typically for many thousands of such random permutations. If the true PIC variance (for original, unshuffled data) is lower than the variance of PICs in \textgreater{} 95\% of random permutations, the null hypothesis of no phylogenetic signal can be rejected at the conventional 95\% confidence level.

In this study, we also use a second statistic, \(D\), which was developed to measure phylogenetic signal in binary data. The \(D\) statistic is described by \textcite{fritz_selectivity_2010}. To summarise briefly, the \(D\) statistic is based on the sum of differences between sister tips and sister clades, \(\Sigma d\). First, differences between values at the tips of the tree are summed. Since \(D\) concerns binary variables, each taxon will either have a 0 or 1 value. At the level of the tips, then, all sister tips will either share the same value (in which case, the difference = 0) or one tip will have a 0 value and the other will have a 1 value (in which case, the difference = 1). Nodes immediately above the tree tips are given the average value of their daughter tips below (which, in a fully bifurcating phylogeny, will either be 0, 0.5 or 1). This process is repeated for all nodes in the tree, until a total sum of differences, \(\Sigma d\), is reached. At two extremes, data may be maximally clumped, such that all 1s are grouped together in the same clade in the tree and likewise for all 0s, or data may be maximally dispersed, such that no two sister tips share the same value (every pair of sisters contains a 1 and a 0, leading to a maximal sum of differences). Lying somewhere in between will be both (i) a distribution that is entirely random relative to phylogenetic structure and (ii) a distribution that is clumped exactly to the degree expected if the character evolved along the tree following a Brownian motion model of evolution. Two permutation procedures are used to determine where these two points lie for a given dataset and phylogenetic tree. Firstly, like Blomberg \emph{et al.}'s permutation test described above, character values are shuffled at random among tips of the tree many times over, thereby destroying phylogenetic signal. The sums of differences are taken from each random permutation to obtain a distribution of sums of differences, given phylogenetic randomness: \(\Sigma d_r\). Then, to obtain a contrasting distribution of sums of differences, the process of character evolution along the tree following a Brownian motion model is simulated many times over. Since Brownian motion is a model of evolution of continuous characters, and what we need here is a distribution of binary character values, and then binarises their values to 0 or 1 at the tips of the tree by observing whether they fall above or below a threshold value. This threshold is set to whatever level will produce the same proportion of 1s and 0s as observed in the real data. The sums of differences are then taken from each simulation, giving a distribution where phylogenetic signal is present: \(\Sigma d_b\). Finally, the \(D\) statistic is determined by scaling the observed sum of differences relative to the means of the two reference distributions just described:

\begin{equation}
D = \frac{\Sigma d_{obs} - mean\left( \Sigma d_{b} \right)}{mean\left( \Sigma d_{r} \right) - mean\left( \Sigma d_{b} \right)}
\end{equation}

Scaling \(D\) in this way provides a standardised statistic with the desirable property that it can be compared between different sets of data, with trees of different sizes and shapes, as with \(K\) for continuous characters. One disadvantage of \(D\), however, is that it requires quite large sample sizes (\textgreater{}50), below which it loses statistical power.

Two \(p\) values determine the statistical significance of \(D\), one each for the null hypotheses that \(D = 0\) (phylogenetic signal present) and \(D = 1\) (the character is distributed randomly relative to phylogenetic structure). These \(p\) values are obtained by comparing the observed \(D\) score to the two distributions of simulated \(D\) scores described above (\(\Sigma d_r\) and \(\Sigma d_b\)). The fraction of randomly simulated \(D\) scores smaller than observed \(D\) is taken as the \(p\) value for \(H_{0(D=1)}\). Likewise, the proportion of the simulated \(D\) scores greater than the observed \(D\) value is the \(p\) value for \(H_{0(D=0)}\).

\hypertarget{materials}{%
\section{Materials}\label{materials}}

Our study measures phylogenetic signal in a variety of types of phonotactic characters, extracted using semi-automated methods from wordlists within the Pama-Nyungan family (Australia). Throughout, we take the \emph{doculect} to be our unit of study. A doculect is a language variety as documented in a given resource \autocites{cysouw_towards_2007}{good_languoid_2013}. That is to say, we treat each wordlist as its own unit of study, without making any claims about the status of the documented language variety's status as a language or dialect. This agnosticism is advantageous in phylogenetic studies, since the terms `language' and `dialect' imply something about the relationship of a documented language variety to other documented language varieties, and a commitment to one term or the other therefore represents a phylogenetic assumption.

\hypertarget{sample}{%
\subsection{Language sample}\label{sample}}

Pama-Nyungan is by far the largest language family on the Australian continent, covering nearly 90\% of its landmass (everywhere except for three areas: part of the Top End, part of the Kimberley, and the whole of Tasmania) and encompassing around two-thirds of the languages present at the time of European settlement \autocite[p.~817]{bowern_computational_2012}. Pama-Nyungan was first proposed and named by Kenneth Hale \autocite[p.~136]{wurm_aboriginal_1963} and it has been the subject of considerable historical linguistic study since this time. Although the family has presented some challenges for historical linguistics, the phylogenetic unity of Pama-Nyungan has been established on traditional historical linguistic grounds \autocite{alpher_pama-nyungan:_2004} with many subgroups identified within \autocites[for example,][]{ogrady_languages_1966}{wurm_languages_1972}{austin_proto-kanyara_1981}. For an overview of the history of Pama-Nyungan classification, see \textcite[ch.~1--5]{bowern_australian_2004} and \textcite{koch_historical_2014}. \textcite{bowern_computational_2012} perform a computational phylogenetic analysis of Pama-Nyungan using lexical data from 194 language varieties, providing for the first time a fully bifurcating phylogeny of the entire Pama-Nyungan family. \textcite{bouckaert_origin_2018} subsequently perform a phylogeographic analysis using the same dataset, but refined and expanded to 306 language varieties and including a geographic element to estimate the point of origin and spread pattern of the family through time and space.

The Pama-Nyungan family provides an excellent test case for this study. It holds practical advantages which make the task of phonological comparison easier, but it also provides us with a deliberately high bar to clear from a theoretical perspective. Both of these features are a result of the unusual degree of phonological homogeneity observed among Australian languages. Australian languages have been noted for a degree of similarity between phonological inventories of contrastive segments that is exceptional and unexpected in light of the phylogenetic and geographical breadth of the family, the level of diversity observed in vocabulary and aspects of grammar, and the level of phonological diversity found in comparably-sized families of languages elsewhere in the world. This has been noted as early as \textcite{schmidt_gliederung_1919} and in more recent times by \textcite{capell_new_1956}, \textcite{voegelin_obtaining_1963}, \textcite{dixon_languages_1980}, \textcite{busby_distribution_1982}, \textcite{hamilton_phonetic_1996}, \textcite{baker_word_2014}, \textcite{bowern_standard_2017} and \textcite{round_segment_2020}, among others. This curious level of homogeneity extends to phonotactics too \autocites{dixon_languages_1980}{hamilton_phonetic_1996}{baker_word_2014}{round_phonotactics_2020}.

On one hand, the abundance of similar phonological inventories makes the task of comparison between them easier, because it limits the problem of \emph{dataset sparsity}. Consider a character coding the frequency of some sequence of two segments \(xy\) in a language: This character can only be compared between languages that contain both \(x\) and \(y\) segments in their inventories. If a language lacks either segment in its inventory, then the character will be coded as absent or missing (as distinct from 0, where a language possesses both segments but never permits them in sequence). We expect fewer missing values in Australia, where languages tend to share a large proportion of equivalent segments, when compared to other parts of the world where we would expect to see many more missing values.

On the other hand, an ostensibly high degree of phonological homogeneity, in spite of considerable phylogenetic diversity, presents challenges for historical linguistics. \textcite[p.~141]{baker_word_2014} and \textcite[p.~103]{alpher_pama-nyungan:_2004} have both written on the difficulties for historical reconstruction in Australia because of this. Moreover, a phylogeny implies some degree of historical divergence, but in the case of Australian languages, there would appear to be little by way of phonological divergence, let alone divergences which are phylogenetically patterned. We therefore choose to study an Australian language family as a deliberately difficult test case, where we expect the bar be set high with respect to detecting phylogenetic signal.

\textcite{gasser_revisiting_2014} counter prevailing views on Australian phonological homogeneity. They find that common assumptions, of the kind discussed above and commonly found repeated in reference grammars, mask a degree of variation which is otherwise revealed by, firstly, extracting data on segmental inventories directly from wordlists and, secondly, considering segmental frequencies extracted from wordlists. This result motivates our current approach; here, we are also concerned with matters of frequency, extracted directly from language wordlists. However, we look at different kinds of characters, pertaining not to single segments but to phonotactics, and consider them with respect to their phylogenetic implications.

\hypertarget{wordlists}{%
\subsection{Wordlists}\label{wordlists}}

Our Pama-Nyungan phonotactic data is extracted from 111 wordlists which are part of a database under development by the last author \autocite{round_ausphon-lexicon_2017}, extending the Chirila resources for Australian languages \autocite{bowern_chirila:_2016}. In this study we restrict our attention to the most accurate sources available, and use only lexical data that is compiled by trained linguists and for which the underlying dataset is available in published or archived form. Additionally, we restrict our sample to wordlists containing a minimum of 250 words. We include this cut-off since measurement accuracy is a concern for smaller wordlists. A documented wordlist is necessarily only a subset of the complete lexicon of a language and it is unclear how big a wordlist must be before frequency statistics begin to stabilise around a sufficient level of accuracy. There is some work in this space concerning frequences of single segments \autocite{dockum_swadesh_2019}, suggesting a rapid decline in the accuracy of phoneme frequencies as wordlists drop below 250 words. Longer wordlists will always be better, however we select 250 words as a reasonable compromise which maintains a generally broad coverage of Pama-Nyungan languages. We return to the subject of wordlist sizes and potential implications for our results in Section \ref{overall-robustness}

Bibliographic details for all underlying data is available in Section S2 of the Supplementary Information. Owing to differences in the length of primary sources, there is considerable diversity in the size of the lexicons we use. As shown in Figure \ref{fig:lex-size}, the difference from smallest to largest is over an order of magnitude (min. 250, max. 4955), with the middle fifty percent between 509 and 1367 items. Mean lexicon size is 1112 (\(SD\) 916))

\begin{figure}

{\centering \includegraphics[width=1\linewidth]{fig/wordlist_sizes} 

}

\caption{Lexicon sizes.}\label{fig:lex-size}
\end{figure}

Original source data, which is typically orthographic and, if digital, is sometimes mixed with metadata or other extraneous material, has undergone extensive data scrubbing, conversion to phonemic form using language-specific orthography profiles \autocite{moran_unicode_2018}, and additional automated and manual error checking. These procedures ensure basic data cleanliness. Separately however, it has long been recognised that the segmental-phonological analysis of languages is a non-deterministic process \autocites{chao_non-uniqueness_1934}{hockett_problem_1963}{hyman_universals_2008}{dresher_contrastive_2009}. Two linguists faced with the same data may produce different analyses, not due to error but due to different applications of the very many analytic criteria that figure into any analysis of segments. Consequently, the cross-linguistic phonological record varies not only according to language facts per se, but due also to variation in the practice of linguistic analysis. Recent literature \autocites{lass_vowel_1984}{hyman_universals_2008}{van_der_hulst_phonological_2017}{round_matthew_2017}{kiparsky_formal_2018} emphasises the value of normalizing source descriptions prior to the analysis of cross-linguistic phonological datasets. This is not an information destroying process---it does not `standardise' languages---but it may shift information from one part of the representation (e.g., contrast between individual symbols) to another (e.g., contrasts between sequences of symbols), in order that information is located in a comparable way across the languages in the dataset, and therefore is more amenable to comparative analysis. Our wordlist data is normalised is this sense. Complex segments are split into simple sequences (e.g., prenasalised stops are split into a homorganic nasal + stop sequence); long vowels are represented as a sequence of identical short vowels, and vowel-glide-vowel sequences in which the glide is homorganic with either vowel are normalised to vowel-vowel; fortis consonants are represented as a sequence of identical short consonants, and positionally neutralised fortis/lenis stops as singletons; laminal consonants which do not figure in a pre-palatal versus dental opposition are represented as palatal, and rhotic glides which do not figure in an alveolar versus post-alveolar opposition are represented as post-alveolar \autocites[see also][]{round_phonemic_2019}{round_australian_2019}. The various phonotactic character sets were extracted from these normalised, comparably segmented wordlists.

\hypertarget{ref-phylogeny}{%
\subsection{Reference phylogeny}\label{ref-phylogeny}}

The reference phylogeny we use is a maximum clade credibility tree\footnote{Bayesian phylogenetic methods return not a single phylogenetic tree but a posterior distribution of many possible trees. These trees can be summarised into a single maximum clade credibility tree with confidence levels for each node in the tree, pertaining to how frequently that node appears in the posterior sample. It is the maximum clade credibility tree that we use for a reference phylogeny in this study. See \textcite{bowern_computational_2012} and \textcite{bouckaert_origin_2018} for a full explanation of the methods used to infer the phylogenies considered in this section.} of 285 Pama-Nyungan language varieties inferred using lexical cognate characters by the second author (Figure S1, Supplementary Information). It was inferred independently of this study, prior to this study's conception and without the involvement of the first and third authors. It was inferred using the same Stochastic Dollo model as \textcite{bowern_computational_2012}, but with an expanded and refined dataset. Further details of the model and phylogeny construction are described in \textcite{bowern_computational_2012}, \textcite{bowern_pama-nyungan_2015} and \textcite{bouckaert_origin_2018}. The cognate data used to infer the reference phylogeny is available on Zenodo \autocite{bowern_pama-nyungan_2018}. See Section S1 of the Supplementary Information for more information on the reference phylogeny.

We considered a reference tree from a newer phylogeographic analysis of Pama-Nyungan based on largely the same data plus further expansion to 304 doculects and continued refinement \autocite{bouckaert_origin_2018}, however, we opted against its use for this particular study. The reason for this is that, although Bayesian inference of phylogenetic tree topology is considered generally robust to the levels of lexical borrowing observed among Pama-Nyungan languages \autocites{greenhill_does_2009}{bowern_does_2011}, borrowing still has the effect of reducing branch lengths across the tree \autocite{greenhill_does_2009}. This effect, and consequently the accuracy of branch length estimates, is equally applicable to both trees considered here. However, the geographic element in the phylogeographic study uses, in part, branch lengths to model geographic dispersal. The posterior distribution of trees, which is jointly informed by cognate data and geography, may therefore show a bias towards geographically proximal languages whose apparent divergence times have been reduced by high rates of borrowing. Thus, although branch length estimates will be impacted by borrowing in any phylogenetic study of Pama-Nyungan, there is more chance of borrowing affecting topology in the phylogeographic study.

We consider it unlikely that the overall conclusions of the study would be altered by the choice of which version of the Pama-Nyungan phylogeny we use as a reference tree. Each of the studies referenced above produced highly congruent Pama-Nyungan phylogenies \autocite[see][ for a detailed comparison]{bouckaert_origin_2018}. Furthermore, \textcite{bouckaert_origin_2018} features fixed clade priors based on subgroups identified in earlier studies, so topological differences are constrained to some extent by design. Nevertheless, the accuracy of the reference tree is a key assumption of the methods we use in this study, and thus phylogenetic uncertainty is an important consideration. We return to this point in Section \ref{overall-robustness} and evaluate the overall robustness of our results to phylogenetic uncertainty by replicating a subset of phylognetic signal testing over a posterior sample of trees.

As discussed above, we treat each wordlist in our study as its own doculect. The reference tree in this study was inferred using a similar approach, while remaining less-commital about the particular status of the unit of analysis. Resources were sometimes combined for a particular language, but they are also frequently broken up into separate units, particularly when the resources come from different authors and different time periods. We have taken care to match the wordlists in this study to their exact or best corresponding tip in the reference tree. In most cases, the wordlists we use here are the same as those used to infer the reference phylogeny. In other cases, we use a different source to the one used in the reference phylogeny but there is, nevertheless, a straightforward one-to-one mapping between the language variety our wordlist represents and a corresponding tip in the tree. In one case, our Mudburra source \autocite{nash_mudburra_1988} matches neither of the sources for the two Mudburra tips in the reference phylogeny. However, the two varieties in the reference phylogeny have the same date. This entails that when either of them is removed from the tree, the exact same result is obtained in terms of tree geometry, which is what is significant for our investigation. Accordingly, we remove one and match our source to the other.

\hypertarget{phy-sig-bin}{%
\section{Phylogenetic signal in binary phonotactic data}\label{phy-sig-bin}}

In the simplest case, the phonotactics of different languages may be compared in terms of which sequences of two segments (\emph{biphones}) they permit and which they do not. If claims about the relative homogeneity of phonotactic constraints in Australian languages holds, then we would expect this kind of comparison to yield little, if any, phylogenetic signal.

In this test, we construct the dataset as follows: We automatically extract from all wordlists every unique sequence of two segments---or more accurately, sequences of \(xy\) where each of \(x\) and \(y\) is either a phonological segment or a word boundary `\#'. Each sequence becomes a character (variable) in the dataset, for which every language receives a binary value: 1 if the sequence \(xy\) is found in the language's wordlist (even if only once); 0 if the language contains both segments in its inventory but the sequence \(xy\) never appears in its wordlist; or, NA (not applicable, missing) if the language does not contain one (or both) of either segment \(x\) or \(y\) in its inventory (and therefore \emph{a prioi} cannot contain the sequence \(xy\)). Binary data of this kind represents sequence permissibility: Where a language contains both segments in its inventory, it will either permit them to appear together in sequence or it will not. In this respect, the information encoded by these characters is similar to what one might find in the phonotactics section of a descriptive grammar, where one often encounters a description in prose and/or a basic tabulation of which segments are permitted and where, within syllable and word structures. However, this kind of information is also, in a sense, quite coarse-grained, since there are only two possible values. A sequence which is very common in one language will be coded in exactly the same way as a sequence which only appears a handful of times in another language.

We apply the \(D\) test individually to each character in the dataset that meets two conditions: at least 50 non-missing values (due to the aforementioned reliability issue with sample sizes smaller than this) and at least one instance of variation (we do not test characters where all languages share identical 1 or 0 values). Given the extensive history of description of Australian languages as phonotactically homogenous, our prior expectation is that testing binary data will fail to yield significant phylogenetic signal. Indeed, we might expect that \(D\) will fall significantly below 0, indicating that values are clumped among tips on the reference phylogeny even more conservatively than would be expected if they had evolved in the same phylogenetic pattern as lexical data.

To evaluate the statistical significance of \(D\) for any given character, a \(p\) value is estimated for each of two null hypotheses: The null hypothesis that \(D = 1\) (\(H_{0(D=1)}\), character values are distributed randomly with regards to phylogeny) and the null hypothesis that \(D = 0\) (\(H_{0(D=0)}\), character values are distributed as could be expected if the character has evolved along the phylogeny according to a Brownian motion model). Each \(p\) value is calculated using a randomization procedure: A random distribution of \(D\) scores is acquired by randomly shuffling character values among tips on the tree for 10,000 permutations. The conventional cutoff for statistical significance is \(p = 0.05\). Here, we use the corresponding Bonferroni-corrected cutoff of 0.025.\footnote{Bonferroni correction is used because the conventional threshold for statistical significance, 0.05, represents the expected chance of a false discovery (false positive). This figure is known as the type I error rate (\(\alpha\)). The chance of a false discovery is multiplied when multiple tests are carried out. Bonferroni correction, which involves dividing the threshold for statistical significance by the number of tests being conducted, ensures that the chance of observing a false positive in any of the set of tests remains at the conventional rate, \(\alpha = 0.05\). In our case, two null hypotheses are tested for each character, hence we divide the threshold for statistical significance by two, ensuring the chance of a false positive for any particular character is 0.05.} For any given character, there are six possible results:
- \(D\) is significantly below 0. Character values are even more tightly clumped among sisters than Brownian motion alone would lead us to expect.
- \(D\) is significantly below 1 and not significantly different from 0. The data patterns phylogenetically, i.e., there is phylogenetic signal.
- \(D\) is significantly above 0 and below 1. The data is neither clearly random nor clearly phylogenetic.
- \(D\) is significantly above 0 and not significantly different from 1. It is consistent with randomness, not phylogeny.
- \(D\) is significantly above 1. It is even more dispersed than expected via a random process.
- \(D\) is not significantly distinct from 0 nor 1. The patterning of the data is indeterminate, and cannot be distinguished from randomness nor from Brownian phylogenetic evolution.

To summarise, the testing procedure proceeds as follows. For each binary biphone character:
- If the character has at least 50 non-NA values, then:
- If the character has at least one `1' and one `0' value (i.e.~not every value is identical), then:
- Calculate \(D\), and:
- Conduct randomisation procedure to calculate \(p\) for \(H_0: D = 0\), and:
- Conduct randomisation procedure to calculate \(p\) for \(H_0: D = 1\), and:
- A result is interpreted from the combination of \(D\) and two \(p\) values.

\hypertarget{phy-sig-bin-results}{%
\subsection{Results for binary phonotactic data}\label{phy-sig-bin-results}}

We estimate \(D\) for 415 biphone characters using a script based on the \emph{phylo.d} function in the \emph{caper} package \autocite{orme_caper:_2013}, implemented in the statistical software \emph{R} \autocite{r_core_team_r:_2017}\footnote{The dataset for this and subsequent tests are available on Zenodo at \url{http://doi.org/10.5281/zenodo.3610089}. This repository also includes a full table of results and the R scripts used to perform the analysis and produce figures for the paper. See Section S3 of the Supplementary Information for usage instructions and a full description of these materials.}. As described in Section \ref{phylo-sig}, \(D\) is calibrated by simulating character evolution under two models---one where the character evolves at random relative to phylogeny and one where the character evolves following a Brownian motion threshold model. In this study, we conduct 10,000 permutations of each model for each character. The 415 \(D\) values cluster centrally around a mean of \texttt{mean(d\_test\_table\$D)}. The distribution is leptokurtic (kurtosis = \texttt{kurtosis(d\_test\_table\$D)}), meaning there are more outliers relative to a normal distribution, making the distribution appear as a tall, narrow peak with long tails (Figure \ref{fig:d-density}). The standard deviation is large (\texttt{sd(d\_test\_table\$D)}).

\begin{figure}

{\centering \includegraphics[width=0.66\linewidth]{fig/d-density} 

}

\caption{Density of $D$ estimates for binary biphone characters. Dotted lines mark $D=0$, the phylogenetic expectation, and $D=1$, the random expectation. Mean $D$ for all characters is 0.43, marked in red.}\label{fig:d-density}
\end{figure}

The \(D\) test was of indeterminate significance for half of all characters (238 characters, 57\% of the dataset). The \(D\) scores for 157 characters (38\% of the total dataset) show evidence of phylogenetic signal. Just 16 characters (4\%) show the opposite result, where the character is consistent with randomness and there is no phylogenetic signal present. Both null hypotheses are rejected for the remaining 4 characters. Of these, 3 are more clumped than their phylogenetic expectation and \texttt{sum(d\_test\_table\ ==\ "0\ \textless{}\ D\ \textless{}\ 1\ (both\ H0s\ rejected)")} fall somewhere between phylogenetic and random expectations (\(0 < D < 1\)). None are more dispersed than the random expectation. The distribution of these results among different biphone characters is plotted in Figure \ref{fig:d-swatch}. Note that we expect around 5\% of null hypothesis rejections (approximately 18 of 180 rejected null hypotheses) to be false discoveries. Nevertheless, when considering the whole dataset as an ensemble rather than each character individually, a general result can be discerned. The clearest conclusion is that binary, permissibility-based characters tend to be low yielding in information, giving a statistically significant outcome in fewer than half of cases. Nevertheless, where a significant result can be determined, phylogenetic signal does tend to be present---to a degree that is perhaps surprising in light of previous literature describing the relative homogeneity of Australian phonotactic restrictions and their lack of utility in historical endeavours. This result suggests there may be a greater degree of historical information contained in Pama-Nyungan phonotactics than previously thought. However, it may be that a finer-grained approach to data extraction is needed in order to detect it.

These results can be compared to two earlier studies performing the same test on much smaller samples of languages. In the first, \textcite{macklin-cordes_high-definition_2015} find no evidence for phylogenetic signal in the Yolngu subgroup of Pama-Nyungan---rather, data are significantly over-clumped, suggesting a higher degree of conservatism in phonotactic restrictions relative to lexical data. They fail to reject a null hypothesis of \(D = 0\) for Ngumpin-Yapa, suggesting there may be a degree of phylogenetic signal in the Ngumpin-Yapa dataset. However, the pilot study results should be treated with caution---particularly the failure to reject the \(D = 0\) null hypothesis in the case of Ngumpin-Yapa---due to the small sample sizes (10 languages for Ngumpin-Yapa, 7 for Yolngu), well below the minimum of 50 taxa recommended by \textcite{fritz_selectivity_2010}. \textcite{dockum_phylogeny_2018} performs the same analysis using biphone characters from 20 Tai lects of the Kra-Dai family. In contrast to \textcite{macklin-cordes_high-definition_2015}, Dockum finds some evidence of phylogenetic signal in the Tai data and suggests perhaps the earlier result was due to insufficient variation in that particular language sample rather than a limitation of binary biphone characters per se. Although the low information yield from binary data is to be expected, our results here appear to support Dockum's conclusion.

\begin{figure}

{\centering \includegraphics[width=0.9\linewidth]{fig/d-sig-swatch} 

}

\caption{Phylogenetic signal significance testing for binary biphone characters. This grid colour-codes each biphone character according to the results of its respective significance tests. The grid is arranged such that the vertical axis represents the first segment of the biphone and the horizontal axis represents the second segment. Besides a tendency for phonotactic restrictions at word boundaries to show phylogenetic signal, few patterns stand out.}\label{fig:d-swatch}
\end{figure}

\hypertarget{phy-sig-bin-robustness}{%
\subsection{Robustness checks}\label{phy-sig-bin-robustness}}

Given a sufficient number of taxa for which data are available (\textgreater{}50), \(D\) scores should reflect a degree of phylogenetic signal present in the data, independently of tree size (the number of taxa) and shape (branching patterns). To check this, in Figure \ref{fig:d-scatterplots}(a) for each character we plot its \(D\) score against the number of doculects for which it had non-missing values. Irrespective of the number of doculects that supply non-missing values, the \(D\) scores appear to cluster centrally around mean \(D\) for the dataset, suggesting that \(D\) is not being unduly affected by missing values for particular characters. A second check leads to rather different results, however. We check whether skewed distributions of character values affects \(D\) scores (Figure \ref{fig:d-scatterplots}(b)). Here, we consider the distribution of 1s and 0s for each character and plot \(D\) against how skewed the distribution of character values is towards a particular value. For example, a character where there are 107 `1' values in the dataset and only 4 `0' values will have a skewing rating of 0.963964 (the count of `1' values, 107, divided by 111 total observations). Here, we find that when the ratio of 1s and 0s for a character is highly unequal, estimates of \(D\) tend towards extreme magnitudes while also being unrevealing, that is, statistically distinguishable neither from 0 nor 1.\footnote{Note that it is a desirable feature of the \(D\) test that it should return a lack of significance when there is a near-complete lack of variability in the data for the test to evaluate. This is an issue with the data, not the test.} As described above, Australian languages are known for homogeneity in phonological inventories and phonotactic restrictions, so consequently there are many characters with skewed distributions affecting the results. In Figures \ref{fig:d-density-filtered}--\ref{fig:d-swatch-filtered}, we plot a subset of the \(D\) test results, restricted to characters with less skewing.

\begin{figure}
\includegraphics[width=1\linewidth]{fig/d-scatterplots} \caption{Scatterplot of $D$ scores against (a) the number of doculects with non-missing values for each character and (b) the skewing of the distribution of 1s and 0s for each character. $D$ clusters evenly around the mean regardless of the number of missing values. Variation in $D$ accelerates greatly among characters where all but 1 or a few doculects share the same value, but the results are overwhelmingly not significant.}\label{fig:d-scatterplots}
\end{figure}

\begin{figure}

{\centering \includegraphics[width=0.66\linewidth]{fig/d-density-filtered} 

}

\caption{Density of $D$ estimates for binary biphone characters where character values are skewed less than 97-to-3.}\label{fig:d-density-filtered}
\end{figure}

\begin{figure}

{\centering \includegraphics[width=0.75\linewidth]{fig/d-swatch-filtered} 

}

\caption{$D$ scores for binary biphone characters where character values are skewed less than 97-to-3. Red shades show a character's proximity to $D=0$---darker red indicates stronger phylogenetic signal. Blue shades show a character's proximity to $D=1$, where character values are distributed randomly. This heat grid shows the proximity of each individual character's $D$ value to $D=0$---its expectation if the character evolved along the phylogeny following a Brownian motion process (the vertical axis represents the first segment of the biphone, the horizontal axis represents the second segment).}\label{fig:d-swatch-filtered}
\end{figure}

\hypertarget{phy-sig-cont}{%
\section{Phylogenetic signal in continuous phonotactic data}\label{phy-sig-cont}}

We test whether a higher degree of phylogenetic signal is detectable in continuous-valued biphone characters. As in the previous test, we take every possible sequence of two segments, or biphones, in our sample of 111 Pama-Nyungan wordlists. In this case, however, rather than simply coding for the presence of a biphone in a language's lexicon, we consider the relative frequencies of transitions between the segments in that biphone across the language's lexicon. For each biphone character, we take two values: The Markov chain forward transition probability---that is, for a biphone \(xy\), the probability of \(x\) being followed by \(y\), normalised over all instances of \(x\). This captures, if only in a basic way, our awareness that words do not consist of strings of independent segments, but rather the probability of observing some segment is very much dependent on what came before it. Secondly, we take Markov chain backward transition probabilities---that is, for the biphone \(xy\), the probability of \(y\) being preceeded by \(x\), normalised over all instances of \(y\) in the lexicon. The frequency characters we extract come from wordlists. This is advantageous in that they are somewhat independent of word frequency effects since each word is counted only once, in contrast to frequencies extracted from language corpora. On the other hand, speakers show sensitivity to phoneme frequencies in language use (for example, when coining novel words) \autocites{coleman_stochastic_1997}{zuraw_patterned_2000}{ernestus_predicting_2003}{albright_rules_2003}{eddington_spanish_2004}{hayes_stochastic_2006}{gordon_phonological_2016} so word frequency will likely have some effect on phoneme and biphone frequency even in a wordlist. Investigation of phylogenetic signal in frequency characters extracted from corpora versus wordlists may be a possibility for future study.

We quantify phylogenetic signal by estimating \(K\) \autocite{blomberg_testing_2003} individually for each character, using the \emph{multiPhylosignal} function, in the \emph{picante} package \autocite{kembel_picante:_2010}, in \emph{R} statistical software. The \(K\) test has somewhat greater statistical power than the \(D\) test, enabling us to apply the test to characters with as few as 20 non-missing values. Calculation of \(K\) works with non-zero values only, so zero values (where the language contains both segments \(x\) and \(y\) but \(x\) is never followed by \(y\), or vice versa) are considered not applicable and removed from calculation. A total of 490 characters (245 biphone forward transition probabilities and 245 backward transition probabilities) meet the criterion of at least 20 languages with non-missing and non-zero values for testing. Subsequently, to evaluate whether the level of phylogenetic signal is significant for a given character, we conduct Blomberg, Garland and Ives' (2003) randomization procedure with 10,000 random permutations per character.

To summarise, this testing procedure proceeds as follows. For each biphone frequency character:
- If the character has at least 20 non-NA values, then:
- If the character has at least two unique frequency values (i.e.~not every character value is the same), then:
- Calculate \(K\), and:
- Conduct randomisation procedure to calculate \(p\) for \(H_0: K = 0\).

Mean \(K\) for all 490 characters is 0.54 (\(SD\) 0.21) (Figure \ref{fig:k-density}). This is comparable to certain physiological traits presented as examples of biological traits with a high degree of phylogenetic signal by \textcite{blomberg_testing_2003}, for example, \(K=0.55\) for log body mass of primates. Using the \textcite{blomberg_testing_2003} randomisation procedure, we find a statistically significant degree of phylogenetic signal for 354 of 490 characters (180 forward transition characters, 174 backward transition characters), or 72\% of the total dataset.

\begin{figure}

{\centering \includegraphics[width=0.66\linewidth]{fig/k-density} 

}

\caption{Density of $K$ scores for all frequency-based biphone character (frequencies of both forward and backward transitions between segments).}\label{fig:k-density}
\end{figure}

We consider whether phylogenetic signal is higher or lower in certain kinds of biphone characters. Figure \ref{fig:k-swatch} shows a matrix of \(K\) scores for forward transition characters, with rows and columns arranged by phonological natural class. No clear pattern stands out.

\begin{figure}

{\centering \includegraphics[width=0.75\linewidth]{fig/k-swatch} 

}

\caption{Phylogenetic signal for forward transition frequencies. This heat grid shows $K$ scores for biphone characters (forward transition frequencies only). Each square represents a biphone (where the first segment is listed on the vertical axis and second segment on the horizontal axis). Data points are taken from the frequencies of each biphone, $xy$, over the total frequency of segment $x$ in each language, and then phylogenetic signal $K$ is measured for each biphone. Darker red shades indicate a stronger degree of phylogenetic signa. As with the $D$ test, no clear pattern of high versus low $K$ scores stands out, although there is a high degree of phylogenetic signal in the dataset overall.}\label{fig:k-swatch}
\end{figure}

\hypertarget{phy-sig-cont-robustness}{%
\subsection{Robustness checks}\label{phy-sig-cont-robustness}}

Although \(K\) is intended to be a measure of phylogenetic signal that is independent of tree size and shape, tree size and shape can have some effect on results in practice \autocite{munkemuller_how_2012}. We wish to check that the Pama-Nyungan tree does not contain any unusual properties that could cause either the K statistic or the randomization procedure to perform unexpectedly. To do this, we allow simulated characters to evolve specifically along the Pama-Nyungan reference tree. We vary the model of evolution, between perfect Brownian motion along the entire tree and pure randomness generated directly at the tips of the tree, by mixing different strengths of Brownian phylogenetic signal and non-phylogenetic noise. 1000 traits are simulated at each percentage point interval for 100,000 total simulated traits (in other words, 1000 traits simulated with 100\% Brownian motion, then 1000 traits simulated with 99\% Brownian motion and 1\% randomness, and so on, until the traits evolve 100\% at random). Each simulated trait is then tested for statistical significance using the randomisation procedure described in Section \ref{phylo-sig}, with 1000 repetitions to determine a \(p\) value. In a robust testing scenario, \(K\) will scale appropriately between 0 and 1 according to the level of Brownian motion and random noise being simulated, and the randomisation procedure will distinguish between traits with and without a significant degree of phylogenetic signal with a satisfactory amount of Type I (false positive) and Type II (false negative) errors.

The results are plotted in Figure \ref{fig:k-simulation-plots}. The \(K\) statistic shows a considerable degree of variability but, in the absence of substantial random noise, centres slightly below \(K=1\) which suggests the statistic is behaving as expected (if not slightly conservatively) when phylogenetic signal is present. For characters whose simulated evolution is near-random, the baseline of \(K\) seems to be elevated a little by our particular reference tree, with non-phylogenetic simulated characters ranging from the expected \(K=0\) to around \(K=0.3\). This should be kept in mind when interpreting \(K\) scores across our results. As for the randomisation procedure, Figure \ref{fig:k-simulation-plots}(b) shows the percentage of simulated traits that were identified as having a significant degree of phylogenetic signal (\(p < 0.05\)) at a given level of Brownian motion mixed with random noise. Above around 65\% Brownian motion, there are no Type II errors. The ability to detect significant phylogenetic signal drops as the level of random noise increases beyond 35\%, though overall the test's sensitivity seems acceptable. At the opposite extreme, where characters are simulated completely at random (and, therefore, there is no phylogenetic signal to detect) the randomisation procedure falsely detects phylogenetic signal 5.2\% of the time, very close to the expected false discovery rate of 5\% (given the conventional threshold for statistical significance of \(\alpha = 0.05\)). On the basis of these simulations, we are satisfied that randomisation procedure is sufficiently robust, given the particular size and shape of the Pama-Nyungan reference phylogeny.

\begin{figure}

{\centering \includegraphics[width=0.9\linewidth]{fig/phylosim} 

}

\caption{Behaviour of the $K$ statistic and randomization procedure with the Pama-Nyungan reference phylogeny. Artificial characters are simulated evolving along the phylogeny with varying levels of non-Brownian noise. Where a pure Brownian motion process operates, $K$ averages around 1, as expected. Where there is no Brownian process at all (and therefore no phylogenetic signal) $K$ is elevated to around 0.2---likely an artefact of this particular tree size and shape.}\label{fig:k-simulation-plots}
\end{figure}

As a final check, we consider whether the \(K\) statistic might be affected by the quantity of missing or `not applicable' values for a given character. We inspect this visually by plotting, for all biphone characters, the relation between a biphone's \(K\) score and the number of language varieties with non-missing data points on which \(K\) was calculated (Figure \ref{fig:k-scatterplot}). When \(K\) is calculated on fewer than around 40 non-missing values, the statistic shows a wider degree of variability. In addition, phylogenetic signal is deemed statistically significant for fewer characters in this range, suggesting that the quantity of missing values is affecting the statistical power of the test. However, all \(K\) scores cluster centrally around the mean regardless of the number of languages with non-missing values, suggesting that the mean \(K\) we observe for the dataset overall is not significantly affected by missing data.

\begin{figure}

{\centering \includegraphics[width=0.66\linewidth]{fig/k-scatterplot} 

}

\caption{Estimates of $K$ for forward and backward transition frequency characters plotted against the number of doculects with non-missing values for each character. Although some statistical power is lost and variability increases among characters with the most missing values, $K$ scores cluster evenly around mean $K$ (0.52).}\label{fig:k-scatterplot}
\end{figure}

\hypertarget{fwd-vs-bkwd}{%
\subsection{Forward transitions versus backward transitions}\label{fwd-vs-bkwd}}

\begin{figure}

{\centering \includegraphics[width=0.66\linewidth]{fig/k-fwd-bkwd} 

}

\caption{Distribution of $K$ scores for forward transition frequencies versus backward transition frequencies. We find no significant difference between these character types.}\label{fig:k-fwd-vs-bkwd}
\end{figure}

We find no significant difference in the means of \(K\) for forward transition characters (mean \(K =\) 0.54) and backward transition characters (mean \(K =\) 0.53) (\(t=\) 0.44, \(df=\) 487.98, \(p=\) 0.661, 95\% CI {[}-0.03, 0.05{]}). The distributions of \(K\) scores for forward and backward transitions are plotted in Figure \ref{fig:k-fwd-vs-bkwd}, showing a high degree of overlap between the two.

\hypertarget{norm-characters}{%
\subsection{Normalisation of character values}\label{norm-characters}}

Visual inspection of the density plots for each character shows there is a tendency for character data to be negatively skewed (the weight of the distribution is left-of-centre), although this is not universally the case. To test whether the particular, heavy-tailed nature of the data has an effect on tests for phylogenetic signal, we apply Tukey's Ladder of Powers transformation to each character in the dataset and re-run both the \(K\) test and randomization procedure. This is a power transformation, which makes the data fit a normal distribution as closely as possible. It does this by finding the power transformation value, \(\lambda\), that maximises the \(W\) statistic of the Shapiro-Wilk test for normality for each character individually. For our purposes, this transformation is effectively a change in the evolutionary model: A Brownian motion process is still assumed---a character value may wander up or down with equal probability---but, in this model, character values shift up or down along a transformed scale.

Mean \(K\) for normalised character data is 0.55 (\(SD\) 0.1966404). Of 490 characters, 408 or 83\% (208 forward transitions, 200 backward transitions) contain phylogenetic signal significantly above the random expectation. There is no statistically significant difference between mean \(K\) for untransformed data (0.54) versus mean \(K\) for normalised data (\(t=\) 0.59, \(df=\) 973.66, \(p=\) 0.558, 95\% CI {[}-0.02, 0.03{]})---see Figure \ref{fig:orig-vs-nrmlzd}.

\begin{figure}

{\centering \includegraphics[width=0.66\linewidth]{fig/k-orig-norm} 

}

\caption{Distributions of $K$ for untransformed character values and their normalised counterparts. We find no significant difference between these distributions.}\label{fig:orig-vs-nrmlzd}
\end{figure}

\hypertarget{phy-sig-classes}{%
\section{Phylogenetic signal in natural-class-based characters}\label{phy-sig-classes}}

One limitation of analysing phylogenetic signal in biphone characters is the assumption that every biphone character is a statistically independent observation. In historical linguistic processes, however, phonological segments rarely behave independently. Rather, sound changes are applied to whole sound classes, thereby affecting any one of various cross-cutting sets of phonological segments (and, therefore, biphone characters we have used in this study).

To account for this non-independence and more faithfully model what we know about how phonotactic systems operate in a language, we extract forward and backward transition probabilities for sequences of phonological features. For the purposes of this experiment, word boundaries are counted as a class and vowels are reduced to a single `vowel' class. Three sets of characters are extracted: foward and backward transition probabilities between natural classes based on place of articulation (segments belonging to the following classes: word boundary, labial, dental, alveolar, retroflex, palatal, velar, glottal, vowel); forward and backward transition probabilities between natural classes based on major places of articulation, where coronal contrasts have been collapsed (word boundary, labial, apical, laminal, velar, vowel); and natural classes based on manner of articulation (word boundary, obstruent, nasal, vibrant, lateral, glide, rhotic glide, vowel). The choice of natural classes is based on well-established principles of organisation among segments in Australian languages \autocites{dixon_languages_1980}{hamilton_phonetic_1996}{baker_word_2014}{round_segment_2020}{round_phonotactics_2020}.

\begin{table}

\caption{\label{tab:k-natural-classes-summary}Summary of $K$ analysis for forward and backward transition frequencies between different natural classes. The two rightmost columns indicate the total number of characters analysed and the percentage of those characters with a significant degree of phylogenetic signal according to the randomization procedure.}
\centering
\begin{tabular}[t]{lrrr}
\toprule
Classes & Mean K & n characters & significant (\%)\\
\midrule
Place & 0.61 & 126 & 94\\
Major place & 0.62 & 96 & 74\\
Manner & 0.59 & 88 & 66\\
\bottomrule
\end{tabular}
\end{table}

Table \ref{tab:k-natural-classes-summary} presents mean \(K\) and the proportion of significant characters for each of these three natural class-based datasets. All show highly similar distributions (Figure \ref{fig:k-natural-classes}). There is no statistically significant difference in the means of \(K\) for the three feature types, according to a one-way ANOVA (\(F\)(, ) = 0.46, \(p\) = 0.6288925). An Anderson-Darling k-sample test, which tests the hypothesis that \(k\) independent samples come from a common, unspecified distribution (i.e., no prior assumption about normality) also finds no significant difference in the distributions of \(K\) scores for the three natural class-based datasets (\(AD=\) 2.15, \(T.AD=\) 0.14, \(p=\) 0.34).

\begin{figure}

{\centering \includegraphics[width=0.66\linewidth]{fig/k-natural-classes} 

}

\caption{Comparison of $K$ scores for transitions between different kinds of natural classes. The differences between all three distributions are not statistically significant.}\label{fig:k-natural-classes}
\end{figure}

\hypertarget{classes-vs-biphones}{%
\subsection{Natural class-based characters versus biphones}\label{classes-vs-biphones}}

\begin{figure}

{\centering \includegraphics[width=0.66\linewidth]{fig/k-biphones-classes} 

}

\caption{Distributions of $K$ scores for biphone characters, coding the relative frequencies of transitions between phonological segments, and natural class-based characters, coding the relative frequencies of transitions between natural classes of segments. Phylogenetic signal is higher overall in the natural class-based dataset than the biphone-based dataset}\label{fig:k-biphones-vs-features}
\end{figure}

We compare the degree of phylogenetic signal in the biphone data tested in Section \ref{phy-sig-cont} to the natural class-based data tested here (see Figure \ref{fig:k-biphones-vs-features}). Mean \(K\) for all natural class-based characters is \texttt{round(mean(k\_all\_classes\$K),\ 2)}, which is significantly higher than the mean \(K\) for biphone data, \texttt{round(biphone\_meanK,\ 2)} (\(t=\) 4.37, \(df=\) 618.89, \(p=\) 0, 95\% CI {[}0.04, 0.1{]}). The Kolmogorov-Smirnov test, which is more sensitive to the overall shape of the distribution than a t-test comparison of means, also finds a significant distinction between \(K\) for biphone characters and \(K\) for natural class-based characters (\(D=\) 0.2, \(p < 0.001\)).

\hypertarget{discussion}{%
\section{Discussion}\label{discussion}}

Computational phylogenetic methods are increasingly commonplace in historical linguistics. However, there has been relatively less consideration of the range of data types that might profitably be used with computational phylogenetic methods, beyond traditional, manually-assembled sets of lexical cognate data. In this study, we have considered the potential utility of quantitative phonotactic data for historical linguistics, for the reasons that quantitative phonotactic data is (i) readily extractible from basic wordlists, and (ii) may show certain kinds of historical conservatism, where the historical signal in more traditional lexical data would be affected by borrowing and lexical innovation.

We extracted frequencies of transitions between phonological segments in scrubbed and comparably segmented wordlists representing 111 Pama-Nyungan language varieties. As points of comparison, we extracted two additional datasets: Firstly, a binarised version of the dataset, which simply records whether or not particular two-segment sequences, a.k.a. biphones, are present in a language's wordlist. This is to emulate, in a simple sense, the kind of information which is often recorded in the phonology section of published descriptive grammars. We also extracted frequencies for transitions between natural classes of sounds. This is to account (at least, to some partial degree) for the fact that phonological segments tend not to evolve independently but pattern into natural classes, thereby limiting the independence of biphone-based variables.

To test whether historical information is preserved in our phonotactic datasets, we tested for phylogenetic signal, that is, the degree to which variance in the data reflects the evolutionary history of the 111 language varieties. We took an independent phylogenetic tree, inferred by the second author using lexical cognate data, and assumed a simple Brownian motion model of evolution. Our first key finding is that a significant degree of phylogenetic signal is detected in all three datasets---binary, segment-based and sound class-based. Finding phylogenetic signal in the binary dataset is somewhat surprising, given previous descriptions of homogeneity in the phonotactics of Australian languages. Our second key finding is that phylogenetic signal is significantly stronger in the higher-definition, frequency-based datasets than it is in binary dataset. In turn, phylogenetic signal in the sound class-based dataset is significantly stronger than the segment-based dataset. We took a closer look at certain comparisons within datasets, namely, whether there is a difference between forward and backward transitions, different kinds of sound classes, or between original (heavy-tailed) data and transformed data (to more closely fit a normal distribution). In all three cases, no significant differences were found.

\hypertarget{overall-robustness}{%
\subsection{Overall robustness}\label{overall-robustness}}

One important assumption in this study is that the tree being used as a reference phylogeny is an accurate depiction of the phylogeny for the languages in question. In the absence of time travel, it is impossible to observe directly the true Pama-Nyungan phylogeny and thus surely satisfy this assumption. Instead, we must rely on best available data and methods to infer the phylogeny. As described in Section \ref{ref-phylogeny}, the Pama-Nyungan phylogeny we use in this study was inferred by the second author using Bayesian computational phylogenetic methods, which produce a posterior sample of possible trees. The specific reference tree we use in this study is a best possible summation of this posterior sample as determined by the \emph{maximum clade credibility} method. Naturally, there is a degree of uncertainty associated with the topology of the maximum clade credibility tree. It may be the case that a better representation of the true Pama-Nyungan phylogeny exists among the many slight permutations contained within the posteior tree sample.

To evaluate the robustness of our results against phylogenetic uncertainty, we repeat the \(K\) test for phylogenetic signal on a subset of sound-class-based characters and each of a subset of 100 trees selected from the posterior sample. The subset of characters includes forward and backward transitions between place features, plus forward and backward transitions for manner features. The trees selected are the 100 best trees in the posterior sample, based on the maximum clade credibility metric. In this way, we capture a degree of uncertainty in the topology and branch lengths of the Pama-Nyungan phylogeny, while restricting our attention to a subset of the most credible alternatives.

\begin{figure}

{\centering \includegraphics[width=0.66\linewidth]{fig/k-mcct-posterior} 

}

\caption{Comparison of $K$ statistics using a 100-tree posterior sample versus the maximum clade credibility tree alone.}\label{fig:tree-uncertainty}
\end{figure}

214 sound class characters in total are tested, giving 21,400 \(K\) statistics in total (each character multiplied by 100 trees). The mean of these \(K\) scores is 0.59, which compares to a mean \(K\) of 0.6 for the same characters applied only to the maximum clade credibility tree as per \ref{phy-sig-classes}. This difference is not statistically significant (\(t=\) -0.95, \(df=\) 217.17, \(p=\) 0.342, 95\% CI {[}-0.05, 0.02{]}). The distributions of these \(K\) scores are illustrated in Figure \ref{fig:tree-uncertainty}.

\begin{figure}

{\centering \includegraphics[width=1\linewidth]{fig/wordlist_subset_sizes} 

}

\caption{Wordlists ranked by size. (A) shows every second wordlist in the language sample. (B) shows the middle 50\% of wordlists. Each subset contains the same number of wordlists but discrepancy in their size is greatly reduce in (B).}\label{fig:wordlist-subset-sizes}
\end{figure}

\begin{figure}

{\centering \includegraphics[width=0.66\linewidth]{fig/k-IQR-vs-Every2nd} 

}

\caption{Comparison of $K$ statistics using only wordlists falling within the 25th an 75th quantiles (middle 50\%) for wordlist size, versus a sample of every 2nd wordlist (when ranked in order of size).}\label{fig:wordlist-uncertainty}
\end{figure}

In Section \ref{wordlists} we mentioned that our wordlists vary significantly is size. The length of a wordlist can correlate with many other linguistic properties that it has. For instance, in our data, wordlist length explains XX\% of the variance in the lists' mean phonemic word length (log-log regression, PVAL), that is, longer lists tend to contain longer words (presumably since shorter lists are weighted towards more basic, shorter vocabulary items). The existence of correlations like this means that it is not possible simply to `counterbalance' the length of wordlists by sub-sampling or resampling their items so that the resampled lists all have the same length. For example, the resampled lists that derive from longer underlying lists would still contain longer words. Nevertheless, it would still desirable to know whether our results in Sections \ref{phy-sig-bin}--\ref{phy-sig-classes} are unduly influenced by the disparities in our wordlist lengths, by manipulating it in a controlled fashion. To do this, we extracted two different language sub-samples from our dataset. For the first, we ranked all wordlists by size and selected every second list, producing a sample half the original size but with the same disparity in lengths. For the second, we ranked all wordlists by size and selected the middle 50\% of the ranking, again producing a sample of half the size but this time with heavily reduced disparity, as shown in Figure \ref{fig:wordlist-subset-sizes}. Wordlists in the middle 50\% range in length from 528 to 1361 (mean 872).

For these samples, we ran \(K\) tests on place and manner natural class characters. Our reasoning is that if wordlist disparity strongly affects the estimation of phylogenetic signal, then we should see a clear difference in the results. Mean \(K\) for the middle 50\% of wordlists is 0.77, which is somewhat greater than the mean for every second wordlist 0.7. This difference is statistically significant (\(t=\) 3.29, \(df=\) 421.21, \(p=\) 0.001, 95\% CI {[}0.03, 0.12{]}). This suggests that disparities in wordlist length are somewhat degrading the phylogenetic signal in our data, although there remains broad similarity between them, as pictured in Figure \ref{fig:wordlist-uncertainty}.

One attributing factor for this small degradation in phylogenetic signal may be measurement error among the bottom quartile of small wordlists. Throughout this study, we have assumed that all character values are accurate and do not account for measurement error. Accounting for measurement error when testing for phylogenetic signal is an area of active development in comparative biology \autocite{zheng_new_2009}. In future studies, this degradation in phylogenetic signal could be investigated by relating variation in \(K\) statistics carefully to various linguistic properties that correlate with the length of wordlists.

\hypertarget{limitations}{%
\subsection{Limitations}\label{limitations}}

Any investigation of phylogenetic signal in essence is an investigation of cross-linguistic (dis)similarity. Accordingly, whenever our representations of linguistic facts are altered, then the phylogenetic signal detectable in them will almost certainly change to some degree. In this paper our focus has been trained on the initial, fundamental question of whether phylogenetic signal is detectable in maximally simple phonotactic characters. However, as work like ours increases, one priority will be to investigate how investigators' choices about how data is represented affects results.

Relevant for the current study, in Section \ref{wordlists} we described a process of normalisation. The motivation for this was to attempt to minimize certain aspects of variation in phonological representation that can arise from variation in how different linguists analyses the same essential facts \autocites{chao_non-uniqueness_1934}{hockett_problem_1963}{hyman_universals_2008}{dresher_contrastive_2009}. While normalisation per se ought to improve the quality of cross-linguistic comparisons that the data enables, there is still the question of which targets one ought to normalise the data towards, on what effect that choice can have. For example, a reviewer asks whether our choice to split up complex segments might amplify phylogenetic signal if it leads to certain phylogenetically distributed complex segments counting instead as biphones. This can be answered in three ways. First, in the general case, since splitting segments changes representations, it will alter aspects of (dis)similarity in the data, and so is very likely to affect phylogenetic signal in some manner. Second, in this particular case, the reviewer is likely to be correct, due to details of our method. Any cross-linguistically rare, complex segment would likely get excluded from our dataset. This is because, although it would figure in certain biphones, we have made use only of biphones that reach a minimal level of recurrence across our language sample, and thus the biphones containing the rare segment quite likely would not qualify. However, if we split this complex segment into two segments, thus into a biphone, the resulting biphone may well have sufficient cross-linguistic recurrence to qualify for inclusion in our dataset, and subsequently may contribute to raising phylogenetic signal. A final observation on thi point is that such questions, about how choices in data representation interact with the results from corresponding quantitative analysis, are made tractable by our method of data preparation. Unlike many state-of-the-art cross-linguistic datasets, in which values for each language are hand-coded and thus incapable of being `recalculated' under altered assumptions, our phonotactic characters are generated algorithmically from an underlying, very rich dataset. With a change to algorithmic parameters, we can systematically split segments or glue them together, neutralise them or keep them distinct, and document what we have done and how. As mentioned, in this paper our focus is on the simple existence of phylogenetic signal. Our methods, though, naturally extend to enable comprehensive checking of such interactions between data choices and results. Ultimately, as a discipline, we would like this to be true for all typological research, not just phylogenetics \autocite{round_matthew_2017}. An advantage of our general approach, is that it open the doors to this rigorous mode of inquiry.

A further limitation of this study relates to the assumption that the data being tested for phylogenetic signal are independent of the data that was used to infer the reference phylogeny. In this study, the wordlists from which we extracted phonotactic characters contain, as a small subset, the basic vocabulary items from which lexical cognate characters were inferred and subsequently used to build the reference phylogeny. It is unclear exactly to what degree this inclusion of basic vocabulary compromises the independence of our reference tree and phonotactic data. A reviewer points out that cognate data and phonotactic data are still somewhat independent, even when extracted from identical wordlists, since phonotactic attributes are not directly called on to make cognacy judgements. Nevertheless, sound change affects both phonotactics and cognate identification, so some degree of non-indepenence is to be expected. To ascertain whether this effect is significant, future studies could parameterise the inclusion/exclusion of basic vocabulary from the phonotactic data.

One reviewer raises the correlation between phylogeny and geography. A noted limitation of phylogenetic comparative methods is the inability to account for geography as a possible confound \autocite{sookias_deep_2018} and this limitation applies to this study. Although we leave it as a priority for future work, the task of disentangling phylogeny and geography is not intractable. For example, \textcite{freckleton_space_2009} present a method for quantifying the relative degree of spatial versus phylogenetic effects in comparative characters.

One final point to note is that recent research suggests that phylogenetic signal can be inflated when character values evolve according to a Lévy process, where a character value can wander as per a Brownian motion process, but with the addition of discontinuous paths (i.e., sudden jumps in the character's value) \autocite{uyeda_rethinking_2018}. This is a realistic concern in the linguistic context, where segment frequencies are subject to sudden shifts caused by phonological mergers and splits. The possibility of Lévy-like evolutionary processes is a matter of concern also in comparative biology and methods to investigate it are subject to active development in that field \autocite{uyeda_rethinking_2018}.

\hypertarget{conclusion}{%
\section{Conclusion}\label{conclusion}}

Historical and synchronic comparative linguistics are increasingly making use of phylogenetic methods for the same reasons that led biologist to switch to them several decades ago. Our central contention has been that phylogenetic methods not only give us new ways of studying existing comparative data sets, but open up the possibility to derive insights from new kinds of data. Here we demonstrate the potential for phylogenetically investigating phonotactic data, by showing that it indeed contains the kind of phylogenetic signal which is the prerequisite for a whole spectrum of phylogenetic analyses.

We find significant phylogenetic signal for several hundred phonotactic characters extracted semi-automatically from 111 Pama-Nyungan wordlists, demonstrating that historical information is detectable in phonotactic data, even at the relatively simple level of biphones and despite ostensibly high phonological uniformity. Contrary to the prevailing view in literature on Australian languages, and contrary to the findings of an earlier pilot study on a much smaller language sample, we find that binary characters marking the presence or absence of biphones in a doculect contain enough phylogenetically-patterned variation to detect phylogenetic signal. However, we find that statistical power is relatively low when operating with coarse-grained binary data and quantification of the degree of phylogenetic signal is affected by a large number of low-variation characters, where all but one or a few doculects share the same value. We find stronger phylogenetic signal in biphone characters of forward and backward transition frequencies. This reaffirms the results of earlier work, for the first time on a sample of languages spanning an entire large family and the vast majority of a continent. It also reaffirms earlier findings that Australian phonologies show a greater level of variation than traditionally has been appreciated, once matters of frequency are taken into account. We find a significantly greater level of phylogenetic signal again in characters based on the frequencies of forward and backward transitions between natural sound classes. The sound-class-based approach reduces the quantity of characters available to test, but limits sparsity in the dataset and accounts somewhat better for the role of sound-classes in the evolutionary processes that affect phonotactic patterns in human language. Interestingly, although there exists considerable variation in the level of phylogenetic signal found in individual characters, we find no observable pattern to this variation in segment-based biphone characters nor between the mean levels of phylogenetic signal observed for different kinds of sound classes (e.g., characters concerning place versus manner of articulation).

This work has implications for comparative linguistics, both typological and historical. Firstly, we recommend the use of phylogenetic comparative methods in typological work where the phylogenetic independence of a language sample (or lack thereof) is paramount. In the immediate term, this should be the case for any typological work concerning phonotactics, even in parts of the world such as Australia where phonotactics traditionally have been assumed to be relatively independent of phylogeny. Beyond phonotactics, however, explicit measurements of phylogenetic signal can be made for any set of cross-linguistic data and this can be built into statistical analysis, even in the presence of gaps and uncertainty in phylogenetic knowledge. In two decades of quantitative development in historical linguistics, there has still been relatively limited consideration of the kinds of characters used for inferring linguistic histories. The phonotactic characters presented here can be extracted relatively simply and in large quantities from wordlists, even where a full descriptive grammar is not available. Here, we test only the degree to which patterns of variation in our data match our independent, pre-existing knowledge of the phylogenetic history of the Pama-Nyungan family, however, the results suggest that phonotactic data of this kind could be used where the phylogeny is less certain, either by incorporating phonotactic data into phylogenetic inference directly or by constraining parts of the tree where lexical data on its own returns some doubt.

\hypertarget{Acknowledgements}{%
\subsubsection{Acknowledgements}\label{Acknowledgements}}

JM-C is supported by an Australian Government Research Training Program Scholarship. ER is supported by Australian Research Council (ARC) Discovery Grant DE150101024 and a Language Evolution Grant from the ARC Centre of Excellence for the Dynamics of Language. CB is supported by National Science Foundation Grant NSF1423711. We gratefully acknowledge this support.

% ***************************************************


%CHAPTER 6
%If you are presenting work which has been previously published, acknowledge this here.
% ***************************************************
% How to introduce a previously published chapter
% ***************************************************
%This is an example of how you might introduce a chapter that has been published previously. 
\cleartoevenpage
\pagestyle{empty}	
%Use this command (above) to suppress the header from the preceding chapter.

\noindent
The following publication has been incorporated as Chapter~\ref{Chap:label}.

\noindent
Manuscript in preparation.

\begin{table}[h]
	\centering
	\begin{tabular}{clr}
		\toprule
		Contributor & Statement of contribution & \% \\
		\midrule
		\textbf{Your Name}				& writing of text 					& 70\\
															& proof-reading							& 60 \\
															& theoretical derivations 	& 70\\
															& numerical calculations 		& 100\\
															& preparation of figures 		& 80 \\
															& initial concept						& 10 \\
		\midrule
		Co-author 1								& writing of text 					& 20\\
															& proof-reading							& 10 \\
															& supervision, guidance 		& 20\\
															& theoretical derivations 	& 10\\
															& preparation of figures 		& 20 \\
															& initial concept						& 10 \\
		\bottomrule
	\end{tabular}
\end{table}

Details TBA.


% ***************************************************
% Example of an internal chapter
% ***************************************************
%This is an internal chapter of the thesis.
%If you have a long title, you can supply an abbreviated version to print in the Table of Contents using the optional argument to the \chapter command.
\chapter[Pama-Nyungan tree inference with phonotactics]{Pama-Nyungan tree inference with phonotactics}
\label{ch-pn-treebuilding}	%CREATE YOUR OWN LABEL.
\pagestyle{headings}

% ********* Enter your text below this line: ********
Summary: This paper evaluates whether phylogenetic tree inference in linguistics is strengthened by the inclusion of phonotactic information. We take \textasciitilde{}2k binary phonotactic variables and several hundred frequency variables and combine them with lexical cognate data from 111 Pama-Nyungan languages. The first part of the study explores the evolutionary dynamics of the phonotactic data. This is necessary to ascertain the best evolutionary model with which to infer a tree, since no one has used this kind of data in linguistic tree inference before. The second part of the study compares two models for inferring a Pama-Nyungan phylogeny using Bayesian methods. In one, a phonotactic data partition and lexical cognate partition are used jointly to infer trees. In the other these partitions are kept separate for the purpose of tree inference. Bayes factors for these two models are compared. We find that the combination of phonotactic data with lexical data \textbf{does/does not} significantly strengthen tree inference.

\hypertarget{pn-tree-intro}{%
\section{Introduction}\label{pn-tree-intro}}

\emph{This section clearly needs fleshing out, but actually not too much. Aim is to keep it sharp and concise}.

Background:

Phylogenies in linguistics are a big deal.

Lots of tree building been happening.

Phylogenies are also crucial for advances in comparative langauge sciences, studies of human history generally.

Data mainly limited to cognates. Some use of structural characters, but these tend to suffer from restricted state space.

In biology, \textcite{parins-fukuchi_use_2018} find that combining continuous morphological characters to more traditional, categorical data can strengthen tree inference. An example of integration of continuous morphological data and genomic data \textcite{domel_combining_2019}.

Prev. study \autocite{macklin-cordes_phylogenetic_2020} found phylogenetic signal in phonotactics. The hypothesis was that phonotactic systems are likely to evolve in an historically conservative way, reflect linguistic phylogenies and therefore be useful for tree inference. That finding was encouraging support for this hypothesis but not definitive proof by any means. Just because something has phylogenetic signal does not mean, by itself, that you can infer phylogenetic trees from it. For example, geography often has a pretty strong phylogenetic signal. In this study, we put the hypothesis to the test by attempting to infer a linguistic phylogeny with the aid of phonotactics.

\hypertarget{pn-tree-methods}{%
\section{Data and methods}\label{pn-tree-methods}}

Cognate data comes from \autocite{bouckaert_origin_2018}. Phonotactic data comes from Ausphonlex database of Australian language lexicons \autocite{round_ausphon-lexicon_2017}, which extends the Chirila database \autocite{bowern_chirila_2016} by providing phonemicised wordforms and various parameters for phonemic normalisation choices between wordlists. In this study, we restrict attention to wordlists which i) represent Pama-Nyungan language varieties that are also included in \textcite{bouckaert_origin_2018}, ii) have been published or are publicly accessible in some way, iii) have been compiled by trained linguists and iv) were compiled using some degree of in-person elicitation or audio recordings (reconstitutions using exclusively archival written records were not included). 111 Ausphonlex wordlists meet these criteria. Original wordlist sources and phonemic normalisation choices are listed in the Supplementary Materials.

\emph{Insert map of languages around here. Centroids colour-coded by subgroup.}

From each wordlist, we extract data on the presence and frequencies of \emph{biphones}, sequences of two segments (where each segment is either a phoneme or a word boundary). We extract two datasests. The first is a binary dataset marking the presence or absence of a given biphone in a language. A biphone is marked `1' if it is present in a language's wordlist (even if only once). If the biphone consists of two segments that are part of the language's phonemic inventory (and therefore the biphone could, in principle, occur in the language) but the biphone never occurs, it is marked `0' for absent. If one or both segments in the biphone are not part of the language's phonemic inventory, then it is marked as a gap `-' in the data. The second dataset A language's phonotactic system consists of rules governing how phonemic segments may combine into larger syllables and words. To represent phonotactics, we extract data on the presence and frequencies of \emph{biphones}, two-segment sequences, from language wordlists. The second dataset extracts frequencies of transitions between segments. We extract forward transition frequencies---that is, the frequency of segment \(y\) following segment \(x\), normalised over all instances of \(x\). We also extract backward frequency transitions---the frequency of segment \(x\) preceding segment \(y\), normalised over all instances of \(y\).

We are motivated to extract these frequency datasets for a couple of reasons. Firstly, it allows us to capture a finer grained level of information than binary data would allow. Binary data is more similar to the kind of phonotactic information one might find in a published language grammar, where a description of phonotactics that one would typically encounter involves a series of statements on the (binary) permissibility or otherwise of certain combination of segments. This information does not, however, account for quantitative differences between common, high frequency sequences of segments versus dispreferred sequences that rarely arise in a language's lexicon. There is considerable evidence to suggest that speakers are psychologically attuned to these kinds of phonological frequencies \autocites{coleman_stochastic_1997}{zuraw_patterned_2000}{ernestus_predicting_2003}{albright_rules_2003}{eddington_spanish_2004}{hayes_stochastic_2006}{gordon_phonological_2016}. The second reason is that the relatively rapid, semi-automated extraction of transition frequencies from wordlists captures structural variation between languages at a scale and degree of precision that would be difficult to attain from manual data coding methods (as preferred for the coding of lexical cognate data and grammatical data used in previous linguistic phylogenetic work). \textcite{macklin-cordes_phylogenetic_2020} show that this transition frequency dataset contains stronger phylogenetic signal than its binary equivalent. There is one limitation of the frequency transition data, which is that presently we require positive values to use for tree inference (more on evolutionary models and tree inference below). Biphones of zero frequency (recorded as `0' in the binary dataset) get transformed to gaps in the dataset. By including the binary dataset in this study, we retain a distinction between biphones that are impossible in a language (because one or both of the segments are absent from the language's phonemic inventory) and biphones that are possible in principle but are never observed. Our phonotactic data captures information on which phonemic segments may combine immediately adjacent to one another and the frequencies at which they do so. This is phonotactics in the simplest sense, and does not directly capture phonotactic restrictions that depend on sequences beyond two segments, syllable structure or morpheme boundaries. Nevertheless, \textcite{macklin-cordes_phylogenetic_2020} confirm that this simple level of phonotactic data is sufficieent to detect strong phylogenetic signal.

Another argument might be that our method avoids \emph{observer bias}. We don't have to rely on an expert picking and choosing which parts of a grammatical or lexical system are interesting and worth coding. This is described as an advantage of large-scale extraction of continuous morphological characters in biology too \autocite{wright_systematists_2019}. Another advantage of encoding structural variation with continuous characters over categorical ones: ``phylogenetic error is very high for characters with \ldots{} very high rates of evolution (due to homoplasy of changes). Continuous characters do not display this relationship as strongly due to their large state space, though more research is needed to demonstrate this effect empirically.'' \autocite{wright_bayesian_2014}. Applicable to grammatical variables in linguistic phylogenetic tree inference, which show high rates of evolution and lots of homoplasy, due at least in part to tightly contrained state space \autocite{greenhill_evolutionary_2017}. We don't have to worry about correcting for acquisition bias since the datasets reflect the full range of logically possible biphones in every language. We can include invariant sites (where all values are the same. These don't matter much for topology but are important for dating/branch lengths) and we don't need to correct for ascertainment bias \autocite{leache_short_2015}.

We use a Bayesian computational approach to infer linguistic phylogenies using BEAST phylogenetic software (v1.10.4) \autocite{suchard_bayesian_2018}. This is similar to earlier work on the Pama-Nyungan phylogeny \autocites{bowern_computational_2012}{bouckaert_origin_2018} which used BEAST2 \autocite{bouckaert_beast_2019}. We selected BEAST over BEAST2 because it offers the ability to infer trees with continuous characters. Throughout, we generally try to follow \textcite{bouckaert_origin_2018} as closely as possible. We follow \textcite{bouckaert_origin_2018} in constraining the tree topology using clade priors for well-established and commonly accepted Pama-Nyungan subgroups, as established by \textcite{ogrady_languages_1966}, \textcite{muhlhausler_atlas_1996} and \textcite{koch_languages_2014} and subsequently recovered in computational phylogenetic analysis by \textcite{bowern_computational_2012}. Dating the Pama-Nyungan tree is a central focus of \textcite{bouckaert_origin_2018}, combining lexical cognate data with geographical data and archaeological calibration points to give a best-available estimate of the geographic and temporal point of origin of the family. Accordingly, we retain their calibration prior on the Wati subgroup, which places a 95\% probability of the subgroup's origin dating between 3,000-5,000 years, with most of the probability density skewing towards the younger end of that range (a gamma distribution of \(\alpha = 2\), \(\beta = 359\), with 3,000 year offset) based on a synthesis of archaeological evidence \autocite[see][p.~746]{bouckaert_origin_2018}. We place a prior on the root age of the Pama-Nyungan family centred on a mean of 5,791 years B.P., following the findings of \textcite{bouckaert_origin_2018}. 5,791 years is the mean root age of the posterior for their best supported hypothesis on Pama-Nyungan's origins. We model this as a normal distribution (SD = 730) approximating the 95\% range of posterior root age estimates. One aspect in which we differ from \textcite{bouckaert_origin_2018} is tip dates. \textcite{bouckaert_origin_2018} use a birth-death skyline tree model which allows for tip dates to differ and includes a parameter corresponding to the proportion of total taxa sampled at a given point in time. This is reasonable since they use language sources span over 200 years. In contrast, we assume all tips are contemporaneous. In our case, since we restrict attention to relatively modern sources, any extra precision to be gained from including tip dates is not worth the reduced tree model choice in BEAST and extra computational expence.

\hypertarget{pn-tree-results}{%
\section{Results}\label{pn-tree-results}}

\hypertarget{phonotactic-evo-model}{%
\subsection{Phonotactic evolutionary model}\label{phonotactic-evo-model}}

There have been a bunch of studies using lexical cognate data and some standards are beginning to emerge, for example covarion model seems widely preferred. {[}check state of the art language comparison article{]}. However, this is to the best of our knowledge the first attempt at tree inference with binary biphone characters as we use here {[}unless Gerhart tried it?{]} so we need to do a good deal of preliminary exploration testing various prior settings to get the best supported evolutionary model and set of sensible priors.

For each model specification, 2 independent chains of 25,000,000 iterations, with parameters logged every 10,000 iterations. Log marginal likelihood is calculated using BEAST's path sampling/stepping stone sampling procedure \autocites{baele_improving_2012}{baele_accurate_2013} consisting of 50 path steps of 500,000 iterations, with parameters logged every 10,000 iterations, conducted on each chain then combined to get an overall marginal likelihood. We conducted autocorrelation and convergence checks using Tracer v1.7.1 software \autocite{rambaut_posterior_2018}. Note that the results here are a preliminary exploration of model parameters to determine the best parameter settings for the tree inference presented in Section \ref{pn-tree-combined} below. We do not anticipate that binary biphone characters will produce especially high quality or realistic language phylogenies on their own. The goal is to get a handle on how best to model the evolutionary dynamics of this dataset when used in combination with other sources of evidence.

\hypertarget{site-model}{%
\subsubsection{Site model}\label{site-model}}

We start by evaluating different site models that describe how binary biphone characters evolve through time. For this stage of evaluation, we fix the clock model to a strict clock (no variation in evolutionary rates between branches) and fix the tree model to a simple calibrated Yule tree model with a uniform birth rate prior (Yule tree models do not allow for extinction events). We then test all eight combinations of three site model parameters:

\begin{itemize}
\tightlist
\item
  A simple continuous time Markov chain (CTMC) model (which contains a single estimated parameter that specifies the frequencies with which biphones are gained and lost) versus a covarion model (which allows sites to switch between fast and slow states). The covarion model is the preferred model of lexical cognate evolution in \textcite{bouckaert_corrections_2012}, \textcite{bouckaert_origin_2018} and \textcite{kolipakam_bayesian_2018}, although \textcite[p.~219]{chang_ancestry-constrained_2015} find little difference between them and opt for the increased simplicity of the former model.
\item
  Empirical character state frequencies versus estimated character state frequencies.
\item
  Site homogeneity (fixed evolutionary rates across all character sites) versus heterogeneity (estimated using four gamma distributed categories, following \textcite{kolipakam_bayesian_2018}). For cognate data, \textcite{bouckaert_origin_2018} find a better fit with homgenous rates but \textcite{kolipakam_bayesian_2018} find a better fit with heterogenous ones.
\end{itemize}

We use Bayes factors to determine the best supported site model. Bayes factors give an indication of the support for one model over another and are calculated by calculating the ratio of the log marginal likelihoods of each model. A Bayes factor of 5 to 20 is taken as substantial support, greater than 20 as strong support, and greater than 100 as decisive \autocite{kass_bayes_1995}. We table Bayes factors comparing each combination of site model settings in Table \ref{tab:site-models}. The names of each model indicate site settings as follows: (S)imple CTMC versus (C)ovarion model, e(M)pirical versus e(S)timated character frequencies, (H)omogenous rates versus (G)amma-distributed heterogenous rates. All models contain the suffix ``-SY'' since they all contain a (S)trict clock and calibrated (Y)ule tree prior. So, for example, the model termed ``CMH'' consists of a covarion model with empirical frequencies and homogenous rates across all sites.

\begin{table}

\caption{\label{tab:site-models}Bayes factors for different site models. Each Bayes Factor represents the support for one model (listed left) against another (listed top). A positive value indicates the first model (left) is supported, and conversely, a negative value indicates the second model (top) is supported. A value over 100 is considered decisive.}
\centering
\begin{tabular}[t]{lrrrrrrrr}
\toprule
Site model & SMH-SY & SSH-SY & SMG-SY & SSG-SY & CMH-SY & CSH-SY & CMG-SY & CSG-SY\\
\midrule
SMH-SY & -- & 6 & -1,277 & -1,313 & -47,211 & -98,162 & -94,904 & -168,465\\
SSH-SY & -6 & -- & -1,283 & -1,319 & -47,217 & -98,168 & -94,910 & -168,471\\
SMG-SY & 1,277 & 1,283 & -- & -36 & -45,934 & -96,885 & -93,627 & -167,188\\
SSG-SY & 1,313 & 1,319 & 36 & -- & -45,898 & -96,849 & -93,591 & -167,152\\
\addlinespace
CMH-SY & 47,211 & 47,217 & 45,934 & 45,898 & -- & -50,951 & -47,693 & -121,254\\
CSH-SY & 98,162 & 98,168 & 96,885 & 96,849 & 50,951 & -- & 3,258 & -70,303\\
CMG-SY & 94,904 & 94,910 & 93,627 & 93,591 & 47,693 & -3,258 & -- & -73,561\\
CSG-SY & 168,465 & 168,471 & 167,188 & 167,152 & 121,254 & 70,303 & 73,561 & --\\
\bottomrule
\end{tabular}
\end{table}

The covarion model overwhelmingly outperforms the CTMC model in all instances. Furthermore, there is support for allowing evolutionary rates to vary across character sites. Unfortunately, this great increase in parameters results in a corresponding increase in computational demand. Models with heterogenous rates require 3--4 times as long as equivalent models with fixed rates. Lastly, there is decisive support for estimating character state frequencies rather than simply taking the observed frequencies when the covarion model is used, although the opposite is true with a CTMC model. A covarion model with estimated frequencies and homogenous evolutionary rates will beat a model where rates are allowed to vary but empirical frequencies are used. All up, we determine the best site model to be a covarion model with estimated frequencies and rate heterogeneity.

\emph{Note to self:} Improper {[}0,Inf{]} uniform prior would have been okay for Yule birth rate since we have node calibrations (\url{https://groups.google.com/forum/\#!topic/beast-users/H_PjNgiZMe8}). But I think {[}0,1{]} uniform prior is okay because the birth rate never gets anywhere near upper bound of 1 in the posteriors anyway. \textcite{kolipakam_bayesian_2018} uses bounded {[}0,1{]} birth rate while \textcite{bouckaert_origin_2018} uses {[}0,Inf{]}.

\emph{Optional extension to this:} Would be nice to test stochastic Dollo model, which has been implemented with some success for cognate data in linguistics (although covarion model seems to be winning these days). Stochastic Dollo only allows characters to spring into existance once and any losses are permanent. I wasn't too worried about SD because I figured it's a bit more realistic for cognates, since the state space of possible words is practically infinite (i.e.~the chance of different people inventing the same word for the same thing independently is very low, although of course it does happen sometimes)\footnote{That said, SD isn't super realistic for cognates either since it doesn't allow for borrowing, which appears as two independent origin points when plotted on a phylogenetic tree. This is likely why covarion tends to work better. As an aside, a dream phylogeographic model of cognate evolution would allow for independent points of origin with very low probability (the likelihood of chance resemblances) plus a relatively high probability of an independent point of origin springing up when a language is geographically adjacent to another where the cognate is already present (this wouldn't really reflect an independent point of origin but rather a borrowing). Computationally expensive though.}. By contrast, there are only so many possible biphone combinations, many unrelated/distantly related languages share biphones (consider, for example, shared biphones between English and Pama-Nyungan languages) and it seems fairly unreasonable to assume a single common point of origin for all of them. Nevertheless, it would nice to test to be sure. Unfortunately, I was getting some nasty errors in BEAST that seem difficult to resolve when the SD model is selected. I wasn't really worried about this because I figured covarion is likely more realistic anyway.

\hypertarget{clock-and-tree-model}{%
\subsubsection{Clock and tree model}\label{clock-and-tree-model}}

We take the best performing site model and compare it to the same model with a lognormally-distributed uncorrelated relaxed clock and a birth-death tree prior. This relaxed clock model generally has been found to outperform a strict clock when modelling lexical cognate evolution \autocites{bouckaert_origin_2018}{kolipakam_bayesian_2018}. The birth-death speciation model allows for extinction events and more closely approximates the birth-death skyline model favoured in \textcite{bouckaert_origin_2018}, although a Yule speciation model was preferred in \textcite{bowern_computational_2012} and \textcite{kolipakam_bayesian_2018}.

Bayes factors are presented in Table \ref{tab:tree-models}. The model naming convention is as above. The suffix reflects the clock and tree prior settings: (S)trict clock versus (R)elaxed clock and calibrated (Y)ule versus (B)irth-death speciation.

\emph{AWAITING RESULTS: Currently running on Awoonga computer cluster. Output expected Tue, 16 June.}

\begin{table}

\caption{\label{tab:tree-models}Comparison of models with different clock and tree settings.}
\centering
\begin{tabular}[t]{lrrrr}
\toprule
Clock/tree model & CSG-SY & CSG-RY & CSG-SB & CSG-RB\\
\midrule
CSG-SY & -- & -- & -- & --\\
CSG-RY & -- & -- & -- & --\\
\addlinespace
CSG-SB & -- & -- & -- & --\\
CSG-RB & -- & -- & -- & --\\
\bottomrule
\end{tabular}
\end{table}

\hypertarget{pn-tree-combined}{%
\subsection{Combined cognate and phonotactics tree inference}\label{pn-tree-combined}}

Evolutionary model for phonotactic frequency dataset is more straightforward. We take a standard, lightweight Brownian motion model in which frequency values can wander up or down with equal probability through time. We are limited to this model by software constraints, but that is not a major limitation at this point. Firstly, Brownian motion is a standard starting point in comparable biological studies that jointly infer trees with continuous data. Secondly, it is the same model used in \textcite{macklin-cordes_phylogenetic_2020}. One difference between \textcite{macklin-cordes_phylogenetic_2020} and this study is that \textcite{macklin-cordes_phylogenetic_2020} use raw frequency values whereas we use log-transformed frequency values. We observe that biphone transition frequencies tend to be skewed such that lexicons tend to contain relatively few high frequency biphone transitions and many low frequency transitions. It follows then that these biphone transition frequencies are more likely the outcome of an evolutionary process where characters wander along a skewed, lognormal scale than one in which they wander along a normal distribution (although, in practice, it may not matter too much. \textcite{macklin-cordes_phylogenetic_2020} find no significant difference in phylogenetic signal using raw values versus log-transformed values). These skewed distributions echo the skewed distributions of single segments observed by \textcite{macklin-cordes_re-evaluating_2020}. As \textcite{macklin-cordes_re-evaluating_2020} makes clear, this does not mean that biphone transition frequencies are necessarily drawn from a lognormal distribution and a more sophisticated maximum likelihood test would be needed to distinguish between the lognormal and several other similarly skewed distribution types. Nevertheless, the lognormal distribution is a sufficient approximation of the skewed distribution of biphone transition frequencies for our purposes in this study.

Thinking briefly about what would be a realistic model of evolution for biphone transition frequencies. We would expect there to be two main forces impacting these frequencies. The first is the introduction of new vocabulary to a language via lexical innovation or borrowing. Each new word entering a lexicon will alter minutely the frequencies of biphone transitions in the language (similarly, transition frequencies will decline as words are replaced or fall out of usage). This is the kind of gradual accumulation of changes that we might expect to follow a Brownian motion-like pattern of evolution (although maybe the rates of going up and down are not equal). Further, since speakers show a preference for high frequency phonotactic sequences over low frequency sequences when coining new words, we might expect this accumulation of changes to follow a kind of `rich get richer' process which would result in the kind of skewed frequency distributions that we observe. Also, when languages borrow vocabulary, the trend is for foreign words with dispreferred phonotactic sequences to shift towards more natively preferred patterns (sometimes gradually over a long period of time, i.e.~look at various French words in English, stress has shifted to English pattern in some but not yet in others), which would strengthen this kind of `rich get richer' process and also keep phonotactic frequency data historically conservative. The second major force on biphone frequencies is sound change. We would expect sound changes to result in sudden jumps in the frequencies of affected biphones, sometimes to 0 or 1. Our binary characters capture some of these effects to a limited extent. For example, perhaps a language has some frequency value for sequences of a nasal followed by a stop with a different place of articulation. If that nasal undergoes place assimilation, the biphone frequency will drop to 0 and thus disappears as a gap in the frequency dataset since evolutionary model requires non-zero values. On the other hand, this assimilation will be recorded in the binary data as a shift from `1' to `0'. In other instances, biphone characters may shift from missing to present and vice versa in both the frequency and binary datasets. For example, if a contrastive vowel length distinction emerges, certain biphones (namely those with long vowels) will go from being a gap in a language's biphone transition frequency data to some positive, non-missing value. In the case of a merger between short and long vowels, the opposite will be true. Our model, at present, simply does not account well for sound change. In this respect, there is an advantage to studying Australian languages, since Australian languages show uniquely constrained variation in phonological inventories {[}REFS{]} (easier to match biphones between languages, less dataset sparsity) and less history of identified sound changes relative to other parts of the world (historical linguists have long turned to sources of historical evidence in other parts of language like morphology etc. {[}REFS{]}). We return to this subject in Section \ref{pn-tree-discussion}.

For the cognate data partition, we approximate as much as possible the best supported priors from \textcite{bouckaert_origin_2018}. We use a covarion model with a relaxed clock and fixed rates across cognate classes.

\emph{RESULTS COMING SOON: Pilot runs complete and show promising results. Ready for immediate implementation when results for binary data model come through. Output expected end June.}

\hypertarget{pn-tree-discussion}{%
\section{Discussion}\label{pn-tree-discussion}}

Discussion in biology regarding combination of morphological and genomic datasets. ``Simultaneous'' approach where both morphological and genomic data are used jointly to infer the tree versus ``scaffolding'' approach where only genomic data is used to infer tree topology, then morphological data is used to assess e.g.~dating (using fossil record) while being constrained to genomic tree topology \autocite{lee_morphological_2015}. Must be aware of the potential circularity of tracing the evolution of characters on a phylogeny which was itself partly based on those characters \autocite{de_queiroz_including_1996}.

Limitations:

\begin{itemize}
\tightlist
\item
  Logical dependencies between variables (because of sound changes, phonotactic restrictions affecting natural classes)
\item
  Logical dependencies between binary/continuous partitions (non-gap in freq data = 1 in binary data. 0 in binary data = gap in freq data)
\item
  Didn't account for sound change
\item
  Limitations of Brownian motion model
\end{itemize}

If we get a negative result (no significant difference between trees inferred with/without phonotactic data partition) then I would speculate that it's probably got a lot to do with the inability of our Brownian motion evolutionary model to capture the effects of sound change, which would manifest as sudden jumps in frequencies.

If we get a positive result, then we would advocate for the use of phonotactic data in combination with other sources of evidence, such as cognate data, to infer linguistic phylogenies.

\begin{itemize}
\tightlist
\item
  Could be used to help resolve phylogenetic conflicts in places where there is more phylogenetic uncertainty. Could be used to help with dating and branch lengths in places where otherwise the topology is quite well understood.
\item
  Could help in under-resourced places that don't have as much lexical data. Studies of Pama-Nyungan phylogeny have benefitted from reasonably extensive cognate coding over nearly 300 meaning classes, but a lot of places will be limited to the scale of Swadesh lists or even less. (The opposite is true in biology, where morphological datasets make up ever shrinking proportion of total combined dataset when combined with genomic datasets that keep getting bigger)
\item
  Could be used for quick and dirty tree inference where some phylogenetic information is required/better than nothing (for example, using phylogenetic comparative methods) but doesn't necessarily have to be perfect. e.g.~could combine with very small lexical datasets/automatic cognate identification. Perhaps could be combined with, e.g.~glottolog classifications to get something consistent with glottolog tree but fully resolved.
\end{itemize}

% ***************************************************


%CHAPTER 7
% ***************************************************
% Conclusion
% ***************************************************
\chapter[Conclusion]{Conclusion}
\label{Chap:Conclusion}

% ********* Enter your text below this line: ********
Conclude your thesis.

% ***************************************************

%CHAPTER 8
% ***************************************************
% Conclusion
% ***************************************************
\chapter[Conclusion]{Conclusion}
\label{ch-conclusion}

% ********* Enter your text below this line: ********
This will be a short, sharp conclusion. No more than a few pages, briefly tying everything together. Est. 2 days' work to write and polish.

Brief summation of research question, themes and motivations (1--2pp).

\begin{itemize}
\item
  Chapter 2 summary.
\item
  Chapter 3, phoneme frequencies paper summary.
\item
  Chapter 4, PCMs in linguistics paper summary.
\item
  Chapter 5, phylogenetic signal in phonotactics paper summary.
\item
  Chapter 6, building trees with phonotactics paper summary.
\item
  Chapter 7, lessons, future directions summary.
\end{itemize}

Optional: Any other brief final thoughts.

One strong final paragraph.

% ***************************************************

% HOW TO ADD ADDITIONAL CHAPTERS
% Step One: Add a new folder called "ChapterX" (X being the chapter number).
% Step Two: Within the folder add a new .tex file by clicking the "New File" button in the Overleaf Menu. Rename the file to a title of your choice.
% Step Three: Copy the Chapter 2 headline and "\input" command located above and insert it below Chapter 2.
% Step Four: Rename the headline to your specific chapter number, change the input command to include the name of the folder you created and the name of the file you created.
% Repeat this process for every chapter.

%CONCLUSION CHAPTER
%\input{./Conclusion/Conclusion.tex}

% ***************************************************
% Bibliography
%****************************************************
%CHOOSE YOUR BIB STYLE AND FILE.
%We have included the following two referencing styles for you to use in your thesis. You can add an alternate style if you prefer.

%Style: apalike = this is an (Author, Year) referencing style similar to APA
%Style: elsarticle-num = this is a numbered referencing style that will display the bibliography in citation order

%To use one of the styles provided ensure the % is removed from the start of the line, and the other option is commented out with a % at the start of the line. The style elsarticle-num is active by default.

%\bibliographystyle{apalike}
%\bibliographystyle{elsarticle-num}

%\bibliography{./References/Bibliography}
\printbibliography

%When you have finished your thesis we recommend that you manually fix any errors in your bibliography. 
%To do this, compile, copy the .bbl into a new .tex file and include this here after commenting out the other bibliography commands. Make corrections in that .tex file.

% ***************************************************
% Appendices
%**************************************************** 
%UNCOMMENT THIS SECTION IF YOU ARE USING APPENDICES.
%Simply adapt the same formatting used for other chapters.
\appendix
% If you need appendix in your thesis then consider the following appendix file (you can add more if you need more) otherwise you should not consider it in your main thesis.
% ***************************************************
% Appendix
% ***************************************************
\chapter{Supplementary figure for Chapter 5}

Quantifying phylogenetic signal requires an independently-derived
reference phylogeny as a yardstick. Our reference phylogeny is a 285-tip
Pama-Nyungan phylogeny inferred by the second author. Figure S1 gives a
112-tip subset of this phylogeny, corresponding to the 112 doculects
used in this study.

\begin{figure}
\includegraphics[width=1\linewidth]{Appendix-A/fig/PN_beast_pruned} \caption[Pama-Nyungan reference phylogeny]{Pama-Nyungan reference phylogeny. Node labels are posterior probabilities, giving an indication of support for each node. Although there is no strict, conventional cut-off, clades with posterior values above 0.5 are considered supported and values above 0.8 are considered strongly supported (Bowern \& Atkinson 2012, p. 829).}\label{fig:plot-ref-tree2}
\end{figure}

As discussed in the main paper, the reference phylogeny was constructed
using Bayesian phylogenetic methods in the software BEAST2
\autocite{bouckaert_beast_2014}. Bayesian phylogenetic methods use a
Markov Chain Monte Carlo (MCMC) procedure to efficiently search the
hypothesis space of possible trees and return a large posterior sample
of similarly credible alternatives (capturing phylogenetic uncertainty).
The tree in Figure S1 above is a maximum clade credibility tree, which
is a summation of the posterior sample where the likelihood of all nodes
in the tree (in terms of how frequently a given node reappears across
the posterior sample) is maximized.

For further details on the Pama-Nyungan phylogeny used as a reference
phylogeny in this study, see \textcite{bowern_pama-nyungan_2015}. See
also \textcite{bowern_computational_2012}, which infers a Pama-Nyungan
phylogeny in the same way, using exactly the same evolutionary model
parameters, but with an earlier iteration of the dataset containing
fewer doculects. Additional discussion of the general process of
constructing language phylogenies in BEAST2 can be found in
\textcite{bouckaert_origin_2018}, although a different evolutionary
model is used.

The reference phylogeny was inferred using lexical cognate data, coded
according to the principles of the Comparative Method. The cognate data
used in the reference phylogeny is publicly available on Zenodo
\autocite{bowern_pama-nyungan_2018} and also as a subset of the
305-language dataset in \textcite{bouckaert_origin_2018}. This latter
source also includes a Perl script for converting multistate cognate
judgements into a binary matrix for use with BEAST2 phylogenetic
software and will include information on underlying sources.

% ***************************************************
% Appendix
% ***************************************************
\chapter{Supplementary information for Chapter 5}

\hypertarget{pama-nyungan-reference-phylogeny}{%
\subsection{Pama-Nyungan reference
phylogeny}\label{-pama-nyungan-reference-phylogeny}}

Quantifying phylogenetic signal requires an independently-derived
reference phylogeny as a yardstick. Our reference phylogeny is a 285-tip
Pama-Nyungan phylogeny inferred by the second author. Figure S1 gives a
111-tip subset of this phylogeny, corresponding to the 111 doculects
used in this study.

\begin{figure}
\includegraphics[width=1\linewidth]{Appendix-B/fig/PN_beast_pruned} \caption{Pama-Nyungan reference phylogeny. Node labels are posterior probabilities, giving an indication of support for each node. Although there is no strict, conventional cut-off, clades with posterior values above 0.5 are considered supported and values above 0.8 are considered strongly supported (Bowern \& Atkinson 2012, p.829).}\label{fig:plot-ref-tree2}
\end{figure}

As discussed in the main paper, the reference phylogeny was constructed
using Bayesian phylogenetic methods in the software BEAST2
\autocite{bouckaert_beast_2014}. Bayesian phylogenetic methods use a
Markov Chain Monte Carlo (MCMC) procedure to efficiently search the
hypothesis space of possible trees and return a large posterior sample
of similarly credible alternatives (capturing phylogenetic uncertainty).
The tree in Figure S1 above is a maximum clade credibility tree, which
is a summation of the posterior sample where the likelihood of all nodes
in the tree (in terms of how frequently a given node reappears across
the posterior sample) is maximized.

For further details on the Pama-Nyungan phylogeny used as a reference
phylogeny in this study, see \textcite{bowern_pama-nyungan_2015}. See
also \textcite{bowern_computational_2012}, which infers a Pama-Nyungan
phylogeny in the same way, using exactly the same evolutionary model
parameters, but with an earlier iteration of the dataset containing
fewer doculects. Additional discussion of the general process of
constructing language phylogenies in BEAST2 can be found in
\textcite{bouckaert_origin_2018}, although a different evolutionary
model is used.

The reference phylogeny was inferred using lexical cognate data, coded
according to the principles of the Comparative Method. The cognate data
used in the reference phylogeny is publicly available on Zenodo
\autocite{bowern_pama-nyungan_2018} and also as a subset of the
305-language dataset in \textcite{bouckaert_origin_2018}. This latter
source also includes a Perl script for converting multistate cognate
judgements into a binary matrix for use with BEAST2 phylogenetic
software and will include information on underlying sources.

\newpage

\hypertarget{phy-sig-wordlist-sources}{%
\subsection*{Wordlist sources}\label{phy-sig-wordlist-sources}}
\addcontentsline{toc}{subsection}{Wordlist sources}

The 111 wordlists used in this study are contained within the Ausphonlex
database, under development by \textcite{round_ausphon-lexicon_2017}.
All underlying wordlist data is available, either publicly in the
CHIRILA database \autocite{bowern_chirila_2016} or elsewhere in
published or archived form. A list of original sources for all wordlists
is presented below.

\textbf{Adnyamathanha}

CHIRILA source: CHIRILA/v2/McEnteeMcKenzie

\fullcite{mcentee_adna-mat-na_1992}

Phonemic normalization: Coda tap normalized as vibrant. Otherwise,
voiced stops, taps and fricatives normalized to lenis obstruents.

\textbf{Mbakwithi}

CHIRILA source: CHIRILA/v1/ASEDA0240

\fullcite{crowley_mbakwithi_1989}

\textbf{Badimaya}

\fullcite{marmion_badimaya_1995}

Phonemic normalization: Double a normalized to long vowel.

\textbf{Pakanh}

\fullcite{hamilton_pakanh_1997}

\textbf{BidyaraGungabula}

\fullcite{breen_bidyara_1973}

\textbf{Bilinarra}

\fullcite{meakins_bilinarra_2013}

\textbf{Biri}

CHIRILA source: CHIRILA/v1/Terrell

\fullcite{terrill_biri_1999}

\textbf{Bularnu}

\fullcite{breen_bularnu_1988}

\textbf{Batyala}

\fullcite{bell_sketch_2003}

\textbf{Dhangu}

\fullcite{zorc_yolngu_2004}

Phonemic normalization: Lenis retroflex stop normalized to retroflex
flap.

\textbf{Dharumbal}

CHIRILA source: CHIRILA/v2/ter02

\fullcite{terrill_dharumbal:_2002}

\textbf{Dhayyi}

\fullcite{wunungmurra_dhalwangu_1993}

Phonemic normalization: Lenis retroflex stop normalized to retroflex
flap; all other voicing is allophonic.

\textbf{Diyari}

\fullcite{austin_grammar_1981}

Phonemic normalization: Phonetic trill-released stop normalized as stop
+ trill. Otherwise, voiced stops normalized as taps.

\textbf{Djabugay}

\fullcite{robertson_jaabugay_1997}

\textbf{Djapu}

CHIRILA source: CHIRILA/v1/mor83

\fullcite{morphy_djapu_1983}

Phonemic normalization: Lenis retroflex stop normalized to retroflex
flap.

\textbf{Djinang}

CHIRILA source: CHIRILA/v1/ASEDA0009

\fullcite{waters_djinang_1988}

Phonemic normalization: Glottal closure normalized to a segment phoneme.

\textbf{Duungidjawu}

CHIRILA source: CHIRILA/v2/K\&W 04

\fullcite{kite_duungidjawu_2004}

\textbf{Dyirbal}

CHIRILA source: CHIRILA/v1/dix72

\fullcite{dixon_dyirbal_1972}

\textbf{Gamilaraay}

CHIRILA source: CHIRILA/v1/ash03

\fullcite{ash_gamilaraay_2003}

\textbf{Gangulu}

CHIRILA source: CHIRILA/v1/Terrell

\fullcite{terrill_biri_1999}

\textbf{Githabul}

CHIRILA source: CHIRILA/v1/cro78

\fullcite{crowley_middle_1978}

\textbf{GuguBadhun}

CHIRILA source: CHIRILA/v1/sut73

\fullcite{sutton_gugu-badhun_1973}

\textbf{Gumbaynggir}

\fullcite{murrbay_aboriginal_and_culture_cooperative_gumbaynggir_2001}

\textbf{Gunya}

CHIRILA source: CHIRILA/v1/dixbla81

\fullcite{breen_margany_1981}

\textbf{Gupapuyngu}

CHIRILA source: CHIRILA/v1/BL

\fullcite{lowe_temporary_1976}

\textbf{Gurindji}

\fullcite{meakins_gurindji_2013}

\textbf{GuuguYimidhirr}

\fullcite{haviland_guugu_1979}

\textbf{Guwamu}

CHIRILA source: CHIRILA/v1/Austin 1980

\fullcite{austin_guwamu_1980}

\textbf{Jaru}

\fullcite{tsunoda_jaru_1981}

Phonemic normalization: iji and uwu normalized as long high vowels.

\textbf{Jiwarli}

CHIRILA source: CHIRILA/v2/ASEDA0435

\fullcite{austin_dictionary_nodate-1}

\textbf{Kalkatungu}

CHIRILA source: CHIRILA/v2/ASEDA0205

\fullcite{blake_kalkatungu_1990}

Phonemic normalization: Double short vowels normalized as long.

\textbf{Karajarri}

\fullcite{mckelson_studies_1989}

\textbf{Kariyarra}

\fullcite{smythe_kariyarra_nodate}

\textbf{Kartujarra}

\fullcite{ogrady_gardudjarra_1988}

\textbf{KokNar}

\fullcite{sommer_koko_nodate}

\textbf{KokoBera}

\fullcite{black_kokoberrin_2007}

\textbf{KuguNganhcara}

CHIRILA source: CHIRILA/v1/ASEDA0021

\fullcite{smith_kugu_1989}

\textbf{Kukatj}

\fullcite{breen_kukatj_1991}

Phonemic normalization: Featureless vowel normalized as schwa.

\textbf{Kukatja}

CHIRILA source: CHIRILA/v1/ASEDA0504

\fullcite{peile_basic_nodate}

\textbf{KukuYalanji}

\fullcite{hershberger_kuku-yalanji_1986}

\textbf{Kurrama}

\fullcite{dench_kurrama_nodate}

\textbf{Kurtjar}

CHIRILA source: CHIRILA/v1/ASEDA0026

\fullcite{black_kurtjar_1988}

Phonemic normalization: Retroflex glide\textasciitilde{}tap normalized
as glide.

\textbf{KuukuYau}

\fullcite{thompson_sand_1988}

\textbf{Linngithigh}

\fullcite{hale_linngithigh_1999}

Phonemic normalization: Trill-released stop normalized as stop + trill.
Prenasalized stops normalized to nasal + lenis stop.

\textbf{Malgana}

\fullcite{gargett_salvage_2011}

\textbf{Malyangapa}

\fullcite{hercus_maljangapa-wadigali_1989}

\textbf{MangalaMcK}

\fullcite{mckelson_mangala_1989}

\textbf{Margany}

CHIRILA source: CHIRILA/v1/bre81

\fullcite{breen_margany_1981}

\textbf{Martuthunira}

\fullcite{dench_martuthunira_1995}

\textbf{Mirniny}

\fullcite{ogrady_mirniny_1988}

\textbf{Mudburra}

\fullcite{nash_mudburra_1988}

\textbf{Muruwari}

CHIRILA source: CHIRILA/v1/ASEDA0252

\fullcite{oates_muruwari_1992}

\textbf{Ngaanyatjarra}

\fullcite{glass_ngaanyatjarra_1988}

\textbf{Ngadjumaya}

\fullcite{wangka_maya_pilbara_aboriginal_language_centre_ngajumaya_2008}

\textbf{Ngamini}

CHIRILA source: CHIRILA/v1/brendn

\fullcite{breen_ngamini_1967}

Phonemic normalization: Phonetic trill-released stop normalized as stop
+ trill. Otherwise, voiced stops normalized as taps.

\textbf{Ngardily}

\fullcite{green_ngardily_1988}

\textbf{Ngarinyman}

\fullcite{jones_ngarinman_2005}

\textbf{Ngarla}

\fullcite{brown_ngarla-english_nodate}

\textbf{Ngarluma}

\fullcite{hale_ngarluma_1989}

\textbf{Ngawun}

CHIRILA source: CHIRILA/v2/BreenMayi

\fullcite{breen_mayi_1981}

\textbf{Nhanta}

CHIRILA source: CHIRILA/v1/ble01

\fullcite{blevins_nhanda:_2001}

\textbf{Yannhangu}

CHIRILA source: CHIRILA/v1/CB-fieldnotes

\fullcite{james_yan-nhangu_2003}

\textbf{Nhirrpi}

CHIRILA source: CHIRILA/v1/bow-nhi

\fullcite{bowern_nhirrpi_1999}

\textbf{Nukunu}

CHIRILA source: CHIRILA/v2/her92

\fullcite{hercus_nukunu_1992}

Phonemic normalization: Voiced retroflex stop normalized to retroflex
tap.

\textbf{Nyamal}

\fullcite{burgman_nyamal_2007}

\textbf{Nyangumarta}

\fullcite{geytenbeek_nyangumarta-english_1991}

\textbf{Nyawaygi}

\fullcite{dixon_nyawaygi_1983}

\textbf{Kunjen}

\fullcite{sommer_ogh_nodate-1}

\textbf{Olkola}

\fullcite{hamilton_uw_1997}

\textbf{UwOykangand}

\fullcite{hamilton_uw_1997}

\textbf{Panyjima}

\fullcite{dench_panyjima_1991-1}

\textbf{Payungu}

CHIRILA source: CHIRILA/v1/ASEDA0394

\fullcite{austin_payungu_nodate}

\textbf{PintupiLuritja}

\fullcite{hansen_pintupi/luritja_1992}

\textbf{PittaPitta}

CHIRILA source: CHIRILA/v1/bla0275

\fullcite{blake_pitta_1990}

\textbf{Purduna}

\fullcite{burgman_burduna_2007}

\textbf{Ritharrngu}

CHIRILA source: CHIRILA/v1/Heath

\fullcite{heath_ritharngu_1976}

\textbf{Paakantyi}

\fullcite{hercus_paakantyi_nodate}

\textbf{KuukThaayorre}

\fullcite{foote_kuuk_1993}

\textbf{Thalanyji}

CHIRILA source: CHIRILA/v2/ASEDA0437

\fullcite{austin_dictionary_nodate-2}

\textbf{Tharrgari}

\fullcite{austin_dictionary_1992-1}

\textbf{Thaynakwith}

\fullcite{fletcher_thanakupis_2007}

\textbf{Umpila}

\fullcite{ogrady_umpila_1988}

\textbf{Bandjalang}

CHIRILA source: CHIRILA/v1/cro78

\fullcite{crowley_middle_1978}

\textbf{WalmajarriHR}

\fullcite{hudson_walmajarri_1993}

\textbf{Wangkatja}

\fullcite{blyth_wangka_2001}

\textbf{Wangkumara}

CHIRILA source: CHIRILA/v1/robnd

\fullcite{robertson_wangkumara_1985}

Phonemic normalization: Double a normalized to long vowel.

\textbf{Warlmanpa}

\fullcite{nash_preliminary_1984}

\textbf{Warlpiri}

CHIRILA source: CHIRILA/v2/WarlpiriDict

\fullcite{schwartz_walpiri_1996}

\textbf{Warluwarra}

\fullcite{breen_warluwara_1990}

Phonemic normalization: Prenasalized stops normalized to nasal + lenis
stop. Tense glides normalized to fricatives. Tense lateral normalized to
double lateral.

\textbf{Warnman}

CHIRILA source: CHIRILA/v2/ASEDA0334

\fullcite{eidwun_warnman_nodate}

\textbf{Wargamay}

CHIRILA source: CHIRILA/v1/dixbla81

\fullcite{dixon_wargamay_1981}

\textbf{Warriyangga}

\fullcite{austin_dictionary_nodate-3}

\textbf{Wajarri}

\fullcite{mackman_wajarri_2012}

\textbf{WembaWemba}

\fullcite{hercus_wembawemba_1992}

\textbf{WesternArrarnta}

\fullcite{breen_introductory_2000}

Phonemic normalization: Labialized consonants normalized to C + w.
Prestopped nasals normalized to stop + nasal sequence. Prepalatalized
consonants normalized to j + C.

\textbf{Wakaya}

\fullcite{breen_wakaya_2006}

\textbf{WikMungkan}

\fullcite{kilham_wik_2011}

\textbf{Wirangu}

\fullcite{hercus_grammar_1999}

Phonemic normalization: Double a normalized to long vowel.

\textbf{Yadhaykenu}

\fullcite{crowley_uradhi_1983}

\textbf{Yalarnnga}

CHIRILA source: CHIRILA/v1/ASEDA0204

\fullcite{breen_yalarnnga_nodate}

\textbf{Yandruwandha}

CHIRILA source: CHIRILA/v1/breyandr

\fullcite{breen_innamincka_2004}

Phonemic normalization: Trill-released stop normalized as stop + trill.
Prestopped laterals normalized to stop + lateral sequence.

\textbf{Yanyuwa}

\fullcite{bradley_yanyuwa_nodate}

Phonemic normalization: Prenasalized stops normalized to nasal + stop
sequence.

\textbf{Yarluyandi}

CHIRILA source: CHIRILA/v1/ASEDA0251

\fullcite{hercus_yarluyandi_nodate}

\textbf{Yaygirr}

CHIRILA source: CHIRILA/v1/morelli2011

\fullcite{morelli_yaygirr_2012}

\textbf{Yidiny}

CHIRILA source: CHIRILA/v2/dix91

\fullcite{dixon_words_1991}

\textbf{Yindjibarndi}

\fullcite{anderson_yindjibarndi_nodate}

\textbf{Yinhawangka}

\fullcite{wangka_maya_pilbara_aboriginal_language_centre_yinhawangka_2008}

\textbf{YirYoront}

\fullcite{alpher_yir-yoront_1991}

\textbf{YortaYorta}

CHIRILA source: CHIRILA/v1/bowmor99

\fullcite{bowe_yorta_1999}

\textbf{Yulparija}

\fullcite{mckelson_yulparija_1989}

\textbf{Yuwaalaraay}

CHIRILA source: CHIRILA/v1/ash03

\fullcite{ash_gamilaraay_2003}

\newpage

\hypertarget{guide-to-code-and-data}{%
\subsection{Guide to code and
data}\label{guide-to-code-and-data}}

The code and data used in this study are publicly accessible on Zenodo
at \url{http://doi.org/10.5281/zenodo.3610089}. Unzipping the file
reveals a directory containing five subdirectories, \texttt{/trees},
\texttt{/data}, \texttt{/R}, \texttt{/results} and \texttt{/fig}.

The \texttt{/trees} subdirectory contains two files:
\texttt{PNY10\_285.(time).sum.tree}, which is a Nexus format tree file
for the Pama-Nyungan maximum clade credibility tree used as a reference
tree throughout the study. \texttt{PNY10\_285.trees} contains the full
posterior sample of trees and was used to check the robustness of our
results against phylogenetic uncertainty.

The \texttt{/data} subdirectory contains 9 comma-separated (csv)
spreadsheets containing all the frequency data used in the study. In all
cases, the first column lists language variety names and the first row
lists characters (variables) for analysis. The \texttt{biphone\_binary}
spreadsheet contains binary permissibility data for the \(D\) test for
phylogenetic signal. A `1' value indicates that the biphone occurs at
least once in that language variety's wordlist. A `0' indicates that the
biphone never appears in the language variety's wordlist. A missing
value (represented by `NA') is entered where one or both of the
phonological segments in the biphone is not part of the language
variety's phonological inventory and therefore would be impossible to
observe. The \texttt{biphone\_fwd} and \texttt{biphone\_bkwd}
spreadsheets give the forward and backward transition probabilities for
each language variety. Once again, missing values occur where the
language lacks entirely one of the segments in a particular biphone.
Otherwise, as discussed in the main paper body, the frequencies given
are the frequencies of the sequency \(xy\), relativised over all
instances of \(x\) (forward transition) or the frequencies of the
sequency \(xy\) relativised over all instances of \(y\) (backward
transition). The remaining spreadsheets give frequencies of transitions
between natural sound classes. Natural classes are split into three
categories, manner, place and major place. The format of the spreadsheet
filenames is
\texttt{\{class\ type\}\_\{transition\ direction\}\_\{file\ creation\ date\}.csv}.
So, for example, the spreadsheet beginning with \texttt{place\_fwd}
gives the forward transition frequencies for transitions between places
of articulation.

The \texttt{/R} subdirectory contains code used to perform the analysis
for the study and create figures for the main text of the paper. The
\texttt{analysis.R} script is written to run in \emph{R} statistical
software \autocite{r_core_team_r_2017}. To run the analysis, the first
step is to set the \emph{R} working directory to the \texttt{/R}
subdirectory. It is important to keep the file structure of the S2
directory intact, since the script requires access to the \texttt{data},
\texttt{trees} and \texttt{results} subdirectories. The first lines of
the script load all its required packages. If any packages are missing
from the machine, these will need to be installed. All packages used are
standard packages available on the CRAN network
(\url{https://cran.r-project.org}), and installation is straightforward
using the \texttt{install.packages("package-name")} command in the R
console. The script can be run from the R console using the command
\texttt{source("analysis.R")}. It has been run succcessfully
(approximately 45 minutes runtime) on a 2015 Macbook Pro with 8GB
memory, with the following R session info:

\begin{verbatim}
R version 3.6.2 (2019-12-12)
Platform: x86_64-apple-darwin15.6.0 (64-bit)
Running under: macOS Catalina 10.15.4

Matrix products: default
BLAS:   /System/Library/Frameworks/Accelerate.framework/Versions/A/Frameworks/vecLib.framework/Versions/A/libBLAS.dylib
LAPACK: /Library/Frameworks/R.framework/Versions/3.6/Resources/lib/libRlapack.dylib

locale:
[1] en_AU.UTF-8/en_AU.UTF-8/en_AU.UTF-8/C/en_AU.UTF-8/en_AU.UTF-8

attached base packages:
[1] stats     graphics  grDevices utils     datasets  methods   base     

other attached packages:
 [1] kSamples_1.2-9    SuppDists_1.1-9.5 phylosignal_1.3   phylobase_0.8.10 
 [5] rcompanion_2.3.25 picante_1.8.1     nlme_3.1-145      vegan_2.5-6      
 [9] lattice_0.20-40   permute_0.9-5     e1071_1.7-3       reshape2_1.4.3   
[13] caper_1.0.1       mvtnorm_1.1-0     MASS_7.3-51.5     ape_5.3          
[17] forcats_0.5.0     stringr_1.4.0     dplyr_0.8.4       purrr_0.3.3      
[21] readr_1.3.1       tidyr_1.0.2       tibble_2.1.3      ggplot2_3.3.0    
[25] tidyverse_1.3.0  

loaded via a namespace (and not attached):
  [1] TH.data_1.0-10     colorspace_1.4-1   deldir_0.1-25      seqinr_3.6-1      
  [5] class_7.3-15       modeltools_0.2-23  fs_1.3.2           rstudioapi_0.11   
  [9] fansi_0.4.1        lubridate_1.7.4    coin_1.3-1         xml2_1.2.2        
 [13] codetools_0.2-16   splines_3.6.2      libcoin_1.0-5      ade4_1.7-15       
 [17] jsonlite_1.6.1     broom_0.5.5        cluster_2.1.0      dbplyr_1.4.2      
 [21] shiny_1.4.0        compiler_3.6.2     httr_1.4.1         backports_1.1.5   
 [25] assertthat_0.2.1   Matrix_1.2-18      fastmap_1.0.1      lazyeval_0.2.2    
 [29] cli_2.0.2          later_1.0.0        htmltools_0.4.0    prettyunits_1.1.1 
 [33] tools_3.6.2        igraph_1.2.4.2     coda_0.19-3        gtable_0.3.0      
 [37] glue_1.3.1         gmodels_2.18.1     Rcpp_1.0.3         raster_3.0-12     
 [41] cellranger_1.1.0   vctrs_0.2.3        spdep_1.1-3        gdata_2.18.0      
 [45] lmtest_0.9-37      adephylo_1.1-11    rvest_0.3.5        mime_0.9          
 [49] lifecycle_0.1.0    gtools_3.8.1       XML_3.99-0.3       LearnBayes_2.15.1 
 [53] zoo_1.8-7          scales_1.1.0       hms_0.5.3          promises_1.1.0    
 [57] parallel_3.6.2     sandwich_2.5-1     expm_0.999-4       EMT_1.1           
 [61] stringi_1.4.6      nortest_1.0-4      boot_1.3-24        spData_0.3.3      
 [65] rlang_0.4.5        pkgconfig_2.0.3    matrixStats_0.56.0 rncl_0.8.4        
 [69] sf_0.8-1           tidyselect_1.0.0   plyr_1.8.6         magrittr_1.5      
 [73] R6_2.4.1           DescTools_0.99.34  generics_0.0.2     multcompView_0.1-8
 [77] multcomp_1.4-12    DBI_1.1.0          pillar_1.4.3       haven_2.2.0       
 [81] withr_2.1.2        mgcv_1.8-31        units_0.6-6        sp_1.4-1          
 [85] survival_3.1-8     modelr_0.1.6       crayon_1.3.4       KernSmooth_2.23-16
 [89] uuid_0.1-4         progress_1.2.2     RNeXML_2.4.3       adegenet_2.1.2    
 [93] grid_3.6.2         readxl_1.3.1       classInt_0.4-2     reprex_0.3.0      
 [97] digest_0.6.25      xtable_1.8-4       httpuv_1.5.2       stats4_3.6.2      
[101] munsell_0.5.0       
\end{verbatim}

Note that the \texttt{analysis.R} script contains the minimum script
required to reproduce the analysis and output results files in the
\texttt{/results} subdirectory. In addition, it contains a good deal of
commented-out code that can be used for basic inspection of the results
and production of summary statistics. This code can be uncommented or
copied into the R console at user discretion. Runtime of the extra code
is minimal, though it will produce a much more verbose output in R's
console if run all at once.

The R script \texttt{modified\_caper\_funcs.R} contains some
minimally-modified versions of functions in the \texttt{caper} package
that are used in the \(D\) test for phylogenetic signal in binary data.
They have been tweaked to improve vectorisation of the original
functions (in order to run the test over a large series of characters
rather than a single character at a time). This script is read by the
\texttt{analysis.R} script. Nothing needs to be done directly in the R
console.

The script \texttt{tree\_uncertainty.R} contains code for replicating
part of the study (place and manner sound class characters) on a
100-tree subset of the posterior sample contained in
\texttt{PN\_SDollo.nex}. Its runtime is around 9.5 hours on the same
machine described above. The script
\texttt{wordlist\_size\_uncertainty.R} contains code for replicating the
same part of the study on two subsets of languages: The middle 50\% of
wordlists when ranked by size and every 2nd wordlist when ranked by
size. The runtime for this script is around 20 minutes. Note that due to
random permutations in the methodology, exact replication is only
possible if a random seed is set. To ensure the seed is set at the
correct time, each script should be run in a clean R session.
Alternatively, the analysis can be reproduced with new random numbers by
changing or removing the \texttt{set.seed} command. Although we expect
the overall results of the study to remain the same, there will be
slight differences in values that rely on stochastic processes for their
calculation (for example, \(p\) values that are calculated via
bootstrapping).

Finally, the \texttt{create\_figs.R} script contains code used to
produce figures for the main text body. The script saves each figure as
a PDF file in the \texttt{/fig} subdirectory. Figures are produced with
the \texttt{ggplot2} package, using the system of visualisation
described by \textcite{wilkinson_grammar_2005}.

The \texttt{/results} subdirectory contains original csv spreadsheets of
results, generated as output from \texttt{analysis.R} and used in the
study. Results from the \(D\) test for phylogenetic signal in binary
phonotactic data are contained in the spreadsheet
\texttt{D\_test\_results\_2020-06-06.csv}. The \texttt{biphone} column
lists the biphone character tested. Note that the underscore in each
biphone label is purely to aid readability and carries no linguistic
meaning (it is not intended to look like the environment of a generative
phonological rule). The \texttt{languages} column gives the number of
languages with non-missing values for which phylogenetic signal was
tested for that particular character. \texttt{count\_0} and
\texttt{count\_1} columns give the number of observed 0 values and 1
values respectively. \texttt{D} gives the observed \(D\) statistic for
each character and the \texttt{pval\_0} and \texttt{pval\_1} columns
give the uncorrected \(p\) values for the two null hypotheses that
\(D = 0\) and \(D = 1\) respectively. The columns \texttt{pval0\_sig}
and \texttt{pval1\_sig} give text descriptions interpreting the
significance of the \(p\) values. If a \(p\) value is small, the
character will be either significantly more clumped or significantly
more dispersed than the null hypothesis. Otherwise, the character will
be consistent with the null hypothesis, either the phylogenetic null
hypothesis (\(D = 0\)) or randomness null hypothesis (\(D = 1\)).
Finally, the \texttt{result} column gives an overall interpretation of
the result of the \(D\) test and two accompanying \(p\) values. There
are six possible categories a character may fall into:

\begin{itemize}
\item
  \begin{enumerate}
  \def\labelenumi{(\roman{enumi})}
  \tightlist
  \item
    More clumped than the phylogenetic null hypothesis, listed as
    \texttt{more\ clumped}.
  \end{enumerate}
\item
  \begin{enumerate}
  \def\labelenumi{(\roman{enumi})}
  \setcounter{enumi}{1}
  \tightlist
  \item
    consistent with the phylogenetic null hypothesis and more clumped
    than the random null hypothesis (i.e.~there is significant
    phylogenetic signal), listed as \texttt{phylogenetic}.
  \end{enumerate}
\item
  \begin{enumerate}
  \def\labelenumi{(\roman{enumi})}
  \setcounter{enumi}{2}
  \tightlist
  \item
    consistent with both null hypotheses, so a result cannot be
    determined either way, listed as
    \texttt{indeterminate\ (neither\ H0\ rejected)}.
  \end{enumerate}
\item
  \begin{enumerate}
  \def\labelenumi{(\roman{enumi})}
  \setcounter{enumi}{3}
  \tightlist
  \item
    inconsistent with both null hypotheses, and \(0 < D < 1\), listed as
    \texttt{0\ \textless{}\ D\ \textless{}\ 1\ (both\ H0s\ rejected)}.
  \end{enumerate}
\item
  \begin{enumerate}
  \def\labelenumi{(\alph{enumi})}
  \setcounter{enumi}{21}
  \tightlist
  \item
    consistent with the randomness null hypothesis and more dispersed
    than the phylogenetic null hypothesis, listed as \texttt{random}.
  \end{enumerate}
\end{itemize}

The sixth possible category a character can fall into, which is written
into the \texttt{analysis.R} script, is: more dispersed than the
randomness null hypothesis, listed as \texttt{more\ dispersed}. However,
in this study, we find no characters that fall into this category and
thus it does not appear in the spreadsheet of results.

Results files for the \(K\) tests are given in spreadsheets named
according to the format
\texttt{K\_\{character\ type\}\_\{transition\ direction\}\_results\_\{date\ of\ analysis\}\_.csv}.
There are four spreadsheets of results for biphone characters: Original
forward and backward transition frequencies, and normalised forward and
backward transition frequencies. For natural class-based characters,
there are two spreadsheets corresponding to forward and backward
transition frequencies for each of three class types: manner, place and
major place.

\newpage

\section*{References}

\printbibliography[keyword=inA2,heading=none]


% ***************************************************
% Appendix
% ***************************************************
\chapter{Supplementary information for Chapter 6}

To do.



% ***************************************************
% Examples
%**************************************************** 
% The following files are only for examples and you should not include them in your final thesis.
%% ***************************************************
% Example of Citations
% ***************************************************
%This example is provided for your reference only. DO NOT INCLUDE IN YOUR FINAL THESIS. 
\chapter{Example of Citations}

This text is only for Bibliography testing purposes. This is a book by Cawvey et al.~\cite{Cawvey2017}. There are few journal articles~\cite{Pakzad2018, Adachi2007} in this bibliography. This is a proceeding paper~\cite{J.Fenwick2010}.  This is a PhD thesis by Peerling~\cite{PeerlingPhD1999}. These are few other types of citations~\cite{Panis2004}.
%% ***************************************************
% Example of Code; how to import Code in pdf.
% ***************************************************
%This example is provided for your reference only. DO NOT INCLUDE IN YOUR FINAL THESIS. 
\chapter{Example of Code}

\section{Find the greatest number from a list of numbers in \textit{Python}}

\begin{verbatim}

a=[1,2,3,4,6,7,99,88,999]
    max= 0
    for i in a:
        if i > max:
            max=i
    print(max)

\end{verbatim}
%% ***************************************************
% Example of Equations
% ***************************************************
%This example is provided for your reference only. DO NOT INCLUDE IN YOUR FINAL THESIS. 
\chapter{Example of Equations}

\begin{equation}
    E = mc^2
\end{equation}


\begin{align}
    a = {}& b + c\\
    x = {}& y + z
\end{align}


\begin{equation}
    \begin{split}
        a = {}& b + c\\
            {}& + d + e
    \end{split} 
\nonumber
\end{equation}
    
    
\begin{equation} 
sinx+cosx=1 
\end{equation}



%% ***************************************************
% Example of Figures
% ***************************************************
%This example is provided for your reference only. DO NOT INCLUDE IN YOUR FINAL THESIS. 

%\begin{figure} : If you put no command after \begin{table} then this figure will be printed anywhere in your pdf where LaTex finds the free space to put it. If you wish to specify where the figure goes you have to give a command in LaTex after  \begin{figure}. 
%
%There are different commands to put the figure in different positions e.g. \begin{table}[h] LaTex will print the figure in the same position that you put it in your source file, if there is enough space to print it
%N.B. it is important to note that commanding LaTeX to add a figure in a specific place may result in formatting issues.

\chapter{Example of Figures}

\begin{figure}[h]
\begin{center}
\includegraphics[width=1.0\textwidth]{Examples/FigureUQ}
\caption{The University Of Queensland}
\label{Fig:1}
\end{center}
\end{figure}

The Figure~\ref{Fig:1} represents beauty of the UQ campus.
%% ***************************************************
% Example of Flow Charts
% ***************************************************
%This example is provided for your reference only. DO NOT INCLUDE IN YOUR FINAL THESIS. 
\chapter{Example of Flow Charts}
\begin{figure}[h]
\caption{Flow Chart}
\label{Fig:FlowChart}
\begin{center}
\tikzstyle{decision} = [diamond, draw, fill=white!10, 
    text width=4.5em, text badly centered, node distance=3cm, inner sep=0pt]
\tikzstyle{block} = [rectangle, draw, fill=white!20, 
    text width=5em, text centered, rounded corners, minimum height=4em]
\tikzstyle{line} = [draw, -latex']
\tikzstyle{cloud} = [draw, ellipse,fill=white!20, node distance=3cm,
    minimum height=2em]
    
\begin{tikzpicture}[node distance = 2cm, auto]
    % Place nodes
    \node [block] (Step1) {initialize model};
    \node [cloud, left of=Step1] (expert) {expert};
    \node [cloud, right of=Step1] (system) {system};
    \node [block, below of=Step1] (identify) {identify candidate models};
    \node [block, below of=identify] (evaluate) {evaluate candidate models};
    \node [block, left of=evaluate, node distance=3cm] (update) {update model};
    \node [decision, below of=evaluate] (decide) {is best candidate better?};
    \node [block, below of=decide, node distance=3cm] (stop) {stop};
    % Draw edges
    \path [line] (Step1) -- (identify);
    \path [line] (identify) -- (evaluate);
    \path [line] (evaluate) -- (decide);
    \path [line] (decide) -| node [near start] {yes} (update);
    \path [line] (update) |- (identify);
    \path [line] (decide) -- node {no}(stop);
    \path [line,dashed] (expert) -- (Step1);
    \path [line,dashed] (system) -- (Step1);
    \path [line,dashed] (system) |- (evaluate);
\end{tikzpicture}
\end{center}
\end{figure}

Flow chart~\ref{Fig:FlowChart} is a simple example.
%% ***************************************************
% Example of Tables
% ***************************************************
%This example is provided for your reference only. DO NOT INCLUDE IN YOUR FINAL THESIS. 
\chapter{Example of Tables}

Here is a really simple table~\ref{Table}.


%\begin{table} : If you put no command after \begin{table} then this table will be printed anywhere in your pdf where LaTex finds the free space to put it. If you wish to specify where the table goes you have to give a command in LaTex after  \begin{table}. 
%
%There are different commands to put the table in different positions e.g. \begin{table}[h] LaTex will print the table in the same position that you put it in your source file, if there is enough space to print it
%N.B. it is important to note that commanding LaTeX to add a table in a specific place may result in formatting issues.

\begin{table}[h]
\caption{Name of the Australian Cities}
\begin{center}
\begin{tabular}{{|c|c|c|}}
\hline
\textbf{Number}& \textbf {Name}\\
\hline
      1& Brisbane\\
      2& Sydney\\
      3& Melbourne\\
      4& Canberra\\
      5& Perth\\
      6& Adelaide\\
      7& Hobart\\
      8& Darwin\\
\hline
\end{tabular}
\end{center}
\label{Table}
\end{table}


% ***************************************************
% Back Matter
%**************************************************** 
%COMMENT OUT IF YOU DO NOT WISH TO INCLUDE BACK MATTER.
%% ***************************************************
% Back Matter
% ***************************************************
% ADD AN ENDQUOTE HERE. If you do not wish to, delete this file.
\backmatter

\normalfont
\cleartooddpage

\pagestyle{empty}

\begin{table}[b!]
\begin{center}
% ********* Enter your quote within {} brackets: ********
\textit{Endquote goes here.}

% ********************************************************
\end{center}
\begin{flushright}
% ********* Enter your text below, as indicated: ********
Author of quote,\\
Source of quote

% ********************************************************
\end{flushright}
\end{table}

\end{document}
