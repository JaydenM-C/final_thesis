\chapter[Literature review]{Literature Review}
\label{Chap:lit-review}

% ***************************************************
% Literature Review
% ***************************************************

This chapter surveys existing literature, firstly to illuminate the motivation for this thesis and position the research questions of the following chapters in their academic context, and secondly to situate this study in a broader context of deep-time human and linguistic history in the continent of Sahul. The chapter proceeds in two parts, addressing each of these goals in turn. Section \ref{academic-context} starts by tracing the origin and development of historical linguistics as a field. Particular focus is given to rapid developments in the past two decades, which increasingly enable integration of historical linguistics with other aspects of human history, rapidly shifting the kinds of questions that can be investigated with comparative linguistic data. This topic is subsequently taken up again in Chapter \textbf{CHAP REF} with a particular concentration on the fundamental task of comparison in historical linguistics and beyond. With this context in mind, Section \ref{sahul-context} turns to a particular world domain, namely the continent of Sahul. The section begins with an overview of the geospatial and ecological context of Sahul and our current best understanding of the earliest human history in the continent. The section proceeds with a picture of the current linguistic diversity in Sahul and the historical linguistic hypothesis posited for how the present linguistic ecology came to be. I then discuss how the current historical linguistic picture reconciles with our general understanding of human history in Sahul as discussed previously. Finally, a general phonological profile of the languages of Sahul is provided. As the studies presented in the following chapters use data from Australian languages only, Australian languages will be the focus of these sections of the literature review. Nevertheless, a brief overview of languages in New Guinea is provided as well in order to situate discussion of future research directions in Chapter \textbf{CHAP REF}

\hypertarget{academic-context}{%
\section{Linguistic history, comparison and evolution}\label{academic-context}}

\hypertarget{phylogenetic-thinking-in-historical-linguistics}{%
\subsection{Phylogenetic thinking in historical linguistics}\label{phylogenetic-thinking-in-historical-linguistics}}

Charting patterns of descent through time is a task common to several fields across the biological and cultural sciences. Occasionally, this is possible through direct observation (e.g.~examination of historical or archaeological records) or controlled laboratory experiments (e.g.~observing microbic evolutionary processes in a lab setting). Often, however, questions relate to a historical past that cannot be observed directly (without time travel, at least). In these cases, the best that can be done is indirect inference. The main method of historical inference is comparison. Extant entities, be they organisms, langauges, populations etc., are compared and historical relationships between entities are inferred. To this end, a countless array of comparative data types and theoretical models have been employed, summoning what knowledge we have about how historical processes operate. Some studies may have the benefit of a partial historical record as a scaffold (e.g.~biological data from modern species combined with a partial fossil record). Indirect historical inference via comparison is particularly familiar in linguistics, given the ephemeral nature of spoken language. Clearly the global body of written language is extensive. But even where written language extends back farthest, it remains a relatively recent human invention, covering perhaps 10\% of the timespan of human language. In many parts of the world, including all of Sahul, written language has only been introduced in the past few centuries, and there remain some languages with no written records at all. Likewise, audio and audio-visual records of human language extend back only 150 years or so.

Phylogenetics is a particular subtype of historical inference, focussing on questions of phylogeny. That is, the tree-like evolutionary relationships that develop from the assumption that some entites are descended from a shared common ancestor. Biologists have charted patterns of descent in phylogenetic trees since Darwin, and so to have linguists for languages for approximately as long. A phylogenetic tree is not the only way schematise historical relationships, nor is a phylogenetic schematisation necessarily mutually exclusive to other, non-tree-like historical approaches. It has, however, proved a highly successful tool for furthering understanding of all sorts of historical questions for over a century and a half.

In the last two decades, as computational phylogenetic methods have entered the historical linguistic mainstream, there has been a resurgence of discussion on the supposed analogies between linguistic evolution and biological evolution \autocite[e.g.][]{atkinson_curious_2005}. But how analogous is linguistic evolution to biological evolution really? And can modern computational phylogenetic methods in biology be applied to linguistics through analogy? Here, I argue for continued uptake and development of computational phylogenetic methods in linguistics but through a shifted paradigm that focusses less on direct analogy between linguistic and biological evolution and instead concentrates on the underlying mathematical elements that may or may not be common to both.

Discussions comparing biological evolution to linguistic evolution typically begin with Darwin, who seems to be the first to propose an explicit analogy between biological evolution and linguistic evolution \autocite[p.~422]{darwin_origin_1859}. Later, \textcite[p.~57]{darwin_descent_1871} states that ``the formation of different languages and of distinct species, and the proofs that both have been developed through a gradual process, are curiously the same.'' And this oft-quoted passage is a common starting point in discussion of linguistic and biological evolution today \autocites[e.g.][]{atkinson_curious_2005}{bromham_curiously_2017}{mesoudi_pursuing_2017} In a footnote to this passage, Darwin refers to \textcite[ch.~13]{lyell_geological_1863}, who compares the descent of languages from generation to generation, with modification from processes such as lexical innovation, to ``the force of inheritance in the organic world'' \autocite[p.~457]{lyell_geological_1863}. Phylogenetic concepts were being actively developed in linguistics around the same time. \textcite{schleicher_ersten_1853} is credited with creating the earliest phylogenetic tree-like model of languages some years prior to Darwin's famous tree illustrations in \emph{The Origin of Species}, and highlighted the similarity of his earlier `Stammbaum' model to Darwin's trees \autocite{schleicher_darwinsche_1863}.

This early period of relatively shared phylogenetic thinking subsequently diverged into two quite independent streams of thought in the 20th century. Although biology and linguistics share more similarities in their methodological histories than is perhaps appreciated (as I discuss in Chapter \textbf{CHAP REF}), rapid advances in genetics and biology's subsequent quantitative turn did lead to a considerable gulf developing between the fields in the second half of the 20th century in particular. With new-found access to large volumes of genetic data and expanding computational capacity, biologists increasingly turned to quantitative algorithms to infer phylogenetic relationships and test evolutionary hypotheses \autocite{atkinson_curious_2005}. In contrast, historical linguistics tended to remain reliant on manual, expert linguistic judgements for data acquisition \autocite{nunn_comparative_2011}. The linguistic \emph{Comparative Method}\footnote{Throughout, I capitalise the term `Comparative Method' to distinguish the specific historical linguistic methodology that goes by this name from the generic `comparative method', which could refer to any of the various methodologies in comparative fields of science.} remained the methodological bedrock of historical linguistics throughout the entire 20th century and arguably remains the `gold standard' of the field today \autocites{chang_ancestry-constrained_2015}{bouckaert_origin_2018}{kolipakam_bayesian_2018}. Following \textcite{thomason_language_1988} and \textcite{campbell_historical_2004}, the Comparative Method can be summarised briefly as follows. Firstly, sets of likely cognate word forms are assembled. First pass cognacy judgements will likely be based on a fairly superficial level of similarity, with subsequent refinement later on. Known borrowings and other chance resemblances should be removed as much as possible. From these cognate sets, it is possible to identify sound correspondences between languages (i.e.~where segment \(x\) in language A and segment \(y\) in language B consistently reoccur in the same context). Proto-phonemes can be deduced by reconciling sound correspondences with sound change processes (e.g.~various split and merger processes, lenition, elision etc.) and, ultimately, a lexicon of proto-word forms can be established. Whether or not it is explicitly stated, reconstruction of proto-phonemes and proto-word forms effectively involves the assumption of a kind of \emph{weighted parsimony} evolutionary model. That is, the most parsimonious reconstructions are favoured, both in terms of the absolute number of hypothesised sound changes and the phisiological and typological plausibility of those sound changes (e.g.~the lenition of a voiceless stop to its voiced counterpart is a highly common, plausible sound change across the world, but a sound change from \emph{*t} to \emph{k}, although famously attested in several Polynesian languages \autocite{blust_t_2004}, is rare and unusual). The Comparative Method has been applied successfully the world over and boasts a robust track record over a century and a half. However, it remains, for the most part, a time-consuming manual process that requires considerable expertise in the particular languages being studied. There has been a substantial amount of development on automating step 1---the assemblage of candidate cognate sets on the basis of similarity \autocites{list_potential_2017}{rama_are_2018}{list_sequence_2018}. But identification of spurious matches due to borrowing and missed matches due to semantic shifts still requires manual intervention, ideally with localised knowledge of trade and contact and so on. With regards to sound change identification, there is some work towards automation in this space as well \autocites{steiner_pipeline_2011}{brown_sound_2013}{bouchard-cote_automated_2013}{hruschka_detecting_2015}, but there is, as yet, no commonly accepted quantitative standard giving the likelihood of sound changes in the world's languages and it seems fair to say that this area of research remains in a relatively youthful stage of development. There is a phylogenetic outcome to the Comparative Method. To classify languages into families and subgroups (or classify an individual language as a family-level or subgroup-level isolate on its own), optionally with higher-level phyla or various intermediate groupings, is to define a phylogenetic tree structure. The difference between these kinds of more traditional language classifications via the comparative method and a fully bifurcating phylogenetic tree, as is familiar in biology or computational phylogenetic studies such as \textcite{bowern_computational_2012}, is that more traditional linguistic classifications only infer branching events at particular levels (the level of the family, the subgroup, etc.) and lack the detailed internal structure of a fully resovled, bifurcating phylogeny. Accordingly, multiple languages will branch out from a single common ancestor node in a rake-like pattern. The tree structure in the Glottolog database gives a good illustration of this \autocite{hammarstrom_glottolog_2020}.

One early, quantitative departure from the Comparative Method in historical linguistics is the method of \emph{lexicostatistics}, developed by \textcite{swadesh_lexico-statistic_1952}. The lexicostatistical method involves the inference of historical relatedness by identifying cognates and quantifying pairwise percentages of shared vocabulary between languages. Simple clustering algorithms are then used to identify families and subgroups. \textcite{swadesh_towards_1955} later presented a method termed \emph{glottochronology} for dating linguistic divergence events. Glottochronology effectively involves an evolutionary \emph{strict clock} model, which assumes a constant, universal rate of lexical replacement. By assuming a particular rate of lexical replacement, phylogenetic branch lengths can be calculated and dates can be calibrated. Glottochronology and the lexicostatistical approach fell into disfavour due to a rejection of Swadesh's \emph{universal constant} theory of lexical replacement \autocite{blust_why_2000} and the unrealistic assumption that similarity necessarily equals relatedness \autocite{bowern_computational_2018}. Another weakness is an inability to quantify uncertainty in the results \autocite{atkinson_curious_2005}. Nevertheless, the lexicostatistical method did provide some utility in Australia, by enabling a rough, overall classification of language families and subgroups that could serve as a preliminary sketch until such time that more robust, future comparative work could be conducted. This seems to be the spirit in which \textcite{ogrady_languages_1966} present their lexicostatistical analysis of Australia.

The turn of the 21st century brought with it the beginnings of a new quantitative era in historical linguistics, with a surge of interest in adapting computational phylogenetic methods from evolutionary biology to infer answers to linguistic questions. Two decades into the century, these methods are increasingly commonplace in the mainstream historical linguistics scene (\textbf{ref to ICHL workshop?}) and \emph{linguistic phylogenetics} arguably constitutes a subfield of study in its own right. This surge towards computational phylogenetics has been driven by a multitude of factors. Some factors are technological. Computational resources have increased in power and also decreased in cost, leading to greater accessibility. Improvements in Internet connectivity and greater awareness of open science principles have led to a proliferation of large scale, open datasets. An additional motivation is the desire for greater empirical rigour in the field (e.g.~testability, quantification of uncertainty, reproducibility) \autocites{atkinson_curious_2005}{mcmahon_finding_2003}{nunn_comparative_2011}.

Much of linguistic phylogenetic work has focused on the fundamental task of phylogenetic tree inference. Several early studies focused on tree inference in the relatively well-studied confines of Indo-European \autocites{gray_language-tree_2003}{atkinson_words_2005}{nicholls_dated_2008} followed later by \textcite{ryder_missing_2011}, \textcite{bouckaert_mapping_2012} and \textcite{chang_ancestry-constrained_2015}. Other early examples concern Austronesian \autocite{gray_language_2000} and Bantu \autocites{holden_bantu_2002}{holden_rapid_2006}. Computational phylogenetic tree inference remains a priority. More recent studies have expanded coverage to the Aslian \autocite{dunn_aslian_2011}, Arawak \autocite{walker_bayesian_2011}, Pama-Nyungan \autocites{bowern_computational_2012}{bouckaert_origin_2018}, Tupí-Guaraní \autocite{michael_bayesian_2015}, Dravidian \autocite{kolipakam_bayesian_2018}, Sino-Tibetan \autocite{sagart_dated_2019}, Dene-Yeniseian \autocites{sicoli_linguistic_2014}[but c.f.][]{yanovich_phylogenetic_2020} and Turkic \autocite{savelyev_bayesian_2020} families (not an exhaustive list). Presently, the maturation of linguistic phylogenetics and of a broader field of evolutionary linguistics generally \autocites{dediu_language_2016}{nolle_language_2020} has brought a widened focus on the diversity of linguistic hypotheses that can be tested with an arsenal of computational phylogenetic methods. Typological features can be studied with the use of a phylogenetic tree and phylogenetic comparative methods \autocite[e.g.][]{dunn_evolved_2011} and this will be the topic of Chapter \textbf{CHAP REF}. Linguistic phylogenetic trees and phylogenetic methods can be combined with other kinds of data (such as geospatial information) and evidence from other fields such as archaeology to investigate ancient human migration movements \autocites[e.g.][]{gray_language_2009}{bouckaert_mapping_2012}{bouckaert_origin_2018}. Other studies investigate evolutionary questions in a computational way, without explicitly including phylogenetic trees or methods, e.g.~the evolution of sound systems with respect to ancient changes in human physiology \autocite{blasi_human_2019}, the evolution of sound systems with respect to ecological environment \autocite{everett_climate_2015}, sound-meaning association biases \autocite{blasi_soundmeaning_2016}, and emotion semantics \autocite{jackson_emotion_2019}. Although one can draw a distinction between studies of language evolution and linguistic phylogenetics, or as \textcite{haspelmath_human_2020} puts it, the ``evolution of linguisticality'' (i.e.~the biological capacity for language) and the ``evolution of languages'', the two can also be tied and can inform one another \autocites{blasi_soundmeaning_2016}{nolle_language_2020}. On this basis, \textcite{segovia-martin_eco-evo-devo_2020} argues for a unified \emph{eco-evo-devo} framework.

The uptake of computational phylogenetic methods in linguistics has not been accepted uncritically in all quarters. \textcite[p.~520]{atkinson_curious_2005} reports a ``curious'' aversion to phylogenetic methods in historical linguistics, speculating that perhaps some are ``haunted \ldots{} by the ghost of glottochronology past'' (see below). \textcite{bowern_computational_2018} groups criticism of phylogenetic methods in linguistics into three categories. The first of these is the contention that an analogy between linguistic evolution and biological evolution is invalid; languages do not evolve like species and, therefore, methods for inferring the evolution of species cannot be applied justly to linguistic data \autocites{andersen_synchrony_2006}{blench_new_2015}\footnote{See also Alexander Lehrman, quoted by \textcite[p.~1326]{balter_search_2004}.}. It is true, of course, that langauges do not evolve in perfectly analogous ways to species and many phylogenetic methods in biology carry in-built assumptions that do not hold in linguistics. In Chapter \textbf{CHAP REF}, however, I make the case that many of the supposed incompatibilities of linguistic data with phylogenetic methods, e.g.~horizontal transmission in language contact situations, non-independence between correlated or logically dependent features, and uncertainty around clock models or various other parameters, are not as unique to linguistics as often assumed and there is often either existing literature or active development on addressing particular challenges applicable to linguistics without abandoning phylogenetics or quantitative methods all together. It is useful to take a step back from direct analogising between linguistics and evolutionary biology, and instead focus on selecting the right tools for the right job in terms of linking historical processes to basic mathematical functions. Naturally, there will still be limitations and uncertainty but these can be made explicit and evaluated empirically. I intend on demonstrating this process in the remaining chapters of this thesis.

A second line of criticism relates to the validity of applying quantitative methods to linguistic data, stemming from (arguably erroneous) associations to lexicostatistics and glottochronology \autocites{eska_recent_2004}{holm_new_2007}. The problems with lexicostatistics and glottochronology are well described and these methods have been discredited for several decades \autocite[see][pp.~285--286]{bowern_computational_2018}. The history of computational phylogenetic methods, though, is essentially independent from lexicostatistics and glottochronology and the methods share little in common besides their generally quantitative nature. If lexicostatistics and glottochronology are rejected, it does not follow that computational phylogenetic methods must be rejected as well. To give some specific examples, computational phylogenetic methods need not carry the assumption that there is a constant rate of lexical replacement, as assumed in glottochronology. An array of clock models exist. These can allow, for example, different parts of the lexicon to evolve at faster or slower rates, and different rates of change in different subparts of a phylogenetic tree. Further, different clock models can be evaluated and compared empirically. Furthermore, phylogenetic methods rely on vastly more sophisticated evolutionary models to determine branching events, and do not simply cluster languages based on a similarity metric.

A third line of criticism concerns a reliance on lexical data with phylogenetic methods, rather than reconstructed sound changes or morphological changes. Despite some efforts to infer phylogenies with datasets of various typological features \autocite[e.g.][]{dunn_structural_2005}, the vast bulk of phylogenetic studies in linguistics do rely on lexical cognate data. The extent to which this is a problem is unclear. \textcite{bowern_computational_2018} argues that there is no evidence to suggest that lexical data are subject to an especially different evolutionary process than other parts of language. She also questions the primacy of sound change evidence over evidence from other parts of language in historical linguistics, and points out that lexical data and sound change data are not completely independent, since sound change reconstruction relies on the assemblage of lexical cognate data in the first place. I would argue that a priority for the methodological development of linguistic phylogenetics is to evaluate empirically the evolutionary dynamics of linguistic features beyond lexical data. \textcite{greenhill_evolutionary_2017} give a particularly clear-eyed example of this, evaluating evolutionary rates of change of grammatical features versus lexical ones and finding, contrary to prior expectations, that the grammatical features evolved faster. In a similar vein, I present an evaluation of the phylogenetics of phonotactic data in Chapter \textbf{CHAP REF}.

In addition to the threads of criticism just discussed, there are a couple of additional points of criticism to address. One particularly reoccurring point is the apparent limitation or inadequacy of a family-tree-based model which only shows vertical patterns of inheritance and not horizontal diffusion through, for example, linguistic contact and borrowing \autocites{bateman_speaking_1990}{donohue_new_2012}{gould_urchin_1987}. Countering this concern, \textcite{bowern_historical_2010} claims that this is confusing the family tree, which is essentially a visualization tool, with the methods themselves. Further, while there is still much work to be done (and not only in linguistics), phylogenetic methods are capable of modelling assumptions about horizontal admixture, not to mention quantifying uncertainty. In addition, \textcite{greenhill_does_2009} conduct a simulation study to test whether phylogenetic methods are invalidated by horizontal diffusion. They find that, to the contrary, their Bayesian phylogenetic method of choice is quite robust to areal borrowing between languages (albeit with some caveats relating to dating phylogenetic branches). Robustness to borrowing is also extensively evaluated in \textcite{bouckaert_origin_2018}.

An additional point relates to reduction of hugely complex linguistic systems into neat datasets of numerical or binarized characters for phylogenetic analysis, without excessively compromising the dataset's ability to be meaningful and informative in a linguistic sense. Further, there is the question of whether linguistic variables can be assumed to be independent and equivalent to the degree required of phylogenetic algorithms, given the complex interplay between linguistic features and non-random, directional patterns of linguistic change that can be observed \autocite{heggarty_interdisciplinary_2006}. I would argue that these are not insurmountable issues. Reducing complex information systems into statistical models is a challenge for all fields and this does not uniquely invalidate phylogenetic methods in linguistics. These are, however, exactly the kinds of questions that linguistic phylogeneticists should consider. A lack of independence between phonotactic characters is a genuine limitation of the studies presented in the following chapters of this thesis. I return to this topic in the Discussion chapter.

Lastly before moving on, I will make the point that, although the Comparative Method carries an unparalleled legacy of success in historical linguistics, no method is perfect and the Comparative Method has its own limitations and flaws \autocite[see][]{durie_comparative_1996}. The neogrammarian assumption of the regularity of sound change, which is a central assumption of the Comparative Method, is an assumption about an evolutionary process that has been subject to some debate and apparent exceptions {[}REFS{]}. Furthermore, the Comparative Method can lack transparency. Analytical decisions during cognate identification can be opaque and difficult for others to evaluate without an extensive background knowledge in the languages being studied. There is presently no straightforward way to incorporate uncertainty into a manual analysis using the Comparative Method. Complete accuracy with regards to identifying borrowed lexicon cannot be guaranteed. In contrast, phylogenetic model choices can be declared explicitly for evaluation by others and statistical uncertainty in the resutls can be quantified. This is not to make the case that the Comparative Method is flawed or that computational phylogenetic methods are superior (I would argue they serve varied functions and can coexist, rather than stand as direct competitors), only that no method is totally free of assumptions, limitations or problematic aspects, and this should be kept in mind before dismissing any particular method outright.

To summarise this section, analogies have been drawn between linguistic evolution and biological evolution, in both directions, for practically as long as phylogenetic tree diagrams have existed. The early history of these academic fields in particular is somewhat intertwined. Although methods in biology and linguistics diverged substantially in the 20th century, phylogenetic thinking, in essence, never really left historical linguistics. The Comparative Method infers patterns of the fundamental evolutionary process of descent with modification and classifies languages in a phylogenetic structure. In recent decades, historical linguistics has taken a quantitative turn and embraced some of the computational phylogenetic methods pioneered by evolutionary biology. Linguistic phylogenetics is still a young subfield and there exists huge scope for methodological devleopment, of which this thesis is a part. Debate continues over the validity of applying phylogenetic methods to linguistic data, partly due to mistaken associations with the tarnished legacy of lexicostatistics and glottochronology. But there are also valid questions around linguistic data structures and their application in evolutionary models. This discussion is important, and need not result in dichotomized `for' and `against' camps of thought. In the chapters that follow, I engage with some of these questions through an empirical evaluation of phonotactic data in phylogenetic methods. In this way, I follow the suggestion of \textcite[p.~2299]{greenhill_does_2009}, who recommend that rather than engaging in ``armchair speculation'' on the suitability of phylogenetic methods ``on \emph{a priori} grounds'', it is more beneficial to engage with these methods and linguistic theory, crunch the numbers, and quantify how useful given methods are for testing various historical hypotheses.

\hypertarget{phylogenetic-preliminaries}{%
\subsubsection{Phylogenetic preliminaries}\label{phylogenetic-preliminaries}}

Computational phylogenetic methods are not monolithic and carry the weight of their own history of methodological development. In Chapter \textbf{CHAP REF} I discuss the methodological development of phylogenetic comparative methods in comparative biology in particular. Here, I briefly summarise the main statistical frameworks in which phylogenetic tree inference has taken place.

Parsimony.

Maximum likelihood.

Bayesian methods.

Notes on trait evolution terminology. Homology vs homoplasy. convergent evolution.

Beyond trees. Split decomposition.

\hypertarget{sahul-context}{%
\section{Sahul context}\label{sahul-context}}

\hypertarget{geospatial-context-and-deep-time-human-history}{%
\subsection{Geospatial context and deep-time human history}\label{geospatial-context-and-deep-time-human-history}}

Brief note on the geology and geography of Sahul and how it has changed through the ages.

History of human occupation. Theories for the peopling of Sahul and existing work triangulating archaeological, linguistic and genomic evidence.

Malaspinas, A.-S. et al.~A genomic history of Aboriginal Australia. Nature 538, 207--214 (2016).

Tobler, R. et al.~Aboriginal mitogenomes reveal 50,000 years of regionalism in Australia. Nature 544, 180--184 (2017).

\hypertarget{languages-of-sahul-and-historical-linguistic-context}{%
\subsection{Languages of Sahul and historical linguistic context}\label{languages-of-sahul-and-historical-linguistic-context}}

The following section gives a brief overview of the genealogical classifications and phonological systems of languages in Sahul. I dedicate the most attention to the languages of Australia, since Australian languages make up the language samples in the subsequent chapters of this thesis. Each Australian subsection is followed by a brief note on the languages of New Guinea, making up the rest of Sahul. By doing so, I aim to place the studies in the following chapters in their broader continental context and foreshadow some points to be addressed in the Discussion chapter further on.

\hypertarget{languages-of-australia}{%
\subsubsection{Languages of Australia}\label{languages-of-australia}}

Somewhere between 250--300 (though perhaps more) distinct languages are understood to have been present in Australia at the time of European colonisation in the 18th century, which can be subdivided further into around 700-800 language varieties when dialectal variation is considered \autocite{koch_languages_2014}. These linguistic communities are characterised by small populations, with a high degree of multilingualism facilitating communication between groups \autocite{koch_languages_2014}.

Colonisation has led to devastating shifts in the linguistic landscape. Of around 120 languages presently spoken in some capacity, only a dozen are estimated to be in a ``relatively strong'' position of sustainability with regards to childhood first language speakers, excluding new language varieties such as Kriol \autocite[p.~42]{australian_government_department_of_infrastructure_transport_regional_development_and_communications_national_2020}. The effects of colonisation and language loss for the affected communities is a deeply important and multifaceted topic demanding of its own dedicated discussion. I do not wish to give an unjustly inadequate summary of this subject here. For the purposes of this thesis, I make note only of the direct consequences of colonisation and language loss for studies such as this one. Fundamentally, the rapid loss of languages in Australia and lack of value placed on Indigenous languages by colonists has led to Australian languages being under-resourced (with regard to documentation, grammatical description, dictionaries, research) and under-represented in global cross-linguistic research. For many languages, existing resources are old and of variable quality (e.g.~wordlists compiled by early settlers with no linguistic training). This limits what can be achieved with methods that demand high fidelity data and/or a high quantity of data. It can mean a relative lack of statistical representation in otherwise global typological datasets, which then affects our ability to make global scale generalisations about language. Consequently, overcoming these challenges is a priority for methodological development in large-scale comparative linguistics.

A broad understanding of the phylogenetics of Australian languages has been established through several decades of historical linguistic work using the \emph{Comparative Method}---the `gold standard' of historical linguistic methodology since the late 19th century {[}REFS to `gold standard'{]}. By far the largest family is Pama-Nyungan, which encompasses around two thirds of Australia's pre-colonial linguistic diversity and nearly 90\% of its landmass \autocite[p.~817]{bowern_computational_2012}. The Pama-Nyungan family was first formally proposed by \textcite{ogrady_languages_1966} as part of a continent-wide classification of Australian languages on lexicostatistical grounds. However, \textcite{hale_classification_1964} generally is credited with coining the language family's name, which is a compound of words for `person' in the southwestern and northeastern extremes of the continent \autocite[see also][]{wurm_aboriginal_1963}. This work was preceded by various earlier, approximately analogous groupings identified on the basis of lexical similarities \autocites[pp.~207--212]{grey_journals_1841}[p.~iv]{moorhouse_vocabulary_1846} and lexical cognacy (more systematically established than in earlier work) with a few phonological and grammatical features \autocites{schmidt_gliederung_1919}{kroeber_relationships_1923}. Capell's \autocites{capell_structure_1937}{capell_classification_1940} primarily typological classification of Australian languages can be included in these earlier precedents too, if one interprets Capell's description of `Common Australian' in the same way \textcite{koch_methodological_2004} does---that is, a proto-language ancestral to most but not all Australian language groups, and distinct from any hypothetical proto-Australian language, which Capell terms `Original Australian' (Koch's interpretation seems valid, based on Capell's \autocite*{capell_history_1979} later work). Despite some skepticism on the validity of the Comparative Method in Australia \autocites{dixon_australian_2002}[but see][ for a counter-argument]{ogrady_coherence_2004}, there has been an abundance of work establishing subgroups within Pama-Nyungan via the Comparative Method in the decades since \textcites{ogrady_languages_1966}[see an overview and several examples in][]{bowern_australian_2004}. There are presently around 36 identified subgroups within Pama-Nyungan plus several isolates \autocite[following][]{bowern_pama-nyungan_2018}, however, not all of these have been established thoroughly with the Comparative Method and the precise number may differ depending on the analyst and the standard of evidence they accept. In recent times, computational phylogenetic methods have been applied to discern higher order groupings above the level of these subgroups, as well as an estimated age and an estimated geographic point of origin for the Pama-Nyungan family. \textcite{bowern_computational_2012} presents computational phylogenetic evidence for four high-level divisions, clustering geographically into southeastern, northern, central and western groups. \textcite{bouckaert_origin_2018} recovers these same high-level groupings and further estimates a point of origin in the Gulf of Carpentaria region, dated to the mid-Holocene period, between approximately 4,500--7,000 years ago.

The little over 10\% of the remaining Australian mainland once Pama-Nyungan is accounted for is home to around 27 language families and isolates \autocite{evans_non-pama-nyungan_2003}. These are located in two discontinuous areas of Australia's north---one in the Kimberley region and one in the Top End. In addition, Tasmania is home to at least one and possibly more independent language families, though the precise number of languages present in Tasmania at the time of colonisation and their family structure is difficult to discern, owing to the particularly devastating effects of colonisation and paucity of linguistic documentation on the island \autocite{bowern_riddle_2012}. In the north, there is scope for discrepancies on the exact number of families depending on certain classification choices. For example, Evans' figure of 27 families includes an `unclassified' group containing likely at least two family-level isolates \autocite[p.~11]{evans_non-pama-nyungan_2003}. Other factors affecting this figure include the treatment of Daly languages, assumed to be a single family by \textcite{tryon_daly_1974} but which have subsequently been argued to consist of up to four separate families \autocite[ch.~4--7]{evans_non-pama-nyungan_2003}, the reclassification of several languages into an expanded Gunwinyguan family and possibly even a larger `Arnhem' family \autocite[p.~14--15]{evans_non-pama-nyungan_2003}, and the treatment of Tangkic languages, Garrwan languages and Yanyuwa, all of which have been argued variously for status within or outside of Pama-Nyungan at various points in time \autocite[p.~12]{evans_non-pama-nyungan_2003}. Whatever their precise classifications, it is clear that an exceptionally high degree of high order linguistic phylogenetic diversity is packed into a particularly small geographic area, in stark contrast to the geographic extensiveness of Pama-Nyungan. It has not been conclusively demonstrated through historical linguistic means how the various non-Pama-Nyungan families relate to one another nor how they relate to Pama-Nyungan, nor whether all Australian families group together under a single `proto-Australian' ancestor to the exclusion of all other languages in the world. Nevertheless, curiously similar phonologies are present across Pama-Nyungan and non-Pama-Nyungan areas (discussed below in Section \ref{phonological-overview-of-australia}) and a proto-Australian root \autocite[e.g.~as proposed by][]{dixon_languages_1980} seems generally assumed by most \autocite[e.g.][]{hamilton_phonetic_1996}. \textcite{evans_non-pama-nyungan_2003} discusses some possibilities for family relationships underneath a proto-Australian root. Proposals include the following: A `rake model', in which all 27 or so non-Pama-Nyungan families and Pama-Nyungan descend equally from the same root with no intervening higher-order structure \autocite{ogrady_languages_1966}. A `binary model', in which proto-Australian splits into two sisters, proto-Pama-Nyungan and proto-non-Pama-Nyungan \autocite{heath_linguistic_1978}. This is approximately equivalent to Capell's \autocite*{capell_new_1956} typological division between prefixing and non-prefixing languages (though there are examples of prefixing Pama-Nyungan languages and non-prefixing non-Pama-Nyungan languages). A `Pama-Nyungan offshoot model', in which Pama-Nyungan, Garrwan, Tangkic and Gunwinyguan families are contained beneath a higher order `proto-macro-Pama-Nyungan' ancestor, and thereby more closely related to one another than to other non-Pama-Nyungan families \autocites{ogrady_preliminaries_1979}{evans_cradle_1997}{nicholas_enigma_1998}. Several decades of progress notwithstanding, relating non-Pama-Nyungan languages to one another, to Pama-Nyungan, and to a putative proto-Australian root remains an active priority for historical linguistics research in Australia.

This thesis concentrates primarily on the Pama-Nyungan family. Chapters \textbf{CHAP REF}--\textbf{CHAP REF} use lexical data from 112 exclusively Pama-Nyungan language varieties. These varieties represent 26 of the 36 \textbf{Check} Pama-Nyungan subgroups identified in \textcite{bowern_pama-nyungan_2018}. In Chapter \textbf{CHAP REF}, which does not require a pre-existing phylogenetic reference tree, an additional XX langauges are included representing XX non-Pama-Nyungan families. All the languages included in the language samples in Chapters \textbf{CHAP REF}--\textbf{CHAP REF} are mapped in Figure \textbf{FIG}.

{[}INSERT MAP ABOUT HERE{]}

\textbf{New Guinea}

\textbf{Linking linguistics, geography and human prehistory}

Reconciling with other fields. \textcite{bouckaert_origin_2018} dates are quite a bit younger than both the Malaspinas et al (2016) paper and work on Mitochondrial dating that came out around the same time. CB's thoughts here: \url{http://pamanyungan.compevol.auckland.ac.nz/implications/}

The linguistic prehistory of Sahul, the continent of Australia and New Guinea, is an enigma which is mostly yet to be cracked by historical linguistics. The question of if and how Australian languages might relate to their Papuan neighbours to the north has been the subject of speculation for many years \autocites[e.g.][]{ogrady_languages_1966}{wurm_papuan_1975}. But, besides a handful of tentative cognates \autocite{foley_papuan_1986} and shared structural features \autocites{nichols_sprung_1997}{reesink_explaining_2009}, conclusive evidence for a connection remains elusive. Notwithstanding the many small-scale language families (plus two large ones---Pama-Nyungan and Trans-New Guinea) which have been successfully identified via the standard comparative method in historical linguistics, the diagnosis of deeper historical relations is handicapped by several factors in this part of the world: A lack of adequate description \autocite{bowern_computational_2012}, absence of pre-colonial written sources \autocite{foley_papuan_1986}, apparent homogeneity of phonological systems \autocite{baker_word_2014} and millennia of extensive horizontal diffusion \autocites{foley_papuan_1986}{dixon_australian_2002} to name a few. Even in ideal circumstances, the scope of the linguistic comparative method is commonly cited as being limited to approximately ten thousand years, at which point historical signal becomes indistinguishable from noise in lexical data \autocite{nichols_sprung_1997}. This limits the method to a window of time well after rising seas inundated Lake Carpentaria approximately 12,000 years ago and only just before the inundation of Torres Strait approximately 8000 years ago.

Recent advances give cause for optimism that we may be able to extend the time-depth at which historical linguistics can operate and unravel something of the prehistory of Sahul. Methodological advances include the adaptation of Bayesian phylogenetic methods to infer the evolutionary history of language families. For example, \textcite{bowern_computational_2012} and subsequently \textcite{bouckaert_origin_2018} use phylogenetic methods, in combination with cognate identification via the traditional comparative method, to infer the internal branching of the Pama-Nyungan family. As for data, and the possible erosion of historical signal after 10,000 years in lexical data specifically, two observations motivate the approach of the present study: Firstly, linguistics has well and truly entered `the age of big data' and computational methods now enable us to extract minute threads of significance from large volumes of data. This enables us to compare more data points across wider groups of languages quickly and efficiently, and identify trends and correlations which would otherwise escape the attention of even the most highly trained human eye. This thesis evaluates the profitability of complemneting lexical data (which continues to form the backbone of most historical linguistic work) with other kinds of data too.

\hypertarget{phonological-overview-of-australia}{%
\subsection{Phonological overview of Australia}\label{phonological-overview-of-australia}}

Australian language phonologies are typologically distinctive. Descriptions of the nature of language phonologies in Australia make reference to a uniquely constrained set of characteristics and the phonologies of individual Australian langauges are frequently described in terms of a specific, narrow set of parameters of variation that follow from the broadly understood profile of what Australian phonologies look like \autocites[e.g.][p.~11]{goddard_grammar_1985}[p.~52]{evans_grammar_1995}[p.~7]{breen_grammar_2007}[p.~23]{gaby_grammar_2006}. This apparent homogeniety of phonological systems across Australia stands out as unusual for such a geographically expansive and phylogenetically diverse area, when compared to the phonological diversity observed in other parts of the world of similar areal and genealogical scope \autocites{maddieson_patterns_1984}{mielke_emergence_2008}. The first instance of this apparently constrained nature of Australian phonological variation appears to be \textcite{schmidt_gliederung_1919}. Subsequent references include \textcite{capell_new_1956}, \textcite{voegelin_obtaining_1963}, \textcite{dixon_languages_1980}, \textcite{busby_distribution_1982}, \textcite{hamilton_phonetic_1996}, \textcite{butcher_australian_2006}, \textcite{dixon_australian_2002}, \textcite{baker_word_2014}, and \textcite{round_segment_2021}, among others. There is, however, some effort to push back against this consensus and illuminate Australian phonological variation to a degree that is perhaps underappreciated within the field. \textcite{bowern_standard_2017} makes a case for treating generalised claims about ``Standard Average Australian'' language characteristics with caution, finding that many such characteristics described in the literature suffer from a lack of testability. In this light, \textcite{gasser_revisiting_2014} attempt to re-examine previous typological claims about Australian langauge phonologies in a more quantifiable way, using lexical data from 120 Australian language varieties. They concede the homogeneity of Australian segmental inventories, but demonstrate that there is considerably greater variation between languages with regards to the frequencies of phonological segments, rather than simply the segmental inventories themselves.

Notwithstanding these points of qualification on the generalisability of Australian language phonologies (and Australian languages generally), there are some common characteristics which serve to illustrate, in a basic sense, the general structure of phonological inventories found across Australia. The notion of a kind of prototypical ``Standard Average Australian'' phonological inventory, as it would be termed by \textcite{gasser_revisiting_2014} and \textcite{bowern_standard_2017}, is described by \textcite{dixon_languages_1980}, \textcite{hamilton_phonetic_1996} and \textcite{fletcher_sound_2014} and includes the following characteristics.

\begin{itemize}
\tightlist
\item
  4--6 contrastive place distinctions, forming three natural classes: two classes of coronals (apical and laminal), while bilabial and dorsal segments pattern together to form a single \emph{peripheral} class. The total number of place distinctions in a given langugage sums between 4--6 depending on whether the language makes one or two place distinctions within the apical and laminal classes.
\item
  A lack of contrastive manner distinctions among obstruents. Frication, where present, is typically non-contrastive. There are two areal exceptions to this. Languages in the Cape York and Daly River regions have developed a class of fricatives, which are historically derived from stops \autocite[p.~125]{dixon_languages_1980}.
\item
  No voicing contrast. Exceptions are found in some non-Pama-Nyungan families, where there is a contrastive \emph{fortis} and \emph{lenis} distinction among stops that is variously analysed as either a voicing contrast or gemination.
\item
  A contrastive nasal segment at each contrastive place of articulation.
\item
  A point of variation is whether there is a single contrastive lateral or a series of laterals corresponding to each of the 2--4 contrastive coronal places of articulation.
\item
  A palatal glide plus a contrast between a rhotic glide and a trill/tap segment.
\item
  3 contrastive vowels (in a standard, triangular system in the vowel space---high-front, high-back and low) with an optional additional contrast for vowel length. A famous exception is the Arandic subgroup of languages, which are argued to have a 2-vowel system, whereby the roundedness of the formerly contrastive /u/ vowel has spread to become a feature of the neighbouring consonants {[}REF{]}. Larger vowel systems are also found in non-Pama-Nyungan language families.
\end{itemize}

Beyond segmental inventories, remarkable consistency among Australian languages has been described in phonetics \autocite{dixon_languages_1980} and phonotactics \autocites{dixon_languages_1980}{hamilton_phonetic_1996}{baker_word_2014}. Turning now to phonotactics, \textcite{hamilton_phonetic_1996} describes an ostensibly considerable degree of variation between languages. However, he finds that this variation is ``systematic and highly constrained'' \autocite[p.~29]{hamilton_phonetic_1996}. Hamilton refers to this underlying similarity, in which the superficial phonotactic variation that does exist can be understood within the tightly prescribed bounds of a uniform, continent-wide system, as \emph{concert}.

One particularly notable feature of Australian phonotactic systems is that they cannot be described easily using the syllable as the fundamental unit of organisation, where phonotactic rules and constraints correspond to prosodic positions within the syllable \autocites[e.g.][]{ito_syllable_1988}{goldsmith_autosegmental_1990}. Australian phonotactic systems are better served by a disyllabic structure as the fundamental unit of organisation and, resultingly, few Australian languages permit words of fewer than two syllables \autocite{dixon_languages_1980}. \textcite[p.~75]{hamilton_phonetic_1996} presents two disyllabic templates, following \textcite{dixon_languages_1980}, presented in \eqref{eq:disyll-template-1} and \eqref{eq:disyll-template-2} below, where \(C_{init}\) is a word-initial consonant, \(C_{inter}\) is an intervocalic consonant, \(C_{fin}\) is an optional word-final consonant, \(C_1\) is a pre-consonantal consonant and \(C_2\) is a post-consonantal consonant.

\begin{equation}
C_{init}VC_{inter}V(C_{fin})
\label{eq:disyll-template-1}
\end{equation}

\begin{equation}
C_{init}VC_1.C_2V(C_{fin})
\label{eq:disyll-template-2}
\end{equation}

The consonantal slots in \eqref{eq:disyll-template-1} and \eqref{eq:disyll-template-2} are characterised by the following salient points, as described by \textcite{dixon_languages_1980}, \textcite{hamilton_phonetic_1996} and \textcite{baker_word_2014}.

\begin{itemize}
\tightlist
\item
  Of all consonant positions, \(C_{inter}\) is generally the only one in which any consonant may appear.
\item
  Apicals are typically restricted from appearing in \(C_{init}\) position, and peripherals are preferred over laminals. Obstruents are preferred over sonorants for both \(C_{init}\) and \(C_2\) positions.
\item
  Conversely, sonorants are preferred over obstruents in \(C_{fin}\) and \(C_1\) positions. Coronal consonants are also preferred in this position over peripherals in these positions.
\item
  Many languages do not include the \(C_{fin}\) position at all, with words necessarily being vowel-final, e.g.~Anindhilyakwa \autocite{van_egmond_enindhilyakwa_2012}, Kayardild \autocite{evans_grammar_1995}, and Diyari \autocite{austin_grammar_1981}.
\end{itemize}

Homorganic nasal+stop clusters are almost universal, although Thargari \autocite{klokeid_thargari_1969} is an exception identified by \textcite{hamilton_phonetic_1996}. One particularly distinctive feature of Australian language phonotactics is the commonality of a rich array of heterorganic nasal+stop clusters. These are typically restricted in frequency and type to a greater degree than their homorganic counterparts \autocite[pp.~78--82]{hamilton_phonetic_1996}, but are nevertheless noteworthy given the rarity and susceptibility of heterorganic nasal+stop clusters to assimilation in other parts of the world \autocite[p.~144]{baker_word_2014}. Lateral+stop clusters are also common, and in this case, heterorganic combinations are more common while homorganic clusters are dispreferred or restricted all together.

In examining the overall typology of Australian language phonologies and phonotactics, one point to consider is the generally non-deterministic nature of phonological analysis \autocites{chao_non-uniqueness_1934}{hockett_problem_1963}{hyman_universals_2008}{dresher_contrastive_2009}. As I discuss in Chapter \textbf{CHAP REF}, there are many criteria that are factored into the determination of a language's segmental inventory and, consequently, multiple possible solutions depending on the application and analytic order of a particular linguist. This has implications for the broad typology of Australian phonology and phonotactics, with particular regards to the impression of homogeneity in these systems. One such criterion for segmental-phonological analysis in an individual language is phonotactic parsimony. For example, by one analysis, Alawa features a contrastive series of prenaslised stop segments \autocite{sharpe_alawa_1972} and, likewise, the same has been described for Tiwi \autocite{lee_tiwi_1987}. The motivation for these analyses is that nasal+stop sequences would otherwise be found in the \(C_1\) position of the disyllabic templates above. These so-called complex segments enable the preservation of a parsimonious phonotacic analysis and a phonotactic analysis that is in concert with fellow Australian languages. However, this is at the expense of creating expanded segmental inventories for Tiwi and Alawa that diverge from the standard average Australian characteristics described above. An alternative analysis of Alawa and Tiwi phonologies might place a lower value on phonotactic criteria and, for example, place a relatively greater emphasis on finding minimally contrastive pairs. Word-initial nasal+stop sequences could be analysed as separate segments, producing smaller, simpler semgental inventories that are more in keeping with the typological standard. This, however, would require acceptance of a greater degree of phonotactic variation in Australia and revision of disyllabic templates \eqref{eq:disyll-template-1} and \eqref{eq:disyll-template-2} if these templates are to be held as applicable Australia-wide. These examples serve to illustrate the point that, although Australian phonologies and phonotactics are often held to be homogenous to a considerable degree, various exceptions to the norm do exist and, furthermore, where and how variation manifests is, in part, a product of the analytic decisions of the individual typologist and/or the individual linguists who document languages.

\hypertarget{phonological-overview-of-new-guinea}{%
\subsubsection{Phonological overview of New Guinea}\label{phonological-overview-of-new-guinea}}

Although no Papuan language data is included in the Chapters that follow, attention is now turned briefly to the phonological characteristics of languages in New Guinea, to provide a more complete phonological context of Sahul. I return to some of the points of difference discussed below in Chapter \textbf{CHAP REF}, and consider the implications for future, Sahul-wide studies of linguistic prehistory.

BRIEF paragraph or two on phonology, following mainly Foley (2018). Points of difference to Aus (more variation)

Paragraph on phonotactics.

Will return to implications for future study in Discussion chapter.

\hypertarget{chapter-synthesis}{%
\section{Chapter synthesis}\label{chapter-synthesis}}

BRIEF. To do.


% ***************************************************