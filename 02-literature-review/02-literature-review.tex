\chapter[Literature review]{Literature Review}
\label{ch-lit-review}

% ***************************************************
% Literature Review
% ***************************************************

This chapter surveys existing literature, firstly to illuminate the motivation for this thesis and position the research questions of the following chapters in their scientific context, and secondly to situate this study in a broader context of deep-time human and linguistic history in the continent of Sahul. The chapter proceeds in two parts, addressing each of these goals in turn. Section \ref{academic-context} starts by tracing the origin and development of historical linguistics as a field. Particular focus is given to rapid developments in the past two decades, which increasingly enable integration of historical linguistics with other aspects of human history, rapidly shifting the kinds of questions that can be investigated with comparative linguistic data. This topic is subsequently taken up again in Chapter \ref{ch-pcms} with a particular concentration on the fundamental task of comparison in historical linguistics and beyond. With this context in mind, Section \ref{sahul-context} turns to a particular world domain, namely the continent of Sahul. The section begins with an overview of the geospatial and ecological context of Sahul and our current best understanding of the earliest human history in the continent. The section proceeds with a picture of the current linguistic diversity in Sahul and the historical linguistic hypothesis posited for how the present linguistic ecology came to be. I then discuss how the current historical linguistic picture reconciles with our general understanding of human history in Sahul as discussed previously. Finally, a general phonological profile of the languages of Australia is provided. As the studies presented in the following chapters use data from Australian languages only, Australian languages will be the focus of these sections of the literature review.

\hypertarget{academic-context}{%
\section{Phylogenetic thinking in historical linguistics}\label{academic-context}}

Charting patterns of descent through time is a task common to several fields across the biological and cultural sciences. Occasionally, this is possible through direct observation (e.g.~examination of historical or archaeological records) or controlled laboratory experiments (e.g.~observing microbial evolutionary processes in a lab setting). Often, however, questions relate to a historical past that cannot be observed directly (without time travel, at least). In these cases, the best that can be done is indirect inference. The main method of historical inference is comparison. Extant entities, be they organisms, languages, populations, etc., are compared and historical relationships between entities are inferred. To this end, a countless array of comparative data types and theoretical models have been employed, summoning what knowledge we have about how historical processes operate. Some studies may have the benefit of a partial historical record as a scaffold (e.g.~biological data from modern species combined with a partial fossil record). Indirect historical inference via comparison is particularly familiar in linguistics, given the ephemeral nature of spoken language. Clearly, the global body of written language is extensive. But even where written language extends back farthest it remains a relatively recent human invention, covering around 5,000 years \autocite{weiss_comparative_2014}, which represents at most around 10\% of the timespan of human language generally \autocite[and probably much less.][]{pagel_q_2017}. In many parts of the world, including all of Sahul, written language has only been introduced in the past few centuries, and there remain some languages with no written records at all. Likewise, audio and audio-visual records of human language extend back only 150 years or so.

Phylogenetics is a particular subtype of historical science, focusing on questions of phylogeny. That is, the tree-like evolutionary relationships that develop from the assumption that some entities are descended from a shared common ancestor. Biologists have charted patterns of descent in phylogenetic trees since Darwin, and so too have linguists for languages for approximately as long. A phylogenetic tree is not the only way schematise historical relationships, nor is a phylogenetic schematisation necessarily mutually exclusive with other, non-tree-like historical approaches. It has, however, proved a highly successful tool for furthering understanding of all sorts of historical questions for over a century and a half.

In the last two decades, as computational phylogenetic methods have entered the historical linguistic mainstream, there has been a resurgence of discussion on the supposed analogies between linguistic evolution and biological evolution \autocite[e.g.][]{atkinson_curious_2005}. But how analogous is linguistic evolution to biological evolution really? And can modern computational phylogenetic methods in biology be applied to linguistics through analogy? Here, I argue for continued uptake and development of computational phylogenetic methods in linguistics but through a shifted paradigm that focusses less on direct analogy between linguistic and biological evolution and instead concentrates on the underlying mathematical elements that may or may not be common to both.

Discussions comparing biological evolution to linguistic evolution typically begin with Darwin, who seems to be the first to propose an explicit analogy between biological evolution and linguistic evolution \autocite[p.~422]{darwin_origin_1859}. Later, \textcite[p.~57]{darwin_descent_1871} states that ``the formation of different languages and of distinct species, and the proofs that both have been developed through a gradual process, are curiously the same.'' And this oft-quoted passage is a common starting point in discussion of linguistic and biological evolution today \autocites[e.g.][]{atkinson_curious_2005}{bromham_curiously_2017}{mesoudi_pursuing_2017} In a footnote to this passage, Darwin refers to \textcite[ch.~13]{lyell_geological_1863}, who compares the descent of languages from generation to generation, with modification via processes such as lexical innovation, to ``the force of inheritance in the organic world'' \autocite[p.~457]{lyell_geological_1863}. Phylogenetic concepts were being actively developed in linguistics around the same time. \textcite{schleicher_ersten_1853} is credited with creating the earliest phylogenetic tree-like model of languages some years prior to Darwin's famous tree illustrations in \emph{The Origin of Species}, and highlighted the similarity of his earlier 'Stammbaum' model to Darwin's trees \autocite{schleicher_darwinsche_1863}.

This early period of relatively shared phylogenetic thinking subsequently diverged into two quite independent streams of thought in the 20th century. Although biology and linguistics share more similarities in their methodological histories than is perhaps appreciated (as I discuss in Chapter \ref{ch-pcms}), rapid advances in genetics and biology's subsequent quantitative turn did lead to a considerable gulf developing between the fields in the second half of the 20th century in particular. With new-found access to large volumes of genetic data and expanding computational capacity, biologists increasingly turned to quantitative algorithms to infer phylogenetic relationships and test evolutionary hypotheses \autocite{atkinson_curious_2005}. In contrast, historical linguistics tended to remain reliant on manual, expert linguistic judgements for data acquisition \autocite{nunn_comparative_2011}. The linguistic \emph{Comparative Method}\footnote{Throughout, I capitalise the term `Comparative Method' to distinguish the specific historical linguistic methodology that goes by this name from the generic `comparative method', which could refer to any of the various methodologies in comparative fields of science.} remained the methodological bedrock of historical linguistics throughout the entire 20th century and arguably remains the ``gold standard'' of the field today \autocite[p.~712]{dunn_structural_2008}. Following \textcite{thomason_language_1988} and \textcite{campbell_historical_2004}, the Comparative Method can be summarised briefly as follows. Firstly, sets of likely cognate word forms are assembled. First pass cognacy judgements will likely be based on a fairly superficial level of similarity, with subsequent refinement later on. Known borrowings and other chance resemblances should be removed as much as possible. From these cognate sets, it is possible to identify sound correspondences between languages (i.e.~where segment \(x\) in language A and segment \(y\) in language B consistently recur in the same context). Proto-phonemes can be deduced by reconciling sound correspondences with sound change processes (e.g.~various categorical split and merger processes, and substantive changes such as lenition, elision etc.) and, ultimately, a lexicon of proto-word forms can be established. Whether or not it is explicitly stated, reconstruction of proto-phonemes and proto-word forms effectively involves the assumption of a kind of \emph{weighted parsimony} evolutionary model. That is, the most parsimonious reconstructions are favoured, both in terms of the absolute number of hypothesised sound changes and the physiological and typological plausibility of those sound changes (e.g.~the lenition of a voiceless stop to its voiced counterpart is a highly common, plausible sound change across the world, but a sound change from \emph{*t} to \emph{k}, although famously attested in several Polynesian languages \autocite{blust_t_2004}, is rare and unusual). The Comparative Method has been applied successfully the world over and boasts a robust track record over a century and a half. However, it remains, for the most part, a time-consuming manual process that requires considerable expertise in the particular languages being studied. There has been a substantial amount of development on automating step 1---the assemblage of candidate cognate sets on the basis of similarity \autocites{list_potential_2017}{rama_are_2018}{list_sequence_2018}. But identification of spurious matches due to borrowing and missed matches due to semantic shifts still requires manual intervention, ideally with localised knowledge of sociolinguistic context, migration patterns, trade practices and so on. A recent paper attempts to identify borrowings computationally, but with mixed results \autocite{miller_using_2020}. With regards to identification of sound correspondences and sound change, there is some work towards automation in this space as well \autocites{steiner_pipeline_2011}{brown_sound_2013}{bouchard-cote_automated_2013}{hruschka_detecting_2015}{list_automatic_2018}, but this task also remains at a relatively youthful stage of development. One limitation is that, although we know in general a good deal about which kinds of sound changes are common and which are rare, there is no definitive, prior reference that quantifies exactly the likelihood of particular sound changes in the world's languages. \textcite{wu_computer-assisted_2020} present a current best practice ``computer-assisted framework'' (as distinct from a fully automated or fully manual one) for the Comparative Method. Nevertheless, the Comparative Method remains largely a labour- and resource-intensive process.

The eventual output of the Comparative Method is phylogenetic in nature. To classify languages into families and subgroups (or classify an individual language as a family-level or subgroup-level isolate on its own), optionally with higher-level phyla or various intermediate groupings, is to define a phylogenetic tree structure. The difference between these kinds of more traditional language classifications, obtained via the comparative method, and a fully bifurcating phylogenetic tree, as familiar in biology or computational phylogenetic studies such as \textcite{bowern_computational_2012}, is that more traditional linguistic classifications only infer branching events at particular levels (the level of the family, the subgroup, etc.) and lack the very detailed internal structure of a fully resolved, bifurcating phylogeny. Accordingly, multiple languages will branch out from a single common ancestor node in a rake-like pattern. The tree structure in the Glottolog database \autocite{hammarstrom_glottolog_2020}, which reflects traditional subgroupings, gives a good illustration of this.

One early, quantitative departure from the Comparative Method in historical linguistics is the method of \emph{lexicostatistics}, developed by \textcite{swadesh_lexico-statistic_1952}. The lexicostatistical method involves the inference of historical relatedness by identifying cognates and quantifying pairwise percentages of shared vocabulary between languages. Simple clustering algorithms are then used to identify families and subgroups. \textcite{swadesh_towards_1955} later presented a method termed \emph{glottochronology} for dating linguistic divergence events. Glottochronology effectively involves an evolutionary \emph{strict clock} model, which assumes a constant, universal rate of lexical replacement. By assuming a particular rate of lexical replacement, phylogenetic branch lengths can be calculated and dates can be calibrated. However, glottochronology and the lexicostatistical approach fell into disfavour following a rejection of Swadesh's \emph{universal constant} theory of lexical replacement \autocite{blust_why_2000} and the unrealistic corollary that similarity necessarily equals relatedness \autocite{bowern_computational_2018}. Another weakness was an inability to quantify uncertainty in the results \autocite{atkinson_curious_2005}. Nevertheless, the lexicostatistical method did provide some utility in Australia, by enabling a rough, overall classification of language families and subgroups that could serve as a preliminary sketch until such time that more robust, future comparative work could be conducted. This seems to be the spirit in which \textcite{ogrady_languages_1966} present their lexicostatistical analysis of Australia.

The turn of the 21st century brought with it the beginnings of a new quantitative era in historical linguistics, with a surge of interest in adapting computational phylogenetic methods from evolutionary biology to infer answers to linguistic questions. Two decades into the century, these methods are increasingly commonplace in the mainstream historical linguistics scene and \emph{linguistic phylogenetics} arguably constitutes a subfield of study in its own right\footnote{Illustrating this point, the International Conference on Historical Linguistics in July 2019, Canberra, Australia, featured a dedicated session on ``Computational and phylogenetic historical linguistics''.}. This surge towards computational phylogenetics has been driven by a multitude of factors. Some factors are technological. Computational resources have increased in power and also decreased in cost, leading to greater accessibility. Improvements in Internet connectivity and greater awareness of open science principles have led to a proliferation of large scale, open datasets. An additional motivation is the desire for greater empirical rigour in the field (e.g.~testability, quantification of uncertainty, reproducibility) \autocites{atkinson_curious_2005}{mcmahon_finding_2003}{nunn_comparative_2011}.

Much of linguistic phylogenetic work has focused on the fundamental task of phylogenetic tree inference. Several early studies focused on tree inference in the relatively well-studied confines of Indo-European \autocites{gray_language-tree_2003}{atkinson_words_2005}{nicholls_dated_2008} followed later by \textcite{ryder_missing_2011}, \textcite{bouckaert_mapping_2012} and \textcite{chang_ancestry-constrained_2015}. Other early examples include studies of Austronesian \autocite{gray_language_2000} and Bantu \autocites{holden_bantu_2002}{holden_rapid_2006}. Computational phylogenetic tree inference remains a priority. More recent examples include (but are not limited to) studies of the following families: Aslian \autocite{dunn_aslian_2011}, Arawak \autocite{walker_bayesian_2011}, Pama-Nyungan \autocites{bowern_computational_2012}{bouckaert_origin_2018}, Tupí-Guaraní \autocite{michael_bayesian_2015}, Chapacuran \autocite{birchall_combined_2016}, Dravidian \autocite{kolipakam_bayesian_2018}, Sino-Tibetan \autocite{sagart_dated_2019}, Dene-Yeniseian \autocites{sicoli_linguistic_2014}[but c.f.][]{yanovich_phylogenetic_2020} and Turkic \autocite{savelyev_bayesian_2020}. Presently, the maturation of linguistic phylogenetics and of a broader field of evolutionary linguistics generally \autocites{dediu_language_2016}{nolle_language_2020} has brought a widened focus on the diversity of linguistic hypotheses that can be tested with an arsenal of computational phylogenetic methods. Typological features can be studied with the use of a phylogenetic tree and phylogenetic comparative methods \autocite[e.g.][]{dunn_evolved_2011} and this will be the topic of Chapter \ref{ch-pcms}. Linguistic phylogenetic trees and phylogenetic methods can be combined with other kinds of data (such as geospatial information) and evidence from other fields such as archaeology to investigate ancient human migration movements \autocites[e.g.][]{gray_language_2009}{bouckaert_mapping_2012}{bouckaert_origin_2018}. Other studies investigate evolutionary questions in a computational way, without explicitly including phylogenetic trees or methods, e.g.~the evolution of sound systems with respect to ancient changes in human physiology \autocite{blasi_human_2019}, the evolution of sound systems with respect to ecological environment \autocite{everett_climate_2015}, sound-meaning association biases \autocite{blasi_soundmeaning_2016}, and emotion semantics \autocite{jackson_emotion_2019}. Although one can draw a distinction between studies of language evolution and linguistic phylogenetics, or as \textcite{haspelmath_human_2020} puts it, the ``evolution of linguisticality'' (i.e.~the biological capacity for language) and the ``evolution of languages'', the two can also be tied and can inform one another \autocites{blasi_soundmeaning_2016}{nolle_language_2020}. On this basis, \textcite{segovia-martin_eco-evo-devo_2020} argues for a unified \emph{eco-evo-devo} framework.

The uptake of computational phylogenetic methods in linguistics has not been accepted uncritically in all quarters. \textcite[p.~520]{atkinson_curious_2005} reports a ``curious'' aversion to phylogenetic methods in historical linguistics, speculating that perhaps some are ``haunted \ldots{} by the ghost of glottochronology past'' (see below). \textcite{bowern_computational_2018} groups criticism of phylogenetic methods in linguistics into three categories. The first of these is the contention that an analogy between linguistic evolution and biological evolution is invalid; languages do not evolve like species and, therefore, methods for inferring the evolution of species cannot be applied justly to linguistic data \autocites{andersen_synchrony_2006}{blench_new_2015}\footnote{See also Alexander Lehrman, quoted by \textcite[p.~1326]{balter_search_2004}.}. It is true, of course, that languages do not evolve in perfectly analogous ways to species and many phylogenetic methods applied in biology carry in-built assumptions that do not hold in linguistics. In Chapter \ref{ch-pcms}, however, I make the case that many of the supposed incompatibilities of linguistic data with phylogenetic methods, e.g.~horizontal transmission in language contact situations, non-independence between correlated or logically dependent features, and uncertainty around clock models or various other parameters, are not as unique to linguistics as often assumed and there is often either existing literature or active development on addressing particular challenges applicable to linguistics without abandoning phylogenetics or quantitative methods altogether. It is useful to take a step back from direct analogising between linguistics and evolutionary biology, and instead focus on selecting the right tools for the right job by linking historical processes to basic mathematical functions. Naturally, there will still be limitations and uncertainty but these can be made explicit and evaluated empirically. In the remaining chapters of this thesis, I demonstrate this approach.

A second line of criticism relates to the validity of applying quantitative methods to linguistic data, stemming from (arguably erroneous) associations to lexicostatistics and glottochronology \autocites{eska_recent_2004}{holm_new_2007}. The problems with lexicostatistics and glottochronology are well described and these methods have been discredited for several decades \autocite[see][pp.~285--286]{bowern_computational_2018}. The history of computational phylogenetic methods, though, is essentially independent from lexicostatistics and glottochronology and the methods share little in common besides their generally quantitative nature. If lexicostatistics and glottochronology are rejected, it does not follow that computational phylogenetic methods must be rejected as well. To give some specific examples, computational phylogenetic methods need not carry the assumption that there is a constant rate of lexical replacement, as assumed in glottochronology. An array of clock models exist. These can allow, for example, different parts of the lexicon to evolve at faster or slower rates, and different rates of change in different subparts of a phylogenetic tree. Further, different clock models can be evaluated and compared empirically. Furthermore, phylogenetic methods rely on vastly more sophisticated evolutionary models to determine branching events, and do not simply cluster languages based on a similarity metric.

A third line of criticism concerns a reliance on lexical data within phylogenetic methods, rather than reconstructed sound changes or morphological changes. Despite some efforts to infer phylogenies with datasets of various typological features \autocites[e.g.][]{dunn_structural_2005}{dunn_structural_2008}{sicoli_linguistic_2014}, the vast bulk of phylogenetic studies in linguistics do rely on lexical cognate data. The extent to which this is a problem is unclear. \textcite{bowern_computational_2018} argues that there is no evidence to suggest that lexical data are subject to an especially different evolutionary process than other parts of language. She also questions the primacy of sound change evidence over evidence from other parts of language in historical linguistics, and points out that lexical data and sound change data are not completely independent, since sound change reconstruction relies on the assemblage of lexical cognate data in the first place. I would argue that, to address these questions, a priority for the methodological development of linguistic phylogenetics is to evaluate empirically the evolutionary dynamics of linguistic features beyond lexical data. \textcite{greenhill_evolutionary_2017} give a particularly clear-eyed example of this, evaluating evolutionary rates of change of grammatical features versus lexical ones and finding, contrary to prior expectations, that the grammatical features evolved faster. In a similar vein, I present an evaluation of the phylogenetic patterning of phonotactic data in Chapter \ref{ch-phylo-signal}.

A fourth point of criticism, recurring particularly frequently, is the apparent limitation or inadequacy of a family-tree-based model which only shows vertical patterns of inheritance and not horizontal diffusion through, for example, linguistic contact and borrowing \autocites{bateman_speaking_1990}{donohue_new_2012}{gould_urchin_1987}. Countering this concern, \textcite{bowern_historical_2010} claims that this is confusing the family tree, which is essentially a visualisation tool, with the methods themselves. Further, while there is still much work to be done (and not only in linguistics), phylogenetic methods are capable of modelling assumptions about horizontal admixture, not to mention quantifying uncertainty. In addition, \textcite{greenhill_does_2009} conduct a simulation study to test whether phylogenetic methods are invalidated by horizontal diffusion. They find that, to the contrary, their Bayesian phylogenetic method of choice is quite robust to areal borrowing between languages (albeit with some caveats relating to dating phylogenetic branches). Robustness to borrowing is also extensively evaluated in \textcite{bouckaert_origin_2018}.

A final source of criticism relates to reduction of hugely complex linguistic systems into neat datasets of numerical or binarised characters for phylogenetic analysis, without excessively compromising the dataset's ability to be meaningful and informative in a linguistic sense. Further, there is the question of whether linguistic variables can be assumed to be independent and equivalent to the degree required of phylogenetic algorithms, given the complex interplay between linguistic features and non-random, directional patterns of linguistic change that can be observed \autocites{heggarty_interdisciplinary_2006}{round_design_2015}. I would argue that these are not insurmountable issues. Reducing complex information systems into statistical models is a challenge for all fields \autocite{round_comparability_2020} and this does not uniquely invalidate phylogenetic methods in linguistics. These are, however, exactly the kinds of questions that linguistic phylogeneticists should consider. A lack of independence between phonotactic characters is a genuine limitation of the studies presented in the following chapters of this thesis. I return to this topic in the Discussion chapter.

Before moving on, I will make the point that, although the Comparative Method carries an unparalleled legacy of success in historical linguistics, no method is perfect and the Comparative Method has its own limitations and flaws \autocite[see][]{durie_comparative_1996}. The neogrammarian assumption of the regularity of sound change \autocite{osthoff_morphologische_1878}, which is a crucial assumption of the Comparative Method \autocite[p.~166]{campbell_historical_2004}, is an assumption about an evolutionary process that has been subject to long-running debate \autocite{labov_regularity_2020} and apparent exceptions \autocites[see][]{chen_sound_1975}{wang_lexicon_1977}{durie_comparative_1996}{miceli_where_nodate}. Furthermore, the Comparative Method can lack transparency. Analytical decisions during cognate identification can be opaque and difficult for others to evaluate without an extensive background knowledge in the languages being studied. There is presently no straightforward way to incorporate uncertainty into a manual analysis using the Comparative Method. Complete accuracy with regards to identifying borrowed lexicon cannot be guaranteed. In contrast, phylogenetic model choices can be declared explicitly for evaluation by others and statistical uncertainty in the results can be quantified. This is not to make the case that the Comparative Method is flawed or that computational phylogenetic methods are superior (I would argue they serve varied functions and can coexist, rather than stand as direct competitors), only that no method is free of assumptions, limitations or problematic aspects, and this should be kept in mind before dismissing any particular method outright.

To summarise this section, analogies have been drawn between linguistic evolution and biological evolution, in both directions, for practically as long as phylogenetic tree diagrams have existed. The early history of these academic fields in particular is somewhat intertwined. Although methods in biology and linguistics diverged substantially in the 20th century, phylogenetic thinking, in essence, never really left historical linguistics. The Comparative Method infers patterns of the fundamental evolutionary process of descent with modification and classifies languages in a phylogenetic structure. In recent decades, historical linguistics has taken a quantitative turn and embraced some of the computational phylogenetic methods pioneered by evolutionary biology. Linguistic phylogenetics is still a young subfield and there exists huge scope for methodological development, of which this thesis is a part. Debate continues over the validity of applying phylogenetic methods to linguistic data, partly due to mistaken associations with the tarnished legacy of lexicostatistics and glottochronology. But there are also valid questions around linguistic data structures and their application in evolutionary models. This discussion is important, and need not result in dichotomised `for' and `against' camps of thought. In the chapters that follow, I engage with some of these questions through an empirical evaluation of phonotactic data in phylogenetic methods. In this way, I follow the suggestion of \textcite[p.~2299]{greenhill_does_2009}, who recommend that rather than engaging in ``armchair speculation'' on the suitability of phylogenetic methods ``on \emph{a priori} grounds'', it is more beneficial to engage with these methods and linguistic theory, crunch the numbers, and quantify how useful given methods are for testing various historical hypotheses.

\hypertarget{languages-of-australia}{%
\section{Languages of Australia}\label{languages-of-australia}}

This section gives a brief overview of the genealogical classifications and phonological systems of Australian languages, which make up the language samples in subsequent chapters of this thesis. This is followed in the next section with some contextual information on Sahul, the continent which includes Australia and the island of New Guinea. By doing so, I aim to place the studies in the following chapters in their broader continental context and foreshadow some points to be addressed in the Discussion chapter further on.

Somewhere between 250--300 (though perhaps more) distinct languages are understood to have been present in Australia at the time of European colonisation in the 18th century, which can be subdivided further into around 700-800 language varieties when dialectal variation is considered \autocite{koch_languages_2014}. These linguistic communities are characterised by small populations, with a high degree of multilingualism facilitating communication between groups \autocite{koch_languages_2014}.

Colonisation has led to devastating shifts in the linguistic landscape. Of around 120 languages presently spoken in some capacity, only a dozen are estimated to be in a ``relatively strong'' position of sustainability with regards to childhood first language speakers, excluding new language varieties such as Kriol \autocite[p.~42]{australian_government_department_of_infrastructure_transport_regional_development_and_communications_national_2020}. The effects of colonisation and language loss on the affected communities is a deeply important and multifaceted topic demanding of its own dedicated discussion. I do not wish to give an unjustly inadequate summary of this subject here. For the purposes of this thesis, I make note only of the direct consequences of colonisation and language loss for studies such as this one. Fundamentally, the rapid loss of languages in Australia and lack of value placed on Indigenous languages by colonists has led to Australian languages being under-resourced (with regard to documentation, grammatical description, dictionaries, research) and under-represented in global cross-linguistic research. For many languages, existing resources are old and of variable quality (e.g.~wordlists compiled by early settlers with no linguistic training). This limits what can be achieved with methods that demand high fidelity data and/or a high quantity of data. It can mean a relative lack of statistical representation in otherwise global typological datasets, which then affects our ability to make global scale generalisations about language. Consequently, overcoming these challenges is a priority for methodological development in large-scale comparative linguistics.

A broad understanding of the phylogenetics of Australian languages has been established through several decades of historical linguistic work using the Comparative Method. By far the largest family is Pama-Nyungan, which encompasses around two thirds of Australia's pre-colonial linguistic diversity and nearly 90\% of its landmass \autocite[p.~817]{bowern_computational_2012}. The Pama-Nyungan family was first formally proposed by \textcite{ogrady_languages_1966} as part of a continent-wide classification of Australian languages on lexicostatistical grounds. However, \textcite{hale_classification_1964} generally is credited with coining the language family's name, which is a compound of words for `person' in the southwestern and northeastern extremes of the continent \autocite[see also][]{wurm_aboriginal_1963}. This work was preceded by various earlier, approximately analogous groupings identified on the basis of lexical similarities \autocites[pp.~207--212]{grey_journals_1841}[p.~iv]{moorhouse_vocabulary_1846} and lexical cognacy (more systematically established than in earlier work) with a few phonological and grammatical features \autocites{schmidt_gliederung_1919}{kroeber_relationships_1923}. Capell's \autocites{capell_structure_1937}{capell_classification_1940} primarily typological classification of Australian languages can be included in these earlier precedents too, if one interprets Capell's description of `Common Australian' in the same way \textcite{koch_methodological_2004} does---that is, a proto-language ancestral to most but not all Australian language groups, and distinct from any hypothetical proto-Australian language, which Capell terms `Original Australian' (Koch's interpretation seems valid, based on Capell's \autocite*{capell_history_1979} later work). Despite some skepticism on the validity of the Comparative Method in Australia \autocites{dixon_australian_2002}[but see][ for a counter-argument]{ogrady_coherence_2004}, there has been an abundance of work establishing subgroups within Pama-Nyungan via the Comparative Method in the decades since \textcites{ogrady_languages_1966}[see an overview and several examples in][]{bowern_australian_2004}. There are presently around 36 identified subgroups within Pama-Nyungan plus several isolates \autocite[following][]{bowern_pama-nyungan_2018}, however, not all of these have been established thoroughly with the Comparative Method and the precise number may differ depending on the analyst and the standard of evidence they accept. In recent times, computational phylogenetic methods have been applied to discern higher order groupings above the level of these subgroups, as well as an estimated age and an estimated geographic point of origin for the Pama-Nyungan family. \textcite{bowern_computational_2012} presents computational phylogenetic evidence for four high-level divisions, clustering geographically into southeastern, northern, central and western groups. \textcite{bouckaert_origin_2018} recovers these same high-level groupings and further estimates a geographic point of origin in the Gulf of Carpentaria region, dated to the mid-Holocene period, between approximately 4,500--7,000 years ago.

The little over 10\% of the remaining Australian mainland once Pama-Nyungan is accounted for is home to around 27 language families and isolates \autocite{evans_non-pama-nyungan_2003}. These are located in two discontinuous areas of Australia's north---one in the Kimberley region and one in the Top End. In addition, Tasmania is home to at least one and possibly more independent language families, though the precise number of languages present in Tasmania at the time of colonisation and their family structure is difficult to discern, owing to the particularly devastating effects of colonisation and paucity of linguistic documentation on the island \autocite{bowern_riddle_2012}. In the north, there is scope for discrepancies on the exact number of families depending on certain classification choices. For example, Evans' figure of 27 families includes an `unclassified' group containing likely at least two family-level isolates \autocite[p.~11]{evans_non-pama-nyungan_2003}. Other factors affecting this figure include the treatment of Daly languages, assumed to be a single family by \textcite{tryon_daly_1974} but which have subsequently been argued to consist of up to four separate families \autocite[ch.~4--7]{evans_non-pama-nyungan_2003}; the reclassification of several languages into an expanded Gunwinyguan family and possibly even a larger `Arnhem' family \autocite[p.~14--15]{evans_non-pama-nyungan_2003}; and the place of the Tangkic languages and possible neighbouring isolate Minkin \autocites{evans_minkin_1990}{memmott_fission_2016}, the Garrwan languages, and Yanyuwa, all of which have been argued variously as either within or outside of Pama-Nyungan \autocite[p.~12]{evans_non-pama-nyungan_2003}. Whatever their precise classifications, it is clear that an exceptional degree of high order linguistic phylogenetic diversity is packed into a particularly small geographic area, in stark contrast to the geographic extensiveness of Pama-Nyungan. It has not been conclusively demonstrated via the Comparative Method how the various non-Pama-Nyungan families relate to one another nor how they relate to Pama-Nyungan. Nevertheless, curiously similar phonologies are present across Pama-Nyungan and non-Pama-Nyungan areas (discussed below in Section \ref{australian-language-phonologies}) and it is commonly assumed that all Australian language families group together under a putative `proto-Australian' ancestor, to the exclusion of all other world languages \autocite[e.g.@dixon\_languages\_1980;][]{hamilton_phonetic_1996}. In addition, \textcite{harvey_reconstructing_2017} argue for proto-Australian on the basis of evidence from nominal prefixes. \textcite{evans_non-pama-nyungan_2003} discusses some possibilities for family relationships underneath a proto-Australian root. Proposals include the following: A `rake model', in which all 27 or so non-Pama-Nyungan families and Pama-Nyungan descend equally from the same root with no intervening higher-order structure \autocite{ogrady_languages_1966}. A `binary model', in which proto-Australian splits into two sisters, proto-Pama-Nyungan and proto-non-Pama-Nyungan \autocite{heath_linguistic_1978}. This is approximately equivalent to Capell's \autocite*{capell_new_1956} typological division between prefixing and non-prefixing languages (though there are examples of prefixing Pama-Nyungan languages and non-prefixing non-Pama-Nyungan languages). A `Pama-Nyungan offshoot model', in which Pama-Nyungan, Garrwan, Tangkic and Gunwinyguan families are contained beneath a higher order `proto-macro-Pama-Nyungan' ancestor, and thereby more closely related to one another than to other non-Pama-Nyungan families \autocites{ogrady_preliminaries_1979}{evans_cradle_1997}{evans_enigma_1998}. Several decades of progress notwithstanding, relating non-Pama-Nyungan languages to one another, to Pama-Nyungan, and to a putative proto-Australian root remains an active priority for historical linguistics research in Australia.

This thesis concentrates primarily on the Pama-Nyungan family. Chapters \ref{ch-phylo-signal}--\ref{ch-tree-inference} use lexical data from 112 exclusively Pama-Nyungan language varieties. These varieties represent 26 of the 36 Pama-Nyungan subgroups and 3 isolates identified in \textcite{bowern_pama-nyungan_2018}. In Chapter \ref{ch-phon-freqs}, which does not require a pre-existing phylogenetic reference tree, an additional 54 languages are included representing all major non-Pama-Nyungan families.

\hypertarget{australian-language-phonologies}{%
\subsection{Australian language phonologies}\label{australian-language-phonologies}}

Australian language phonologies are typologically distinctive. Descriptions of the nature of language phonologies in Australia make reference to a uniquely constrained set of characteristics and the phonologies of individual Australian languages are frequently described in terms of a specific, narrow set of parameters of variation that follow from the broadly understood profile of what Australian phonologies look like \autocites[e.g.][p.~11]{goddard_grammar_1985}[p.~52]{evans_grammar_1995}[p.~7]{breen_grammar_2007}[p.~23]{gaby_grammar_2006}. This apparent homogeneity of phonological systems across Australia stands out as unusual for such a geographically expansive and phylogenetically diverse area, when compared to the phonological diversity observed in other parts of the world of similar areal and genealogical scope \autocites{maddieson_patterns_1984}{mielke_emergence_2008}. The first instance of this apparently constrained nature of Australian phonological variation appears to be \textcite{schmidt_gliederung_1919}. Subsequent references include \textcite{capell_new_1956}, \textcite{voegelin_obtaining_1963}, \textcite{dixon_languages_1980}, \textcite{busby_distribution_1982}, \textcite{hamilton_phonetic_1996}, \textcite{butcher_australian_2006}, \textcite{dixon_australian_2002}, and \textcite{baker_word_2014}, among others. There is, however, some effort to push back against this consensus and illuminate Australian phonological variation to a degree that is perhaps underappreciated within the field. \textcite{bowern_standard_2017} makes a case for treating generalised claims about ``Standard Average Australian'' language characteristics with caution, finding that many such characteristics described in the literature suffer from a lack of testability. In this light, \textcite{gasser_revisiting_2014} attempt to re-examine previous typological claims about Australian langauge phonologies in a more quantifiable way, using lexical data from 120 Australian language varieties. They concede the homogeneity of Australian segmental inventories, but demonstrate that there is considerably greater variation between languages with regards to the frequencies of phonological segments, rather than simply the segmental inventories themselves. Most recently, \textcite{round_segment_2021} quantifies some of the key characteristics traditionally said to define Australian language phonologies within six major divisions of the Pama-Nyungan family plus 13 other Australian language families. He finds that five parameters of variation are sufficient to describe the consonantal phoneme inventories of around two thirds of the language sample. These parameters are:

\begin{itemize}
\tightlist
\item
  One versus two contrastive apical places of articulation.
\item
  One versus two contrastive laminal places of articulation.
\item
  The presence or absence of a contrastive glottal stop.
\item
  Whether there is a single lateral phoneme, a maximal number of laterals equal to the number of contrastive coronal places of articulation, or some `submaximal' number of laterals in between.
\item
  One versus two contrastive series of plosives.
\end{itemize}

For the approximately two thirds of Australian languages that fit within these five parameters of variation, the common characteristics that define the rest of the phoneme inventory are as follows:

\begin{itemize}
\tightlist
\item
  A contrastive labial and dorsal place of articulation (which are often analysed as forming their own `peripheral' natural class).
\item
  A maximal set of contrastive nasals, one at each supralaryngealplace of articulation.
\item
  A labio-velar and palatal glide.
\item
  Two contrastive `rhotic' phonemes---an alveolar tap/trill and a retroflex glide.
\item
  No contrastive fricatives (notable in light of the cross-linguistic abundance of fricatives elsewhere in the world).
\end{itemize}

Quantification of these parameters reveals some interesting nuance to previous characterisations of Australian phonologies. For example, although Australian languages seem roughly evenly split on the first two points (one versus two apical places and one versus two laminal places) in raw terms, only the laminals parameter is split relatively evenly at the family level (or level of major subclades, in the case of Pama-Nyungan). Languages with a single apical series are present in relatively few language families but happen to be common among two of the largest subclades of Pama-Nyungan. Thus, although there appears to be a relatively even split between individual languages with one versus two apical series, \textcite{round_segment_2021} demonstrates that the lack of an apical distinction is relatively rare at the genealogical level. With regards to the other parameters of variation, glottal stops are restricted genealogically and in gross terms, being present only in a few relatively small language families and two Pama-Nyungan subgroups, all in the north of the country. The subject of plosive contrasts in Australian languages is more complex. Languages vary in the phonetic factors by which plosive contrasts are maintained, and this contrast is sometimes phonologically analysed as a singleton/geminate distinction rather than a voiced/voiceless distinction or some other fortis/lenis distinction \autocite[for a more detailed overview, see][]{round_segment_2021}.

A variety of other segments are found in the remaining one third of Australian languages that do not fit this paradigm. Languages with contrastive fricatives are found in languages in the Cape York and Daly River regions, the former of which have been analysed as historically deriving from stops \autocite[p.~199]{dixon_languages_1980}. Other exceptions are a small number of languages (around 2\% for each) that are either missing a nasal phoneme at a supralaryngeal place of articulation or lack one of the rhotic tap/trill or glide phonemes \autocite{round_segment_2021}. In addition, various complex segment types have been argued to exist in a number of languages, not always with uniform agreement \autocite[see][]{round_segment_2021}. A classic example is the series of labialised consonants proposed for Arandic languages \autocites{wilkins_mparntwe_1989}{breen_wonders_2001}.

With regards to vowels, Australian languages are most typically described as featuring a triangular three-vowel system, optionally with a length distinction. Other common variations are the addition of mid vowels and the addition of a non-low central vowel. Rarer exceptions are nasalised vowels in Anguthimri \autocite{crowley_mpakwithi_1981}, contrastive rounding distinctions between vowels of identical height and frontness in particular Daly, Giimbiyu and Paman languages \autocite{round_segment_2021}, and phonemic diphthongs (more typically analysed as vowel-glide-vowel sequences in Australian languages) in Mbarrumbathama \autocite{verstraete_mbarrumbathama_2019}.

Beyond segmental inventories, remarkable consistency among Australian languages has been described in phonetics \autocite{dixon_languages_1980} and phonotactics \autocites{dixon_languages_1980}{hamilton_phonetic_1996}{baker_word_2014}. Turning now to phonotactics, \textcite{hamilton_phonetic_1996} describes an ostensibly considerable degree of variation between languages. However, he finds that this variation is ``systematic and highly constrained'' \autocite[p.~29]{hamilton_phonetic_1996}. Hamilton refers to this underlying similarity, in which the superficial phonotactic variation that does exist can be understood within the tightly prescribed bounds of a uniform, continent-wide system, as \emph{concert}. \textcite{round_phonotactics_2021} quantifies the phonotactic characteristics used to describe Australian languages and the systematic parameters of variation identified by \textcite{hamilton_phonetic_1996}, in a similar vein to his quantification of phonemic generalisations described above.

One notable feature of Australian phonotactic systems is that they cannot be described easily using the syllable as the fundamental unit of organisation, where phonotactic rules and constraints are localised to prosodic positions within the syllable \autocites[e.g.][]{ito_syllable_1988}{goldsmith_autosegmental_1990}. Australian phonotactic systems are better served by a disyllabic structure as the fundamental unit of organisation and, resultingly, few Australian languages permit words of fewer than two syllables \autocite{dixon_languages_1980}. \textcite[p.~75]{hamilton_phonetic_1996} presents two disyllabic templates, following \textcite{dixon_languages_1980}, presented in \eqref{eq:disyll-template-1} and \eqref{eq:disyll-template-2} below, where \(C_{init}\) is a word-initial consonant, \(C_{inter}\) is an intervocalic consonant, \(C_{fin}\) is an optional word-final consonant, \(C_1\) is a pre-consonantal consonant and \(C_2\) is a post-consonantal consonant.

\begin{equation}
C_{init}VC_{inter}V(C_{fin})
\label{eq:disyll-template-1}
\end{equation}

\begin{equation}
C_{init}VC_1.C_2V(C_{fin})
\label{eq:disyll-template-2}
\end{equation}

The consonantal slots in \eqref{eq:disyll-template-1} and \eqref{eq:disyll-template-2} are characterised by the following salient points, as described by \textcite{dixon_languages_1980}, \textcite{hamilton_phonetic_1996} and \textcite{baker_word_2014}.

\begin{itemize}
\tightlist
\item
  Of all consonant positions, \(C_{inter}\) is generally the only one in which all consonants may appear.
\item
  Apicals are typically restricted from appearing in \(C_{init}\) position, and peripherals are preferred over laminals. Obstruents are preferred over sonorants for both \(C_{init}\) and \(C_2\) positions.
\item
  Conversely, sonorants are preferred over obstruents in \(C_{fin}\) and \(C_1\) positions. Coronal consonants are also preferred in this position over peripherals in these positions.
\item
  Many languages do not include the \(C_{fin}\) position at all, with words necessarily being vowel-final, e.g.~Anindhilyakwa \autocite{van_egmond_enindhilyakwa_2012}, Kayardild \autocite{evans_grammar_1995}, and Diyari \autocite{austin_grammar_1981}.
\end{itemize}

Homorganic nasal+stop clusters are almost universal, although Thargari \autocite{klokeid_thargari_1969} is an unusual exception, where homorganic nasal+stop sequences have developed into a series of fortis stop phonemes, so almost all the remaining nasal+stop clusters are heterorganic \autocite{austin_grammar_1981}. One particularly distinctive feature of Australian language phonotactics is the commonality of a rich array of heterorganic nasal+stop clusters \autocites{baker_word_2014}{fletcher_sound_2014}{round_phonotactics_2021}. These are typically restricted in frequency and type to a greater degree than their homorganic counterparts \autocite[pp.~78--82]{hamilton_phonetic_1996}, but are nevertheless noteworthy given the rarity and susceptibility of heterorganic nasal+stop clusters to assimilation in other parts of the world \autocite[p.~144]{baker_word_2014}. Lateral+stop clusters are also common, and in this case, heterorganic combinations are more common while homorganic clusters are dispreferred or restricted all together.

In examining the overall typology of Australian language phonologies and phonotactics, one point to consider is the generally non-deterministic nature of phonological analysis \ref{ch-phylo-signal}, there are many criteria that are factored into the determination of a language's segmental inventory and, consequently, multiple possible solutions depending on the application and analytic order of a particular linguist. This has implications for the broad typology of Australian phonology and phonotactics, with particular regards to the impression of homogeneity in these systems. One such criterion for segmental-phonological analysis in an individual language is phonotactic parsimony. For example, by one analysis, Alawa features a contrastive series of prenaslised stop segments \autocite{sharpe_alawa_1972} and, likewise, the same has been described for Tiwi \autocite{lee_tiwi_1987}. The motivation for these analyses is that nasal+stop sequences would otherwise be found in the \(C_1\) position of the disyllabic templates above. These so-called complex segments enable the preservation of a parsimonious phonotactic analysis and a phonotactic analysis that is in concert with fellow Australian languages. However, this is at the expense of creating expanded segmental inventories for Tiwi and Alawa that diverge from the standard average Australian characteristics described above. An alternative analysis of Alawa and Tiwi phonologies might place a lower value on phonotactic criteria and, for example, place a relatively greater emphasis on finding minimally contrastive pairs. Word-initial nasal+stop sequences could be analysed as separate segments, producing smaller, simpler segmental inventories that are more in keeping with the typological standard. This, however, would require acceptance of a greater degree of phonotactic variation in Australia and revision of disyllabic templates \eqref{eq:disyll-template-1} and \eqref{eq:disyll-template-2} if these templates are to be held as applicable Australia-wide. \textcite{round_phonotactics_2021} gives an extended analysis of prenasalised stops and also prestopped laterals and prestopped nasals that have been proposed in various languages around the country via systematic assessment of the distribution of complex segments in each word position and comparison to the permissibility of other comparable clusters in each language. He finds that, for the languages for which stop+lateral sequences have been analysed as unitary prestopped lateral segments, these sequences are licensed to appear in the same positions as other bisegmental clusters and, therefore, are equally consistent with being analysed as either single segments or bisegmental clusters on phonotactic grounds. The same is true for stop+nasal sequences, which have been analysed as prestopped nasals, in languages of the Thura-Yura and Karnic subgroups. However, several other Pama-Nyungan subgroups (Arandic, Paman, Kulin, plus Djinang of the Yolngu subgroup) permit stop+nasal sequences in positions in which single nasal segments are permitted but not other clusters, thus stop+nasal segments pattern phonotactically more like single segments in these languages. Together, these examples serve to illustrate the point that, although Australian phonologies and phonotactics are often held to be homogeneous to a considerable degree, various exceptions to the norm do exist and, furthermore, where and how variation manifests is, in part, a product of the analytic decisions of the individual typologist and/or the individual linguists who document languages. I take these issues into account explicitly in the comparative studies that follow in subsequent chapters.

\hypertarget{sahul-context}{%
\section{Australian languages within the continent of Sahul}\label{sahul-context}}

Sahul is the continent that prinicipally includes the Australian mainland, Tasmania and the island of New Guinea. These form a single landmass when sea levels are low, as was the case throughout most of Sahul's human history. The current separation of Tasmania and New Guinea from the Australian mainland is the result of rapidly rising sea levels around the beginning of the Holocene era. Specifically, this includes the inundation of the Arafura Sill (a land bridge connecting Arnhem Land to New Guinea) and Lake Carpentaria (an enormous fresh water body situated in the area of the present-day Gulf of Carpentaria) around 13 ka; the inundation of Bass Strait, separating Tasmania from the Australian mainland approximately 11 ka; and finally, the inundation of Torres Strait, separating Cape York from New Guinea approximately 8 ka \autocites{reeves_sedimentary_2008}{williams_sea-level_2018}. These dates are of interest here in a couple of respects. Firstly, for at least five sixths of the history of occupation, Sahul was a single landmass (see below for discussion of human migration). Secondly, this trio of inundation events tantalisingly overlap with the maximal limit for linguistic reconstruction via the Comparative Method, which is commonly cited as around 10,000 years \autocite[p.~135]{nichols_sprung_1997}.

The initial peopling of Sahul is an active and rapidly developing area of interest. The earliest date of human arrival in the continent has been revised substantially in recent years. The earliest archaeological evidence currently comes from the Madjedbebe rock shelter in Arnhem Land, dated to around 65 ka \autocites{clarkson_human_2017}[see also][]{florin_first_2020}, which pushed back earlier estimates by 5,000--18,000 years at the time \autocites{oconnell_process_2015}{clarkson_archaeology_2015}. One of the challenges for dating the earliest migration to Sahul has been Australia's relatively fragmented archaeological record \autocite{bradshaw_minimum_2019} but, in addition to continuing archaeological work, the triangulation of evidence from ancient DNA, computational modelling and linguistic evidence is helping to clarify the picture. Another recent advance includes identification of Australia's first underwater archaeological site \autocite{benjamin_aboriginal_2020}. Genomic evidence generally points to an early intial colonisation of Sahul, consistent with archaeological evidence, and subsequent isolation from populations outside of Sahul in the tens of thousands of years that followed \autocites{hudjashov_revealing_2007}{nagle_aboriginal_2017}{pedro_papuan_2020}. In addition, computational modelling has been used to simulate possible migration routes taken to reach Sahul from Sunda to the north. \textcite{bird_early_2019} model historical coastlines and ocean drift and find that, even with low sea levels, humans would be unlikely to reach Sahul by random drift. Rather, \textcite{bird_early_2019} conclude that the first migrants to Sahul likely had the capacity to undertake multi-day open sea voyages---remarkable, given the antiquity of such voyages and the implications for our understanding of technological capacity of early modern humans. \textcite{bradshaw_minimum_2019} model the minimum founding population size that would be required for this initial migration and find that between 1,300 and 1,550 people would be required at a minimum. These individuals would either have to migrate all at once or in smaller voyages of at least 130 people successively over 700--900 years. There are two main proposed routes that these early seafarers might have taken to reach Sahul, proposed by \textcite{birdsell_recalibration_1977} and debated ever since. These are a northern route into New Guinea via Sulawesi, or a southern route into north-western Australia via Timor and Bali. \textcite{bradshaw_minimum_2019} determine that the northern route would be easier and therefore more likely to support a successful migration, requiring as few as three crossings, all of which could be made in 2--3 days (assuming good conditions) and all while maintaining sight of the destination island. \textcite{kealy_least-cost_2018} similarly favours the northern route as the easier of the two, though, as \textcite{bradshaw_minimum_2019} points out, more complex routes would not necessarily be beyond the first migrants depending on their level of seafaring skills and technology.

In a landmark study, \textcite{malaspinas_genomic_2016} inferred more detailed migration movements within Sahul. They find an initial diversification of Papuans from Indigenous Australians at 25--40 ka, a common ancestor to all Indigenous Australians in their sample 10--32 ka, and, significantly, a population expansion out of Austrlia's north-east within the last 10,000 years, potentially consistent with the expansion of Pama-Nyungan languages. \textcite{malaspinas_genomic_2016} compared their genomic evidence to a linguistic phylogeny inferred from lexical cognate data. Curiously, the topologies of these trees are highly congruent but apply on different timescales, the genomic tree being at least twice as old as the linguistic one. The explanation for this discrepancy in timing is uncertain. A subsequent study of mitochondrial DNA by \textcite{tobler_aboriginal_2017} similarly finds evidence for an initial ancient migration followed by rapid spread around Australia following the coasts. However, unlike \textcite{malaspinas_genomic_2016}, \textcite{tobler_aboriginal_2017} find that populations remained remarkably geographically static for the 50,000 years or so following initial migration and settlement. They dispute the evidence for a more recent Holocene-era expansion, saying the ``genetic signal remains ambiguous at best'' \autocite[p.~183]{tobler_aboriginal_2017}.

This discussion draws back importantly to the discussion of the Pama-Nyungan family of languages in Section \ref{languages-of-australia} above. The unresolved question of whether or not genetic evidence favours a rapid Holocene expansion relates to the linguistic question of how the Pama-Nyungan family came to dominate so much of the Australian conteninent. Note that there is no \emph{a priori} reason why a rapid linguistic expansion should necessarily be accompanied by a rapid genetic expansion as well, but this would have implications for the method of linguistic expansion. It could be the case that some kind of social context arises in which there is rapid language shift but speakers nevertheless remain in their ancestral homelands, as opposed to, say, a conquest situation that would leave a genetic signature. In the case of Pama-Nyungan, several theories have been put forward to explain its expansion, such as a rapid mid-Holocene spread of small tool technology and/or social/ceremonial restructuring \autocite{evans_enigma_1998}, or expansion from refugia from the Last Glacial Maximum as climactic conditions became more favourable in the early Holocene \autocite{bouckaert_origin_2018}. \textcite{bouckaert_origin_2018} conducts a phylogeographic study using lexical cognate data and finds support for a geographic point of origin around the Gulf of Carpentaria. This point of origin is consistent with the Holocene expansion identified by \textcite{malaspinas_genomic_2016}, but, once again, the inferred age of the linguistic phylogeny is much younger than the age inferred from genomic evidence in \textcite{malaspinas_genomic_2016}. The age of Pama-Nyungan inferred by \textcite{bouckaert_origin_2018} is more consistent with a rapid mid-Holocene expansion faciliated through technological and cultural advantages.

Taking a step back, the linguistic prehistory of Sahul remains enigmatic in historical linguistics. The question of if and how Australian languages might relate to their Papuan neighbours to the north has been the subject of speculation for many years \autocites[e.g.][]{ogrady_languages_1966}{wurm_papuan_1975}. But, besides a handful of tentative cognates \autocite{foley_papuan_1986} and shared structural features \autocites{nichols_sprung_1997}{reesink_explaining_2009}, conclusive evidence for a connection remains elusive. Of course, there has been much successful genealogical classification within Sahul---many small families plus the very large Pama-Nyungan family have been well established in Australia and, similarly, many small families plus the very large Trans-New Guinea family have been identified in New Guinea. Nevertheless, the diagnosis of deeper historical relations is handicapped by several factors in this part of the world. These limitations include a lack of adequate description \autocite{bowern_computational_2012}, an absence of pre-colonial written sources \autocite{foley_papuan_1986}, apparent homogeneity of phonological systems \autocites{baker_word_2014}{round_segment_2021}{round_phonotactics_2021} and millennia of extensive horizontal diffusion \autocites{foley_papuan_1986}{dixon_australian_2002} \autocite[although the extent to which this horizontal diffusion exists may be disputed. c.f.][]{bowern_does_2011}. Even in ideal circumstances, the maximal time-depth of the Comparative Method is usually assumed to be around 10,000 years, as noted above. Although this overlaps with the land bridge between Cape York and New Guinea, Arnhem Land had already been separated by the Gulf of Carpentaria and Tasmania had already been separated from the Australian mainland by 10 ka.

Recent methodological advances, as discussed in Section \ref{academic-context}, give cause for optimism that we may be able to extend the time-depth at which historical linguistics can operate and unravel something of the prehistory of Sahul. As for data, and the possible erosion of historical signal after 10,000 years in lexical data specifically, two observations motivate the approach of the present study: Firstly, linguistics has well and truly entered `the age of big data' and computational methods now enable us to extract minute threads of significance from large volumes of data. This enables us to compare more data points across wider groups of languages quickly and efficiently, and identify trends and correlations which would otherwise escape the attention of even the most highly trained human eye. This thesis evaluates the profitability of complementing lexical data (which continues to form the backbone of most historical linguistic work) with other kinds of data too---namely, phonotactics.

Amid present uncertainty and conflicting findings, it seems clear that linguistics has an important place within the wider science of human history. It offers a unique perspective and level of granularity within a particular time-depth that is not necessarily captured by genomic evidence alone. As seen in the case of the expansion of Pama-Nyungan, the combination of linguistic evidence with genomic evidence offers the potential to infer not just who was where and when, but also to give a more nuanced view of why, since the combination of evidence can indicate the method of expansion in a way that either piece of evidence in isolation could not. Continuing advances in both fields will help to resolve current ambiguities and the future of this multidisciplinary endeavour looks bright.

\hypertarget{lit-rev-conclusion}{%
\section{Conclusion}\label{lit-rev-conclusion}}

This chapter began with a walk through the early history of historical linguistics as a discipline, and the development of phylogenetic thinking within. The phylogenetic tree, a graphical devices for charting past relationships of shared descent with modification, shares its early origin in linguistics as well as biology. Methodologically, both fields divided substantially in the 20th century, with the Comparative Method the enduring gold standard in historical linguistics. Meanwhile, advances in genetics led biologists to embrace quantitative historical methods. In recent times, computational phylogenetics has taken an increasingly mainstream place within historical linguistics. Section \ref{academic-context} listed numerous examples of linguistic phylogenies inferred through computational phylogenetic methods. Beyond tree inference, however, Section \ref{sahul-context} shows how linguistic phylogenetics is playing an increasingly valuable role in multidisciplinary investigations of deep-time human history.

Section \ref{languages-of-australia} and Section \ref{australian-language-phonologies} give an overview of the Australian languages that are the focus of this thesis and their phonologies. The most numerically and geographically expansive family, Pama-Nyungan, is of particular interest since it is the basis of the language samples in Chapters \ref{ch-phon-freqs}--\ref{ch-tree-inference}. This ties into the discussion of Sahul's human history in Section \ref{sahul-context}, since the precise nature and timing of Pama-Nyungan's dramatic expansion across the Australian mainland remains uncertain and subject to active investigation. Continuing linguistic resesarch will be essential to advancing our understanding on this topic. Linguists should not fall into the self-depreciating assumption that geneticists will `come in and solve everything'.\footnote{This may be an obvious point to some but I will admit it was not immediately obvious to me.} Firstly, as seen in Section \ref{sahul-context}, ancient DNA research produces uncertainty and conflicting evidence of its own and has its own limitations (e.g.~expense and difficulty of data collection). Moreover, the linguistic record will have its own story to tell, which may or may not be correlated with other sources of evidence, offering nuanced insights into human history.

The goal of this thesis is to contribute to the methodological development of linguistic phylogenetics by investigating a novel source of data---quantitative phonotactic information extracted from Australian language wordlists. In doing so, however, I aim not simply to analogise between biological evolution and linguistic evolution but to develop a model of phylogenetic phonotactics critically, and where necessary from the ground up, using presently available quantitative tools. The first step in this process is a re-evaluation of individual phoneme frequencies in Australian languages, which follows in the next chapter.

% ***************************************************